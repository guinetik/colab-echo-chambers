\section{Objetivo Geral}
Investigar o fenômeno das câmaras de eco no aplicativo de cidadania brasileiro Colab e propor estratégias para detectar, compreender e mitigar esse fenômeno em redes sociais de grafo direcionado. O estudo tem como objetivo aprofundar a compreensão das câmaras de eco no contexto do Colab, explorando seu impacto na percepção, engajamento cívico e polarização política. Ao combinar insights da modelagem baseada em agentes, epidemiologia digital, neurologia clássica apoiados em uma fenomenologia exploratória alinhada com o pensamento de Heidegger, espera-se contribuir para a promoção de uma maior diversidade de perspectivas e o fortalecimento dos valores democráticos no contexto das plataformas de mídia social.

A pesquisa será dividida em quatro etapas:

\begin{enumerate}
  \item Análise exploratória da rede e interações dos usuários no Colab.
  \item Proposição de heurísticas para detecção de câmaras de eco.
  \item Investigação da formação de câmaras de eco por meio de simulações computacionais.
  \item Desenvolvimento de uma aplicação web para demonstrar as heurísticas de detecção de câmaras de eco e o treinamento de modelos de aprendizado de máquina para classificação de postagens.
\end{enumerate}

\section{Objetivos Específicos}
\begin{enumerate}
	\item Analisar a topologia e estrutura do grafo da rede social Colab bem como a frequência de postagens e interações dos usuários, mapeando o engajamento e as conexões dos usuários na plataforma. Identificar comunidades distintas e utilizar métricas de análise de redes sociais para compreender a estrutura da rede e identificar os nós mais influentes. Essa análise exploratória fornecerá uma base sólida para a compreensão inicial das características da rede Colab e sua relação com as câmaras de eco.

	\item Adaptar as heurísticas relevantes da metodologia de \citeonline{2023_Atiqi_BOOK}, como o Parâmetro Global de Câmara de Eco (GEC), o Coeficiente de Agrupamento de Exposição (ECC) e a Exposição Média (AE) - tradicionalmente utilizados para mensurar a difusão de notícias em uma rede - para a funcionalidade de zeladoria pública do Colab. Essa adaptação permitirá a aplicação dessas métricas específicas para o contexto da plataforma, enriquecendo a compreensão das câmaras de eco e sua influência nas discussões de zeladoria pública.

	\item Classificar as postagens dos usuários utilizando técnicas avançadas de análise de sentimento e aprendizado de máquina supervisionado. Essa classificação permitirá criar métricas personalizadas, como o \textit{score} e a \textit{persona}, que subsidiarão o cálculo dos parâmetros GEC, ECC e AE. Essa abordagem proporcionará uma avaliação mais precisa do conteúdo das postagens e sua relevância para a formação das câmaras de eco, contribuindo para a compreensão das dinâmicas e dos perfis de usuários envolvidos.

	\item Propor um modelo algorítmico de detecção de câmaras de eco baseado em heurísticas inspiradas em técnicas tradicionais de análise de redes e teoria dos grafos. Esse modelo proporcionará uma metodologia sistemática e eficiente para a detecção automatizada das câmaras de eco na rede Colab, facilitando a identificação e a compreensão desse fenômeno complexo.

	\item Investigar a formação de câmaras de eco no Colab por meio de Modelagem Baseada em Agentes (ABM), simulações de Monte Carlo e modelos de epidemiologia digital. Desenvolver modelos de agentes que representam os usuários da plataforma e suas interações. Explorar intervenções potenciais, como sugestões e campanhas de informação, com o objetivo de incentivar os usuários a se envolverem com conteúdos fora de suas câmaras de eco. Essa abordagem permitirá uma simulação dinâmica e realista das interações entre os usuários, fornecendo insights sobre a formação das câmaras de eco e suas possíveis soluções.

	\item Desenvolver uma aplicação web para demonstrar as heurísticas de detecção de câmaras de eco e o treinamento de modelos de aprendizado de máquina para classificação de postagens. Essa aplicação prática das heurísticas desenvolvidas permitirá a avaliação em tempo real das câmaras de eco na plataforma Colab, facilitando a visualização e compreensão desse fenômeno para os usuários e tomadores de decisão.
\end{enumerate}

O resultado do estudo será condensado no Colab:GraphScan, um aplicativo especializado que permitirá a detecção automatizada de câmaras de eco na plataforma Colab. O aplicativo possibilitará o treinamento de modelos de aprendizado de máquina para classificação de postagens e fornecerá uma interface de visualização das câmaras de eco detectadas. Essa ferramenta prática permitirá a avaliação em tempo real do fenômeno das câmaras de eco, facilitando sua compreensão para os usuários e tomadores de decisão.