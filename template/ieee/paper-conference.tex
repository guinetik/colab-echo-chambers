% This template has been tested with IEEEtran of 2015.

% !TeX spellcheck = en-US
% LTeX: language=en-US
% !TeX encoding = utf8
% !TeX program = pdflatex
% !BIB program = bibtex
% -*- coding:utf-8 mod:LaTeX -*-

% DO NOT DOWNLOAD IEEEtran.cls - Use the one of your LaTeX distribution
% For the final version, replace "draftcls" by "final"
\documentclass[conference,a4paper]{IEEEtran}[2015/08/26]

% Balance the last page
% The pbalance package (see https://ctan.org/pkg/pbalance) "just works" (in contrast to balance.sty or other solutions)
\usepackage{pbalance}

% backticks (`) are rendered as such in verbatim environments.
% See following links for details:
%   - https://tex.stackexchange.com/a/341057/9075
%   - https://tex.stackexchange.com/a/47451/9075
%   - https://tex.stackexchange.com/a/166791/9075
\usepackage{upquote}

% Set English as language and allow to write hyphenated"=words
%
% Even though `american`, `english` and `USenglish` are synonyms for babel package (according to https://tex.stackexchange.com/questions/12775/babel-english-american-usenglish), the llncs document class is prepared to avoid the overriding of certain names (such as "Abstract." -> "Abstract" or "Fig." -> "Figure") when using `english`, but not when using the other 2.
% english has to go last to set it as default language
\usepackage[ngerman,main=english]{babel}
%
% Hint by http://tex.stackexchange.com/a/321066/9075 -> enable "= as dashes
\addto\extrasenglish{\languageshorthands{ngerman}\useshorthands{"}}

% Links behave as they should. Enables "\url{...}" for URL typesettings.
% Allow URL breaks also at a hyphen, even though it might be confusing: Is the "-" part of the address or just a hyphen?
% See https://tex.stackexchange.com/a/3034/9075.
\usepackage[hyphens]{url}

% When activated, use text font as url font, not the monospaced one.
% For all options see https://tex.stackexchange.com/a/261435/9075.
% \urlstyle{same}

% Improve wrapping of URLs - hint by http://tex.stackexchange.com/a/10419/9075
\makeatletter
\g@addto@macro{\UrlBreaks}{\UrlOrds}
\makeatother

% nicer // - solution by http://tex.stackexchange.com/a/98470/9075
% DO NOT ACTIVATE -> prevents line breaks
%\makeatletter
%\def\Url@twoslashes{\mathchar`\/\@ifnextchar/{\kern-.2em}{}}
%\g@addto@macro\UrlSpecials{\do\/{\Url@twoslashes}}
%\makeatother


% use nicer font for code
\usepackage[zerostyle=b,scaled=.75]{newtxtt}

% Has to be loaded AFTER any font packages. See https://tex.stackexchange.com/a/2869/9075.
\usepackage[T1]{fontenc}

% Character protrusion and font expansion. See http://www.ctan.org/tex-archive/macros/latex/contrib/microtype/

\usepackage[
  babel=true, % Enable language-specific kerning. Take language-settings from the languge of the current document (see Section 6 of microtype.pdf)
  expansion=alltext,
  protrusion=alltext-nott, % Ensure that at listings, there is no change at the margin of the listing
  final % Always enable microtype, even if in draft mode. This helps finding bad boxes quickly.
        % In the standard configuration, this template is always in the final mode, so this option only makes a difference if "pros" use the draft mode
]{microtype}

% \texttt{test -- test} keeps the "--" as "--" (and does not convert it to an en dash)
\DisableLigatures{encoding = T1, family = tt* }

%\DeclareMicrotypeSet*[tracking]{my}{ font = */*/*/sc/* }%
%\SetTracking{ encoding = *, shape = sc }{ 45 }
% Source: http://homepage.ruhr-uni-bochum.de/Georg.Verweyen/pakete.html
% Deactiviated, because does not look good

\usepackage{graphicx}

% Diagonal lines in a table - http://tex.stackexchange.com/questions/17745/diagonal-lines-in-table-cell
% Slashbox is not available in texlive (due to licensing) and also gives bad results. Thus, we use diagbox
\usepackage{diagbox}

\usepackage{xcolor}

% Code Listings
\usepackage{listings}

\definecolor{eclipseStrings}{RGB}{42,0.0,255}
\definecolor{eclipseKeywords}{RGB}{127,0,85}
\colorlet{numb}{magenta!60!black}

% JSON definition
% Source: https://tex.stackexchange.com/a/433961/9075

\lstdefinelanguage{json}{
    basicstyle=\normalfont\ttfamily,
    commentstyle=\color{eclipseStrings}, % style of comment
    stringstyle=\color{eclipseKeywords}, % style of strings
    numbers=left,
    numberstyle=\scriptsize,
    stepnumber=1,
    numbersep=8pt,
    showstringspaces=false,
    breaklines=true,
    frame=lines,
    % backgroundcolor=\color{gray}, %only if you like
    string=[s]{"}{"},
    comment=[l]{:\ "},
    morecomment=[l]{:"},
    literate=
        *{0}{{{\color{numb}0}}}{1}
         {1}{{{\color{numb}1}}}{1}
         {2}{{{\color{numb}2}}}{1}
         {3}{{{\color{numb}3}}}{1}
         {4}{{{\color{numb}4}}}{1}
         {5}{{{\color{numb}5}}}{1}
         {6}{{{\color{numb}6}}}{1}
         {7}{{{\color{numb}7}}}{1}
         {8}{{{\color{numb}8}}}{1}
         {9}{{{\color{numb}9}}}{1}
}

\lstset{
  % everything between (* *) is a latex command
  escapeinside={(*}{*)},
  %
  language=json,
  %
  showstringspaces=false,
  %
  extendedchars=true,
  %
  basicstyle=\footnotesize\ttfamily,
  %
  commentstyle=\slshape,
  %
  % default: \rmfamily
  stringstyle=\ttfamily,
  %
  breaklines=true,
  %
  breakatwhitespace=true,
  %
  % alternative: fixed
  columns=flexible,
  %
  numbers=left,
  %
  numberstyle=\tiny,
  %
  basewidth=.5em,
  %
  xleftmargin=.5cm,
  %
  % aboveskip=0mm,
  %
  % belowskip=0mm,
  %
  captionpos=b
}

% Enable Umlauts when using \lstinputputlisting.
% See https://stackoverflow.com/a/29260603/873282 für details.
% listingsutf8 did not work in June 2020.
\lstset{literate=
  {á}{{\'a}}1 {é}{{\'e}}1 {í}{{\'i}}1 {ó}{{\'o}}1 {ú}{{\'u}}1
  {Á}{{\'A}}1 {É}{{\'E}}1 {Í}{{\'I}}1 {Ó}{{\'O}}1 {Ú}{{\'U}}1
  {à}{{\`a}}1 {è}{{\`e}}1 {ì}{{\`i}}1 {ò}{{\`o}}1 {ù}{{\`u}}1
  {À}{{\`A}}1 {È}{{\'E}}1 {Ì}{{\`I}}1 {Ò}{{\`O}}1 {Ù}{{\`U}}1
  {ä}{{\"a}}1 {ë}{{\"e}}1 {ï}{{\"i}}1 {ö}{{\"o}}1 {ü}{{\"u}}1
  {Ä}{{\"A}}1 {Ë}{{\"E}}1 {Ï}{{\"I}}1 {Ö}{{\"O}}1 {Ü}{{\"U}}1
  {â}{{\^a}}1 {ê}{{\^e}}1 {î}{{\^i}}1 {ô}{{\^o}}1 {û}{{\^u}}1
  {Â}{{\^A}}1 {Ê}{{\^E}}1 {Î}{{\^I}}1 {Ô}{{\^O}}1 {Û}{{\^U}}1
  {Ã}{{\~A}}1 {ã}{{\~a}}1 {Õ}{{\~O}}1 {õ}{{\~o}}1
  {œ}{{\oe}}1 {Œ}{{\OE}}1 {æ}{{\ae}}1 {Æ}{{\AE}}1 {ß}{{\ss}}1
  {ű}{{\H{u}}}1 {Ű}{{\H{U}}}1 {ő}{{\H{o}}}1 {Ő}{{\H{O}}}1
  {ç}{{\c c}}1 {Ç}{{\c C}}1 {ø}{{\o}}1 {å}{{\r a}}1 {Å}{{\r A}}1
}

% For easy quotations: \enquote{text}
% This package is very smart when nesting is applied, otherwise textcmds (see below) provides a shorter command
\usepackage[autostyle=true]{csquotes}

% Enable using "`quote"' - see https://tex.stackexchange.com/a/150954/9075
\defineshorthand{"`}{\openautoquote}
\defineshorthand{"'}{\closeautoquote}

% Nicer tables (\toprule, \midrule, \bottomrule)
\usepackage{booktabs}

% Extended enumerate, such as \begin{compactenum}
\usepackage{paralist}

% Bibliopgraphy enhancements
%  - enable \cite[prenote][]{ref}
%  - enable \cite{ref1,ref2}
% Alternative: \usepackage{cite}, which enables \cite{ref1, ref2} only (otherwise: Error message: "White space in argument")

% Doc: http://texdoc.net/natbib
\usepackage[%
  square,        % for square brackets
  comma,         % use commas as separators
  numbers,       % for numerical citations;
  %sort           % orders multiple citations into the sequence in which they appear in the list of references;
  sort&compress % as sort but in addition multiple numerical citations
                  % are compressed if possible (as 3-6, 15);
]{natbib}

% Same fontsize as without natbib
\renewcommand{\bibfont}{\normalfont\footnotesize}

% Enable hyperlinked author names in the case of \citet
% Source: https://tex.stackexchange.com/a/76075/9075
\usepackage{etoolbox}
\makeatletter
\patchcmd{\NAT@test}{\else \NAT@nm}{\else \NAT@hyper@{\NAT@nm}}{}{}
\makeatother

% Enable nice comments
\usepackage{pdfcomment}

\newcommand{\commentontext}[2]{\colorbox{yellow!60}{#1}\pdfcomment[color={0.234 0.867 0.211},hoffset=-6pt,voffset=10pt,opacity=0.5]{#2}}
\newcommand{\commentatside}[1]{\pdfcomment[color={0.045 0.278 0.643},icon=Note]{#1}}

% Compatibality with packages todo, easy-todo, todonotes
\newcommand{\todo}[1]{\commentatside{#1}}

% Compatiblity with package fixmetodonotes
\newcommand{\TODO}[1]{\commentatside{#1}}

% Put footnotes below floats
% Source: https://tex.stackexchange.com/a/32993/9075
\usepackage{stfloats}
\fnbelowfloat

\usepackage[group-minimum-digits=4,per-mode=fraction]{siunitx}
\addto\extrasgerman{\sisetup{locale = DE}}

% Enable that parameters of \cref{}, \ref{}, \cite{}, ... are linked so that a reader can click on the number an jump to the target in the document
\usepackage{hyperref}

% Enable hyperref without colors and without bookmarks
\hypersetup{
  hidelinks,
  colorlinks=true,
  allcolors=black,
  pdfstartview=Fit,
  breaklinks=true
}

% Enable correct jumping to figures when referencing
\usepackage[all]{hypcap}

\usepackage[caption=false,font=footnotesize]{subfig}

\usepackage[incolumn]{mindflow}

% Extensions for references inside the document (\cref{fig:sample}, ...)
% Enable usage \cref{...} and \Cref{...} instead of \ref: Type of reference included in the link
% That means, "Figure 5" is a full link instead of just "5".
\usepackage[capitalise,nameinlink,noabbrev]{cleveref}

\crefname{listing}{Listing}{Listings}
\Crefname{listing}{Listing}{Listings}
\crefname{lstlisting}{Listing}{Listings}
\Crefname{lstlisting}{Listing}{Listings}

\usepackage{lipsum}

% For demonstration purposes only
% These packages can be removed when all examples have been deleted
\usepackage[math]{blindtext}
\usepackage{mwe}
\usepackage[realmainfile]{currfile}
\usepackage{tcolorbox}
\tcbuselibrary{listings}

%introduce \powerset - hint by http://matheplanet.com/matheplanet/nuke/html/viewtopic.php?topic=136492&post_id=997377
\DeclareFontFamily{U}{MnSymbolC}{}
\DeclareSymbolFont{MnSyC}{U}{MnSymbolC}{m}{n}
\DeclareFontShape{U}{MnSymbolC}{m}{n}{
  <-6>    MnSymbolC5
  <6-7>   MnSymbolC6
  <7-8>   MnSymbolC7
  <8-9>   MnSymbolC8
  <9-10>  MnSymbolC9
  <10-12> MnSymbolC10
  <12->   MnSymbolC12%
}{}
\DeclareMathSymbol{\powerset}{\mathord}{MnSyC}{180}

\usepackage{xspace}
%\newcommand{\eg}{e.\,g.\xspace}
%\newcommand{\ie}{i.\,e.\xspace}
\newcommand{\eg}{e.\,g.,\ }
\newcommand{\ie}{i.\,e.,\ }

% Enable hyphenation at other places as the dash.
% Example: applicaiton\hydash specific
\makeatletter
\newcommand{\hydash}{\penalty\@M-\hskip\z@skip}
% Definition of "= taken from http://mirror.ctan.org/macros/latex/contrib/babel-contrib/german/ngermanb.dtx
\makeatother

% Add manual adapted hyphenation of English words
% See https://ctan.org/pkg/hyphenex and https://tex.stackexchange.com/a/22892/9075 for details
% Does not work on MiKTeX, therefore disabled - issue reported at https://github.com/MiKTeX/miktex-packaging/issues/271
% \input{ushyphex}

% correct bad hyphenation here
\hyphenation{op-tical net-works semi-conduc-tor}

% Enable copy and paste of text from the PDF
% Only required for pdflatex. It "just works" in the case of lualatex.
% Alternative: cmap or mmap package
% mmap enables mathematical symbols, but does not work with the newtx font set
% See: https://tex.stackexchange.com/a/64457/9075
% Other solutions outlined at http://goemonx.blogspot.de/2012/01/pdflatex-ligaturen-und-copynpaste.html and http://tex.stackexchange.com/questions/4397/make-ligatures-in-linux-libertine-copyable-and-searchable
% Trouble shooting outlined at https://tex.stackexchange.com/a/100618/9075
%
% According to https://tex.stackexchange.com/q/451235/9075 this is the way to go
\input glyphtounicode
\pdfgentounicode=1

\begin{document}
% Enable following command if you need to typeset "IEEEpubid".
% See https://bytefreaks.net/tag/ieeeoverridecommandlockouts for details.
%\IEEEoverridecommandlockouts

\title{Quick start for LaTeXing with IEEEtran.cls for\\ IEEE Computer Society Conferences}

\author{%
  \IEEEauthorblockN{First Author, Second Author}
  \IEEEauthorblockA{University of Examples, Germany\\
    \{lastname\}@example.org}
  \and
  \IEEEauthorblockN{Third Author}
  \IEEEauthorblockA{School of Electrical and\\Computer Examples\\
    Georgia Institute of Examples\\
    Atlanta, Georgia 30332--0250\\
    \url{http://www.example.org}}
}

% use for special paper notices
%\IEEEspecialpapernotice{(Invited Paper)}

% make the title area
\maketitle

% In case you want to add a copyright statement.
% Works only in the compsoc conference mode.
%
% Source: https://tex.stackexchange.com/a/325013/9075
%
% All possible solutions:
%  - https://tex.stackexchange.com/a/325013/9075
%  - https://tex.stackexchange.com/a/279134/9075
%  - https://tex.stackexchange.com/q/279789/9075 (TikZ)
%  - https://tex.stackexchange.com/a/200330/9075 - for non-compsocc papers
\iffalse
  \IEEEoverridecommandlockouts
  \IEEEpubid{\begin{minipage}{\textwidth}\ \\[12pt] \centering
      1551-3203 \copyright 2015 IEEE.
      Personal use is permitted, but republication/redistribution requires IEEE permission.
      \\
      See \url{https://www.ieee.org/publications_standards/publications/rights/index.html} for more information.
    \end{minipage}}
\fi

\begin{abstract}
\lipsum[1]
\end{abstract}

% For peer review papers, you can put extra information on the cover
% page as needed:
% \ifCLASSOPTIONpeerreview
% \begin{center} \bfseries EDICS Category: 3-BBND \end{center}
% \fi
%
% For peerreview papers, this IEEEtran command inserts a page break and
% creates the second title. It will be ignored for other modes.
\IEEEpeerreviewmaketitle

\section{Introduction}
\label{sec:introduction}
\lipsum[1-3]\todo{Refine me}

The remainder of the paper starts with a presentation of related work (\cref{sec:relatedwork}).
It is followed by a presentation of hints on \LaTeX{} (\cref{sec:latexhints}).
Finally, a conclusion is drawn and outlook on future work is made (\cref{sec:outlook}).

\section{Related Work}
\label{sec:relatedwork}

Winery~\cite{Winery} is a graphical \commentontext{modeling}{modeling with one ``l'', because of AE} tool.
The whole idea of TOSCA is explained by \citet{Binz2009}.

\section{LaTeX Hints}
\label{sec:latexhints}

% Required for proper example rendering in the compiled PDF
\newcount\LTGbeginlineexample
\newcount\LTGendlineexample
\newenvironment{ltgexample}%
{\LTGbeginlineexample=\numexpr\inputlineno+1\relax}%
{\LTGendlineexample=\numexpr\inputlineno-1\relax%
%
\tcbinputlisting{%
  listing only,
  listing file=\currfilepath,
  colback=green!5!white,
  colframe=green!25,
  coltitle=black!90,
  coltext=black!90,
  left=8mm,
  title=Corresponding \LaTeX{} code of \texttt{\currfilepath},
  listing options={%
    frame=none,
    language={[LaTeX]TeX},
    escapeinside={},
    firstline=\the\LTGbeginlineexample,
    lastline=\the\LTGendlineexample,
    firstnumber=\the\LTGbeginlineexample,
    basewidth=.5em,
    aboveskip=0mm,
    belowskip=0mm,
    numbers=left,
    xleftmargin=0mm,
    numberstyle=\tiny,
    numbersep=8pt%
  }
}
}%

This section contains hints on writing LaTeX.
It focuses on minimal examples, which can be directly adapted to the content

\subsection{Handling of paragraphs}

\begin{ltgexample}
One sentence per line.
This rule is important for the usage of version control systems.
A new line is generated with a blank line.
As you would do in Word:
New paragraphs are generated by pressing enter.
In LaTeX, this does not lead to a new paragraph as LaTeX joins subsequent lines.
In case you want a new paragraph, just press enter twice (!).
This leads to an empty line.
In word, there is the functionality to press shift and enter.
This leads to a hard line break.
The text starts at the beginning of a new line.
In LaTeX, you can do that by using two backslashes (\textbackslash\textbackslash).\\
This is rarely used.

Please do \textit{not} use two backslashes for new paragraphs.
For instance, this sentence belongs to the same paragraph, whereas the last one started a new one.
A long motivation for that is provided at \url{http://loopspace.mathforge.org/HowDidIDoThat/TeX/VCS/#section.3}.
\end{ltgexample}

\subsection{Notes separated from the text}

The package mindflow enables writing down notes and annotations in a way so that they are separated from the main text.

\begin{ltgexample}
\begin{mindflow}
This is a small note.
\end{mindflow}
\end{ltgexample}

\subsection{Hyphenation}

\LaTeX{} automatically hyphenates words.
When using microtype, there should be less hypnetations than in other settings.
It might be necessary to tweak the hyphenations nevertheless.
Here are some hints:

\begin{ltgexample}
In case you write \enquote{application-specific}, then the word will only be hyphenated at the dash.
You can also write \verb1applica\allowbreak{}tion-specific1 (result: applica\allowbreak{}tion-specific), but this is much more effort.

You can now write words containing hyphens which are hyphenated at other places in the word.
For instance, \verb1application"=specific1 gets application"=specific.
This is enabled by an additional configuration of the babel package.
\end{ltgexample}

\subsection{Typesetting Units}

\begin{ltgexample}
Numbers can written plain text (such as 100), by using the siunitx package like that:
\SI{100}{\km\per\hour},
or by using plain \LaTeX{} (and math mode):
$100 \frac{\mathit{km}}{h}$.
\end{ltgexample}

\begin{ltgexample}
\SI{5}{\percent} of \SI{10}{kg}
\end{ltgexample}

\begin{ltgexample}
Numbers are automatically grouped: \num{123456}.
\end{ltgexample}

\subsection{Surrounding Text by Quotes}

\begin{ltgexample}
Please use the \enquote{enquote command} to quote something.
Quoting with "`quote"' or ``quote'' also works.

\end{ltgexample}

\subsection{Cleveref examples}
\label{sec:ex:cref}

Cleveref demonstration: Cref at beginning of sentence, cref in all other cases.

\begin{figure}
    \centering
    \includegraphics[width=.75\linewidth]{example-image-a}
    \caption{Example figure for cref demo}
    \label{fig:ex:cref}
\end{figure}

\begin{figure}
    \centering
    \begin{tabular}{ll}
      \toprule
      Heading1 & Heading2 \\
      \midrule
      One      & Two      \\
      Thee     & Four     \\
      \bottomrule
    \end{tabular}
    \caption{Example table for cref demo}
    \label{tab:ex:cref}
\end{figure}

\begin{ltgexample}
\Cref{fig:ex:cref} shows a simple fact, although \cref{fig:ex:cref} could also show something else.

\Cref{tab:ex:cref} shows a simple fact, although \cref{tab:ex:cref} could also show something else.

\Cref{sec:ex:cref} shows a simple fact, although \cref{sec:ex:cref} could also show something else.
\end{ltgexample}

\subsection{Figures}

\begin{ltgexample}
\Cref{fig:label} shows something interesting.

\begin{figure}
  \centering
  \includegraphics[width=.8\linewidth]{example-image-golden}
  \caption[Simple Figure]{Simple Figure. Based on \citet{mwe}.}
  \label{fig:label}
\end{figure}
\end{ltgexample}


One can span a figure across multiple columns by using \verb+\begin{figure*}+.
See \cref{fig:16x9} as an example.

\begin{ltgexample}
\begin{figure*}
  \centering
  % note that \textwidth is used instead of \linewidth
  % This ensures that the graphics width is 60% of the "page" (text block), and not just 60% of the current text column
  % See https://tex.stackexchange.com/a/17085/9075 for details
  \includegraphics[width=.6\textwidth]{example-image-16x9}
  \caption{16x9 Figure}
  \label{fig:16x9}
\end{figure*}
\end{ltgexample}


\subsection{Sub Figures}

An example of two sub figures is shown in \cref{fig:two_sub_figures}.

\begin{ltgexample}
\begin{figure*}[!b]
    \centering
    \subfloat[Case I]{\includegraphics[width=.4\linewidth]{example-image-a}%
    \label{fig:first_case}}
  \hfil
    \subfloat[Case II]{\includegraphics[width=.4\linewidth]{example-image-b}%
    \label{fig:second_case}}
  \caption{Example figure with two sub figures.}
  \label{fig:two_sub_figures}
\end{figure*}
\end{ltgexample}

Note that often IEEE papers with subfigures do not employ subfigure
captions (using the optional argument to \verb+\subfloat[]+), but instead will
reference/describe all of them (a), (b), etc., within the main caption.
Be aware that for subfig.sty to generate the (a), (b), etc., subfigure
labels, the optional argument to \verb+\subfloat+ must be present. If a
subcaption is not desired, just leave its contents blank,
e.g., \verb+\subfloat[]+.
An example is shown in \cref{fig:two_sub_figures_ieee}.

\begin{ltgexample}
\begin{figure*}[!b]
    \centering
    \subfloat[]{\includegraphics[width=.4\linewidth]{example-image-a}%
    \label{fig:first_case_ieee}}
  \hfil
    \subfloat[]{\includegraphics[width=.4\linewidth]{example-image-b}%
    \label{fig:second_case_ieee}}
  \caption{Example figure with two sub figures. IEEE style. (a) The first case. (b) The second case.}
  \label{fig:two_sub_figures_ieee}
\end{figure*}
\end{ltgexample}

\subsection{Tables}

Note that IEEE does not support \verb+\begin{table}+, one has to use \verb+\begin{figure}+.

\begin{ltgexample}
\begin{figure}
  \caption{Simple Table}
  \label{tab:simple}
  \centering
  \begin{tabular}{ll}
    \toprule
    Heading1 & Heading2 \\
    \midrule
    One      & Two      \\
    Thee     & Four     \\
    \bottomrule
  \end{tabular}
\end{figure}
\end{ltgexample}

\begin{ltgexample}
% Source: https://tex.stackexchange.com/a/468994/9075
\begin{figure}
\caption{Table with diagonal line}
\label{tab:diag}
\begin{center}
\begin{tabular}{|l|c|c|}
\hline
\diagbox[width=10em]{Diag\\Column Head I}{Diag Column\\Head II} & Second & Third \\
\hline
& foo & bar \\
\hline
\end{tabular}
\end{center}
\end{figure}
\end{ltgexample}


\subsection{Source Code}

\begin{ltgexample}
\Cref{lst:XML} shows source code written in XML.
\Cref{line:comment} contains a comment.

\begin{lstlisting}[
  language=XML,
  caption={Example XML Listing},
  label={lst:XML}]
<listing name="example">
  <!-- comment --> (* \label{line:comment} *)
  <content>not interesting</content>
</listing>
\end{lstlisting}
\end{ltgexample}

One can also add \verb+float+ as parameter to have the listing floating.
\Cref{lst:flXML} shows the floating listing.

\begin{ltgexample}
\begin{lstlisting}[
  % one can adjust spacing here if required
  % aboveskip=2.5\baselineskip,
  % belowskip=-.8\baselineskip,
  float,
  language=XML,
  caption={Example XML listing -- placed as floating figure},
  label={lst:flXML}]
<listing name="example">
  Floating
</listing>
\end{lstlisting}
\end{ltgexample}

One can also typeset JSON as shown in \cref{lst:json}.

\begin{ltgexample}
\begin{lstlisting}[
  float,
  language=json,
  caption={Example JSON listing -- placed as floating figure},
  label={lst:json}]
{
  key: "value"
}
\end{lstlisting}
\end{ltgexample}

Java is also possible as shown in \cref{lst:java}.

\begin{ltgexample}
\begin{lstlisting}[
  caption={Example Java listing},
  label=lst:java,
  language=Java,
  float]
public class Hello {
    public static void main (String[] args) {
        System.out.println("Hello World!");
    }
}
\end{lstlisting}
\end{ltgexample}

\subsection{Itemization}

One can list items as follows:

\begin{ltgexample}
\begin{itemize}
\item Item One
\item Item Two
\end{itemize}
\end{ltgexample}

With the package paralist, one can create itemizations with lesser spacing:

\begin{ltgexample}
\begin{compactitem}
\item Item One
\item Item Two
\end{compactitem}
\end{ltgexample}

One can enumerate items as follows:

\begin{ltgexample}
\begin{enumerate}
  \item Item One
  \item Item Two
\end{enumerate}
\end{ltgexample}

With the package paralist, one can create enumerations with lesser spacing:

\begin{ltgexample}
\begin{compactenum}
  \item Item One
  \item Item Two
\end{compactenum}
\end{ltgexample}

With paralist, one can even have all items typset after each other and have them clean in the tex document:

\begin{ltgexample}
\begin{inparaenum}
  \item All these items...
  \item ...appear in one line
  \item This is enabled by the paralist package.
\end{inparaenum}
\end{ltgexample}

\subsection{Other Features}

\begin{ltgexample}
The words \enquote{workflow} and \enquote{dwarflike} can be copied from the PDF and pasted to a text file.
\end{ltgexample}

\begin{ltgexample}
The symbol for powerset is now correct: $\powerset$ and not a Weierstrass p ($\wp$).

$\powerset({1,2,3})$
\end{ltgexample}

\begin{ltgexample}
Brackets work as designed:
<test>
One can also input backquotes in verbatim text: \verb|`test`|.
\end{ltgexample}


\section{Conclusion and Outlook}
\label{sec:outlook}
\lipsum[1-2]

% regular IEEE prefers the singular form
\section*{Acknowledgment}

Identification of funding sources and other support, and thanks to individuals and groups that assisted in the research and the preparation of the work should be included in an acknowledgment section, which is placed just before the reference section in your document \cite{acmart}.

%%% ===============================================================================
%%% Bibliography
%%% ===============================================================================

In the bibliography, use \texttt{\textbackslash textsuperscript} for \enquote{st}, \enquote{nd}, \ldots:
E.g., \enquote{The 2\textsuperscript{nd} conference on examples}.
When you use \href{https://www.jabref.org}{JabRef}, you can use the clean up command to achieve that.
See \url{https://help.jabref.org/en/CleanupEntries} for an overview of the cleanup functionality.

% trigger a \newpage just before the given reference
% number - used to balance the columns on the last page
% adjust value as needed - may need to be readjusted if
% the document is modified later
%\IEEEtriggeratref{8}
% The "triggered" command can be changed if desired:
%\IEEEtriggercmd{\enlargethispage{-5in}}

% Enable to reduce spacing between bibitems (source: https://tex.stackexchange.com/a/25774)
% \def\IEEEbibitemsep{0pt plus .5pt}

\bibliographystyle{IEEEtranN} % IEEEtranN is the natbib compatible bst file
% argument is your BibTeX string definitions and bibliography database(s)
\bibliography{paper}

% Enfore empty line after bibliography
\ \\
%
All links were last followed on October 5, 2020.

\end{document}
