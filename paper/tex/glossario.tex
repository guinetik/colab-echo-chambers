\newword{Algoritmos de classificação}{Algoritmos de classificação são técnicas de aprendizado de máquina que categorizam objetos ou dados em classes predefinidas. Eles são usados em uma variedade de aplicações, desde filtros de spam em e-mails até reconhecimento de imagem e diagnóstico médico.}

\newword{Algoritmos de detecção de comunidades}{Algoritmos de detecção de comunidades, como Louvain, Girvan-Newman e Leiden, são métodos utilizados em análise de redes sociais para identificar grupos de nós que estão mais densamente conectados entre si do que com o restante da rede. Esses algoritmos ajudam a revelar a estrutura modular e a organização hierárquica de redes complexas.}

\newword{Análise de Redes Sociais (ARS)}{Análise de Redes Sociais (ARS) é um campo interdisciplinar que estuda as estruturas sociais através da utilização de teorias de redes e grafos. ARS investiga padrões de relacionamentos, fluxos de informação e difusão de inovações entre indivíduos, grupos, organizações ou mesmo países inteiros.}

\newword{Análise de sentimento}{Análise de sentimento é um campo de estudo que utiliza técnicas de processamento de linguagem natural para identificar, extrair e quantificar as emoções e opiniões expressas em textos. É amplamente utilizado para análise de feedback de clientes, monitoramento de marca e pesquisa de opinião pública.}

\newword{Análise exploratória}{Análise exploratória é um enfoque estatístico usado para descobrir padrões, relações ou anomalias nos dados. Esta abordagem é frequentemente o primeiro passo na análise de dados, preparando o caminho para análises estatísticas mais formais.}

\newword{Aprendizado de máquina (Machine Learning)}{Aprendizado de máquina é um ramo da inteligência artificial que se concentra no desenvolvimento de algoritmos que podem aprender de dados e fazer previsões ou tomar decisões sem ser explicitamente programados para cada tarefa.}

\newword{Arestas (ou arcos)}{Em teoria dos grafos, uma aresta ou arco é um elemento que conecta dois nós (vértices) em um grafo, representando a relação entre eles. Arestas podem ser direcionadas ou não direcionadas, indicando se a relação é unidirecional ou bidirecional.}

\newword{Assortatividade}{Assortatividade é uma medida em análise de redes que indica se nós tendem a se conectar a outros nós que são semelhantes (assortativos) ou diferentes (dissortativos) em alguma característica, como grau, idade, idioma, etc.}

\newword{AUC (área sob a curva ROC)}{AUC, ou área sob a curva ROC, é uma métrica usada para avaliar a performance de modelos de classificação. Uma AUC alta indica que o modelo tem uma boa medida de separabilidade, sendo capaz de distinguir entre as classes positivas e negativas eficientemente.}

\newword{Barômetro Social Hiperlocal}{Barômetro Social Hiperlocal é um indicador que mede a pressão das opiniões e sentimentos de uma comunidade em um nível muito localizado, como um bairro ou pequena cidade, geralmente usando dados de redes sociais e outras fontes digitais para captar o pulso da comunidade.}

\newword{Bias (Vieses), incluindo viés de confirmação}{Bias ou viés em pesquisa refere-se a erros sistemáticos que podem distorcer resultados. O viés de confirmação, em particular, é a tendência de favorecer informações que confirmam as crenças pré-existentes ou hipóteses, ignorando evidências contrárias.}

\newword{Bolhas de filtro (filter bubbles)}{Bolhas de filtro são efeitos de sistemas de personalização online que mostram aos usuários apenas informações alinhadas com seus interesses, histórico e comportamento passado, limitando a exposição a pontos de vista divergentes e potencialmente criando uma câmara de eco.}

\newword{Centralidade}{Centralidade é um conceito em análise de redes que determina a importância de um nó dentro da rede. Existem várias medidas de centralidade, incluindo centralidade de intermediação, eigenvector e grau, cada uma destacando diferentes aspectos da posição de um nó na rede.}

\newword{Câmaras de eco}{Câmaras de eco são ambientes em que uma pessoa é exposta apenas a opiniões que refletem e reforçam as suas próprias, amplificando seus pontos de vista e criando uma ilusão de consenso ou apoio universal.}

\newword{Cliques}{Em análise de redes, cliques referem-se a subconjuntos de nós que estão todos interconectados. Em contextos sociais, uma clique é um grupo onde cada membro conhece e interage com todos os outros membros.}

\newword{Clusterização}{Clusterização é o processo de agrupar um conjunto de objetos de modo que os objetos no mesmo grupo (ou cluster) sejam mais semelhantes (em algum sentido) entre si do que com aqueles em outros grupos. É uma técnica comum em análise de dados para descobrir estruturas e padrões.}

\newword{Colab}{O Colab é um aplicativo brasileiro que promove a participação cidadã na melhoria das cidades. Permite que os usuários relatem problemas e sugiram ideias diretamente para as autoridades, buscando soluções colaborativas para questões urbanas. O objetivo é fortalecer a conexão entre cidadãos e poder público, visando a transparência e a construção de cidades mais sustentáveis e inclusivas.}

\newword{Conectividade}{Conectividade em uma rede refere-se à capacidade dos seus nós de se conectarem uns aos outros. Em redes de computadores, indica a infraestrutura que permite a comunicação entre dispositivos. Em teoria dos grafos, descreve quão bem um grafo está conectado.}

\newword{Curva ROC (Receiver Operating Characteristic)}{A curva ROC é um gráfico que ilustra o desempenho de um sistema classificador binário à medida que seu limiar de discriminação é variado. É usada com a AUC para avaliar a capacidade do modelo de separar as duas classes.}

\newword{Dataset (conjunto de dados)}{Dataset é uma coleção de dados geralmente tabulados, onde cada coluna representa uma variável particular e cada linha corresponde a um determinado registro. Datasets são fundamentais para a análise estatística e treinamento de modelos de aprendizado de máquina.}

\newword{Diâmetro da rede}{Diâmetro da rede é a maior distância entre dois nós em uma rede. Em teoria dos grafos, corresponde ao maior número de arestas em um caminho mais curto entre dois vértices, fornecendo uma medida da extensão linear da rede.}

\newword{E-Government (e-Gov)}{E-Government, ou Governo Eletrônico, refere-se ao uso de tecnologias da informação e comunicação para proporcionar acesso a serviços governamentais, melhorar a eficiência das operações e promover a transparência e participação cidadã.}

\newword{Formato booleano}{Formato booleano é um tipo de dado que pode ter apenas dois valores: verdadeiro ou falso (geralmente representado por 1 ou 0). É amplamente usado em computação, lógica e eletrônica digital.}

\newword{Gamificação}{Gamificação é a aplicação de elementos e princípios de design de jogos em contextos não jogáveis, como negócios ou educação, para aumentar o engajamento e a motivação.}

\newword{Georreferenciação}{Georreferenciação é o processo de associar um objeto físico ou evento a uma localização na superfície terrestre, geralmente através de coordenadas geográficas em um sistema de referência espacial.}

\newword{Gephi}{Gephi é uma plataforma de código aberto para visualização e análise de redes e sistemas complexos. Permite aos usuários explorar, analisar e entender grafos por meio de técnicas de visualização e ferramentas de análise de redes.}

\newword{Google Colaboratory (Google Colab)}{O Google Colaboratory, conhecido como Colab, é um ambiente de desenvolvimento gratuito que permite escrever e executar código Python no navegador. É especialmente útil para aprendizado de máquina, análise de dados e educação, oferecendo fácil acesso a recursos computacionais poderosos, como GPUs, sem necessidade de configuração.}

\newword{Grafo (direcionado, não direcionado)}{Um grafo é uma estrutura de dados composta por nós (vértices) e arestas que conectam pares de nós. Em grafos direcionados, as arestas têm uma direção, indicando uma relação assimétrica entre os nós (como em um diagrama de fluxo), enquanto em grafos não direcionados, as arestas são bidirecionais, sugerindo uma relação simétrica (como em uma rede social).}

\newword{Grau de um vértice (entrada, saída)}{O grau de um vértice em um grafo refere-se ao número de arestas conectadas a ele. Em grafos direcionados, o grau de entrada é o número de arestas que entram no vértice, e o grau de saída é o número de arestas que saem do vértice. Essas medidas são indicativos da importância e da conectividade de um vértice dentro da rede.}

\newword{Heurística}{Heurística é uma abordagem prática para resolver problemas que, embora não seja perfeitamente precisa ou completa, é suficientemente boa para alcançar uma solução em tempo razoável. Heurísticas são frequentemente usadas em situações onde encontrar a resposta ótima é impraticável ou impossível.}

\newword{Histograma}{Um histograma é uma representação gráfica de distribuição de dados que organiza um conjunto de dados em intervalos (bins) e mostra a frequência dos dados em cada intervalo. É comumente utilizado em estatística para dar uma imagem grosseira da distribuição de probabilidade de um conjunto de dados.}

\newword{Homofilia}{Homofilia é o princípio de que a semelhança gera conexão. Em redes sociais, isso significa que indivíduos tendem a formar laços com outros que são semelhantes a eles em algum aspecto significativo, como idade, gênero, classe social ou interesses.}

\newword{Inclusão digital}{Inclusão digital é o esforço para garantir que todos os indivíduos e comunidades, especialmente os mais desfavorecidos, tenham acesso e capacidade de usar as tecnologias de informação e comunicação. Envolve não apenas o acesso físico à tecnologia, mas também a habilidade de usar efetivamente as ferramentas digitais.}

\newword{Inferência Estatística}{Inferência estatística é o processo de tirar conclusões sobre as propriedades de uma população com base em uma amostra dessa população. Utiliza técnicas de probabilidade para estimar parâmetros, testar hipóteses e fazer previsões.}

\newword{Interações sociais}{Interações sociais são trocas entre dois ou mais indivíduos e formam a base das relações sociais e da estrutura social. Em contextos de redes sociais, elas podem incluir comunicação, influência mútua, e comportamentos colaborativos ou competitivos.}

\newword{K-Nearest Neighbors (KNN)}{K-Nearest Neighbors (KNN) é um algoritmo de aprendizado de máquina simples e não paramétrico que classifica novas observações com base na semelhança de características com exemplos no conjunto de treinamento. É amplamente utilizado para classificação e regressão em diversos campos.}

\newword{Layouts (e.g., Force Atlas 2, Kamada-Kawai, Fruchterman-Reingold)}{Layouts em análise de redes sociais referem-se aos algoritmos usados para posicionar os nós de um grafo de maneira a refletir a estrutura da rede e facilitar a compreensão visual. Exemplos incluem Force Atlas 2, Kamada-Kawai e Fruchterman-Reingold, cada um com sua própria lógica para distribuir os nós e arestas.}

\newword{Matriz de Correlação}{Uma matriz de correlação é uma tabela que mostra os coeficientes de correlação entre variáveis. Cada célula na matriz mostra a correlação entre dois elementos, permitindo identificar facilmente o grau de relação linear entre as variáveis de um conjunto de dados.}

\newword{Métricas quantitativas}{Métricas quantitativas são medidas numéricas que quantificam informações e fornecem dados para análise estatística. Em pesquisa, são usadas para avaliar e comparar características ou desempenhos de diferentes entidades ou fenômenos.}

\newword{Modelagem Baseada em Agentes (MBA)}{Modelagem Baseada em Agentes (MBA) é uma abordagem de simulação que utiliza agentes autônomos com comportamentos definidos para modelar fenômenos complexos. Permite estudar interações entre agentes e emergência de padrões em sistemas sociais, econômicos e biológicos.}

\newword{Modelo de previsibilidade}{Um modelo de previsibilidade é uma ferramenta estatística ou de aprendizado de máquina que fornece uma estimativa quantitativa da probabilidade de um evento futuro com base em dados históricos ou atuais.}

\newword{Modelo de regressão}{Um modelo de regressão é uma abordagem estatística que examina a relação entre uma variável dependente e uma ou mais variáveis independentes. É usado para prever, entender e modelar relações entre variáveis.}

\newword{Modularidade}{Em análise de redes, modularidade é uma métrica que mede a força da divisão de uma rede em módulos ou comunidades. Uma alta modularidade indica uma estrutura de rede com grupos densamente interconectados internamente e conexões mais fracas entre os grupos.}

\newword{NetworkX}{NetworkX é uma biblioteca Python projetada para a criação, manipulação e estudo da estrutura, dinâmica e funções de redes complexas. É amplamente utilizada em análise de redes sociais, biológicas e de infraestrutura.}

\newword{Nós (em redes)}{Nós são os elementos fundamentais de uma rede que podem representar indivíduos, organizações, ou outras entidades em uma rede social, computacional ou qualquer outro tipo de rede. Os nós são conectados por arestas, que representam as relações ou interações entre eles.}

\newword{Overfitting}{Overfitting ocorre quando um modelo de aprendizado de máquina é tão complexo que se ajusta perfeitamente aos dados de treinamento, incluindo o ruído ou as flutuações aleatórias, resultando em um desempenho pobre em novos dados devido à falta de generalização.}

\newword{PageRank}{PageRank é um algoritmo desenvolvido pelos fundadores do Google para medir a importância de páginas da web com base na qualidade e quantidade de links que apontam para elas. É um exemplo de algoritmo de centralidade de rede que também pode ser aplicado em outros contextos de redes sociais.}

\newword{PCA (Principal Component Analysis)}{A Análise de Componentes Principais (PCA) é uma técnica estatística que transforma dados de alta dimensão em um espaço de dimensões reduzidas, mantendo a maior parte da variação dos dados. É comumente usada para simplificar a análise de dados e visualização.}

\newword{Personas (Helpers e Complainers)}{Personas são arquétipos que representam os comportamentos, atitudes e motivações de diferentes usuários em um sistema. "Helpers" são aqueles que tendem a oferecer ajuda e soluções, enquanto "Complainers" são mais propensos a expressar insatisfação ou problemas.}

\newword{Polarização (Polarization)}{Polarização refere-se ao processo pelo qual as opiniões de um grupo de indivíduos se dividem em extremos opostos, frequentemente levando a um aumento no conflito social e à redução do consenso ou da deliberação pública.}

\newword{Pré-Processamento}{No contexto de análise de dados, pré-processamento é o conjunto de procedimentos aplicados a dados brutos para prepará-los para análise. Isso pode incluir limpeza, normalização, transformação e redução de dados.}

\newword{Pressão Social}{A pressão social é um fenômeno psicológico onde as opiniões ou comportamentos dos indivíduos são influenciados pela percepção de normas ou preferências expressas por um grupo ou sociedade em geral. Em contextos de pesquisa, é frequentemente analisada para entender como as atitudes e ações são moldadas por fatores sociais e culturais.}

\newword{Processamento de Linguagem Natural (PLN)}{O Processamento de Linguagem Natural, ou PLN, é um campo de estudo interdisciplinar que combina conhecimentos da linguística, ciência da computação e inteligência artificial para criar sistemas capazes de entender e manipular a linguagem humana. O PLN é utilizado em uma variedade de aplicações, como tradução automática, análise de sentimentos, assistentes virtuais e interfaces homem-máquina.}

\newword{Python (linguagem de programação)}{Python é uma linguagem de programação de alto nível, interpretada e de propósito geral, conhecida por sua legibilidade e eficiência. Amplamente utilizada em ciência de dados, desenvolvimento web, automação e muitos outros campos, ela oferece uma sintaxe clara e uma vasta biblioteca padrão, além de suportar múltiplos paradigmas de programação.}

\newword{Regressão Logística}{A regressão logística é uma técnica estatística de modelagem preditiva utilizada para analisar um conjunto de dados no qual existem uma ou mais variáveis independentes que determinam um resultado binário. É amplamente usada em campos como medicina, ciências sociais e marketing para estimar a probabilidade de uma variável dependente categorial, baseando-se na relação entre essa variável dependente e as variáveis independentes.}

\newword{Simulações de Monte Carlo}{As Simulações de Monte Carlo são uma classe de algoritmos computacionais que dependem de amostragem aleatória repetida para obter resultados numéricos. Eles são frequentemente usados para modelar fenômenos físicos e financeiros complexos, onde a previsão analítica é impraticável ou impossível. O nome destes métodos refere-se ao famoso cassino em Mônaco, devido ao elemento de aleatoriedade associado a jogos de azar.}

\newword{Sociometria}{Sociometria é uma técnica quantitativa para medir relações sociais. Foi desenvolvida por Jacob L. Moreno para avaliar a atração ou rejeição entre membros de um grupo, e é usada para descobrir padrões de interações sociais, identificar líderes informais, grupos isolados, ou indivíduos marginalizados dentro de uma comunidade ou organização.}

\newword{Teoria dos Grafos}{A Teoria dos Grafos é um ramo da matemática que estuda as propriedades de grafos, estruturas matemáticas usadas para modelar relações par a par entre objetos. Um grafo é composto por "nós" (ou vértices) que representam os objetos e "arestas" que representam as conexões entre eles. A teoria é amplamente utilizada em ciência da computação, engenharia, biologia e ciências sociais.}

\newword{Vigilância Participativa}{Vigilância Participativa refere-se à prática de cidadãos usarem a tecnologia para coletar e analisar dados sobre questões públicas ou governamentais, frequentemente com o objetivo de influenciar políticas ou práticas. Esse tipo de vigilância se baseia na participação ativa da comunidade e pode ser uma ferramenta para aumentar a transparência e a responsabilidade do poder público.}