\newword{Colab.re}{Colab.re é um aplicativo brasileiro que promove a participação cidadã na melhoria das cidades. Permite que os usuários relatem problemas e sugiram ideias diretamente para as autoridades, buscando soluções colaborativas para questões urbanas. O objetivo é fortalecer a conexão entre cidadãos e poder público, visando a transparência e a construção de cidades mais sustentáveis e inclusivas.}
\newword{Eigenvector}{Em análise de redes, um eigenvector de uma matriz de adjacência de um grafo representa um vetor próprio associado a um valor próprio específico. Em termos simples, um eigenvector é um vetor no qual a importância relativa de um nó em um grafo é determinada pela sua conectividade com outros nós. Os eigenvectors são amplamente utilizados em medidas de centralidade, como a centralidade de eigenvector, que identifica nós importantes com base na sua influência sobre a rede. O cálculo dos eigenvectors é fundamental para entender a estrutura e a dinâmica de redes complexas.}
\newword{Aresta}{Em análise de redes, uma aresta é a conexão entre dois nós em um grafo. Ela representa uma relação ou interação entre os nós e pode ser direcionada ou não direcionada, dependendo da presença ou ausência de uma direção específica. As arestas são essenciais para compreender a estrutura e a dinâmica das redes, bem como para analisar propriedades como conectividade e fluxo de informações.}
\newword{Nó}{Em análise de redes, um nó é um elemento fundamental em um grafo que representa uma entidade individual. Também conhecido como vértice, ele pode representar uma pessoa, um local, um objeto ou qualquer outra entidade relevante para o estudo em questão. Os nós são conectados por arestas, que representam as relações ou interações entre eles. Eles desempenham um papel crucial na análise de redes, permitindo a investigação de propriedades como conectividade, centralidade e fluxo de informações.}
\newword{Gephi}{O Gephi é uma ferramenta de software para visualização e análise de redes. Ele possui recursos para importar dados de redes, organizar nós e arestas, e realizar análises, como medir centralidade e identificar comunidades. É amplamente utilizado em áreas como análise de redes sociais e biológicas para compreender as estruturas complexas das redes.}
\newword{Neo4j}{O Neo4j é um sistema de gerenciamento de banco de dados orientado a grafos. Ele é usado para armazenar e consultar dados conectados de forma eficiente, permitindo representar relacionamentos complexos entre entidades. }