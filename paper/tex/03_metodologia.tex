O conjunto de dados utilizado neste estudo foi obtido no formato CSV do aplicativo brasileiro Colab.re. O conjunto de dados contém listas de arestas, que são listas de conexões entre usuários, descrevendo os relacionamentos entre os usuários (ou seja, quem segue quem) e as postagens dos usuários de 2016 a 2022. A pesquisa foi limitada às cidades de Caruaru, Rio de Janeiro, Recife e Niterói.

Para analisar o conjunto de dados, uma variedade de ferramentas foi utilizada.

O Gephi é um programa de software de código aberto usado para análise exploratória de dados, especificamente para visualização e análise de redes complexas. Neste artigo, o Gephi foi utilizado para criar visualizações para uma análise exploratória de rede. O programa possui uma variedade de ferramentas, incluindo diferentes algoritmos de layout, atributos de nós e arestas e estatísticas. O Gephi permite a personalização das visualizações, possibilitando que os pesquisadores destaquem padrões e estruturas específicas nos dados. Sua interface intuitiva o torna acessível tanto para pesquisadores quanto para profissionais. Utilizando o Gephi neste artigo, nós fomos capazes de criar visualizações que forneceram insights sobre a estrutura da rede em estudo, permitindo a identificação de padrões potenciais e áreas para investigação adicional.

O Google Colaboratory, também conhecido como Colab, é uma plataforma baseada em nuvem que fornece um ambiente de notebook Jupyter gratuito para os usuários escreverem e executarem código Python. O Google Colaboratory permite que os usuários aproveitem o poder de processamento do Google, fornecendo acesso a uma CPU, GPU e TPU de alto desempenho de forma gratuita. Os notebooks do Google Colaboratory podem ser usados para desenvolver e executar modelos de aprendizado de máquina, análise de dados e tarefas de visualização de dados, entre outras coisas. Este artigo utiliza a plataforma para aproveitar seu poder de processamento para tarefas complexas de análise de dados, incluindo análise de rede, visualização e aprendizado de máquina. As características colaborativas do Google Colaboratory também permitem uma colaboração perfeita entre pesquisadores e facilitam o compartilhamento de código e dados com outros usuários. Ao utilizar o Google Colaboratory, este artigo tem como objetivo melhorar a eficiência e acessibilidade da análise de dados para pesquisadores com recursos de computação limitados, incentivando a migração para uma plataforma baseada em nuvem, especialmente se for gratuita. Para fins de desambiguação, este artigo sempre se referirá a ela como Google Colaboratory para evitar confusão com o aplicativo Colab.re, objeto de estudo deste artigo.

Para gerar os modelos de identificação de câmaras de eco, utilizamos o Python como a linguagem de programação principal. As seguintes bibliotecas populares de Python foram utilizadas:

NetworkX: uma biblioteca para criação, manipulação e estudo de redes complexas (ou seja, grafos).
Pandas: uma biblioteca para manipulação e análise de dados, projetada especificamente para lidar com dados em formato tabular.
Matplotlib: uma biblioteca para criar visualizações, como gráficos e histogramas.
Epidemics: um pacote Python para simular e analisar a propagação de doenças usando os modelos SIR e SEIR.
Para implementar as técnicas de análise de rede utilizadas neste estudo, utilizamos a biblioteca NetworkX. Essa biblioteca fornece um conjunto de algoritmos e funções que nos permitem analisar e manipular redes complexas. Especificamente, utilizamos os algoritmos de agrupamento espectral e Louvain para identificar as comunidades dentro da rede.

Para implementar os modelos SIR e SEIR para epidemiologia digital, utilizamos o pacote Epidemics. Esse pacote fornece um conjunto de funções que nos permitem simular e analisar a propagação de doenças usando os modelos SIR e SEIR. Utilizamos esses modelos para simular a propagação de informações dentro da rede e identificar os grupos de usuários mais propensos a estarem em uma câmara de eco.

Em geral, a metodologia utilizada neste estudo envolveu a coleta do conjunto de dados do aplicativo Colab.re, o pré-processamento dos dados usando o Pandas e a análise da rede usando o NetworkX. Em seguida, aplicamos os modelos SIR e SEIR usando o pacote Epidemics para simular a propagação de informações dentro da rede e identificar câmaras de eco. Por fim, utilizamos o Matplotlib para criar visualizações que auxiliam na interpretação dos resultados.