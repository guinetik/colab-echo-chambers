A interação entre o ser humano e sua realidade tecnológica tem sido um tema de reflexão constante. Como argumentado por Martin Heidegger em "O Ser e o Tempo", a qualidade de nossa experiência está intrinsecamente ligada à forma como nos relacionamos com o ambiente, incluindo o uso de tecnologia. Suas observações fenomenológicas revelam que uma interação perfeita com uma ferramenta pode alterar nossa percepção existente, levando-nos a gostar e apreciar algo por reconhecê-lo e dominar seu uso \cite{2012_Silva_DISSERTATION}.

Essa compreensão de Heidegger sobre a interação com as ferramentas tecnológicas encontra ecos em pesquisas recentes em neurofisiologia, psicologia e neuropsicologia, que mostram que o uso de ferramentas para alcançar objetivos distantes desencadeia mudanças nas redes neurais responsáveis por manter um mapa atualizado da forma e postura do corpo, conhecido como "esquema corporal" na neurologia clássica. 

No estudo "Tools for the Body" \cite{2004_Maravita}, que aprofunda a análise da relação entre as ferramentas e o corpo humano, é explorado como as ferramentas tecnológicas podem se tornar extensões do corpo, ampliando nossas habilidades e transformando nossa experiência cotidiana. O autor destaca a importância da interação eficiente com as ferramentas digitais para o sucesso na manipulação do ambiente tecnológico, ressaltando que essa interação pode resultar em uma mudança significativa em nossa percepção e apreciação de marcas e serviços.

Ao considerarmos uma intersecção entre as ideias de Heidegger e o estudo contemporâneo dessas ferramentas tecnológicas, percebemos a relevância do conceito de Zuhandenheit, que descreve a tecnificação das mãos, ou seja, como as ferramentas se integram em nossas atividades cotidianas de forma fluida e transparente. No entanto, essa tecnificação excessiva das mãos em tempos contemporâneos pode levar à inautenticidade, à perda de uma conexão genuína com nosso ser e ao surgimento de um sentimento de ansiedade.

Nesse contexto, a hauntologia de Jacques Derrida oferece uma perspectiva interessante. A hauntologia descreve o estado de assombração causado pela sensação de que o presente não corresponde às promessas do passado. Em relação à tecnologia, a hauntologia nos lembra de como as interações digitais, embora aparentemente numerosas e eficientes, podem falhar em suprir as necessidades humanas básicas de conexão e pertencimento. Assim como Heidegger nos alertou sobre a tecnificação excessiva, Derrida nos convida a refletir sobre como as interações tecnológicas fantasmagóricas nos informam sobre a falta de autenticidade em nosso envolvimento com o mundo digital.

Embora essas plataformas tenham o potencial de aumentar o engajamento político e fomentar o diálogo, também são palco de um fenômeno preocupante: as câmaras de eco. Nas câmaras de eco, os usuários são expostos apenas a informações que confirmam suas crenças e tendências pré-existentes, contribuindo para a polarização das opiniões. Essa fragmentação das fontes de informação, descrita por Heidegger como uma forma de inautenticidade, e o fenômeno das bolhas de filtro, estudado por pesquisadores contemporâneos, compartilham características semelhantes e agravam o problema.

A inautenticidade, conceito central na filosofia de Heidegger, pode ser entendida como a falta de uma relação autêntica com o mundo e consigo mesmo. No contexto das plataformas de mídia social, a inautenticidade pode se manifestar quando os usuários adotam opiniões populares ou polêmicas para se encaixar em determinados grupos ou para ganhar aprovação e validação social. Essa busca por aceitação pode levar à polarização, à medida que as pessoas se alinham cada vez mais com visões extremas para se distinguir e ganhar reconhecimento dentro de suas comunidades online.

Essa tendência à inautenticidade é agravada pelos algoritmos das plataformas de mídia social, que personalizam o conteúdo exibido aos usuários com base em seu comportamento passado. Esses algoritmos, em busca de maximizar o engajamento e o tempo gasto nas plataformas, analisam o histórico de navegação, curtidas, compartilhamentos e interações do usuário para determinar quais conteúdos são mais propensos a serem do seu interesse. Como resultado, os usuários são apresentados principalmente a informações que se alinham com suas preferências e visões de mundo, enquanto conteúdos divergentes ou contraditórios são filtrados ou recebem menos destaque.

O processo de filtragem cria uma "bolha" em torno de cada usuário, onde seu ambiente de informações se torna cada vez mais estreito e alinhado com suas perspectivas existentes. Pesquisas recentes informam a noção que a formação de bolhas de filtro é um processo complexo que envolve vários fatores. De acordo com o estudo de \citeonline{2011_Pariser_BOOK}, as bolhas de filtro são criadas por algoritmos de personalização que selecionam o conteúdo com base no comportamento passado do usuário, suas conexões sociais e suas preferências expressas. Isso resulta em um ambiente de informação que é altamente personalizado e alinhado com as visões existentes do usuário.

No entanto, a formação de bolhas de filtro não é apenas uma consequência direta da personalização, mas também é influenciada pela interação do usuário com a plataforma. O estudo de \citeonline{2018_Aslay} destaca que a exposição do usuário a diferentes pontos de vista é limitada não apenas pelos algoritmos de personalização, mas também pela probabilidade do usuário compartilhar ou interagir com um conteúdo específico. Isso significa que as bolhas de filtro são reforçadas tanto pelos algoritmos de personalização quanto pelo comportamento do usuário.

Além disso, um estudo recente de \citeonline{2022_Boeker} sobre o TikTok oferece uma visão mais detalhada sobre como as bolhas de filtro são criadas nesta plataforma específica. Eles descobriram que o algoritmo de recomendação do TikTok leva em consideração várias características do usuário, como idioma, localização, ações de curtir e seguir, e taxa de visualização de vídeo. Além disso, o algoritmo também considera as tags descritivas atribuídas aos vídeos com base em análises de visão computacional, hashtags mencionadas, descrição do post, som e textos incorporados. Isso sugere que a formação de bolhas de filtro é um processo multifatorial que envolve tanto a personalização algorítmica quanto as características e comportamentos do usuário.

Portanto, o processo de filtragem que cria uma "bolha" em torno de cada usuário é um fenômeno complexo que envolve a interação entre algoritmos de personalização e comportamento do usuário. À medida que os usuários interagem com a plataforma e expressam suas preferências, os algoritmos de personalização adaptam o ambiente de informação para se alinhar cada vez mais com essas preferências. Isso resulta em uma exposição limitada a opiniões divergentes e uma redução na diversidade de pontos de vista, contribuindo para a ampliação da polarização política.

Essa personalização algorítmica pode levar os usuários a acreditar erroneamente que sua visão de mundo é a única válida ou amplamente aceita, exacerbando ainda mais a polarização e a formação de câmaras de eco. A combinação das câmaras de eco e das bolhas de filtro nas plataformas de mídia social torna cada vez mais difícil para os usuários se engajarem em discussões saudáveis, acessar informações imparciais e considerar diferentes pontos de vista.

É importante compreender e abordar o fenômeno das bolhas de filtro juntamente com as câmaras de eco, pois ambos têm implicações significativas para o discurso político e a democracia. A superação desses desafios requer esforços para aumentar a diversidade de perspectivas, garantir o acesso a informações diversas e promover a interação entre usuários com opiniões diferentes. Além disso, é fundamental que os usuários desenvolvam uma consciência crítica em relação à sua interação com as ferramentas tecnológicas, buscando uma relação autêntica com o mundo digital e evitando a dependência excessiva que pode levar à perda de contato com a realidade e à inautenticidade.

O impacto das câmaras de eco no discurso político é significativo. Estudos indicam que usuários expostos a uma maior diversidade de pontos de vista políticos têm maior probabilidade de se envolver em discussões políticas, enquanto aqueles expostos apenas a conteúdos similares aos seus têm menos probabilidade de engajar-se com notícias políticas. Essas câmaras de eco são particularmente acentuadas no Brasil, onde a polarização política tem aumentado desde 2010 \cite{2022_Ortellado}. É importante notar que pesquisas científicas sobre câmaras de eco frequentemente se concentraram na análise de redes sociais públicas, como Twitter, Facebook e Reddit. Diversas pesquisas têm investigado as dinâmicas das câmaras de eco nesses ambientes online. Por exemplo, estudos como o de \cite[p. 224]{2016_Vicario} examinaram o surgimento e a propagação de desinformação e polarização nas redes sociais, destacando a importância de entender esses fenômenos para a sociedade como um todo.

O Colab é uma plataforma brasileira de mídia social que visa promover a democracia participativa, permitindo que os usuários apresentem ideias e propostas para melhorar suas comunidades. Através do Colab, os cidadãos podem compartilhar problemas locais, sugerir soluções, colaborar com outros membros da comunidade e interagir com representantes governamentais. A plataforma tem sido elogiada por sua abordagem inovadora para o envolvimento cívico, incentivando uma participação mais ativa dos cidadãos na tomada de decisões e no aprimoramento de suas localidades. No entanto, assim como outras plataformas de mídia social, o Colab também enfrenta o desafio das câmaras de eco, onde os usuários tendem a seguir e interagir principalmente com aqueles que compartilham suas opiniões políticas, resultando em uma redução de perspectivas e uma possível erosão dos valores democráticos. Compreender as câmaras de eco no contexto do aplicativo Colab apresenta uma oportunidade valiosa de pesquisa considerando que muitas plataformas de mídia social não disponibilizam seus dados para acesso público, tornando desafiador o estudo das câmaras de eco em tais ambientes. Além de uma rede social, o Colab também oferece serviços de eGov. Empresas de eGov, ou governo eletrônico, são organizações que fornecem soluções digitais para apoiar o funcionamento e os serviços do governo. Elas se concentram em utilizar a tecnologia da informação e comunicação (TIC) para melhorar a eficiência, transparência e interação entre o governo e os cidadãos. Essas empresas oferecem uma variedade de serviços, como desenvolvimento de portais governamentais, plataformas de participação cidadã, sistemas de gestão de documentos, soluções de segurança digital, entre outros.

No ambiente atual de plataformas de mídia social e redes online, as interações digitais moldam percepções, opiniões e até mesmo comportamentos dos usuários. Nesse contexto, o aplicativo Colab se apresenta como um microcosmo dessas interações, oferece uma representação vívida de como as comunidades online se formam e operam. Assumindo um papel central nesta dissertação, o Colab é analisado através de um prisma de câmaras de eco, comunidades estreitamente conectadas em que crenças e opiniões semelhantes são reforçadas, muitas vezes excluindo perspectivas divergentes.

Sob esta perspectiva, são estabelecidas as seguintes suposições:

\begin{enumerate}
    \item Suposição 1: Usuários dentro do aplicativo Colab possuem personas discerníveis com base em seu comportamento e sentimentos expressos.
    \item Suposição 2: Usuários com personas similares podem formar conexões dentro da rede.
    \item Suposição 3: Câmaras de eco são, potencialmente, comunidades estreitamente conectadas onde crenças e opiniões similares são reforçadas.
    \item Suposição 4: Se existirem, é provável que as câmaras de eco sejam significativamente isoladas da rede mais ampla, limitando a exposição a visões divergentes.
\end{enumerate}

Partindo dessas suposições, formulamos a seguinte hipótese central:

\begin{citacao}
Se câmaras de eco existirem dentro das comunidades da rede do aplicativo Colab, é provável que estejam ligadas a usuários com interesses e comportamentos semelhantes, formando comunidades isoladas e estreitamente conectadas. Estas comunidades serviriam potencialmente como mecanismos de reforço para suas próprias crenças e opiniões, isolando-as ainda mais da rede geral.
\end{citacao}

Esta pesquisa é orientada pela seguinte questão principal:

\begin{citacao}
    Como um método heurístico quantitativo sintetizado pode ser efetivamente projetado e implementado para identificar e caracterizar câmaras de eco dentro da comunidade de rede do Colab?
\end{citacao}

Para abordar essa questão, a pesquisa será organizada em quatro etapas fundamentais:

\begin{enumerate}
    \item Análise exploratória da rede e interações dos usuários no Colab.
    \item Proposição de heurísticas para detecção de câmaras de eco.
    \item Investigação da formação de câmaras de eco por meio de simulações computacionais.
    \item Desenvolvimento de uma aplicação web para demonstrar as heurísticas de detecção de câmaras de eco e o treinamento de modelos de aprendizado de máquina para classificação de postagens.
  \end{enumerate}

Este trabalho tem como objetivo examinar o fenômeno das câmaras de eco na plataforma de mídia social brasileira Colab, e propor estratégias para mitigar seus efeitos negativos. Para isso, serão realizadas análises de redes na plataforma, identificando as câmaras de eco e seus nós principais. Além disso, serão desenvolvidos algoritmos baseados em heurísticas inspiradas por técnicas tradicionais de análise de redes e teoria dos grafos, além de métodos de vanguarda como modelagem baseada em agentes, estimativas de cadeias markovianas de Monte Carlo baseada em modelos ERGM, e modelos não markovianos baseados em epidemiologia digital, como os modelos SIR e SEIR, para prever a propagação de informações dentro e entre essas câmaras de eco bem como tentar traçar as suas origens.

Ao combinar os o arcabouço teórico da academia com os dados e experiência do Colab, é possível obter uma compreensão mais profunda das câmaras de eco em um contexto específico, isto é, uma rede social fomentada por interesses  de eGov. Isso não apenas aprimora nossa compreensão desses fenômenos sociais complexos, mas também fornece insights valiosos para o governo e aprimora as estratégias de engajamento cidadão nas cidades atendidas pelo Colab. Ao compreender melhor como essas câmaras influenciam as interações políticas e o diálogo cívico, o governo pode tomar medidas mais eficazes para combater a polarização, promover a diversidade de perspectivas e fortalecer a participação democrática. A pesquisa pode fornecer insights práticos que ajudarão o governo a tomar decisões informadas e a desenvolver estratégias eficientes para criar um ambiente político mais saudável e autêntico.