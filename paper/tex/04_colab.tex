O Colab.re é um aplicativo brasileiro de rede social que permite aos usuários se conectarem com pessoas de ideias semelhantes e compartilharem ideias sobre uma variedade de tópicos. Foi fundado em 2012 com o objetivo de criar uma plataforma de vigilância participativa, que é definida como
\begin{citacao}
    "uma forma de inteligência coletiva em que as pessoas se reúnem para monitorar, analisar e agir coletivamente em relação a um problema ou questão compartilhada" \cite[p. 1]{2011_Bryer}.
\end{citacao}

A vigilância participativa tem sido utilizada em diversos contextos, incluindo resposta a desastres, prevenção de crimes e monitoramento ambiental \cite[text]{2009_Girardin}. No caso do Colab.re, o foco está em melhorar a vida urbana, permitindo que os usuários relatem problemas como buracos nas ruas, postes de luz quebrados, poluição, etc às autoridades locais.

Um dos desafios da vigilância participativa é garantir que as informações compartilhadas sejam precisas e imparciais. É aí que o conceito de câmaras de eco se torna relevante. Uma câmara de eco é uma situação em que as crenças de uma pessoa são reforçadas pela exposição a informações que são consistentes com suas opiniões existentes, enquanto informações conflitantes são ignoradas ou rejeitadas (Sunstein, 2007). No contexto do Colab.re, uma câmara de eco poderia levar a relatos imprecisos de problemas, pois os usuários podem compartilhar apenas informações que estejam alinhadas com suas concepções prévias do que constitui um problema em sua comunidade.

O perigo das câmaras de eco em redes sociais é amplamente documentado. Elas podem levar à polarização, onde grupos se tornam mais extremistas em suas opiniões e menos propensos a se envolver com pontos de vista opostos \cite[text]{2001_Sunstein_BOOK}. Isso pode levar, em última instância, a uma quebra na comunicação e à incapacidade de resolver problemas complexos (Levy/Nail, 2020).

Para abordar o problema das câmaras de eco no Colab.re, nossa equipe de pesquisa desenvolveu uma ferramenta de análise de rede que pode identificar grupos de usuários que provavelmente compartilham visões semelhantes e interagem principalmente entre si. O objetivo dessa ferramenta é ajudar os administradores do Colab.re a identificar potenciais câmaras de eco e tomar medidas para garantir que uma variedade diversificada de opiniões seja representada na plataforma.