A \sigla{ARS}{Análise de Redes Sociais} emergiu como uma ferramenta poderosa e cada vez mais popular para analisar a estrutura e a dinâmica das redes sociais. Utilizada para estudar uma variedade de fenômenos, como comportamento organizacional, redes políticas, crime e inovação, a ARS tem demonstrado ser uma metodologia extremamente versátil. No Brasil, a relevância da ARS é evidenciada em múltiplos contextos e áreas de estudo, incluindo o planejamento urbano, a avaliação de políticas públicas, a compreensão das dinâmicas de migração e a análise de preconceitos e divisões sociais nas redes sociais [1, 2, 3, 4, 5].

Um dos aspectos que torna a {ARS} especialmente relevante no Brasil é o alto uso de redes sociais pela população. O Brasil é um dos países com maior número de usuários de redes sociais no mundo, criando um vasto campo de dados que pode ser analisado através da {ARS}. Além disso, a diversidade cultural e regional do Brasil, com suas muitas diferenças locais, proporciona um cenário complexo que a ARS pode ajudar a decifrar. Ao identificar padrões de interação e circulação de informações nas redes sociais, a ARS pode revelar como essas diferenças regionais e culturais se manifestam online.

Além disso, o Brasil enfrenta uma série de questões sociais complexas e uma alta polarização política, aspectos que são frequentemente expressos e amplificados nas redes sociais. A ARS pode ser uma ferramenta valiosa para entender a formação e a dinâmica dessas polarizações, assim como para estudar a formação de grupos de opinião e a disseminação de informações (ou desinformação). Por último, eventos de grande escala, como a Copa do Mundo, as Olimpíadas ou as eleições presidenciais, geram uma enorme quantidade de atividade nas redes sociais, proporcionando oportunidades únicas para a aplicação da ARS.

Diante deste cenário, este capítulo apresenta um resumo breve da de algumas contribuições relevantes que utilizam a ARS no Brasil, começando com uma revisão de suas principais contribuições teóricas e metodológicas. Em seguida, ele discute os desafios atuais e futuros na aplicação desta abordagem no contexto brasileiro, com o objetivo de explorar como a ARS pode continuar a fornecer insights valiosos em meio à constante evolução das redes sociais.

As contribuições teóricas e metodológicas da ARS no contexto brasileiro são diversas e significativas. A ARS tem sido aplicada em uma ampla gama de contextos, desde a análise de redes de colaboração científica até a exploração de redes de interação em plataformas de mídia social.

No campo teórico, a ARS tem contribuído para a compreensão de como as redes sociais influenciam uma variedade de fenômenos sociais. Por exemplo, o estudo de Melo et al. (2023) sobre a rede de interações no Twitter durante as eleições presidenciais de 2018 no Brasil contribuiu para a teoria da formação de grupos de opinião e da polarização política [6]. Da mesma forma, Santos et al. (2023) utilizaram a ARS para analisar a rede de interações no Facebook em uma comunidade quilombola no Brasil. Eles descobriram que a rede é altamente conectada, com uma forte presença de laços familiares e de amizade [7]. Este estudo ilustra como a ARS pode ser usada para entender a dinâmica das redes sociais em comunidades específicas, fornecendo insights valiosos sobre a estrutura e a dinâmica dessas redes.

Outro estudo relevante é o de Borges et al. (2023), que analisou a rede de interações no Twitter durante o \#ProtestodosPintas em Natal (RN) [11]. O estudo encontrou que os significados construídos nas redes sociais sobre o protesto foram em grande parte negativos, com a rede sendo articulada em torno dos perfis @tribunadonorte e @BlogdoBG. Este estudo demonstra como a ARS pode ser usada para analisar a opinião pública e a formação de consenso (ou dissensão) em torno de eventos específicos.

No campo metodológico, a ARS tem contribuído para o desenvolvimento de técnicas avançadas de análise de redes. Por exemplo, o estudo de Silva et al. (2023) sobre a rede de colaboração científica em pesquisas sobre a COVID-19 no Brasil utilizou técnicas avançadas de ARS para analisar a estrutura e a dinâmica da rede [8]. Da mesma forma, o estudo de Fonseca et al. (2023) sobre a disseminação de informações sobre a dengue no Twitter utilizou técnicas de ARS para analisar a propagação de informações e desinformação na rede [9].

A tese de doutorado de Renata Lopes Rosa (2016) também fez uma contribuição significativa para o campo metodológico, propondo novas métricas de sentimentos e afeto para melhorar a análise de sentimentos [10]. A métrica de análise de sentimentos associada a um fator de correção correspondente para n-gramas, tempos verbais, expressões e características pessoais, como idade, gênero e educação, é desenvolvida neste trabalho. As frases são classificadas em intensidade de sentimentos positivos, neutros ou negativos por meio de um novo dicionário de palavras em português e de um novo cálculo de sentimentos. A solução de análise de sentimentos e afeto é aplicada em um sistema de recomendação de músicas, como estudo de caso, que sugere conteúdos de acordo com o estado emocional da pessoa.

No entanto, a aplicação da ARS no contexto brasileiro também apresenta uma série de desafios. Primeiramente, a análise de redes sociais requer acesso a grandes volumes de dados, o que pode ser difícil de obter devido a questões de privacidade e ética. Isso é particularmente relevante no Brasil, onde a legislação de proteção de dados é rigorosa. A pesquisa de Oliveira et al. (2023) destaca esse desafio, discutindo a necessidade de desenvolver competências em ARS entre os profissionais de saúde para melhorar a gestão de sistemas de saúde [12].

Além disso, a análise de redes sociais pode ser complexa e requer habilidades técnicas e metodológicas avançadas. Isso pode ser um desafio no Brasil, onde a formação em métodos quantitativos avançados pode ser limitada. A pesquisa de Oliveira et al. (2023) destaca esse desafio, discutindo a necessidade de desenvolver competências em ARS entre os profissionais de saúde para melhorar a gestão de sistemas de saúde [12].

Outro desafio é a necessidade de adaptar a ARS para lidar com a constante evolução das redes sociais. Como Borges et al. (2023) apontam, as redes sociais estão em constante mudança, com novas plataformas emergindo e padrões de uso mudando ao longo do tempo. Isso requer que os pesquisadores estejam constantemente atualizando suas abordagens e técnicas de análise.

Finalmente, a diversidade cultural e regional do Brasil apresenta um desafio adicional. Como mencionado anteriormente, o Brasil é um país de grande diversidade, com muitas diferenças locais. Isso significa que a análise de redes sociais no Brasil deve levar em consideração essa diversidade, adaptando-se às especificidades de diferentes contextos regionais e culturais.

Apesar desses desafios, a ARS tem um grande potencial para contribuir para a compreensão de uma ampla gama de fenômenos sociais no Brasil. Por exemplo, a pesquisa de Gomes et al. (2023) sobre a rede de colaboração entre os pesquisadores brasileiros em Ciência da Informação demonstrou como a ARS pode ser usada para analisar a estrutura e a dinâmica das redes de colaboração científica [13].

Nesse contexto, uma pesquisa sobre a rede social Colab.re tem um potencial de se inserir de maneira relevante no cenário da Análise de Redes Sociais (ARS) no Brasil, um campo de estudo que tem visto aplicações diversificadas e significativas.

Assim como estudos anteriores, como o de Melo et al. (2023) e Fonseca et al. (2023), que exploraram a dinâmica das redes sociais durante eventos políticos importantes e a disseminação de informações e desinformação, o estudo busca entender a formação e a dinâmica das câmaras de eco. Este trabalho se alinha com os esforços anteriores para entender a polarização e a formação de grupos de opinião nas redes sociais.

Também aborda os desafios associados à aplicação da ARS no Brasil. Como Borges et al. (2023) apontam, as redes sociais estão em constante mudança, com novas plataformas emergindo e padrões de uso mudando ao longo do tempo. Ao focar em uma plataforma específica, a pesquisa do Colab.re responde a essa dinâmica, adicionando mais um pool de dados para o corpo de conhecimento da ARS no Brasil tanto de forma quantitativa, através da disponibilização dos dados, quanto de forma qualitativa, provendo um framework para futuras análises.

Com base nos dados extremamente localizados do Colab.re, nos quais os usuários interagem e criam eventos relacionados a problemas específicos em suas cidades, como buracos nas vias, calçadas irregulares, descarte irregular de lixo, vazamentos de água e iluminação pública, é possível obter insights valiosos tanto no âmbito social quanto político.

Em termos sociais, a análise das interações e dos padrões de engajamento dos usuários pode revelar informações sobre a coesão social e a formação de grupos de interesse em nível local. Por exemplo, ao examinar as postagens sobre problemas específicos, como buracos nas vias, é possível identificar redes de interação entre os usuários que compartilham preocupações semelhantes. Essas redes podem revelar comunidades de interesse e fornecer insights sobre a participação cívica local e a busca de soluções colaborativas para questões cotidianas.

Do ponto de vista político, a análise das interações políticas no Colab.re pode fornecer informações sobre a polarização e a formação de grupos de opinião em nível local. Por exemplo, ao examinar as postagens relacionadas a políticas públicas, é possível identificar padrões de interação entre usuários com diferentes posições políticas. Esses padrões podem ajudar a compreender a dinâmica da polarização política em nível local e como isso influencia a deliberação pública e a tomada de decisões políticas.

Em suma, a análise dos dados do Colab.re sob uma perspectiva da ARS permite adquirir insights valiosos em diversos aspectos sociais e políticos, desde a coesão social em nível local até a dinâmica da polarização política. Ao estabelecer essas conexões, é possível obter uma compreensão mais abrangente e contextualizada das redes sociais e de como elas impactam a vida cotidiana e a tomada de decisões nas comunidades.