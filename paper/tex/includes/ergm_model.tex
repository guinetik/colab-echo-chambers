\begin{codigo}[caption={Exemplo de criação de um modelo ERGM na linguagem R}, label={codigo:ergm_model}, language=R, breaklines=true]
    library(network)
    library(igraph)
    library(ergm)
    # Função para checar reciprocidade
    check_reciprocity <- function(source, target, data) {
      return(any(data$source == target & data$target == source))
    }
    # Carrega os dados
    data <- read.csv("sampled_graph.csv")
    # Aplica a função para checar reciprocidade
    data$mutual <- mapply(check_reciprocity, data$source, data$target, MoreArgs = list(data = data))
    # Converte a coluna para numérica
    data$mutual <- as.numeric(data$mutual)
    # Converte o dataframe para um objeto network que representa um grafo em R
    network <- network(data[, c("source", "target")], directed = TRUE, loops = TRUE)
    # Atribui os atributos ao grafo
    network %v% "location" <- data$location
    network %v% "gender" <- data$gender
    network %v% "age" <- data$age
    network %v% "mutual" <- data$mutual
    # Exibe um resumo do grafo
    summary(network)
    # Criando o modelo
    model <- ergm(network ~ edges 
                  + nodematch("location")
                  + istar(1) + ostar(1)
                  + nodematch("age")
                  + gwidegree(decay=0.2, fixed=T, attr=NULL,cutoff=30,levels=NULL) #Popularity spread (indegree)
                  + gwodegree(decay=0.2, fixed=T, attr=NULL,cutoff=30,levels=NULL) #Acitivty spread (outdegree)
                  + nodefactor("age")
                  + gwdsp(decay = 0.5)
                  + nodeicov("age")
                  + mutual,
                  control = control.ergm(seed = 123))
    # Obtem os coeficientes estimados
    parameters <- coef(model)
    # Simula o modelo
    num_simulations <- 2  # Numero de simulações desejadas
    simulated_networks <- simulate(model, nsim = num_simulations)
    # Exibe os resultados
    print(parameters)
    print(simulated_networks)
\end{codigo}