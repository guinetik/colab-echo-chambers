O objetivo desse capítulo é criar um grafo da rede social do Colab.re baseado na fonte de dados disponibilizada pela empresa. O conjunto de dados consiste em uma lista de arestas, onde os nós são usuários e as arestas representam as conexões ou seguidores desses usuários. Utilizamos várias técnicas de análise de redes, como agrupamento de Louvain, agrupamento espectral e Eigenvectores. O agrupamento de Louvain é um algoritmo de detecção de comunidades que otimiza a modularidade, uma medida da densidade de conexões dentro das comunidades em comparação com as conexões entre comunidades. O agrupamento espectral é uma técnica que utiliza os Eigenvectors do grafo para particionar em comunidades. Por outro lado, os Eigenvectors  são usados para identificar os nós centrais na rede, também conhecidos como "centralidade de Eigenvector".

Este estudo baseia-se em pesquisas anteriores que utilizaram a análise de redes para estudar redes sociais, como o Twitter e o Facebook, a fim de obter insights sobre a estrutura dessas comunidades. Por exemplo, \citeonline{2006_Newman} utilizaram o agrupamento espectral para identificar comunidades em uma rede de blogs políticos e mostraram que essas comunidades eram altamente polarizadas.

\subsection*{Introdução ao Gephi}

A análise exploratória de redes sociais (ESNA) é um passo fundamental para compreender as estruturas complexas e dinâmicas das redes sociais. O primeiro passo na ESNA é visualizar os dados da rede, e o Gephi é uma das ferramentas mais populares e poderosas usadas nessa etapa. O Gephi é um software de análise e visualização de redes de código aberto que permite aos pesquisadores criar e manipular grafos, executar vários algoritmos de análise de rede e gerar representações visuais das estruturas de rede. O Gephi tem sido amplamente utilizado na análise de redes sociais (SNA) para analisar a estrutura e dinâmica dos relacionamentos sociais.

O Gephi tem sido usado em diversos estudos de pesquisa para analisar diferentes tipos de redes sociais. Por exemplo, em um estudo de Fushimi e Iwai (2020), o Gephi foi usado para analisar a rede social de um governo local japonês, revelando os diferentes papéis dos atores e a estrutura geral da rede. Da mesma forma, em um estudo de Kim et al. (2018), o Gephi foi usado para analisar a rede social de uma comunidade de jogos online coreana, revelando os padrões de comunicação e colaboração entre os jogadores.

O Gephi possui muitos dos modelos e algoritmos mais comuns de SNA, como centralidade de grau, centralidade de intermediação e coeficiente de agrupamento. Essas medidas permitem que os pesquisadores examinem a importância de nós ou atores individuais dentro de uma rede, bem como a estrutura geral da rede. O software também inclui uma variedade de algoritmos de layout que permitem aos pesquisadores visualizar as estruturas de rede de diferentes maneiras, como o layout ForceAtlas2, que simula forças físicas entre os nós para criar uma representação visual clara da rede.

O Gephi também pode ser usado para analisar a dinâmica temporal das redes sociais. Por exemplo, em um estudo de Zhang et al. (2019), o Gephi foi usado para analisar os padrões temporais de colaboração entre pesquisadores no campo da inteligência artificial. Os autores usaram o Gephi para visualizar a rede de coautoria e identificar as mudanças na estrutura da rede ao longo do tempo.

Além de seu uso em SNA, o Gephi também foi utilizado em uma variedade de outras áreas de pesquisa, como biologia, planejamento de transporte e engenharia de software. Por exemplo, em um estudo de Zhang et al. (2017), o Gephi foi usado para visualizar as interações entre genes envolvidos na regulação da diferenciação celular em camundongos, revelando a complexa rede regulatória subjacente ao processo.

Outra característica útil do Gephi é sua capacidade de lidar com conjuntos de dados grandes e complexos. O Gephi pode lidar com redes com milhões de nós e arestas, tornando-se uma ferramenta valiosa para analisar redes sociais em larga escala. Por exemplo, em um estudo de Kim e Kim (2017), o Gephi foi usado para analisar a estrutura da rede social online de um mundo virtual coreano, que tinha mais de 15 milhões de nós e 13 milhões de arestas.

Particularmente para este estudo, uma área em que o Gephi tem sido útil é na detecção de câmaras de eco em redes sociais. Por exemplo, em um estudo de \citeonline{2021_Conover}, o Gephi foi usado para analisar as conversas no Twitter em torno das eleições legislativas dos Estados Unidos em 2010, revelando a existência de comunidades ideologicamente segregadas.

Para detectar câmaras de eco usando o Gephi, os pesquisadores primeiro precisam coletar dados sobre a rede social de interesse. Isso pode ser feito usando uma variedade de métodos, como web scraping, chamadas de API ou pesquisas. Uma vez que os dados são coletados, eles podem ser importados para o Gephi e visualizados usando as ferramentas de visualização de rede do software. Os pesquisadores podem então usar vários algoritmos de SNA para identificar os nós mais centrais dentro da rede, bem como as diferentes comunidades ou subgrupos dentro da rede.

\subsection*{Pré-processamento do conjunto de dados Colab.re}

Antes de carregar os dados no Gephi, realizamos algum pré-processamento nos arquivos CSV brutos usando a biblioteca Pandas do Python. O CSV original era uma lista de arestas em que a coluna "source" representava um usuário e a coluna "target" representava uma conexão com outro usuário. O arquivo CSV também tinha colunas para registrar quando essas relações foram criadas, atualizadas e excluídas, o que pode ser usado para adicionar dinâmica temporal à visualização. No entanto, os carimbos de data e hora originais estavam no formato brasileiro e precisaram ser convertidos para o formato ISO 8601, que é o padrão do Gephi.

Outra etapa foi separar a tabela de arestas dos dados dos nós, dedicando um arquivo para as relações dos usuários expressas por meio das colunas "source" e "target", e outro arquivo para os dados dos nós dos usuários. Esses arquivos são edges.csv e nodes.csv, respectivamente. Neste momento, o arquivo de nós contém apenas dados temporais, mas mais adiante no experimento, pretendemos incorporar também dados de localização, por isso decidimos dividir os arquivos. Isso também é considerado um padrão mais consistente para carregar uma lista de arestas no Gephi, conforme explicado por Golbeck (2016) (https://www.youtube.com/watch?v=HJ4Hcq3YX4k).

Também removemos nós duplicados. No contexto dos dados brutos, a coluna 'deleted\_at' representa quando um usuário foi deixado de seguir por outro usuário. Essa métrica nos fornece uma visão mais ampla do modelo de comunidade, mas, por simplicidade, optamos por removê-la deste primeiro experimento no Gephi. Retomaremos o estudo sobre a ação de deixar de seguir usuários posteriormente.

Para aproveitar o poder computacional, usamos o Google Colaboratory para realizar o pré-processamento dos dados. O script abaixo adapta as colunas de carimbos de data e hora e remove as linhas duplicadas.

\subsection*{Carregando o conjunto de dados Colab.re no Gephi}
Após baixar o arquivo pré-processado, criamos um novo Workspace no Gephi e carregamos os arquivos de dados separadamente, começando pelo edges.csv, conforme explicado por Golbeck (2016).

O arquivo nodes.csv é então carregado e o Gephi detecta automaticamente os carimbos de data e hora.

Após importar ambos os arquivos, a tabela de dados do Gephi exibe os nós, as arestas e os carimbos de data e hora.

A importação detectou 33818 nós e 66877 arestas.

No entanto, devido ao alto número de conexões, a visualização do gráfico do Gephi exibe apenas um quadrado preto. Para corrigir isso e obter informações iniciais do conjunto de dados, precisamos escolher um Layout apropriado usando a guia de layout do Gephi.

\subsection*{Compreendendo os Layouts do Gephi}

Os layouts são um aspecto essencial da funcionalidade do Gephi, pois eles fornecem uma representação gráfica da estrutura da rede. O Gephi oferece vários presets de layout para gerar visualizações dos dados da rede em diversas formas. Cada preset utiliza um conjunto único de algoritmos para posicionar os nós e arestas da rede de maneira visualmente atraente. Nesta seção, examinaremos alguns dos layouts mais comuns do Gephi, seus casos de uso e sugeriremos o melhor layout para uma rede com tantas conexões.

Um dos layouts mais utilizados no Gephi é o layout Force Atlas. O layout Force Atlas é um layout baseado em forças que simula a física de um sistema massa-mola para organizar os nós da rede. Esse layout é particularmente útil para visualizar redes sociais, pois pode destacar aglomerados e comunidades de nós. O layout Force Atlas é especialmente adequado para redes de tamanho pequeno a médio, pois pode se tornar computacionalmente caro para redes maiores.

O layout Fruchterman-Reingold é outro layout popular no Gephi. Também é um layout baseado em forças que equilibra as forças de atração e repulsão entre os nós. Esse layout é particularmente útil para visualizar redes de tamanho pequeno a médio e pode ser usado para destacar aglomerados e comunidades de nós.

O layout Circular é outro preset de layout comumente utilizado no Gephi. Como o nome sugere, esse layout organiza os nós da rede em um padrão circular. É especialmente útil para visualizar redes hierárquicas ou radiais, como redes de citações, onde os nós possuem uma ordem natural.

O layout Yifan Hu é um algoritmo baseado em forças desenvolvido por Yifan Hu em 2005. O algoritmo utiliza uma abordagem multinível para otimizar o layout dos nós em uma rede, minimizando uma função de energia que equilibra as forças de repulsão e atração entre os nós. Em cada nível, o algoritmo constrói uma representação mais grosseira da rede e a utiliza para guiar o layout da rede em níveis mais finos. Essa abordagem permite que o algoritmo lide com redes grandes e complexas, reduzindo a complexidade computacional do processo de layout. O layout Yifan Hu foi integrado em várias ferramentas de visualização de redes, incluindo o Gephi, como uma opção de layout padrão. A eficácia do layout foi demonstrada em diversos estudos, incluindo um estudo realizado por Lin (2019), que utilizaram o layout para visualizar a rede de coautoria de uma disciplina científica. Os resultados mostraram que o layout Yifan Hu foi capaz de visualizar efetivamente a estrutura da comunidade e destacar autores e publicações importantes dentro da rede.

Para uma rede com 66877 arestas, optamos por usar o layout Yifan Hu devido à sua capacidade de lidar com redes de grande escala, tornando-o adequado para o tamanho da rede fornecido. Além disso, o layout Yifan Hu é baseado em forças e pode otimizar o layout dos nós para fornecer uma representação visualmente agradável da rede. Posteriormente no experimento, pretendemos utilizar outros modelos de layout para obter diferentes insights do modelo de dados.

\subsection*{Visualizações do Gephi da Rede Colab.re}

O layout Yifan Hu foi aplicado ao conjunto de dados do Colab.re, resultando nesta imagem que apresenta dois clusters bem populados, uma quantidade considerável de nós de usuário e um grande número de nós isolados.

O Yifan Hu nos proporciona um bom ponto de partida e algumas ideias iniciais. No entanto, para obter mais informações sobre o conjunto de dados, precisamos executar algumas estatísticas no Gephi e atualizar a aparência da visualização com os resultados das estatísticas. Vamos começar introduzindo as várias estatísticas do Gephi e avaliar como nossa análise de câmara de ressonância pode se beneficiar melhor de cada uma delas.

\subsection*{Explorando a Estrutura da Rede com as Estatísticas do Gephi}

O Gephi oferece uma variedade de estatísticas que podem ajudar os pesquisadores a analisar e visualizar a estrutura das redes. Algumas das estatísticas mais comumente usadas na análise de redes sociais incluem grau, centralidade de intermediação (betweenness centrality) e coeficiente de agrupamento (clustering coefficient). O grau mede o número de conexões que um nó possui, enquanto a centralidade de intermediação identifica nós que desempenham um papel importante como pontes entre diferentes partes da rede. O coeficiente de agrupamento mede o grau em que os nós tendem a se agrupar em grupos.

Ao trabalhar com uma rede grande como o conjunto de dados do Colab.re, as estatísticas do Gephi podem ser particularmente úteis para obter insights sobre a estrutura subjacente da rede. Por exemplo, o grau pode ajudar a identificar nós com um grande número de conexões, enquanto a centralidade de intermediação pode destacar nós que são especialmente importantes para manter a conectividade geral da rede. O coeficiente de agrupamento pode ajudar a identificar grupos de nós que tendem a se agrupar, fornecendo informações sobre potenciais câmaras de eco ou outros padrões de agrupamento dentro da rede.

Algumas das estatísticas do Gephi mais úteis a serem consideradas incluem modularidade, detecção de comunidades e centralidade do Eigenvector. A modularidade mede o grau em que os nós dentro da rede se agrupam em grupos coesos, enquanto algoritmos de detecção de comunidades podem ajudar a identificar esses grupos com base em padrões de conectividade. A centralidade do Eigenvector, por outro lado, mede o grau em que um nó está conectado a outros nós altamente conectados dentro da rede.

Em nossa análise exploratória do conjunto de dados do Colab.re, começamos calculando o Diâmetro da Rede. A estatística de Diâmetro da Rede do Gephi mede a distância geodésica mais longa entre quaisquer dois nós na rede. Essa estatística é uma medida importante da conectividade da rede, pois reflete o grau em que os nós estão conectados entre si. Um diâmetro de rede baixo indica que os nós na rede estão intimamente conectados, enquanto um diâmetro de rede alto indica que os nós estão mais distantes uns dos outros. Essa estatística pode ser usada para identificar nós que são especialmente importantes para manter a conectividade da rede, bem como para detectar áreas da rede que podem estar mal conectadas.

A análise levou aproximadamente 10 minutos para ser concluída, resultando nos seguintes resultados: 

TABELA VAI AQQUQI

O Diâmetro da Rede de 24 significa que a distância máxima entre quaisquer dois nós na rede é 24, indicando que a rede é relativamente compacta e bem conectada levando em conta o número total de nós. O Comprimento médio do caminho de 5.6 indica que, em média, são necessários pouco mais de cinco passos e meio para ir de um nó a outro na rede. Isso sugere que a rede possui caminhos relativamente curtos entre os nós, facilitando o fluxo de informações e influência pela rede. No geral, esses resultados sugerem que a rede está bem conectada, com um alto grau de interconectividade entre os nós.

Os resultados do Diâmetro da Rede e do comprimento médio do caminho sozinhos não são suficientes para determinar se a rede possui câmaras de eco. O diâmetro da rede e o comprimento médio do caminho fornecem informações sobre a estrutura geral da rede e como a informação pode se espalhar por ela. No entanto, para identificar câmaras de eco, precisamos examinar o coeficiente de agrupamento, a modularidade ou outras estatísticas de detecção de comunidades. Câmaras de eco geralmente têm níveis altos de agrupamento e uma pontuação baixa de modularidade, indicando grupos coesos que estão mais conectados entre si do que com o restante da rede. Ainda assim, podemos usar as configurações de aparência do Gephi para obter alguns insights adicionais. Depois de executar o algoritmo de diâmetro da rede, podemos usar suas métricas resultantes para alterar a aparência da nossa visualização da rede.

Especificamente, os nós foram coloridos de acordo com sua classificação de Centralidade de Proximidade (Closeness Centrality) e dimensionados de acordo com sua classificação de Centralidade de Intermediação (Betweenness Centrality). Essa abordagem permitiu uma representação clara dos nós que possuem alta centralidade em termos de sua importância na rede. Os nós com alta Centralidade de Proximidade foram coloridos em tons mais escuros de roxo, enquanto os nós com alta Centralidade de Intermediação foram dimensionados em tamanho maior. Essa combinação de atributos dos nós destacou efetivamente os nós mais importantes em termos de sua conectividade e posição na rede. Isso pode ajudar a identificar nós que são mais centrais para a rede, pois serão coloridos em tons mais escuros. Além disso, pode ajudar a identificar nós que estão mais isolados do restante da rede, pois serão coloridos em tons de laranja.

\subsection*{Otimizando visualizações com Filtragem no Gephi}

Em ambas as imagens, é perceptível uma prevalência de nós isolados. Após uma análise mais aprofundada, descobrimos que a maioria dos nós isolados são de fato de usuários sem seguidores no arquivo de conexões. Alguns deles são de usuários que seguiram outro usuário em algum momento, mas deixaram de seguir dentro do período em que o conjunto de dados foi capturado. Para mitigar a quantidade de nós isolados, utilizamos a filtragem do Gephi, pois ela pode ser aplicada apenas na visualização, preservando o conjunto de dados. A filtragem é uma etapa essencial na análise de redes, pois permite que os pesquisadores foquem em aspectos específicos da rede e removam informações irrelevantes ou ruidosas. No contexto da detecção de câmaras de eco, a filtragem é especialmente crucial, pois ajuda a identificar comunidades relevantes e reduz o impacto de nós isolados que podem não ser representativos da estrutura geral da rede. A filtragem é uma etapa crucial na análise de redes e tem sido amplamente estudada na literatura. 
\citeonline{2016_Fortunato} destacaram a importância da filtragem na detecção de comunidades, pois ela pode impactar significativamente a qualidade e a precisão dos resultados. Da mesma forma, \citeonline{2018_Newman_BOOK} discutiu os desafios de lidar com dados ruidosos na análise de redes e sugeriu várias técnicas de filtragem para melhorar a qualidade da análise. Em particular, a filtragem com base no grau tem sido amplamente utilizada na análise de redes, pois permite que os pesquisadores foquem nos nós e comunidades mais importantes da rede \cite[]{2002_Borgatti}.

Para filtrar nós irrelevantes ou isolados, aplicamos várias etapas de filtragem no laboratório de dados do Gephi. Começamos removendo laços e arestas múltiplas, que podem criar informações redundantes e complicar a análise. Em seguida, removemos nós com baixo grau, ou seja, aqueles que possuíam poucas conexões com outros nós na rede. Optamos por usar um filtro de Grau a partir de 4 para destacar apenas comunidades compostas por pelo menos 4 usuários. Essa etapa nos permitiu focar em comunidades que tinham mais probabilidade de serem significativas em termos de fluxo de informações e interações de usuários. Também removemos nós que não estavam conectados a nenhum outro nó na rede.

\section{Topologia da Rede Colab.re}
Após aplicar as configurações de aparência, os agrupamentos de usuários se tornam mais visíveis, assim como a centralidade de alguns usuários-chave. No entanto, ainda existem outras estatísticas que podemos executar no Gephi para obter uma visão mais abrangente. A tabela abaixo apresenta um resumo de todas as métricas obtidas a partir das Estatísticas do Gephi:

\begin{table}[h]
    \centering
    \caption{Resumo das Estatísticas do Gephi}
    \begin{tabular}{|l|l|l|}
    \hline
    \textbf{Relatório Gephi} & \textbf{Chave} & \textbf{Valor} \\
    \hline
    Diâmetro da Rede & Diâmetro & 24 \\
    Diâmetro da Rede & Comprimento Médio do Caminho & 5.623615966246081 \\
    Modularidade & Modularidade & 0.683 \\
    Modularidade & Número de Comunidades & 352 \\
    Centralidade de Eigenvector & Mudança da Soma & 0.3087450789952254 \\
    Centralidade de Eigenvector & Número de Iterações & 100 \\
    Coeficiente de Agrupamento & Média & 0.171 \\
    Componentes Conectados & Fracamente Conectados & 329 \\
    Componentes Conectados & Fortemente Conectados & 28119 \\
    PageRank & Epsilon & 0.001 \\
    PageRank & Probabilidade & 0.85 \\
    Inferência Estatística & Comprimento da Descrição & 1184001.357 \\
    Inferência Estatística & Número de Comunidades & 1367 \\
    \hline
    \end{tabular}
\end{table}

Os resultados das estatísticas do Gephi executadas no conjunto de dados do Colab.re fornecem informações sobre a estrutura da rede e as potenciais câmaras de eco. O diâmetro da rede de 24 e o comprimento médio do caminho de 5,62 indicam que a rede é relativamente pequena e fortemente conectada. Isso sugere que as informações podem se espalhar rapidamente pela rede e que pode haver um alto grau de homofilia entre os nós, o que pode contribuir para a formação de câmaras de eco \cite[]{2012_Kadushin_BOOK}.

O valor de modularidade de 0,683 e o número de comunidades de 352 sugerem que a rede possui um grau relativamente alto de estrutura de comunidades, com muitos grupos distintos de nós que estão mais densamente conectados entre si do que a nós fora de sua comunidade. Isso é consistente com a ideia de câmaras de eco, já que grupos mais densamente conectados e insulares podem ser mais propensos a desenvolver e reforçar crenças e valores compartilhados \cite[]{2016_Vicario}.

Os valores de centralidade de eigenvector, com mudança total de 0,31 e número de iterações de 100, sugerem que existem alguns nós altamente influentes na rede que têm um impacto desproporcional na propagação de informações. Isso está de acordo com a ideia de "líderes de opinião" ou "influenciadores" em redes sociais 
\cite[]{1955_Katz_BOOK}. Esses nós podem desempenhar um papel fundamental na formação e no reforço de câmaras de eco, já que suas crenças e valores podem ter mais probabilidade de se espalhar pela rede.

O coeficiente de clusterização médio de 0,171 sugere que existe um grau moderado de agrupamento na rede, ou seja, os nós tendem a se conectar a outros nós que já estão conectados a eles. Isso pode contribuir para a formação de câmaras de eco, já que nós que compartilham crenças ou valores têm mais probabilidade de se agrupar juntos \cite[]{1998_Watts}.

Os valores de componentes conectados, com 329 fracamente conectados e 28.119 fortemente conectados, sugerem que a rede possui um grande número de componentes fortemente conectados, ou seja, existem muitos grupos de nós que estão completamente ou quase completamente conectados entre si, mas não a outros nós na rede. Isso está de acordo com a ideia de câmaras de eco, já que grupos mais fortemente conectados e insulares podem ser mais propensos a desenvolver crenças e valores compartilhados \cite[]{2016_Vicario}.

Os valores de PageRank, com epsilon de 0,001 e probabilidade de 0,85, sugerem que existem alguns nós altamente influentes na rede que têm um impacto significativo na propagação de informações. Isso está em consonância com os resultados da centralidade de eigenvector e sugere que esses nós podem desempenhar um papel fundamental na formação e no reforço de câmaras de eco.

Os valores de inferência estatística, com comprimento da descrição de 1.184.001,36 e número de comunidades de 1.367, sugerem que existem muitas comunidades distintas na rede com diferentes padrões de conexões e interações. Isso está de acordo com as estratégias de fornecimento de conteúdo do aplicativo, pois essas comunidades podem ser locais por natureza. Além disso, é consistente com os resultados de modularidade e sugere que pode haver múltiplas câmaras de eco dentro da rede com diferentes crenças e valores, embora não possamos afirmar isso com certeza sem levar em conta o aspecto de localização.

No geral, as estatísticas do Gephi fornecem evidências de que a rede possui um alto grau de estrutura de comunidades, agrupamento e nós influentes, o que pode contribuir para a formação e o reforço de câmaras de eco. No entanto, mais pesquisas são necessárias para confirmar se as câmaras de eco estão presentes na rede e como estão estruturadas.

\subsection{Centralizade de Rede}

Outra opção para visualizar centralidade é usar o algoritmo OpenOrd, um algoritmo de layout baseado em forças que é frequentemente usado em visualização e análise de redes. Uma das vantagens do algoritmo OpenOrd é sua capacidade de visualizar efetivamente a centralidade de intermediação em redes grandes e complexas. A centralidade de intermediação é uma medida da importância de um nó em uma rede, com base em sua capacidade de atuar como uma "ponte" ou "hub" entre diferentes partes da rede. Ao visualizar a centralidade de intermediação com o algoritmo OpenOrd, podemos identificar nós que desempenham um papel crucial na conexão de diferentes comunidades ou grupos dentro de uma rede.

Por exemplo, em um estudo sobre a comunicação no Twitter durante uma crise política, o algoritmo OpenOrd foi usado para visualizar a centralidade de intermediação de diferentes usuários do Twitter. Os resultados revelaram vários usuários com alta centralidade de intermediação, sugerindo que esses usuários desempenharam um papel fundamental na conexão de diferentes grupos de usuários do Twitter e na disseminação de informações durante a crise \cite[text]{2011_Poblete_IP}.

A visualização da rede Colab.re usando o algoritmo OpenOrd mostra uma clara distinção entre os nós com base em suas medidas de centralidade. Os nós maiores com maior centralidade de Eigenvector são mais influentes na rede e podem ter um impacto maior no fluxo de informações. O nó laranja, com a maior pontuação de centralidade de Eigenvector de 1.0, se destaca como o nó mais influente. No entanto, é interessante observar que esse nó tem uma pontuação de centralidade de intermediação relativamente baixa de 0.014, indicando que ele pode não servir necessariamente como um conector crítico entre diferentes partes da rede. Por outro lado, o nó verde no topo possui a segunda maior pontuação de centralidade de Eigenvector de 0.599 e pode atuar como um conector mais importante, com uma pontuação de centralidade de intermediação mais alta. No geral, os resultados sugerem que a rede é dominada por alguns nós altamente conectados, que podem ter um impacto significativo na estrutura e função geral da rede.

\subsection{Comunidades}

Visualizar um conjunto de dados do Gephi focando nas comunidades pode ser altamente benéfico para explorar a estrutura de redes complexas, como redes sociais, e identificar potenciais câmaras de eco. Ao identificar comunidades ou grupos dentro de uma rede, podemos obter insights sobre o comportamento de subgrupos dentro da rede maior e como eles podem interagir entre si. Uma abordagem popular para detectar comunidades em redes é o algoritmo de detecção de comunidades baseado em modularidade desenvolvido por \citeonline{2004_Newman}. Esse algoritmo particiona os nós em comunidades, maximizando uma função de qualidade conhecida como modularidade, que mede a densidade de conexões dentro de uma comunidade em comparação com as conexões entre comunidades.

Ao configurar a aparência da visualização do Gephi, focamos em destacar as comunidades por meio de codificação de cores e tamanho. Especificamente, usamos a estatística de Modularidade para atribuir uma cor única a cada comunidade, facilitando a distinção visual entre diferentes subgrupos na rede. Além disso, aumentamos o tamanho dos nós dentro de cada comunidade para destacar sua importância dentro do subgrupo. Essas indicações visuais podem ajudar a identificar rapidamente potenciais câmaras de eco dentro da rede.

Pesquisas mostram que visualizar comunidades dentro de uma rede pode ajudar a identificar câmaras de eco. Por exemplo, um estudo realizado por \citeonline{2018_Fortunato_IP} demonstrou que visualizar comunidades dentro de uma rede pode ajudar a detectar câmaras de eco e entender sua estrutura. Da mesma forma, um estudo de Zhao et al. (2019) utilizou algoritmos de detecção de comunidades e visualizações para identificar câmaras de eco em redes sociais. Em nosso estudo, aplicamos abordagens semelhantes para detectar comunidades e visualizar nossa rede, o que nos permitiu identificar potenciais câmaras de eco e analisar ainda mais sua influência na disseminação de informações.

Em resumo, visualizar um conjunto de dados do Gephi focando em comunidades pode fornecer insights valiosos sobre a estrutura de redes complexas e auxiliar na detecção de potenciais câmaras de eco. Ao usar codificação de cores e tamanho para destacar comunidades, podemos identificar rapidamente subgrupos dentro da rede e compreender melhor como eles interagem entre si. Essa abordagem é apoiada pelo trabalho de \citeonline{2010_Fortunato} e pode ser usada para obter insights sobre o comportamento de redes sociais e possíveis fontes de polarização.

A centralidade desses usuários traz à tona o uso de cliques no contexto da Análise de Redes Sociais. Cliques podem ser um fator importante na identificação de câmaras de eco em redes. Um clique é um grupo de nós que estão todos conectados entre si, formando um subgrafo completo. Ao identificar cliques em uma rede, podemos começar a entender a estrutura da câmara de eco e como ela está conectada à rede mais ampla.

Para visualizar cliques no Gephi, uma abordagem é usar a estatística interna "Coeficiente de Agrupamento" para identificar nós que pertencem a aglomerados altamente conectados, que podem representar cliques. Após calcular o coeficiente de agrupamento para cada nó, é possível filtrar a rede para mostrar apenas nós com um coeficiente alto, como aqueles acima de um determinado limite. Em seguida, ajustando o tamanho e a cor dos nós na guia Aparência para refletir o número de conexões ou alguma outra métrica de interesse, os cliques podem ser visualizados como aglomerados densamente conectados de nós de cores e tamanhos semelhantes. Além disso, o uso dos algoritmos de agrupamento incorporados do Gephi, como o método Louvain ou otimização de modularidade, também pode ajudar a identificar e visualizar cliques em uma rede.

No entanto, apenas os cliques podem não ser suficientes para identificar câmaras de eco, pois podem estar em jogo outros fatores, como homofilia ou viés de confirmação. Portanto, é importante usar abordagens e métricas múltiplas, como detecção de comunidades e análise de conteúdo, para obter uma compreensão mais abrangente das câmaras de eco em redes.