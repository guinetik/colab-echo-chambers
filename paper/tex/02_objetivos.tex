\section{Objetivos}

\subsection{Objetivo Geral}
Desenvolver e aplicar um conjunto de métricas quantitativas para prever comportamentos polarizadores e identificar câmaras de eco em comunidades online, utilizando como estudo de caso a rede social do aplicativo Colab.

\subsection{Objetivos Específicos}
\begin{enumerate}
	\item Analisar as interações dos usuários no aplicativo Colab, examinando as principais métricas e padrões de comportamento que contribuem para a polarização.

	\item Conduzir uma análise exploratória da rede do Colab utilizando a ferramenta de visualização Gephi, com o intuito de mapear a estrutura da rede e identificar potenciais subgrupos polarizados.

	\item Implementar uma análise de sentimento das postagens dos usuários, culminando no desenvolvimento de um modelo de regressão de aprendizado de máquina que atribui uma pontuação de sentimento positiva ou negativa a cada postagem.

	\item Classificar as personas dos usuários do Colab através de algoritmos de aprendizado de máquina, utilizando um conjunto de dados de postagens anotadas para treinar o modelo de classificação.

	\item Analisar os tipos de eventos mais criados pelos usuários do Colab, correlacionando as personas dos usuários e as pontuações de sentimento para estabelecer uma métrica de pressão social que mede o impacto de cada evento na rede.

	\item Construir heurísticas para a detecção de câmaras de eco baseadas em métricas tradicionais de análise de redes e novas métricas derivadas da análise de sentimento e classificação de personas.

	\item Aplicar o detector de câmaras de eco desenvolvido nas redes sociais de Niterói, Santo André e Mesquita, para identificar as origens e analisar a polarização dessas redes.
\end{enumerate}

Estes objetivos específicos proporcionam uma estrutura metodológica para a identificação e análise de comportamentos polarizadores e câmaras de eco. A integração das métricas de pressão social com a análise de sentimento e classificação de personas permitirá uma avaliação detalhada da qualidade do debate público e do engajamento cidadão nas redes sociais locais.
