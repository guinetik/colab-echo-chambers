O objetivo deste estudo é aproveitar técnicas de análise de redes e epidemiologia digital para detectar câmaras de eco no aplicativo Colab.re. Especificamente, pretendemos:

1. Construir um grafo de rede social das interações dos usuários dentro do aplicativo Colab.re utilizando os dados da lista de arestas.
2. Aplicar algoritmos de agrupamento espectral e detecção de comunidades para identificar grupos de usuários com padrões de interação semelhantes.
3. Utilizar os modelos SIR e SEIR da epidemiologia digital para simular a disseminação de opiniões ou crenças nos grupos de usuários identificados.
4. Desenvolver um painel em tempo real para monitorar o surgimento e crescimento de câmaras de eco dentro do aplicativo Colab.re com base nos resultados da análise de rede e nos modelos de epidemiologia digital.

Ao alcançar esses objetivos, esperamos fornecer insights sobre a formação e proliferação de câmaras de eco dentro do aplicativo Colab.re e contribuir para o desenvolvimento de estratégias eficazes para combater os impactos negativos das câmaras de eco em comunidades online.