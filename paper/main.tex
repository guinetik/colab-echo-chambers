% ------------------------------------------------------------------------
% ------------------------------------------------------------------------
% ICMC: Modelo de Trabalho Acadêmico (tese de doutorado, dissertação de
% mestrado e trabalhos monográficos em geral) em conformidade com 
% ABNT NBR 14724:2011: Informação e documentação - Trabalhos acadêmicos -
% Apresentação
% ------------------------------------------------------------------------
% ------------------------------------------------------------------------

% Opções: 
%   Qualificação          = qualificacao 
%   Curso                 = doutorado/mestrado/tcc
%   Situação do trabalho  = pre-defesa/pos-defesa (exceto para qualificação)
%   Versão para impressão = impressao
\documentclass[mestrado, pre-defesa]{packages/icmc}
% ---------------------------------------------------------------------------
% Pacotes Opcionais
% ---------------------------------------------------------------------------
\usepackage{rotating}                       % Usado para rotacionar o texto
\usepackage[all,knot,arc,import,poly]{xy}   % Pacote para desenhos gráficos
\usepackage{lipsum}                         % Usado para adicionar texto temporario
\usepackage{pgf-pie}                        % Usado para adicionar gráficos de pizza
\usepackage[section]{placeins}              % Usado para manter imagens e tabelas dentro da seção
\usepackage{subcaption}
\usepackage{xcolor}
\definecolor{maroon}{cmyk}{0, 0.87, 0.68, 0.32}
\definecolor{halfgray}{gray}{0.55}
\definecolor{ipython_frame}{RGB}{207, 207, 207}
\definecolor{ipython_bg}{RGB}{247, 247, 247}
\definecolor{ipython_red}{RGB}{186, 33, 33}
\definecolor{ipython_green}{RGB}{0, 128, 0}
\definecolor{ipython_cyan}{RGB}{64, 128, 128}
\definecolor{ipython_purple}{RGB}{170, 34, 255}
\lstdefinelanguage{Python}{
  morekeywords={access,and,break,class,continue,def,del,elif,else,except,exec,finally,for,from,global,if,import,in,is,lambda,not,or,pass,print,raise,return,try,while},
  morekeywords=[2]{abs,all,any,basestring,bin,bool,bytearray,callable,chr,classmethod,cmp,compile,complex,delattr,dict,dir,divmod,enumerate,eval,execfile,file,filter,float,format,frozenset,getattr,globals,hasattr,hash,help,hex,id,input,int,isinstance,issubclass,iter,len,list,locals,long,map,max,memoryview,min,next,object,oct,open,ord,pow,property,range,raw_input,reduce,reload,repr,reversed,round,set,setattr,slice,sorted,staticmethod,str,sum,super,tuple,type,unichr,unicode,vars,xrange,zip,apply,buffer,coerce,intern},
  sensitive=true,
  morecomment=[l]\#,
  morestring=[b]',
  morestring=[b]",
  morestring=[s]{'''}{'''},
  morestring=[s]{"""}{"""},
  morestring=[s]{r'}{'},
  morestring=[s]{r"}{"},
  morestring=[s]{r'''}{'''},
  morestring=[s]{r"""}{"""},
  morestring=[s]{u'}{'},
  morestring=[s]{u"}{"},
  morestring=[s]{u'''}{'''},
  morestring=[s]{u"""}{"""},
  identifierstyle=\color{black}\ttfamily\ABNTEXfontereduzida,
  commentstyle=\color{ipython_cyan}\ttfamily\ABNTEXfontereduzida,
  stringstyle=\color{ipython_red}\ttfamily\ABNTEXfontereduzida,
  keepspaces=true,
  showspaces=false,
  showstringspaces=false,
  rulecolor=\color{ipython_frame},
  frame=single,
  frameround={t}{t}{t}{t},
  framerule=0.8pt,
  xleftmargin=20.25pt,
  framexleftmargin=21.25pt,
  framexbottommargin=5pt,
  framextopmargin=5pt,
  abovecaptionskip=10pt,
  numbers=left, 
  numbersep=6pt,
  numberstyle=\ttfamily\ABNTEXfontereduzida\color{halfgray},
  backgroundcolor=\color{ipython_bg},
  % extendedchars=true,
  basicstyle=\scriptsize,
  keywordstyle=\color{ipython_green}\ttfamily\ABNTEXfontereduzida,
}
\usepackage{tikz}
\usetikzlibrary{shapes.geometric, arrows,shapes, positioning, calc}               % Minhas configurações pessoais
%\usepackage[subsection]{placeins}          % Usado para manter imagens e tabelas dentro da subseção
% Este pacote pode conflitar com outros pacotes gráficos como o ``pictex''
% Então é necessário usar apenas um dos pacotes conflitantes
\newcommand{\VerbL}{0.52\textwidth}
\newcommand{\LatL}{0.42\textwidth}
% ---------------------------------------------------------------------------

% ---
% Informações de dados para CAPA e FOLHA DE ROSTO
% ---
% Tanto na capa quanto nas folhas de rosto apenas a primeira letra da primeira palavra (ou nomes próprios) devem estar em letra maiúscula, todas as demais devem ser em letra minúscula.
\tituloPT{Câmaras de Eco nas Mídias Sociais: Análise de Rede do Aplicativo Colab.re}
\tituloEN{Echo Chambers in Social Media: Network Analysis of Colab.re app}
\autor[Sousa, J. G. R.]{João Guilherme Ribeiro de Sousa}
\genero{M} % Gênero do autor (M = Masculino / F = Feminino)
\orientador[Orientador]{Prof. Dr.}{Erico Souza Teixeira}
\coorientador{Prof. Dr.}{Onício Leal Neto}
\curso{CCMC}
\data{05}{11}{2023} % Data do depósito
\idioma{PT} % Idioma principal do documento (PT = português / EN = inglês)
% ---


% ---
% RESUMOS
% ---

% Resumo em PORTUGUÊS
% conter no máximo 500 palavras
% conter no mínimo 1 e no máximo 5 palavras-chave
\textoresumo[brazil]{Este estudo analisa as dinâmicas de polarização e câmaras de eco na rede social do aplicativo Colab, uma plataforma de cidadania, especificamente nas comunidades de Santo André (SP), Mesquita e Niterói (RJ). Os resultados revelam a presença de pelo menos uma comunidade polarizada em cada cidade e como as interações dentro da plataforma digital espelham e influenciam os contextos sociais urbanos.

A pesquisa adota a metodologia desenvolvida por Atiqi (2023), aplicando técnicas de modelagem baseada em agentes, originalmente aplicadas em simulações, em uma rede social real. Desenvolvemos heurísticas de detecção que combinam análise de redes sociais, análise de sentimento e classificação de personas utilizando aprendizagem de máquina. Esta abordagem permite uma análise detalhada dos padrões de comunicação e comportamento dos usuários baseado nos seus relacionamentos e o conteúdo criado e compartilhado entre eles.

O estudo não apenas identifica grupos polarizados, mas também fornece insights sobre como esses grupos se formam e interagem dentro do ecossistema da rede. Sob a lente da polarização, emerge uma nova perspectiva em que o Colab pode ser interpretado como um barômetro social hiperlocal. Esta percepção sublinha como as interações na plataforma se tornam um reflexo direto e imediato das necessidades das comunidades urbanas. A análise dessas dinâmicas de pressão social pode oferecer um novo paradigma para agentes governamentais e tomadores de decisão, o que permite um melhor entendimento e uma resposta mais eficaz às demandas locais. O estudo também se aprofunda na fenomenologia das interações no Colab e destaca como a experiência dos usuários na plataforma molda a percepção da realidade urbana e contribui para o entendimento mais amplo das dinâmicas sociais, levantando questões sobre a relação entre o digital e o físico na experiência urbana.}
{Câmaras de Eco; Polarização; Análise de Redes Sociais; Teoria dos Grafos}


% resumo em INGLÊS
% conter no máximo 500 palavras
% conter no mínimo 1 e no máximo 5 palavras-chave
\textoresumo[english]{This study examines the dynamics of polarization and echo chambers in the social network of the Colab app, a citizenship platform, specifically in the communities of Santo André (SP), Mesquita, and Niterói (RJ) in Brazil. The results reveal the presence of at least one polarized community in each city and how interactions within the digital platform mirror and influence urban social contexts.

The research adopts the methodology developed by Atiqi (2023), applying agent-based modeling techniques, originally used in simulations, to a real social network. We developed detection heuristics that combine social network analysis, sentiment analysis, and persona classification using machine learning. This approach allows a detailed analysis of communication patterns and user behavior based on their relationships and the content created and shared among them.

The study not only identifies polarized groups but also provides insights into how these groups form and interact within the network's ecosystem. Under the lens of polarization, a new perspective emerges where Colab can be interpreted as a hyperlocal social barometer. This perception underscores how interactions on the platform become a direct and immediate reflection of the urban communities' needs. Analyzing these dynamics of social pressure can offer a new paradigm for governmental agents and decision-makers, allowing a better understanding and more effective response to local demands. The study also delves into the phenomenology of interactions in Colab and highlights how users' experiences on the platform shape the perception of urban reality and contribute to a broader understanding of social dynamics, raising questions about the relationship between the digital and the physical in urban experience.}
{Echo Chambers; Polarization; Social Network Analysis; Graph Theory}


% ----------------------------------------------------------
% ELEMENTOS PRÉ-TEXTUAIS
% ----------------------------------------------------------

% Inserir a ficha catalográfica
%\incluifichacatalografica{tex/pre-textual/ficha-catalografica.pdf}

% DEDICATÓRIA / AGRADECIMENTO / EPÍGRAFE
\textodedicatoria*{tex/pre-textual/dedicatoria}
\textoagradecimentos*{tex/pre-textual/agradecimentos}
\textoepigrafe*{tex/pre-textual/epigrafe}

% Inclui a lista de figuras
\incluilistadefiguras

% Inclui a lista de tabelas
\incluilistadetabelas

% Inclui a lista de quadros
\incluilistadequadros

% Inclui a lista de algoritmos
%\incluilistadealgoritmos

% Inclui a lista de códigos
\incluilistadecodigos

% Inclui a lista de siglas e abreviaturas
\incluilistadesiglas

% Inclui a lista de símbolos
%\incluilistadesimbolos

% ----
% Início do documento
% ----
\begin{document}
% ----------------------------------------------------------
% ELEMENTOS TEXTUAIS
% ----------------------------------------------------------
\textual

\chapter{Introdução}
\label{chapter:01_introducao}
A interação entre o ser humano e sua realidade tecnológica tem sido um tema de reflexão constante. Como argumentado por Martin Heidegger em sua obra "O Ser e o Tempo", a qualidade da nossa experiência está intrinsecamente ligada à forma como manipulamos nosso ambiente, incluindo a tecnologia. Suas observações fenomenológicas revelam que a interação perfeita com uma ferramenta pode alterar nossa concepção existente, levando-nos a gostar e desfrutar de algo porque o reconhecemos e dominamos seu uso. Essa perspectiva encontra eco em pesquisas neurofisiológicas, psicológicas e neuropsicológicas recentes, que indicam que o uso de ferramentas para alcançar objetivos distantes desencadeia mudanças nas redes neurais responsáveis por manter um mapa atualizado da forma e postura do corpo, conhecido como "esquema corporal" na neurologia clássica.

Além disso, uma análise cuidadosa da relação entre as ferramentas e o corpo humano é apresentada no artigo "Tools for the Body" \cite[1]{2004_Maravita}. Esse estudo explora como as ferramentas tecnológicas podem se tornar extensões do corpo, ampliando nossas habilidades e transformando nossa experiência cotidiana. O autor destaca que a interação eficiente com as ferramentas digitais é essencial para o sucesso da manipulação do ambiente tecnológico, e que essa interação pode resultar em uma mudança significativa em nossa percepção e apreciação de marcas e serviços.

No entanto, a proliferação das plataformas de mídia social tem apresentado desafios para essa interação perfeita entre o ser humano e a tecnologia. Embora essas plataformas tenham o potencial de aumentar o engajamento político e fomentar o diálogo, também são palco de um fenômeno preocupante: as câmaras de eco. Nas câmaras de eco, os usuários são expostos apenas a informações que confirmam suas crenças e tendências pré-existentes, contribuindo para a polarização das opiniões políticas. Essa fragmentação das fontes de informação, descrita por \citeonline{2001_Sunstein_BOOK}, e o fenômeno das bolhas de filtro, estudado por \citeonline{2011_Pariser_BOOK}, compartilham características semelhantes e agravam o problema.

As bolhas de filtro referem-se ao processo pelo qual os algoritmos das plataformas de mídia social personalizam o conteúdo exibido para os usuários com base em seu comportamento passado. Esses algoritmos analisam o histórico de navegação, curtidas, compartilhamentos e interações do usuário para determinar quais conteúdos são mais propensos a serem do seu interesse. Como resultado, os usuários são apresentados principalmente a informações que se alinham com suas preferências e visões de mundo, enquanto conteúdos divergentes ou contraditórios são filtrados ou recebem menos destaque.

Esse processo de filtragem cria uma "bolha" em torno de cada usuário, onde seu ambiente de informações se torna cada vez mais estreito e alinhado com suas perspectivas existentes. À medida que os usuários são expostos principalmente a opiniões semelhantes às suas, há uma redução significativa da diversidade de pontos de vista e uma limitação do acesso a informações que possam desafiar suas crenças. Como resultado, as bolhas de filtro podem contribuir para a ampliação da polarização política, uma vez que os usuários têm menos oportunidades de serem expostos a perspectivas diferentes e debater ideias contraditórias.

Essa personalização algorítmica pode levar os usuários a acreditar erroneamente que sua visão de mundo é a única válida ou amplamente aceita, exacerbando ainda mais a polarização e a formação de câmaras de eco. A combinação das câmaras de eco e das bolhas de filtro nas plataformas de mídia social torna cada vez mais difícil para os usuários se engajarem em discussões saudáveis, acessar informações imparciais e considerar diferentes pontos de vista.

É importante compreender e abordar o fenômeno das bolhas de filtro juntamente com as câmaras de eco, pois ambos têm implicações significativas para o discurso político e a democracia. A superação desses desafios requer esforços para aumentar a diversidade de perspectivas, garantir o acesso a informações diversas e promover a interação entre usuários com opiniões diferentes.

O impacto das câmaras de eco no discurso político é significativo. Estudos indicam que usuários expostos a uma maior diversidade de pontos de vista políticos têm maior probabilidade de se envolver em discussões políticas, enquanto aqueles expostos apenas a conteúdos similares aos seus têm menos probabilidade de engajar-se com notícias políticas. Essas câmaras de eco são particularmente acentuadas no Brasil, onde a polarização política atinge níveis alarmantes, como revelado por um estudo do Ibope Inteligência (2018).

Estudos acadêmicos sobre câmaras de eco frequentemente se concentraram na análise de redes sociais públicas, como Twitter, Facebook e Reddit. Diversas pesquisas têm investigado as dinâmicas das câmaras de eco nesses ambientes online. Por exemplo, estudos como o de \cite[p. 224]{2016_Vicario} examinaram o surgimento e a propagação de desinformação e polarização nas redes sociais, destacando a importância de entender esses fenômenos para a sociedade como um todo.

O Colab.re é uma plataforma brasileira de mídia social que visa promover a democracia participativa, permitindo que os usuários apresentem ideias e propostas para melhorar suas comunidades. Através do Colab.re, os cidadãos podem compartilhar problemas locais, sugerir soluções, colaborar com outros membros da comunidade e interagir com representantes governamentais. A plataforma tem sido elogiada por sua abordagem inovadora para o envolvimento cívico, incentivando uma participação mais ativa dos cidadãos na tomada de decisões e no aprimoramento de suas localidades. No entanto, assim como outras plataformas de mídia social, o Colab.re também enfrenta o desafio das câmaras de eco, onde os usuários tendem a seguir e interagir principalmente com aqueles que compartilham suas opiniões políticas, resultando em uma redução de perspectivas e uma possível erosão dos valores democráticos.

Compreender as câmaras de eco no contexto do aplicativo Colab.re apresenta uma oportunidade valiosa de pesquisa considerando que muitas plataformas de mídia social não disponibilizam seus dados para acesso público, tornando desafiador o estudo das câmaras de eco em tais ambientes.

Além de uma rede social, o Colab.re também oferece serviços de eGov. Empresas de eGov, ou governo eletrônico, são organizações que fornecem soluções digitais para apoiar o funcionamento e os serviços do governo. Elas se concentram em utilizar a tecnologia da informação e comunicação (TIC) para melhorar a eficiência, transparência e interação entre o governo e os cidadãos. Essas empresas oferecem uma variedade de serviços, como desenvolvimento de portais governamentais, plataformas de participação cidadã, sistemas de gestão de documentos, soluções de segurança digital, entre outros.

Ao combinar os insights da academia com os dados e experiência do Colab.re, é possível obter uma compreensão mais profunda das câmaras de eco em um contexto específico de eGov. Isso não apenas aprimora nossa compreensão desses fenômenos sociais complexos, mas também fornece insights valiosos para o governo e aprimora as estratégias de engajamento cidadão nas cidades atendidas pelo Colab.re. Ao compreender melhor como essas câmaras influenciam as interações políticas e o diálogo cívico, o governo pode tomar medidas mais eficazes para combater a polarização, promover a diversidade de perspectivas e fortalecer a participação democrática. A pesquisa pode fornecer insights práticos que ajudarão o governo a tomar decisões informadas e a desenvolver estratégias eficientes para criar um ambiente político mais saudável e envolvente.

Diante desse cenário, este artigo examina as câmaras de eco presentes na plataforma de mídia social brasileira Colab.re. e propõe estratégias para mitigar seus efeitos negativos. Para isso, serão realizadas análises de redes na plataforma, identificando as câmaras de eco e seus nós principais. Além disso, serão desenvolvidos algoritmos baseados em epidemiologia digital, como os modelos SIR e SEIR, para prever a propagação de informações dentro e entre essas câmaras de eco. Por fim, serão exploradas intervenções potenciais, como sugestões e campanhas de informação, com o objetivo de incentivar os usuários a se envolverem com conteúdos fora de suas câmaras de eco.

Em resumo, este artigo busca oferecer uma análise aprofundada do fenômeno das câmaras de eco na plataforma Colab.re, propondo abordagens inovadoras para detectar e mitigar seus efeitos negativos. Ao combinar insights da epidemiologia digital, neurologia clássica e filosofia de Heidegger, espera-se contribuir para a promoção de uma maior diversidade de perspectivas e o fortalecimento dos valores democráticos no contexto das plataformas de mídia social.

\chapter{Metodologia}
\label{chapter:02_metodologia}
\input{tex/02_metodologia}

\chapter{Fundamentação teórica}
\label{chapter:03_networkanalysis}
\input{tex/03_networkanalysis}

\chapter{Uma Visão Geral do Colab}
\label{chapter:04_colab}
A participação cidadã e a colaboração em governos eletrônicos (e-Gov) têm se tornado cada vez mais relevantes na sociedade contemporânea. Com o avanço da tecnologia e a disseminação das redes sociais, surgiram novas formas de engajamento e interação entre cidadãos e governos. Nesse contexto, o aplicativo Colab.re desponta como uma plataforma inovadora que combina elementos de redes sociais com a participação cidadã em questões relacionadas à gestão pública. Neste capítulo, exploraremos o Colab.re sob a perspectiva das redes sociais e e-Gov, analisando suas funcionalidades, impactos sociais, desafios e limitações.
\begin{citacao}
    "uma forma de inteligência coletiva em que as pessoas se reúnem para monitorar, analisar e agir coletivamente em relação a um problema ou questão compartilhada" \cite[p. 1]{2011_Bryer}.
\end{citacao}

\section*{História e Desenvolvimento}
O Colab.re foi lançado em [ano de lançamento] como uma iniciativa pioneira no campo da participação cidadã digital. A plataforma foi desenvolvida com o objetivo de promover a interação entre cidadãos e governos, permitindo que os usuários compartilhem ideias, façam sugestões, denunciem problemas e participem ativamente na construção de políticas públicas. Desde então, o Colab.re tem conquistado espaço em diversas cidades, tornando-se uma ferramenta de referência no campo da democracia digital.

\section{Funcionalidades}

O Colab.re oferece uma variedade de funcionalidades que estimulam a participação cidadã e promovem a interação entre os usuários. Entre as principais funcionalidades do aplicativo, destacam-se:

\begin{itemize}
  \item \textbf{Publicação de ideias e sugestões:} Os usuários podem compartilhar suas ideias e sugestões sobre questões de interesse público. Essas publicações podem abranger diversos temas, desde melhorias na infraestrutura urbana até propostas de políticas sociais.

  \item \textbf{Denúncia de problemas:} O Colab.re permite que os cidadãos denunciem problemas, como buracos nas vias, iluminação pública deficiente, entre outros. Essas denúncias são georreferenciadas, o que facilita a identificação e resolução dos problemas pelas autoridades competentes.

  \item \textbf{Interatividade social:} O aplicativo permite que os usuários sigam outros usuários, sejam seguidos, curtam e comentem as publicações. Essa interatividade promove a formação de uma rede social dentro do Colab.re, ampliando as possibilidades de engajamento e diálogo entre os participantes.

  \item \textbf{Acompanhamento de demandas:} Os usuários podem acompanhar o andamento das demandas e propostas que foram apresentadas. Isso permite que eles estejam cientes das ações tomadas pelo governo em resposta às suas contribuições.
\end{itemize}

Através dessas funcionalidades, o Colab.re busca fortalecer a participação cidadã e criar um ambiente propício para o diálogo entre cidadãos e governos. Além disso, o aplicativo também oferece uma série de benefícios para os usuários, como a possibilidade de influenciar diretamente as políticas públicas e a oportunidade de se conectar com outros cidadãos que compartilham os mesmos interesses.

\section*{Plataformas de Participação cidadã e e-Gov}
O Colab.re se destaca no cenário das plataformas de participação cidadã devido às suas características específicas. Ao compará-lo com outras plataformas similares, é possível observar algumas diferenças significativas. Por exemplo, enquanto algumas plataformas focam principalmente na coleta de dados e informações dos usuários, o Colab.re se concentra em incentivar a participação ativa e o diálogo entre os cidadãos e os governos. Além disso, o Colab.re se destaca pela sua interface amigável e pela capacidade de georreferenciamento das denúncias, o que permite uma maior eficácia na resolução dos problemas reportados.

\section*{Impacto do Colab.re em cidades}
O Colab.re tem demonstrado um impacto significativo nas cidades em que está presente. Por exemplo, em um estudo realizado na cidade de Paragominas-PA, o aplicativo foi utilizado como um dispositivo de participação social na gestão urbana, permitindo que os cidadãos reportassem problemas e propusessem soluções [1]. O estudo constatou que o Colab.re contribuiu para uma maior transparência e eficácia na resolução dos problemas urbanos, além de fortalecer o engajamento cívico na cidade.

Além de Paragominas, o Colab.re também tem sido adotado em outras cidades brasileiras, como São Paulo, Rio de Janeiro e Fortaleza. Através da plataforma, os cidadãos têm a oportunidade de participar ativamente na gestão pública, compartilhando suas opiniões, denunciando problemas e colaborando na construção de uma cidade mais inclusiva e sustentável.

\section*{Estudos de caso do Colab.re na resolução de problemas urbanos}
O Colab.re tem sido utilizado em diversos casos para resolver problemas específicos e melhorar a qualidade de vida nas cidades. Por exemplo, em um estudo de visualização de dados em aplicativos móveis, o Colab.re foi investigado como um exemplo de aplicativo de cidade inteligente [2]. O estudo mostrou que o aplicativo oferece recursos de visualização de dados que permitem aos usuários entender melhor os problemas urbanos e contribuir de forma mais efetiva para a sua resolução.

Outro estudo relatou o caso de um aplicativo que utilizou o Colab.re como plataforma para denúncias de problemas relacionados ao sistema de informações geográficas (GIS) e colaboração coletiva [3]. O estudo destacou a eficácia do Colab.re na coleta de dados georreferenciados e no engajamento dos cidadãos na solução de problemas urbanos.

Esses estudos de caso demonstram como o Colab.re tem sido usado de forma efetiva para resolver problemas específicos e melhorar a qualidade de vida nas cidades.

\section{Dados}

Com base nos dados disponibilizados pelo Colab para esse estudo, analisamos um total de 328.876 eventos criados entre 04/01/2013 e 05/12/2022. Para essa análise foi utilizado a linguagem Python e a biblioteca Pandas. O código fonte está disponível no GitHub.

\subsection*{Modelo de dados}

\begin{table}[ht]
    \centering
    \caption{Modelo de Dados da lista de usuários}
    \label{tab:user_model}
    \begin{tabular}{|l|p{6cm}|}
    \hline
    \textbf{Campo} & \textbf{Descrição} \\
    \hline
    colab\_user\_id & Identificador único do usuário no sistema Colab \\
    gender & Gênero do usuário \\
    birth\_date & Data de nascimento do usuário \\
    city\_id & Identificador único da cidade do usuário \\
    city\_name & Nome da cidade do usuário \\
    state\_id & Identificador único do estado do usuário \\
    state\_name & Nome do estado do usuário \\
    created\_at & Data de criação do registro do usuário \\
    last\_sign\_in\_at & Data da última vez que o usuário fez login \\
    device & Dispositivo utilizado pelo usuário (por exemplo, desktop, mobile) \\
    \hline
    \end{tabular}
\end{table}

\begin{table}[ht]
    \centering
    \caption{Modelo de Dados de eventos reportados}
    \label{tab:event_model}
    \begin{tabular}{|l|p{6cm}|}
    \hline
    \textbf{Campo} & \textbf{Descrição} \\
    \hline
    event\_id & Identificador único do evento \\
    user\_id & Identificador único do usuário relacionado ao evento \\
    description & Descrição do evento \\
    status & Status do evento \\
    created\_at & Data de criação do evento \\
    event\_type\_id & Identificador único do tipo de evento \\
    event\_type\_name & Nome do tipo de evento \\
    \hline
    \end{tabular}
\end{table}

Os dados dos foram disponibilizados em formato CSV, contendo informações sobre os usuários e os eventos reportados. A tabela \autoref{tab:user_model} apresenta o modelo de dados da lista de usuários. A tabela \autoref{tab:event_model} apresenta o modelo de dados de eventos reportados.

\subsection*{Distribuição demográfica}

O quadro \autoref{quadro:usersbygender} apresenta a distribuição demográfica dos usuários por gênero. A maioria dos usuários se declararam do gênero másculino. 


\begin{quadro}[htb]
    \caption{Usuários por gênero}
    \label{quadro:usersbygender}
    \centering
    \pie[
            % Opções de aparência do gráfico
            radius=2.5, % Raio do gráfico de pizza
            text=pin, % Posição do texto no gráfico
            rotate=90 % Rotação do gráfico
        ]{
            30494/Masculino,
            19555/Feminino,
            16/Não Binário,
            335/Desconhecido,
            274/Outro,
            92/Não Informado
        }
\end{quadro}

\subsection{Tipos de Eventos}

\begin{table}[h]
    \centering
    \caption{Tipos de eventos com mais ocorrências}
    \label{tab:tiposevento}
    \begin{tabular}{|l|l|l|}
    \hline
    \textbf{Tipo de Evento} & \textbf{Total de Ocorrências} \\
    \hline
    Entulho na calçada/via pública            & 61.785 \\
    Buraco nas vias                           & 41.200 \\
    Lâmpada apagada à noite                   & 32.907 \\
    Ponto de infração de trânsito recorrente  & 15.873 \\
    Calçada irregular                         & 14.837 \\
    Mato alto                                 & 13.459 \\
    Poda de árvore                            & 12.810 \\
    Descarte irregular de lixo                & 12.685 \\
    Bueiro entupido                           & 8.825  \\
    Vazamento de água                         & 7.433  \\
    Bueiro sem tampa                          & 5.844  \\
    Ocupação irregular de área pública        & 5.714  \\
    Fiação irregular                          & 5.643  \\
    Veículo abandonado                        & 5.335  \\
    Equipamento público danificado            & 4.694  \\
    Esgoto a céu aberto                       & 4.656  \\
    Retirada de árvore                        & 4.437  \\
    Ponto recorrente de poluição sonora       & 4.189  \\
    Bloqueio na via                           & 4.066  \\
    Iluminação pública irregular              & 3.702  \\
    \hline
    \end{tabular}
\end{table}

A tabela \autoref{tab:tiposevento} apresenta os 20 tipos de evento mais reportados pelos usuários. A análise dos dados fornecidos pelos usuários do Colab proporcionou insights valiosos sobre as preocupações e demandas da comunidade. Os eventos mais frequentemente relatados estão intrinsecamente ligados a problemas e irregularidades na infraestrutura urbana, como entulho na calçada/via pública, buraco nas vias, lâmpada apagada à noite, ponto de infração de trânsito recorrente e calçada irregular. Essas ocorrências destacam a importância de investimentos contínuos na manutenção e melhoria da infraestrutura da cidade. Além disso, questões ambientais emergem como uma área de preocupação significativa, com denúncias frequentes de descarte irregular de lixo, desmatamento ilegal, esgoto a céu aberto e mato alto. Tais dados indicam uma conscientização dos usuários em relação à preservação ambiental e ressaltam a necessidade de ações efetivas para aprimorar a gestão dos recursos naturais. O transporte público também é alvo de atenção, com reclamações recorrentes sobre problemas em ônibus, atrasos e superlotação. Esses aspectos exigem uma análise aprofundada das questões relacionadas à mobilidade urbana e podem impulsionar esforços para melhorar a qualidade e eficiência do transporte coletivo. Além disso, eventos relacionados à segurança e vigilância, como pontos de exploração sexual de menores e maus-tratos a animais, refletem a preocupação dos usuários com a proteção e bem-estar da comunidade. Por fim, a ocorrência de eventos envolvendo estabelecimentos comerciais, como falta de alvará e condições sanitárias irregulares, destaca a importância de ações rigorosas de fiscalização e de garantir a conformidade legal por parte dos estabelecimentos. Esses insights fornecem uma visão abrangente das preocupações dos usuários do Colab e podem orientar as prefeituras da cidade na implementação de políticas públicas que visem atender às demandas da comunidade e aprimorar a qualidade de vida em geral.

Os usuários estão engajados e ativos na identificação e denúncia de problemas na infraestrutura urbana, meio ambiente, transporte público e questões sociais. Isso indica uma participação cidadã ativa e um desejo de melhorar as condições de suas comunidades.
Os clientes podem aproveitar essas informações para acompanhar as preocupações e demandas da comunidade, tomar medidas corretivas mais efetivas e aprimorar a qualidade dos serviços e infraestrutura oferecidos.

Os dados fornecem uma visão clara das principais questões enfrentadas pela comunidade, permitindo que as prefeituras priorizem recursos e esforços em áreas críticas, como manutenção da infraestrutura, gestão ambiental, transporte público e segurança. As prefeituras podem usar esses insights para desenvolver políticas públicas mais eficazes, implementar medidas preventivas e corretivas, bem como estabelecer canais de comunicação e interação mais robustos com os cidadãos, fortalecendo a confiança e a participação da comunidade nas decisões governamentais.

\section*{O papel do Colab.re na transparência governamental}
A transparência governamental é um elemento essencial para a democracia e o bom funcionamento das instituições públicas. O Colab.re desempenha um papel importante na promoção da transparência, uma vez que todas as denúncias, propostas e ações tomadas pelo governo são visíveis para os usuários da plataforma. Isso permite que os cidadãos acompanhem o andamento das demandas, fiscalizem as ações do governo e tenham acesso às informações sobre a gestão pública. Essa transparência fortalece a confiança e a accountability entre os governantes e os governados.

\section*{Contribuições do Colab.re para a democracia digital}
A democracia digital refere-se à utilização da tecnologia para fortalecer a participação e o engajamento cívico. Nesse contexto, o Colab.re desempenha um papel significativo ao oferecer uma plataforma acessível e interativa para os cidadãos se envolverem na gestão pública. Através do aplicativo, os cidadãos têm a oportunidade de expressar suas opiniões, contribuir com propostas e denunciar problemas, promovendo assim uma participação mais ampla e inclusiva na tomada de decisões.

\section*{Desafios dos usuários do Colab.re}
Apesar dos benefícios e do potencial do Colab.re, há desafios que os usuários podem enfrentar ao utilizar a plataforma. Por exemplo, a falta de conhecimento ou acesso à tecnologia pode limitar a participação de certos grupos de cidadãos. Além disso, a confiança no governo e a percepção de que as contribuições dos cidadãos são levadas em consideração são fatores-chave para incentivar o engajamento contínuo dos usuários.

\section*{Como o Colab.re pode ser melhorado no futuro}
Para garantir a eficácia contínua do Colab.re e maximizar seu potencial, é importante considerar melhorias e atualizações futuras. Algumas sugestões incluem aprimorar a interface do usuário para torná-la mais intuitiva e amigável, expandir a divulgação e o treinamento para aumentar a conscientização e o acesso ao aplicativo, e estabelecer parcerias com outras instituições e organizações para ampliar o impacto e a cobertura do Colab.re.

\section*{O Colab.re e a participação cidadã durante a pandemia de COVID-19}
A pandemia de COVID-19 trouxe desafios sem precedentes para a participação cidadã e o engajamento social. Nesse contexto, o Colab.re desempenhou um papel importante ao permitir que os cidadãos se envolvessem virtualmente na gestão pública, mesmo durante o distanciamento social. Através do aplicativo, os cidadãos puderam denunciar problemas relacionados à pandemia, compartilhar informações relevantes e contribuir com propostas para enfrentar os desafios impostos pela crise sanitária.

\section*{Conclusão}
Em conclusão, o Colab.re representa uma importante ferramenta no contexto das redes sociais e e-Gov. Com suas funcionalidades inovadoras, o aplicativo tem promovido a participação cidadã, fortalecido a transparência governamental e contribuído para a democracia digital. Apesar dos desafios e limitações, o Colab.re demonstra um impacto significativo nas cidades onde é adotado, melhorando a qualidade de vida dos cidadãos e fortalecendo a governança local. À medida que avançamos em direção a um futuro cada vez mais digital, é essencial continuar explorando o potencial do Colab.re e outras plataformas semelhantes para promover uma participação mais inclusiva e engajada dos cidadãos na gestão pública.

\chapter{Análise exploratória da Rede do Colab}
\label{chapter:05_exploratory}
\input{tex/05_exploratory}

\chapter{Análise de sentimento das postagens do Colab}
\label{chapter:06_sentiment}
Neste capítulo, abordamos a análise de sentimento como uma ferramenta poderosa no contexto do processamento de linguagem natural, focando especialmente nas postagens do Colab. Exploramos como técnicas de aprendizado de máquina, particularmente com o uso do Natural Language Toolkit (NLTK), podem ser empregadas para automatizar e aprimorar a análise de sentimentos. Além disso, discutimos a relevância de identificar diferentes "personas" de usuários, como os \textit{helpers} e \textit{complainers}, e como essa distinção pode influenciar a dinâmica e a polarização dentro de uma plataforma de mídia social. Finalmente, abordamos a relação entre análise de sentimento, polarização e a formação de câmaras de eco, destacando a importância de entender e mitigar esses fenômenos em ambientes digitais.

Talvez seja interessante contextualizar a análise exploratória de sentimento e estabelecer uma conexão com o objeto desse estudo, a detecção de câmaras de eco. Tradicionalmente, a metodologia de \citeonline{2023_Atiqi_BOOK} cria simulações de redes sociais onde usuários respondem a estímulos de notícias da mídia através de curtidas, comentários mas também seguindo usuários que fazem postagens similares. Portanto, o conteúdo gerado e compartilhado pelos usuários é um fator determinante na detecção de câmaras de eco. No contexto do Colab, os usuários não tendem a postar links de notícias, mas a criar eventos de zeladoria pública, indicando problemas na cidade. Portanto, a análise de sentimento das postagens é uma ferramenta valiosa para entender a dinâmica da rede e identificar possíveis câmaras de eco.

Para realizar essa análise, fundamentamos nosso estudo em conceitos e técnicas de processamento de linguagem natural (PLN) e análise de sentimentwos. O PLN é uma área da inteligência artificial que visa capacitar os computadores a entender, interpretar e gerar linguagem humana de forma natural. A análise de sentimentos, por sua vez, é uma subárea do PLN que se concentra em identificar e extrair informações sobre os sentimentos e opiniões expressos em textos \cite []{2009_Bird_BOOK}.

Utilizamos a linguagem de programação Python como base para a implementação do experimento. Como runtime, novamente utilizamos o Google Colab, que nos permite executar o código em nuvem, sem a necessidade de instalar bibliotecas ou configurar o ambiente de desenvolvimento. O código-fonte do experimento está disponível no GitHub\footnote{https://github.com/guinetik/colab-network-ec}.

Além disso, apresentamos uma análise exploratória dos dados para identificar os principais tópicos discutidos no Colab e as principais palavras associadas a cada tópico. Adicionalmente, integramos uma camada de análise de redes ao experimento, utilizando a assortatividade como métrica para medir a homofilia na rede.

Com a aplicação dessas técnicas e o uso da linguagem Python, pudemos extrair insights valiosos sobre os sentimentos presentes nas postagens do Colab. Essas métricas servirão como base para as análises subsequentes e contribuirão para aprimorar a compreensão do ecossistema do Colab.

\section{Heurísticas para análise de sentimentos}

Para realizar uma análise de sentimento das postagens do Colab, adotamos uma abordagem híbrida que combina métodos léxicos tradicionais com técnicas de aprendizado de máquina. A seguir, descrevemos detalhadamente o processo metodológico empregado para calcular o score final de sentimento de cada postagem. Nossa proposta é realizar uma análise de sentimento dos eventos criados pelos usuários, introduzindo o conceito de "score". Esta métrica foi desenvolvida especificamente para classificar as postagens como positivas ou negativas, proporcionando uma visão quantitativa dos sentimentos expressos pelos usuários. 

A metodologia utilizada se baseia na abordagem de \citeonline{2008_Pang}, que sugere capturar a estrutura do discurso em documentos modelando o sentimento global como uma trajetória de sentimentos locais. Especificamente, cada sentença em um documento recebe uma pontuação de sentimento local, que é então suavizada através de convolução com um kernel de suavização. As distâncias entre essas trajetórias de sentimentos permitem refletir as distâncias entre os sentimentos globais dos documentos. Adaptamos essa técnica para classificar os eventos no Colab, criando fundamentos para uma análise mais detalhada e precisa dos sentimentos expressos nas postagens dos usuários. 

Cada postagem recebe um score de sentimento que varia de -1 (negativo) a 1 (positivo), com scores próximos a 0 indicando neutralidade. Esta abordagem quantitativa não só nos permite capturar o sentimento geral expresso nas postagens, mas também entender como esses sentimentos se propagam e interagem dentro da rede.

O processo de análise de sentimento das postagens começa com o carregamento dos dados. Nessa análise, estamos interessados em postagens criadas por usuários das comunidades identificadas na análise do capítulo anterior, nas cidades de Niterói, Santo André e Mesquita. Em seguida, realizamos o pré-processamento do texto, que segue a metodologia apresentada por \citeonline{2009_Bird_BOOK}, como a remoção de caracteres especiais, conversão para letras minúsculas, tokenização e aplicação de técnicas de stemização para reduzir as palavras às suas formas básicas.

Após o pré-processamento cada token do documento pode ser classificado como positivo, negativo ou neutro. Para isso, utilizamos um dicionário léxico que mapeia palavras para scores de sentimento. Os dicionários utilizados são discutidos no \autoref{sec:dicionarios_lexicos}. Utilizando esses dicionários, calculamos o score de sentimento de cada token e a postagem também é analisada pela ferramenta LeIA como discutido no \autoref{sec:ferramenta_leia}. O score composto final é calculado baseado nos scores de sentimento de cada token segundo os dicionários léxicos e o score da postagem como um todo segundo a ferramenta LeIA. O score final de sentimento de cada postagem foi obtido através de uma média ponderada dos scores normalizados dos dicionários léxicos e dos scores compostos da LeIA.

\section{Análise exploratória de Sentimento}

Após a atribuição de scores a cada postagem do dataset, realizamos uma análise exploratória dos dados para entender melhor as características das postagens. Utilizamos a técnica de "bag-of-words" descrita em \citeonline{2013_Mikolov} com a biblioteca CountVectorizer para converter o texto em uma matriz numérica, onde cada coluna representa uma palavra e cada linha representa uma postagem. Com base nessa matriz, identificamos as palavras mais frequentes e as visualizamos em um gráfico de barras e na forma de uma nuvem de palavras. Além disso, utilizamos a biblioteca Word2Vec para criar um modelo de palavras em vetores, que foi usado para visualizar as associações de palavras mais comuns.

\begin{quadro}[!htb]
	\caption{Distribuição de palavras mais frequentes}
	\label{fig:wordcount}
	\centering
	\includegraphics[width=\textwidth]{images/wordcount.png}
	\fautor
\end{quadro}

\begin{figure}[!htb]
	\caption{Núvem de palavras mais utilizadas}
	\label{fig:wordcloud}
	\centering
	\includegraphics[scale=0.5]{images/wordcloud.png}
	\fautor
\end{figure}

\begin{figure}[htb]
	\centering
	\caption{Comparação de núvem de palavras mais usadas}\label{fig:lexicon_tagcloud}
	\begin{subfigure}[b]{0.317\textwidth}
		\includegraphics[width=\textwidth]{images/lexicon_worst_scores_tagcloud.png}
		\caption{Scores Negativos}
		\label{fig:tigre}
	\end{subfigure} ~ %add desired spacing between images, e. g. ~, \quad, \qquad, \hfill etc. %(or a blank line to force the subfigure onto a new line) 
	\begin{subfigure}[b]{0.317\textwidth}
		\includegraphics[width=\textwidth]{images/lexicon_best_scores_tagcloud.png}
		\caption{Scores Positivos} \label{fig:leao}
	\end{subfigure} ~ %add desired spacing between images, e. g. ~, \quad, \qquad, \hfill etc. %(or a blank line to force the subfigure onto a new line)
	\fautor
\end{figure}

\subsection{Análise das postagens em Niterói, Santo André e Mesquita}

\begin{quadro}[!htb]
	\caption{Distribuição dos scores de sentimento nas postagens da rede das 3 cidades selecionadas. Cada ponto representa um usuário e a cor indica o score médio de sentimento de suas postagens (verde para positivo, vermelho para negativo e laranja para neutro)}
	\label{fig:scores_scatterplot}
	\centering
	\includegraphics[scale=0.70]{images/scores_scatterplot.png}
	\fautor
\end{quadro}

Após a seleção das postagens dos usuários das comunidades das cidades de Niterói, Santo André e Mesquita, conforme detalhado no capítulo anterior, foi possível identificar um total de 132.846 eventos de zeladoria criados por membros dessas comunidades. Esta vasta quantidade de dados nos ofereceu uma oportunidade única para aprofundar nossa análise de sentimentos e entender melhor as emoções e opiniões expressas pelos usuários.

Ao aplicar a análise de sentimentos nas postagens, foi essencial considerar a extensão dos textos. Uma postagem mais longa tem naturalmente mais palavras e, consequentemente, uma maior soma de polaridades. Para contornar essa característica e garantir uma análise justa, optamos por normalizar os scores com base no número de tokens ou palavras presentes na frase original. Esta abordagem permitiu que cada palavra contribuísse proporcionalmente para o score final da postagem, independentemente do seu tamanho.

Ao examinar os melhores scores, notamos algumas tendências interessantes. Primeiramente, as postagens com os scores mais elevados tendem a ter uma combinação de sentimentos neutros e positivos, conforme indicado pelas métricas do LeIA. Por exemplo, a postagem do evento 100876, originada de Niterói, apresenta uma combinação de 75,1\% de conteúdo neutro e 16,8\% de conteúdo positivo, resultando em um score composto de 0,671. Isso sugere que, mesmo quando os usuários estão apresentando informações factuais ou descritivas, há uma inclinação positiva em suas expressões.

Além disso, é notável que, mesmo com variações nos scores derivados dos dicionários léxicos, o LeIA consistentemente percebeu essas postagens como altamente positivas. Isso pode ser atribuído à capacidade do LeIA de capturar nuances e contextos específicos da língua portuguesa, como a influência de emojis e a presença de negações.

Outro ponto de destaque é a variedade de temas abordados nas postagens com os melhores scores. Enquanto algumas postagens focam em questões de zeladoria como o acúmulo de lixo, outras discutem colaborações entre organizações privadas e públicas. Isso reforça a ideia de que a plataforma Colab é um espaço diversificado, onde os usuários se sentem empoderados para discutir uma ampla gama de tópicos relacionados à melhoria de suas comunidades.

Em resumo, a análise dos melhores scores nos proporcionou uma visão mais clara das emoções e opiniões dos usuários nas comunidades selecionadas. Estes insights são fundamentais para entender as motivações dos usuários ao interagir na plataforma e podem ser usados para orientar futuras estratégias de engajamento e moderação.

\begin{quadro}[!htb]
	\caption{Score médio de sentimento por número de usuários}
	\label{fig:average_score_by_number_of_users}
	\centering
	\includegraphics[scale=0.70]{images/average_score_by_number_of_users.png}
	\fautor
\end{quadro}

Ao analisar os piores scores, é evidente que as postagens refletem um alto grau de insatisfação e frustração dos usuários em relação a questões específicas de zeladoria em suas comunidades. Estas postagens, oriundas das cidades de Niterói e Rio de Janeiro, destacam-se não apenas pelo conteúdo negativo, mas também pela intensidade das emoções expressas.

A postagem do evento 218258, por exemplo, menciona a repetição de reclamações feitas pelo usuário sem a devida solução, indicando um sentimento de desamparo e descontentamento com a resposta (ou falta dela) das autoridades competentes. O score derivado do dicionário léxico para esta postagem foi de -2.857, enquanto o LeIA identificou uma predominância de conteúdo neutro (73,5\%), mas com uma porcentagem significativa de conteúdo negativo (19,9\%). O score composto, que combina ambas as métricas, resultou em -0.957, refletindo a natureza altamente negativa da postagem.

Da mesma forma, a postagem do evento 156161 expressa indignação com a situação de uma rua específica, usando palavras em caixa alta para enfatizar o descontentamento. O uso de termos como "ABSURDO" e "CAOS" sugere uma forte emoção negativa. O LeIA capturou essa nuance, atribuindo um score negativo de -0.9930, enquanto o score do dicionário léxico foi de -0.167. A combinação de ambos resultou em um score composto de -0.956.

É interessante notar que, mesmo nas postagens com os piores scores, ainda há uma presença de conteúdo neutro e, em alguns casos, até mesmo positivo. Por exemplo, a postagem do evento 239454, apesar de expressar frustração com tentativas repetidas de comunicação, ainda contém uma saudação cordial ("Olá prezados"). Isso sugere que, mesmo em meio à insatisfação, os usuários ainda buscam manter um tom respeitoso e construtivo em suas comunicações.

Em resumo, as postagens com os piores scores ilustram claramente os desafios e frustrações enfrentados pelos usuários em suas comunidades. Estas postagens são valiosas, pois destacam áreas que requerem atenção imediata e melhorias por parte das autoridades locais. Além disso, a análise de sentimentos fornece uma ferramenta poderosa para identificar e compreender essas preocupações, permitindo uma resposta mais eficaz e empática por parte dos tomadores de decisão.

\begin{quadro}[!htb]
	\caption{Distribuição de quantidade de postagens por score}
	\label{fig:score_distribution}
	\centering
	\includegraphics[scale=0.90]{images/score_distribution.png}
	\fautor
\end{quadro}

Após obter o score de sentimento de cada postagem, o próximo passo foi agregar as postagens de cada usuário. Ao calcular a média ponderada do score de sentimento pelo número de postagens, conseguimos criar um score geral que reflete o sentimento médio de todas as postagens de um usuário. Esta métrica agregada nos forneceu uma visão mais clara do panorama geral dos sentimentos expressos nas redes de Niterói, Santo André e Mesquita.

\begin{figure}[htbp]
	\centering
	\begin{subfigure}{0.45\textwidth}
		\includegraphics[width=\linewidth]{images/average_leia_by_number_of_users.png}
		\caption{Score Composto LeIA}
		\label{fig:average_leia_by_number_of_users}
	\end{subfigure}
	\hfill
	\begin{subfigure}{0.45\textwidth}
		\includegraphics[width=\linewidth]{images/average_neutral_by_number_of_users.png}
		\caption{Neutros}
		\label{fig:average_neutral_by_number_of_users}
	\end{subfigure}
	\vskip\baselineskip
	\begin{subfigure}{0.45\textwidth}
		\includegraphics[width=\linewidth]{images/average_positive_by_number_of_users.png}
		\caption{Positivos}
		\label{fig:imagem3}
	\end{subfigure}
	\hfill
	\begin{subfigure}{0.45\textwidth}
		\includegraphics[width=\linewidth]{images/average_negative_by_number_of_users.png}
		\caption{Negativos}
		\label{fig:average_negative_by_number_of_users}
	\end{subfigure}
	\caption{Distribuição dos sentimentos médios em relação ao número de usuários. Os gráficos apresentam uma análise detalhada dos sentimentos, incluindo o score composto pelo LeIA, bem como as categorias neutras, positivas e negativas.}
	\label{fig:average_sentiment_by_number_of_users}
\end{figure}

Com os scores de sentimentos devidamente ajustados, partimos para uma análise mais detalhada dos dados. Uma observação inicial revelou que, ao contrário da premissa inicial de que a maioria das postagens tinha um tom negativo, a distribuição de sentimentos, quando consideramos cada postagem individualmente, mostra que 55.9\% das postagens são positivas, 24\% são negativas e 20.2\% são neutras. No entanto, quando agregamos todas as postagens de um único usuário, observamos que cerca de 67.3\% dos usuários têm postagens majoritariamente positivas, 17.7\% são majoritariamente negativas e 15\% são neutras.

Isso sugere que, embora possa haver uma percepção predominante de que os usuários expressam insatisfação ou preocupações em suas postagens, a realidade é mais matizada. A maioria dos usuários, de fato, tende a compartilhar feedbacks ou observações positivas sobre suas comunidades. Isso pode ser um reflexo de uma série de fatores: talvez os usuários estejam mais inclinados a compartilhar experiências positivas para promover a coesão comunitária, ou talvez as plataformas de mídia social, como o Colab, estejam se tornando espaços onde as pessoas desejam destacar o que está funcionando bem, em vez de apenas apontar problemas.

No entanto, os 24\% de postagens negativas não devem ser negligenciados. Mesmo que representem uma minoria em relação às postagens positivas, elas são cruciais para entender as áreas de preocupação e insatisfação dos usuários. Essas postagens podem ser extremamente valiosas para os tomadores de decisão, pois fornecem insights diretos sobre onde as intervenções podem ser mais necessárias.

As postagens neutras, que compõem 20.2\% do total, também são intrigantes. Elas podem representar uma variedade de conteúdos, desde solicitações de informação até comentários que não expressam uma opinião clara em uma direção ou outra. Essas postagens podem servir como um lembrete de que nem todo feedback pode ser facilmente categorizado como positivo ou negativo, e que a neutralidade em si pode oferecer insights sobre as áreas onde os sentimentos dos usuários são mistos ou incertos.

\begin{quadro}[!htb]
	\caption{Scores de sentimento por quantidade de postagens}
	\label{fig:pie_sentiment_breakdown}
	\centering
	\includegraphics[scale=0.5]{images/pie_sentiment_breakdown.png}
	\fautor
\end{quadro}

Ao analisar os eventos postados, observamos que "Entulho na calçada/via pública" é um dos tópicos mais frequentemente mencionados, com um total de 44.273 postagens. No entanto, é interessante notar que a maioria dessas postagens é classificada como neutra, com 35.391 postagens, seguida por 5.712 postagens positivas e 3.170 postagens negativas. Isso sugere que, embora haja uma alta incidência de relatos sobre entulho nas vias, muitos usuários não expressam uma opinião fortemente positiva ou negativa sobre o assunto.

Outro evento que se destaca é "Lâmpada apagada à noite", com um total de 8.701 postagens. Deste total, 3.212 postagens foram classificadas como neutras, 2.296 como positivas e 3.193 como negativas. Isso indica que há uma divisão nas opiniões dos usuários sobre a questão da iluminação pública à noite, com muitos reconhecendo os esforços para resolver o problema, mas também uma quantidade significativa de postagens expressando insatisfação.

Por outro lado, "Ponto de travessia irregular" teve um total de 46 postagens, das quais 31 foram classificadas como negativas, 6 como neutras e 6 como positivas. Isso sugere que há uma preocupação predominante com a segurança dos pontos de travessia, e que muitos usuários veem isso como uma área que precisa de melhorias.

Em resumo, os dados refletem as preocupações dos cidadãos em relação a diferentes aspectos da infraestrutura e serviços públicos. Enquanto alguns eventos são amplamente reportados e recebem uma mistura de feedback positivo, negativo e neutro, outros eventos, embora menos frequentemente mencionados, destacam-se por ter uma opinião predominantemente negativa ou positiva.

\begin{quadro}[!htb]
	\caption{Distribuição dos 10 eventos mais comuns nas redes das 3 cidades.}
	\label{fig:pie_most_common_events}
	\centering
	\includegraphics[width=0.8\textwidth]{images/pie_most_common_events.png}
	\fautor
\end{quadro}

Ao expandir nossa análise para considerar a dimensão de gênero, observamos padrões distintos na distribuição de sentimentos entre diferentes grupos. Para as usuárias identificadas como femininas, notamos que a maioria das postagens (8.207) tem um sentimento negativo, seguido por 3.894 postagens positivas e 3.531 neutras. Isso sugere que, embora as mulheres estejam ativamente envolvidas na plataforma, elas tendem a expressar mais preocupações ou insatisfações em suas postagens do que sentimentos positivos ou neutros. Os usuários masculinos, por outro lado, apresentam um padrão diferente. Com 56.011 postagens neutras, 34.836 negativas e 22.513 positivas, vemos que a maioria das postagens masculinas é neutra.

Isso pode indicar que os homens na plataforma tendem a ser mais informativos ou questionadores em suas postagens, sem expressar uma opinião clara em uma direção ou outra. Quanto aos usuários não binários, a amostra é extremamente pequena, com apenas uma postagem neutra registrada. Isso pode ser devido a uma representação limitada desse grupo na plataforma ou à relutância em se identificar devido a preocupações de privacidade. Os usuários que optaram por não informar seu gênero ou se identificaram como "outros" também têm uma presença na plataforma, embora em números menores em comparação com os gêneros masculino e feminino. Para os não informados, a distribuição é de 101 postagens negativas, 42 positivas e 37 neutras. Já para os que se identificam como "outros", temos 1.780 postagens negativas, 526 positivas e 1.356 neutras.

Essa análise por gênero destaca a importância de considerar as diversas perspectivas e experiências dos usuários ao avaliar o sentimento nas postagens. Cada grupo traz uma lente única para a plataforma, e entender essas nuances pode ajudar a criar estratégias de engajamento mais eficazes e a responder de forma mais adequada às preocupações e feedbacks dos usuários.

\begin{quadro}[!htb]
	\caption{Distribuição dos scores de sentimento por gênero.}
	\label{fig:sentiment_by_gender}
	\centering
	\includegraphics[scale=0.8]{images/sentiment_by_gender.png}
	\fautor
\end{quadro}

Aprofundando ainda mais nossa análise, decidimos explorar a relação entre a análise de sentimentos e a estrutura da rede social do Colab. A ideia era entender como os sentimentos expressos nas postagens poderiam influenciar ou ser influenciados pelas conexões e interações entre os usuários. Neste contexto, a análise de redes sociais, combinada com a análise de sentimentos, pode oferecer insights valiosos sobre a formação e a dinâmica de câmaras de eco dentro da plataforma.

Um conceito fundamental na análise de redes é a assortatividade, que mede a tendência de nós em uma rede se conectarem a outros nós que são semelhantes em alguma característica específica. No nosso caso, estávamos interessados em entender se usuários com sentimentos semelhantes tendem a se conectar e interagir mais entre si.

\begin{table}[h]
	\centering
	\begin{tabular}{|l|c|}
		\hline
		\textbf{Atributo} & \textbf{Valor de Assortatividade} \\
		\hline
		Tipo de Evento    & 0.015                             \\
		\hline
		Idade             & 0.015                             \\
		\hline
		Gênero            & 0.025                             \\
		\hline
		Escolaridade      & 0.01987                           \\
		\hline
		Raça              & 0.025                             \\
		\hline
		Score Lexicon     & 0.01489                           \\
		\hline
		Score Composto    & 0.01489                           \\
		\hline
		Score Positivo    & 0.01487                           \\
		\hline
		Score Negativo    & 0.01489                           \\
		\hline
		Score Neutro      & 0.01489                           \\
		\hline
		Cidade            & 0.02506                           \\
		\hline
	\end{tabular}
	\caption{Assortatividade por Atributo}
\end{table}

Para entender a influência dos sentimentos nas conexões entre os usuários, começamos por calcular a assortatividade da rede em relação a várias características, incluindo o tipo de evento, idade, gênero, educação, raça, score médio de sentimento e cidade. A assortatividade nos fornece uma métrica quantitativa que indica se a rede exibe uma tendência de homofilia, ou seja, se nós semelhantes tendem a se conectar entre si.

Os resultados foram reveladores. Os resultados foram reveladores. A assortatividade para todas as características listadas variou entre 0.01487 e 0.02506, indicando uma tendência moderada de homofilia na rede. Em outras palavras, há uma leve tendência para usuários com características semelhantes se conectarem entre si.

O atributo 'gênero' e 'raça' apresentaram valores de assortatividade de 0.025, sugerindo que os usuários tendem a se conectar mais frequentemente com outros usuários do mesmo gênero ou raça. Da mesma forma, o atributo 'cidade' também apresentou um valor semelhante de 0.02506, indicando que os usuários têm uma propensão a se conectar com outros que residem na mesma cidade. Isso era esperado, pois é natural que usuários da mesma cidade tenham mais probabilidade de se conectar e interagir entre si, dada a proximidade geográfica e os problemas comuns enfrentados.

Estes resultados têm implicações significativas para a dinâmica da rede social do Colab. A formação de câmaras de eco, especialmente em relação ao sentimento expresso nas postagens, pode influenciar a disseminação de informações, a percepção dos problemas e as soluções propostas. Por exemplo, se um grupo de usuários consistentemente posta com um sentimento negativo sobre um determinado tipo de evento, isso pode influenciar a percepção de outros usuários sobre a gravidade ou prevalência desse problema. Além disso, a presença de câmaras de eco pode ter implicações para a eficácia das intervenções ou políticas implementadas com base no feedback dos usuários. Se a plataforma estiver dominada por vozes particularmente positivas ou negativas, isso pode distorcer a percepção dos tomadores de decisão sobre as necessidades e prioridades da comunidade.

\begin{quadro}[!htb]
	\caption{Assortatividade por Atributo}
	\label{fig:assortativity_by_attribute}
	\centering
	\includegraphics[scale=0.75]{images/assortativity_by_attribute.png}
	\fautor
\end{quadro}

Ao analisar os dez usuários com o maior número de postagens na rede, podemos identificar padrões e características que nos ajudam a entender melhor a dinâmica da plataforma e os comportamentos dos usuários mais ativos. Estes usuários, devido à sua alta atividade, têm o potencial de influenciar significativamente a percepção e o sentimento geral da comunidade.

\begin{table}[h]
	\centering
	\caption{Detalhes dos usuários mais ativos}
	\begin{tabular}{|c|c|c|c|c|c|c|c|c|}
		\hline
		ID     & Centralidade & Seguidores & Seguindo & Eventos & Score & Gênero & Idade & Raça \\
		\hline
		318649 & 0.0585          & 24         & 5        & 12287.0   & -0.011      & M      & 38    & -    \\
		\hline
		240336 & 0.0532          & 13         & 4        & 11609.0   & 0.03        & M      & 49    & P    \\
		\hline
		425243 & 0.0422          & 15         & 15       & 5310.0    & -0.001      & M      & 48    & P    \\
		\hline
		216238 & 0.1354          & 72         & 17       & 4490.0    & -0.055      & M      & 59    & P    \\
		\hline
		186310 & 0.2552          & 133        & 406      & 4185.0    & -0.052      & M      & 52    & N    \\
		\hline
		76184  & 0.0447          & 74         & 56       & 4031.0    & -0.079      & M      & 31    & N    \\
		\hline
		43341  & 0.0261          & 64         & 24       & 3621.0    & -0.165      & M      & 40    & B    \\
		\hline
		194422 & 0.0234          & 28         & 14       & 2911.0    & -0.329      & O      & 28    & -    \\
		\hline
		253059 & 0.0624          & 26         & 6        & 1698.0    & 0.082       & M      & 48    & B    \\
		\hline
		200628 & 0.0341          & 43         & 0        & 1550.0    & -0.393      & M      & 36    & B    \\
		\hline
	\end{tabular}
\end{table}

Os dois usuários que lideram em número de postagens, com IDs 318649 e 240336, demonstram uma presença significativa na rede, com 12287 e 11609 postagens, respectivamente. No entanto, a influência na rede não se restringe apenas ao volume de postagens. A centralidade de autovetor, que indica a influência de um nó na rede, mostra que o usuário com ID 186310, apesar de estar em quinto lugar em número de postagens, é um dos mais influentes, com uma centralidade de 0.2552. Este usuário destaca-se não só pela sua influência, mas também pelo seu alto número de seguidores, 133, e pelo fato de seguir 406 outros usuários.

O sentimento geral das postagens, representado pelo score médio, varia entre os usuários. Por exemplo, enquanto o usuário com ID 253059 tem um score médio positivo de 0.082, indicando uma tendência a postagens mais positivas, o usuário com ID 194422 tem o score mais negativo de -0.329, sugerindo postagens com um tom mais crítico ou descontente.

Além disso, é interessante observar a diversidade demográfica entre os usuários mais ativos. Temos representantes de diferentes faixas etárias, desde os 28 anos do usuário com ID 194422 até os 59 anos do usuário com ID 216238. Em termos de raça, há uma variedade, com usuários identificados como brancos, negros e pardos. O gênero predominante entre os mais ativos é masculino, com uma exceção identificada como "outros".

A presença de uma ampla gama de scores de sentimento entre os usuários mais ativos é uma indicação saudável de uma comunidade vibrante e multifacetada. Em muitas plataformas online, é comum encontrar usuários que repetidamente postam conteúdo com uma única tonalidade, seja ela positiva, negativa ou neutra. No entanto, a diversidade observada aqui sugere que esses usuários estão engajados em uma variedade de tópicos e situações, refletindo uma gama mais ampla de experiências e sentimentos.

Além disso, essa variedade pode ser vista como um indicativo de autenticidade. Usuários que consistentemente postam com um único tom podem ser percebidos como tendenciosos ou até mesmo como bots. Por outro lado, aqueles cujas postagens refletem uma variedade de sentimentos são mais propensos a serem vistos como genuínos e confiáveis por outros membros da comunidade.

Isso também destaca a importância de não fazer suposições apressadas sobre os usuários com base apenas em sua atividade. Enquanto um alto volume de postagens pode sugerir um usuário muito engajado, a verdadeira natureza de seu engajamento só pode ser compreendida ao se considerar o conteúdo e o sentimento dessas postagens.

A diversidade de sentimentos também pode ser um indicativo de que a plataforma está servindo a seu propósito de fornecer um espaço para discussão e feedback. Se todos os usuários mais ativos tivessem sentimentos uniformemente positivos ou negativos, poderia ser um sinal de que a plataforma está se tornando uma câmara de eco, onde apenas certas opiniões são expressas e reforçadas.

Além disso, essa diversidade pode ser benéfica para os administradores ou moderadores da plataforma. Ao monitorar os sentimentos variados dos usuários mais ativos, eles podem obter insights valiosos sobre áreas de preocupação, bem como aspectos da plataforma ou da comunidade que estão funcionando bem. Isso pode informar decisões sobre modificações na plataforma, campanhas de engajamento ou iniciativas de moderação.

Por fim, é essencial reconhecer que, em qualquer comunidade, a diversidade de opiniões e experiências enriquece o diálogo e a troca de ideias. A presença de usuários ativos com uma variedade de sentimentos sugere uma comunidade dinâmica e engajada, onde os membros se sentem livres para expressar suas opiniões e compartilhar suas experiências, sejam elas positivas, negativas ou neutras.

\begin{quadro}[!htb]
	\caption{Eigencentrality vs. Número de Posts}
	\label{fig:eigencentrality_vs_number_of_posts}
	\centering
	\includegraphics[scale=0.70]{images/eigencentrality_vs_number_of_posts.png}
	\fautor
\end{quadro}

Ao longo deste experimento, exploramos a o complexo microcosmo representado pelos sentimentos e opiniões expressas pelos usuários da plataforma Colab. A análise de sentimentos, quando aplicada a plataformas de mídia social, pode revelar insights profundos sobre as percepções, preocupações e satisfações dos usuários. No caso do Colab, essa análise nos permitiu entender melhor a dinâmica da comunidade e as emoções subjacentes às postagens dos usuários.

Primeiramente, a presença de uma distribuição quase equilibrada de sentimentos positivos e negativos desafia a noção comum de que plataformas de feedback tendem a ser dominadas por críticas. Isso sugere que o Colab não é apenas um espaço para reclamações, mas também um fórum onde os usuários reconhecem e apreciam as soluções e melhorias.

A análise dos usuários mais ativos e suas postagens revelou uma diversidade saudável de sentimentos, refutando a ideia de que os usuários mais ativos são unidimensionais em suas postagens. Esta diversidade é um testemunho da autenticidade e do engajamento genuíno dos usuários com a plataforma. Além disso, destaca a importância de considerar o conteúdo e o sentimento das postagens, em vez de se basear apenas no volume de atividade.

A presença de câmaras de eco, onde usuários com sentimentos semelhantes tendem a se agrupar, é uma preocupação em muitas plataformas online. No entanto, a diversidade de sentimentos observada no Colab sugere que a plataforma tem evitado, até certo ponto, essa armadilha. Isso é crucial para garantir que a plataforma continue a ser um espaço para discussão aberta e feedback construtivo.

A análise de redes sociais combinada com a análise de sentimentos também revelou padrões interessantes de conexão e interação entre os usuários. A tendência de usuários com sentimentos semelhantes se conectarem entre si tem implicações significativas para a disseminação de informações e a formação de opiniões dentro da plataforma.

Em conclusão, a análise de sentimentos no Colab ofereceu uma janela única para o coração e a mente da comunidade. Revelou uma comunidade vibrante e engajada, onde os membros se sentem empoderados para expressar suas opiniões e compartilhar suas experiências. À medida que a plataforma continua a crescer e evoluir, será essencial continuar monitorando e compreendendo esses sentimentos para garantir que o Colab permaneça um espaço inclusivo, autêntico e valioso para todos os seus membros.

\section{Análise de Sentimento com Regressão Supervisionada}
\label{sec:analise_de_sentimento_com_regressao_supervisionada}

Até o momento, nossa abordagem para a análise de sentimentos nas postagens do Colab foi baseada em técnicas manuais e ferramentas como o NLTK. Embora tenhamos obtido insights valiosos e estabelecido um score para cada postagem, é evidente que essa metodologia possui limitações em termos de escalabilidade e sustentabilidade, especialmente considerando o volume crescente de dados gerados diariamente na plataforma. Para superar esses desafios e otimizar o processo de análise de sentimentos, é imperativo adotar uma abordagem mais automatizada e robusta. Neste contexto, o aprendizado de máquina emerge como uma solução promissora. Ao utilizar um subconjunto das postagens que já classificamos manualmente - um conjunto de treinamento de aproximadamente 4.000 postagens - podemos treinar um modelo de aprendizado de máquina para realizar a classificação de sentimentos de forma autônoma. Esta abordagem não apenas acelera o processo de análise, mas também garante uma maior consistência e precisão na classificação. No próximo segmento, exploraremos em detalhes a implementação e os benefícios desta metodologia baseada em aprendizado de máquina, delineando como ela pode melhorar nossa capacidade de entender e interpretar os sentimentos expressos pelos usuários do Colab.

Nessa seção, descrevemos a metodologia empregada para a análise de dados e a construção do modelo de aprendizado de máquina e apresentamos uma abordagem para realizar a análise de sentimento em postagens do Colab. Este modelo foi treinado usando um conjunto de dados de treinamento que consiste em postagens de usuário rotuladas como positivas ou negativas. O modelo então aprende a associar certas palavras e frases a sentimentos positivos ou negativos.

A distinção entre classificação e regressão é um pilar central no aprendizado de máquina supervisionado. A classificação é destinada à atribuição de categorias discretas a instâncias de dados, ao passo que a regressão prediz elementos contínuos, oferecendo um espectro de possibilidades (\citeonline{2017_Shen_IC}).

No domínio da análise de sentimentos, a prática convencional muitas vezes simplifica a complexidade dos sentimentos humanos a categorias binárias. Todavia, sentimentos são intrinsecamente graduais e multidimensionais. Ao adotar uma escala contínua de -1 a 1 para representar sentimentos, abraçamos a sutileza e a riqueza dessas variações.

A opção pela regressão como abordagem modelagem foi catalisada pelos insights revelados na análise exploratória de sentimentos do capítulo anterior, na qual identificamos um espectro completo de sentimentos. Preservar essa continuidade permite-nos capturar as nuances mais finas nas postagens, concedendo-nos também a flexibilidade para futuras adaptações metodológicas sem a perda de detalhes valiosos dos dados originais.

Para fortalecer nosso conjunto de dados de treinamento e expandir nossa compreensão dos sentimentos manifestados nas postagens do Colab, incluímos uma seleção aleatória de 2000 eventos adicionais. Esta estratégia promove a diversidade e representa melhor a gama de sentimentos expressos, evitando vieses que poderiam surgir da seleção de extremos. Combinando esses eventos com os dados das postagens das três cidades estudadas anteriormente, consolidamos um conjunto de treinamento robusto com 4000 instâncias.

Esperamos que o algoritmo, ao ser treinado com essa fonte de dados, aprenda a correlacionar padrões linguísticos com sentimentos positivos ou negativos, propiciando uma classificação automática refinada dos sentimentos nas postagens do Colab. Esta abordagem de aprendizado supervisionado utiliza os dados de treinamento como alicerces para a criação de um modelo generalista capaz de captar a gama de sentimentos expressos nas postagens.

Depois da extração de recursos e da análise exploratória, dividimos os dados em conjuntos de treino e validação, seguido pela padronização dos mesmos para assegurar uniformidade na escala (\cite{2000_Jain}). Testamos uma gama de modelos regressivos para discernir o mais apropriado para nossa aplicação:

\begin{itemize}
\item \textbf{RandomForestRegressor}: Este algoritmo constrói um ensemble de árvores de decisão e usa a média de suas previsões para gerar o resultado final. É notável pela sua robustez em lidar com grandes volumes de dados e pela habilidade em captar interações complexas entre as variáveis, minimizando o risco de overfitting \cite[5-32]{2001_Breiman}.
\item \textbf{LinearRegression}: Um dos métodos mais fundamentais e amplamente aplicados, a Regressão Linear busca estabelecer uma relação linear entre variáveis dependentes e independentes.
\item \textbf{DecisionTreeRegressor}: Este modelo segmenta o espaço de dados em subconjuntos baseados em características e utiliza a média dos valores dos subconjuntos para fazer previsões. Sua interpretabilidade é uma vantagem, embora possa sucumbir ao overfitting se não for devidamente ajustado \cite[81-106]{1986_Quinlan}.
\item \textbf{K-Nearest Neighbors (KNN)}: O KNN faz previsões com base na semelhança entre instâncias de dados, considerando os 'k' exemplos mais próximos no conjunto de treinamento para inferir resultados.
\end{itemize}

Em um campo desafiador como a análise de sentimentos, especialmente quando focado na identificação de câmaras de eco, é imperativo selecionar um modelo de aprendizado de máquina que possa capturar com precisão a gama de sentimentos nos dados. Avaliamos o desempenho dos modelos utilizando métricas como MSE, MAE e o coeficiente de determinação (R\^2), que são mais indicativas do que a acurácia para tarefas de regressão.

O Random Forest Regressor destaca-se como a opção mais promissora, exibindo um equilíbrio ideal entre acurácia e generalização. O Decision Tree Regressor, apesar de um R\^2 quase perfeito no treinamento, evidencia overfitting devido à discrepância em validação. O KNN não alcançou o desempenho esperado em comparação aos outros modelos.

\begin{table}[h]
	\centering
	\begin{tabular}{|l|c|c|c|c|c|c|}
		\hline
		\textbf{Modelo} & \textbf{Training R$^2$} & \textbf{Validation R$^2$} & \textbf{MSE} & \textbf{MAE} & \textbf{RMSE} & \textbf{Tempo (s)} \\
		\hline
		Random Forest   & 0.9619                  & 0.7228                    & 0.0847       & 0.1683       & 0.2910        & 1192.96            \\
		\hline
		Decision Tree   & 0.9999                  & 0.5107                    & 0.1496       & 0.1894       & 0.3867        & 33.65              \\
		\hline
		KNN             & 0.4932                  & 0.1727                    & 0.2529       & 0.3871       & 0.5029        & 133.97             \\
		\hline
	\end{tabular}
	\caption{Desempenho dos modelos de regressão nos dados de treinamento e validação.}
	\label{tab:model_performance}
\end{table}	

A partir desses resultados, podemos concluir que o Random Forest Regressor é o modelo mais promissor entre os testados, com o melhor equilíbrio entre desempenho nos dados de treinamento e validação. O Decision Tree Regressor parece estar overfitting, dado o R\^2 quase perfeito nos dados de treinamento e o desempenho significativamente pior nos dados de validação. O KNN, por outro lado, tem um desempenho geralmente mais fraco em comparação com os outros modelos.

Em análise de sentimentos, a precisão é um desiderato, especialmente ao explorar dinâmicas sociais em plataformas como o Colab. Iniciamos este capítulo com um experimento nas mensagens dos usuários de três cidades, utilizando uma heurística baseada em dicionários de sentimentos, polaridade de emojis e ferramentas como o LeIA. A transparência dessa metodologia inicial é uma vantagem, mas também pode introduzir vieses. O trânsito para um modelo de aprendizado de máquina, especificamente o Random Forest, tem como objetivo mitigar tais vieses e melhorar a objetividade.

A preparação e a qualidade dos dados são vitais para o sucesso do modelo de aprendizado. Ao utilizar os scores de sentimentos do experimento inicial como entradas para o modelo regressivo, estabelecemos um processo sequencial e integrado de análises. Este estudo sublinha a importância de métodos adaptáveis na análise de sentimentos, especialmente ao lidar com fenômenos sociais complexos. Com os insights adquiridos, o próximo passo é classificar os usuários em personas, utilizando o score de sentimentos, tópicos de interesse e modelos de classificação para desvelar as tendências de polarização e interação no Colab.

A análise de sentimentos, particularmente em plataformas dinâmicas como o Colab, é uma tarefa intrincada que exige uma abordagem meticulosa. No início deste capítulo, conduzimos um experimento para classificar as mensagens dos usuários de Niterói, Santo André e Mesquita. Utilizando uma heurística baseada em dicionários léxicos, polaridade de emojis e a ferramenta LeIA, atribuímos scores de sentimentos às postagens. Este processo é transparente, permitindo-nos codificar cada etapa da atribuição de scores como uma caixa branca. No entanto, essa transparência também pode introduzir viéses, sejam eles conscientes ou não.

A decisão de transformar essa abordagem inicial em um modelo de aprendizado de máquina foi motivada pela perspectiva da engenharia de software. Ao adotar a regressão supervisionada, buscamos criar um sistema mais objetivo e menos suscetível a viéses humanos. A escolha do Random Forest Regressor, que demonstrou desempenho superior, reitera a necessidade de algoritmos robustos para capturar padrões complexos nos dados. Além disso, manter os scores de sentimento em um espectro contínuo, em vez de categorias discretas, permitiu uma representação mais fiel e granular dos sentimentos.

A qualidade dos dados e sua preparação são fundamentais para o sucesso de qualquer modelo de aprendizado de máquina. Ao usar os scores de sentimentos derivados do experimento inicial como insumo para o treinamento do modelo de regressão, demonstramos a interconexão e a sequencialidade dos experimentos. Esta pesquisa destaca a importância de abordagens adaptativas em análise de sentimentos, especialmente ao abordar fenômenos sociais complexos. Com os insights obtidos através desta modelagem de sentimentos, o próximo passo é a classificação dos usuários em personas. Utilizando o score de sentimento, os tópicos mais comentados e um modelo de classificação, buscaremos entender melhor os perfis dos usuários. Esta compreensão será crucial para a subsequente análise de redes, visando a detecção de câmaras de eco e aprofundando nossa compreensão sobre as dinâmicas de interação e polarização no Colab.

\section{Análise de Sentimento e Personas: Uma Nova Perspectiva}

Nossa jornada analítica no Colab começou com uma avaliação meticulosa das interações dos usuários, empregando técnicas sofisticadas de processamento de linguagem natural e análise de redes. Este processo gerou um sistema de pontuação capaz de capturar a polaridade dos sentimentos - positivos, negativos ou neutros - em cada postagem. Esta avaliação considerou não só o léxico e emojis utilizados, mas também os jargões e nuances específicas da plataforma.

Este sistema não só desvendou a natureza das interações na plataforma, revelando tendências e sentimentos dominantes, como também permitiu, ao ser combinado com análise de redes, vislumbrar a disseminação de sentimentos através das conexões entre os usuários. Esta sinergia entre análise de sentimentos e de redes elucidou a dinâmica complexa das interações comunitárias.

Com a análise de sentimentos estabelecida, o foco ampliou-se naturalmente para a classificação de personas, com o intuito de decifrar padrões comportamentais mais abrangentes dos usuários. Esta etapa avançada da pesquisa almejou transcender a análise individual de postagens para compreender como os usuários, em suas conexões e influências mútuas, coalescem em arquétipos mais amplos - as personas.

Essas personas são construções representativas que sintetizam os traços e comportamentos de grupos de usuários. No ecossistema do Colab, identificamos principalmente duas personas: os \textit{helpers} e os \textit{complainers}. Os Helpers se caracterizam pelo engajamento colaborativo e proativo, frequentemente auxiliando e partilhando informações valiosas. Já os Complainers tendem a adotar uma postura mais crítica, expressando descontentamento e críticas frequentes.

Ambas as personas são fundamentais para a dinâmica do Colab. Os Helpers fomentam o apoio mútuo e a resolução colaborativa de problemas, enquanto os Complainers podem ser catalisadores de mudança, destacando áreas que necessitam de atenção e melhoria. Contudo, quando essas personas se polarizam, surgem riscos de formação de câmaras de eco, locais onde as opiniões se reforçam mutuamente, limitando a diversidade de perspectivas.

Através da análise de sentimentos e subsequente classificação em personas, procuramos desvendar como os usuários se distribuem na rede do Colab e os padrões emergentes. Esse entendimento é crucial para identificar a existência de "bolhas de personas" e para investigar a dinâmica que as sustenta: Será um fenômeno intrínseco ao comportamento dos usuários ou as plataformas de mídia social estão, através de viéses algorítmicos, contribuindo ativamente para essa segregação? No decorrer desta seção, abordaremos as metodologias aplicadas para discernir e classificar as personas \textit{helpers} e \textit{complainers} no Colab. Ao aprofundar-se na implementação desses arquétipos, realçaremos sua relevância para a análise de dados e para o aprimoramento estratégico da plataforma.

\subsection{Helpers}

A persona \textit{helper} é caracterizada por um comportamento proativo e colaborativo em uma comunidade online. Esses indivíduos são frequentemente encontrados respondendo a perguntas, oferecendo conselhos e compartilhando informações úteis com outros membros da comunidade. Eles tendem a expressar sentimentos positivos em suas postagens e são motivados pelo desejo melhorar o coletivo e contribuir para a comunidade.

No Colab, por exemplo, muitos usuários realizam postagens frequentes reportando buracos nas vias, estruturas danificadas, alertam para situações de alagamento e deslizamento de terrenos. Muitos desses usuários adotam formatos padronizados para suas postagens, incluindo fotos, localização e descrição detalhada do problema. Além disso, eles frequentemente interagem com outros usuários, fornecendo informações adicionais ou atualizações sobre o status do problema.

Os \textit{helpers} são fundamentais para o sucesso de qualquer comunidade online, pois eles ajudam a criar um ambiente de apoio e colaboração. Eles são frequentemente vistos como líderes informais ou especialistas em suas respectivas áreas de interesse. Eles podem ser motivados por uma variedade de fatores, incluindo o desejo de compartilhar conhecimento, a satisfação de ajudar os outros, ou o reconhecimento e respeito que recebem da comunidade.

\subsection{Complainers}

A persona \textit{complainer} é caracterizada por um comportamento mais crítico ou negativo em uma comunidade online. Esses indivíduos são frequentemente encontrados expressando insatisfação, fazendo reclamações ou criticando ações das agências públicas. Eles tendem a expressar sentimentos negativos em suas postagens e são motivados por uma variedade de fatores, incluindo frustração, descontentamento ou a necessidade de expressar suas opiniões. Também é importante notar que a maioria das postagens de \textit{complainers} não são necessariamente negativas, pois os usuários estão de fato reportando problemas nas cidades, mas podem ser percebidas como tal devido ao seu tom ou conteúdo. Um outro fator interessante é que muitas das postagens de \textit{complainers} são direcionadas a assuntos mais individualizados como reportes nas vicinidades de suas residências ou locais de trabalho. Notavelmente ao analisar os comentários, esses usuários tendem a comentar ativamente em postagens criadas por outros usuários, muitas vezes expressando apoio ou concordância com o autor da postagem, mas sempre destacando a ineficiência ou ineficácia dos órgãos públicos.

No Colab, muitos usuários realizam postagens se referindo a prefeitura e outros órgãos sarcasticamente, fazendo críticas e reclamações sobre a falta de manutenção de vias, atrasos em obras, entre outros. Outro ponto recorrente diz respeito a perturbação sonora e uma certa correlação com alguns preconceitos musicais. Alguns usuários com scores particularmente negativos tendem a cobrar os orgãos públicos pelos impostos pagos que são, na opinião desses usuários, mal administrados, assim como outras tarifas como de transporte público ou de energia elétrica.

Os \textit{complainers} desempenham um papel importante em qualquer comunidade online, pois eles ajudam a identificar problemas, desafios ou áreas de melhoria. Embora suas postagens possam ser percebidas como negativas, elas podem fornecer feedback valioso que pode ser usado para melhorar a comunidade. No entanto, é importante gerenciar e responder adequadamente a esses usuários para evitar a criação de um ambiente negativo ou tóxico.

\subsection{Personas e papéis nas comunidades do Colab}

A presença das personas \textit{helpers} e \textit{complainers} dentro do Colab é altamente relevante para o ecossistema dessa plataforma colaborativa. Ambas as personas desempenham papéis distintos e complementares que podem influenciar a experiência do usuário e fornecer insights valiosos para o aprimoramento contínuo do aplicativo. A seguir, são apresentados os argumentos que sustentam a relevância dessas personas específicas:

Os Helpers desempenham um papel fundamental no Colab, pois contribuem ativamente para a comunidade, oferecendo suporte, compartilhando conhecimento e fornecendo soluções para os desafios enfrentados pelos usuários. Eles ajudam a fomentar a colaboração e o aprendizado coletivo, tornando-se recursos valiosos para aqueles que precisam de assistência ou orientação. Sua presença cria um ambiente propício para troca de ideias, resolução de problemas e crescimento mútuo. Ao compartilhar suas habilidades e conhecimentos, os Helpers estabelecem uma cultura de generosidade e reciprocidade dentro da comunidade do Colab. Eles inspiram outros usuários a se envolverem ativamente, encorajando a participação e a colaboração entre os membros. Além disso, a presença de Helpers é essencial para garantir que novos usuários se sintam bem-vindos e apoiados, promovendo assim um ambiente inclusivo e acolhedor.

Embora os Complainers possam ser vistos como usuários críticos ou negativos, sua presença é igualmente importante para o Colab. Essas personas desempenham o papel de destacar problemas, lacunas e áreas de melhoria dentro do aplicativo. Ao expressar suas preocupações e insatisfações, eles fornecem um feedback valioso que pode impulsionar o aprimoramento contínuo da plataforma.

Os Complainers atuam como "sentinelas" da comunidade, chamando a atenção para questões que podem ter sido negligenciadas ou passado despercebidas. Suas críticas construtivas podem levar a melhorias significativas na usabilidade, funcionalidade e qualidade geral do Colab. Além disso, ao abordar e resolver essas preocupações, a equipe responsável pelo desenvolvimento do aplicativo demonstra seu compromisso com a satisfação e o engajamento dos usuários.

A interação entre as personas \textit{helpers} e \textit{complainers} no Colab é uma relação simbiótica que impulsiona o crescimento e o aprimoramento contínuo da plataforma. Os Helpers oferecem suporte, orientação e soluções, tornando o ambiente colaborativo e enriquecedor. Por outro lado, os Complainers fornecem feedback crítico e identificam áreas de melhoria, promovendo a evolução e aprimoramento do aplicativo. Essa sinergia entre essas personas complementares é essencial para criar uma comunidade vibrante, responsiva e em constante aprimoramento no Colab.

A partir de uma análise exploratória de sentimento, foram identificadas as personas \textit{helpers} e \textit{complainers} dentro do Colab. No entanto, é importante ressaltar que existem outras personas que também podem ser identificadas, como por exemplo, a persona "Explorer", que tem um perfil voltado para a exploração da cidade e reporte dos problemas identificados. Além disso, uma nova persona que poderia ser considerada é a persona "Innovator", alguém que está sempre em busca de novas soluções e ideias inovadoras para melhorar a experiência dos usuários no Colab. A escolha das duas personas mencionadas anteriormente foi baseada nossa capacidade de mensurar e classificar essas personas através da analise de sentimento de suas postagens e também em seu potencial de gerar câmaras de eco, pois representam posições mais antagônicas dentro das comunidades da rede.

Nas próximas páginas, detalharemos as técnicas e metodologias utilizadas para identificar e classificar as personas \textit{helpers} e \textit{complainers} dentro do Colab, fornecendo uma visão mais aprofundada sobre a implementação dessas personas e sua contribuição para a análise de dados e aprimoramento da plataforma.

\section{Classificação de usuário por Persona}

A construção de um modelo de classificação para análise de sentimentos em plataformas como o Colab é uma tarefa intrincada. A linguagem utilizada pelos usuários é repleta de nuances, jargões específicos, emojis e outros elementos que podem influenciar a interpretação do sentimento. Além disso, a diversidade demográfica dos usuários traz uma riqueza de perspectivas e formas de expressão. O principal desafio é desenvolver um modelo que seja sensível a essas nuances, evitando o overfitting, e que possa generalizar bem para postagens não vistas anteriormente.

Para assegurar uma representação abrangente e equitativa da comunidade do Colab, 5000 postagens foram meticulosamente selecionadas com base em critérios demográficos. Esta seleção garantiu que todas as faixas etárias, gêneros e raças estivessem representados de forma proporcional à distribuição desses grupos entre os usuários do aplicativo. Além disso, levamos em consideração o tamanho das postagens, garantindo uma mistura de postagens longas e curtas. Cada postagem foi então manualmente classificada como \textit{helper} ou \textit{complainer}, considerando-se o conteúdo, a intenção e as nuances da linguagem.

Do conjunto inicial de postagens, 4000 foram destinadas ao treinamento, mantendo um equilíbrio de 2000 postagens para cada categoria (\textit{helper} e \textit{complainer}). As 1000 postagens restantes foram reservadas como conjunto de teste. Esta separação assegura que o modelo seja treinado e avaliado em conjuntos distintos, minimizando o risco de overfitting.

\section{Análise de Redes baseada em Persona}

A jornada até aqui foi marcada por uma série de experimentos meticulosos, todos com um objetivo comum: construir heurísticas robustas para a detecção de câmaras de eco na rede do Colab. Através da aplicação do modelo de classificação, conseguimos uma visão detalhada das personas dos usuários, revelando nuances sobre como eles interagem e se expressam na plataforma. Esta análise nos mostrou que, apesar de existirem críticos e insatisfeitos, a maioria dos usuários busca colaborar e contribuir de forma positiva.

Ao integrar essas personas na análise de redes, conseguimos ir além da simples classificação. Observamos como as personas se manifestam na estrutura da rede, como se conectam e quais posições ocupam. Esta análise nos deu insights valiosos sobre a dinâmica das interações e sobre como diferentes tipos de usuários influenciam e são influenciados dentro da rede.

No entanto, o mais importante é o que fazemos com esses insights. A detecção de câmaras de eco é mais do que um exercício acadêmico; é uma busca para entender como as opiniões são formadas e reforçadas em ambientes digitais. Ao identificar e compreender essas câmaras, temos a oportunidade de promover diálogos mais abertos e inclusivos, onde diferentes vozes são ouvidas e onde a informação circula de forma mais equilibrada. Com os resultados obtidos, estamos um passo mais perto de tornar o Colab, e outras plataformas digitais, espaços mais saudáveis e representativos para todos os seus usuários.

\section{Aplicação e resultados do Modelo de Classificação}

Após a meticulosa construção e validação do modelo de classificação, chegou o momento de aplicá-lo ao conjunto de dados real e observar sua capacidade em campo. Utilizamos a rede composta pelas três cidades com maior interação, conforme identificado no \autoref{chapter:05_exploratory}, e o dataset contendo postagens dos usuários dessas cidades relacionadas a eventos de zeladoria pública. Ao rodar o modelo em todas as postagens, cada uma recebeu um score de sentimento e uma persona atribuída. Posteriormente, para determinar a persona de cada usuário, compilamos todas as postagens classificadas desse usuário e, através de uma média, calculamos sua persona predominante. Esta abordagem nos permite não apenas entender o sentimento expresso em postagens individuais, mas também obter uma visão mais ampla do perfil comportamental dos usuários no Colab.

\begin{quadro}[htb]
    \centering
    \includegraphics[width=0.5\textwidth]{images/personas_pie.png}
    \caption{Distribuição de Personas}
    \label{fig:personas_pie}
\end{quadro}

\begin{quadro}[htb]
    \centering
    \includegraphics[width=0.95\textwidth]{images/personas_city.png}
    \caption{Distribuição de Personas por Cidade}
    \label{fig:personas_city}
\end{quadro}

A distribuição de personas revela uma predominância de usuários classificados como \textit{helper} (82.2\%) em comparação aos \textit{complainer} (17.8\%). Esta distribuição sugere que a maioria dos usuários do Colab tende a ser mais colaborativa e propositiva em suas postagens, enquanto uma minoria expressa insatisfações ou críticas. 

\begin{quadro}[htb]
    \centering
    \includegraphics[width=0.7\textwidth]{images/persona_gender.png}
    \caption{Distribuição de Personas por Gênero}
    \label{fig:persona_gender}
\end{quadro}

\begin{quadro}[htb]
    \centering
    \includegraphics[width=0.7\textwidth]{images/persona_education.png}
    \caption{Distribuição de Personas por Escolaridade}
    \label{fig:persona_education}
\end{quadro}

\begin{quadro}[htb]
    \centering
    \includegraphics[width=0.7\textwidth]{images/persona_race.png}
    \caption{Distribuição de Personas por Raça}
    \label{f
	ig:persona_race}
\end{quadro}

Ao analisar a distribuição por cidade, observa-se que Niterói possui a maior proporção de \textit{helpers} (1759) em comparação com \textit{complainers} (582). Em contraste, Mesquita apresenta uma proporção mais equilibrada, com 532 \textit{helpers} e 78 \textit{complainers}. Santo André, por sua vez, segue uma tendência semelhante a Niterói, com 1194 \textit{helpers} e 424 \textit{complainers}. 

O espectro de gênero mostra que ambos os gêneros, masculino e feminino, têm uma proporção maior de \textit{helpers} em relação aos \textit{complainers}. No entanto, é interessante notar que, embora haja uma representação mínima de gêneros não binários e outros, eles também seguem essa tendência. 

Quando observamos a escolaridade, os usuários com formação de bacharelado lideram tanto na categoria \textit{helper} quanto na \textit{complainer}. No entanto, é notável que, em todas as categorias de escolaridade, os \textit{helpers} superam os \textit{complainers}. Em relação à raça, a maioria dos \textit{helpers} e \textit{complainers} se identifica como brancos, seguidos por pardos. As demais categorias raciais apresentam números menores, mas ainda assim, a tendência de mais \textit{helpers} do que \textit{complainers} persiste.

\begin{figure}[h]
    \centering
    \includegraphics[width=1\textwidth]{images/personas_network.png} 
    \caption{Rede de usuários das 3 cidades com mais interações no Colab. Os usuários são agrupados de acordo com suas comunidades. Os nós são coloridos de acordo com a persona identificada: azul para helper, vermelho para complainer e laranja para não identificado.}
    \label{fig:personas_network}
\end{figure}

Após a classificação das postagens e a determinação das personas dos usuários, buscamos entender como essas personas se relacionam com a estrutura da rede de interações no Colab. A ideia é investigar se há padrões de comportamento ou tendências associadas a determinadas personas no contexto das interações na plataforma. Para isso, integramos o atributo de persona ao grafo, permitindo que cada nó (usuário) carregasse consigo sua persona predominante. Esta integração nos possibilita explorar correlações entre as métricas da rede e as personas, fornecendo insights sobre como diferentes tipos de usuários interagem e se posicionam dentro da rede.

A análise de redes revela um grafo direcionado com 6904 nós e 25785 arestas. A presença de 110 comunidades sugere uma estrutura de rede diversificada. A modulação, com uma pontuação de 0.6331, indica uma estrutura de comunidade bem definida. A assortatividade em relação à 'persona' é positiva (0.311), indicando que usuários com personas semelhantes tendem a se conectar entre si. Ao agrupar os nós da rede por 'persona', observa-se que, em média, os \textit{helpers} têm graus de entrada e saída mais altos e uma maior centralidade de eigenvector em comparação com os \textit{complainers}. Isso sugere que os \textit{helpers} podem ser mais centrais e influentes na rede do que os \textit{complainers}.

Ao combinar as métricas de análise de redes com as personas do usuário em uma matriz de correlação, percebemos que há uma relação positiva moderada entre o grau de entrada e o grau de saída. Na prática, o grau de entrada representa quantos usuários estão seguindo um determinado usuário, enquanto o grau de saída indica quantos usuários esse indivíduo está seguindo. Além disso, observa-se uma relação positiva entre o grau de entrada e a centralidade de eigenvector. A centralidade de eigenvector é uma métrica que identifica a influência de um nó na rede, considerando a qualidade das suas conexões. Ou seja, não apenas quantas conexões ele tem, mas também quão influentes são os nós com os quais ele está conectado.

Ao aprofundar a análise das métricas apresentadas, percebemos nuances distintas na distribuição e comportamento das personas nas três cidades em foco: Niterói, Santo André e Mesquita. A primeira observação notável é a variação na proporção de usuários classificados em cada cidade, sendo Mesquita e Santo André as que apresentam as maiores proporções, com aproximadamente 85\% de seus usuários classificados, enquanto Niterói possui cerca de 48\%.

\begin{figure}[h]
    \centering
    \includegraphics[width=1\textwidth]{images/network_personas_niteroi.png}
    \caption{Distribuição de personas na rede de usuários de Niterói.}
    \label{fig:network_personas_niteroi}
\end{figure}

Niterói, apesar de ter uma menor proporção de usuários classificados, mostra uma predominância significativa de \textit{helpers} sobre \textit{complainers}, uma característica compartilhada com Santo André. Mesquita, por outro lado, apresenta uma proporção ainda mais inclinada para \textit{helpers}, com 87,21\% de seus usuários classificados enquadrando-se nesta categoria. Este dado sugere que a dinâmica de interação em Mesquita pode ser mais colaborativa e positiva, o que pode influenciar a formação de comunidades e a disseminação de informações.

A análise da assortatividade em relação à 'persona' em cada cidade revela que usuários com personas semelhantes tendem a se conectar entre si, com a assortatividade variando de 0.007 em Niterói a 0.071 em Santo André. Este fenômeno pode ser um indicativo da formação de câmaras de eco, onde usuários com opiniões e comportamentos semelhantes tendem a se agrupar, reforçando suas visões e reduzindo a exposição à diversidade de pensamentos.

\begin{figure}[h]
    \centering
    \includegraphics[width=1\textwidth]{images/network_personas_sandre.png}
    \caption{Distribuição de personas na rede de usuários de Santo André.}
    \label{fig:network_personas_sandre}
\end{figure}

Ao observar as métricas de centralidade, notamos que, em média, os \textit{helpers} apresentam graus de entrada e saída mais altos e uma maior centralidade de eigenvector em comparação com os \textit{complainers} em todas as cidades. Este padrão sugere que os \textit{helpers} podem ocupar posições mais centrais e influentes na rede, atuando como hubs de informação e interação. A influência elevada dos \textit{helpers} pode ser um fator determinante na modulação do ambiente online, direcionando o tom e o conteúdo das discussões.

A modulação, medida que indica a estrutura de comunidade bem definida, varia entre as cidades, sendo mais alta na rede que engloba as três cidades e mais baixa em Mesquita. Esta variação pode ser interpretada como uma diferença na coesão das comunidades e na forma como os usuários interagem entre si. Cidades com modulação mais alta podem apresentar comunidades mais isoladas, o que pode favorecer a formação de câmaras de eco.

A matriz de correlação entre as métricas de centralidade e o valor da persona revela relações interessantes. Em todas as cidades, observa-se uma correlação positiva entre o grau de entrada e o grau de saída, bem como entre o grau de entrada e a centralidade de eigenvector. No entanto, a correlação entre o valor da persona e as métricas de centralidade é, em geral, negativa, sugerindo que os \textit{complainers} podem ter uma tendência a ser menos centrais e influentes na rede.

\begin{figure}[h]
    \centering
    \includegraphics[width=1\textwidth]{images/network_personas_mesquita.png}
    \caption{Distribuição de personas na rede de usuários de Mesquita.}
    \label{fig:network_personas_mesquita}
\end{figure}

A diversidade na estrutura de comunidades, evidenciada pelo número e tamanho das comunidades em cada cidade, sugere diferentes dinâmicas de agrupamento. Mesquita, por exemplo, apresenta o maior número de comunidades, mas com um tamanho médio menor, indicando uma possível fragmentação dos usuários em grupos menores. Esta fragmentação pode ser um terreno fértil para a formação de câmaras de eco, onde opiniões e informações são reforçadas dentro de grupos homogêneos.

Ao considerar o contexto acadêmico, é imperativo refletir sobre como essas descobertas podem contribuir para o entendimento das dinâmicas de interação online e a formação de câmaras de eco. A predominância de \textit{helpers} e sua posição central nas redes podem ser vistas como um mecanismo de resistência contra a polarização e a formação de câmaras de eco, promovendo a diversidade de opiniões e a interação construtiva.

No entanto, a presença de assortatividade positiva e a variação na estrutura de comunidades apontam para a necessidade de estratégias de intervenção e moderação para prevenir a formação de ambientes isolados e polarizados. A compreensão dessas dinâmicas é fundamental para o desenvolvimento de heurísticas e ferramentas que possam detectar e mitigar a formação de câmaras de eco, promovendo um ambiente online mais saudável e inclusivo.

Em conclusão, a análise detalhada das métricas de rede nas três cidades revela padrões e tendências que são essenciais para a compreensão das dinâmicas de interação e a identificação de câmaras de eco. A predominância e influência dos \textit{helpers}, a assortatividade positiva, a diversidade de comunidades e as correlações entre métricas de centralidade e valor de persona são elementos chave que podem guiar futuras pesquisas e intervenções na busca por um ambiente online mais equilibrado e representativo.

A jornada até aqui foi marcada por uma série de experimentos meticulosos, todos com um objetivo comum: construir heurísticas robustas para a detecção de câmaras de eco na rede do Colab. Através da aplicação do modelo de classificação, conseguimos uma visão detalhada das personas dos usuários, revelando nuances sobre como eles interagem e se expressam na plataforma. Esta análise nos mostrou que, apesar de existirem críticos e insatisfeitos, a maioria dos usuários busca colaborar e contribuir de forma positiva.

Ao integrar essas personas na análise de redes, conseguimos ir além da simples classificação. Observamos como as personas se manifestam na estrutura da rede, como se conectam e quais posições ocupam. Esta análise nos deu insights valiosos sobre a dinâmica das interações e sobre como diferentes tipos de usuários influenciam e são influenciados dentro da rede.

No entanto, o mais importante é o que fazemos com esses insights. A detecção de câmaras de eco é mais do que um exercício acadêmico; é uma busca para entender como as opiniões são formadas e reforçadas em ambientes digitais. Ao identificar e compreender essas câmaras, temos a oportunidade de promover diálogos mais abertos e inclusivos, onde diferentes vozes são ouvidas e onde a informação circula de forma mais equilibrada. Com os resultados obtidos, estamos um passo mais perto de tornar o Colab, e outras plataformas digitais, espaços mais saudáveis e representativos para todos os seus usuários.

\section{Homofilia e Câmaras de Eco}

\begin{quadro}[htb]
    \centering
    \includegraphics[width=0.8\textwidth]{images/personas_violin.png}
    \caption{Distribuição de Personas por Score (Violin Plot)}
    \label{fig:personas_violin}
\end{quadro}

A homofilia, termo cunhado por Lazarsfeld e Merton em 1954, refere-se à tendência de indivíduos se associarem a outros que são semelhantes a eles em termos de características, interesses e opiniões. Esse fenômeno é amplamente estudado em diversas áreas, incluindo sociologia, psicologia e ciência das redes, e tem sido observado em várias configurações sociais, desde relacionamentos pessoais até interações online em redes sociais.

A homofilia pode estar intimamente relacionada às câmaras de eco, uma vez que a tendência de buscar semelhanças pode levar à formação de grupos com visões e perspectivas convergentes. Quando os indivíduos se associam principalmente a outros que compartilham suas opiniões, informações e conteúdos que são compartilhados dentro desses grupos tendem a reforçar e amplificar essas visões específicas. Isso cria um ambiente propício para o desenvolvimento de câmaras de eco, onde a diversidade de perspectivas é limitada e as ideias divergentes são escassas. A homofilia pode contribuir para a persistência das câmaras de eco ao restringir a exposição a opiniões contrárias e limitar a troca de informações entre diferentes grupos na rede. Essa dinâmica pode resultar em polarização, falta de entendimento mútuo e até mesmo no fortalecimento de crenças extremas.

Estudos têm destacado a relação entre homofilia e câmaras de eco em diferentes contextos, como a propagação de desinformação em redes sociais ou a formação de bolhas de opinião política. A presença de homofilia nas redes sociais pode contribuir para a criação de "filtros de informação" que reforçam as visões existentes e dificultam a exposição a diferentes perspectivas, aumentando assim a probabilidade de formação de câmaras de eco.

Ao entender o conceito de homofilia e sua relação com as câmaras de eco, podemos explorar métricas e técnicas para identificar e mitigar esses fenômenos nas redes sociais. A análise de métricas de homofilia pode fornecer insights sobre a estrutura das redes sociais e ajudar a compreender como as informações são disseminadas e as opiniões são formadas dentro desses contextos específicos. Essas descobertas podem ser fundamentais para desenvolver estratégias e intervenções que promovam uma maior diversidade de perspectivas, diálogo e compreensão mútua na era digital.

As métricas de homofilia geradas a partir deste experimento podem ajudar a detectar câmaras de eco em um grafo da rede do Colab. Câmaras de eco são fenômenos sociais onde as opiniões e informações são amplificadas ou reforçadas pela comunicação e repetição dentro de um sistema fechado e podem contribuir para a polarização social. Ao identificar e entender estas câmaras de eco, podemos desenvolver estratégias para promover a diversidade de opiniões e a comunicação aberta.

\section{Classificação de personas em Modelagem Baseada em Agentes}

A Modelagem Baseada em Agentes (MBA) é uma técnica computacional que tem ganhado interesse em diversas áreas de pesquisa, devido à sua capacidade de simular a interação de agentes autônomos e observar os resultados emergentes dessas interações. A MBA é particularmente útil para estudar sistemas complexos, onde o comportamento global do sistema não pode ser facilmente deduzido a partir do comportamento individual dos agentes.

Agora, consideremos as personas \textit{helpers} e \textit{complainers}. Estas personas, embora não sejam explicitamente mencionadas por Atiqi, podem ser consideradas como agentes dentro da estrutura de MBA. Os \textit{helpers} podem ser vistos como agentes que buscam transmitir mensagens positivas e úteis, enquanto os \textit{complainers} tendem a transmitir mensagens negativas ou críticas. A interação entre essas duas personas pode levar a diferentes dinâmicas de rede e padrões de comunicação. Por exemplo, se os \textit{helpers} são mais influentes ou numerosos, eles podem criar um ambiente mais positivo e cooperativo na rede social. Por outro lado, se os \textit{complainers} são mais influentes ou numerosos, eles podem criar um ambiente mais negativo e crítico. 

Além disso, a presença de bots também pode influenciar a dinâmica entre \textit{helpers} e \textit{complainers}. Por exemplo, bots programados para agir como \textit{helpers} podem aumentar a positividade e a cooperação na rede, enquanto bots programados para agir como \textit{complainers} podem aumentar a negatividade e a crítica. Portanto, a abordagem de MBA usada por Atiqi se torna uma ferramenta útil para estudar a interação entre \textit{helpers} e \textit{complainers} em redes sociais e entender como essas interações podem influenciar a dinâmica da rede e a formação da opinião pública.

Dentro desse contexto, Atiqi introduz o conceito de "opinião média", que pode ser entendido como uma métrica que reflete a tendência geral ou o sentimento dominante em uma rede social em um determinado momento. A opinião média é calculada com base nas postagens e interações dos usuários, e pode ser influenciada tanto por \textit{helpers} quanto por \textit{complainers}, bem como por outros fatores externos, como notícias ou eventos atuais.

A ideia de calcular a opinião média é crucial para entender a dinâmica de uma rede social. Se, por exemplo, a opinião média em uma rede social é predominantemente positiva, isso pode indicar que os \textit{helpers} estão tendo um impacto maior na formação da opinião pública. Por outro lado, uma opinião média predominantemente negativa pode indicar uma influência maior dos \textit{complainers}.

No contexto da plataforma Colab, podemos adaptar essa ideia para extrair a opinião média dos usuários com base em suas postagens. Ao analisar o conteúdo, a frequência e o sentimento das postagens dos usuários, podemos calcular uma opinião média para diferentes tópicos ou áreas de interesse. Por exemplo, se muitos usuários estão postando sobre problemas de transporte público e expressando insatisfação, a opinião média sobre esse tópico seria negativa.

Para criar heurísticas que nos ajudem a extrair essa opinião média, podemos começar categorizando postagens com base em palavras-chave ou tópicos específicos. Em seguida, podemos analisar o sentimento dessas postagens usando técnicas de processamento de linguagem natural. Ao combinar essas informações com os dados de persona dos usuários, podemos obter uma visão mais completa da opinião média em diferentes tópicos.

No próximo capítulo, introduziremos o conceito de barômetro social hiperlocal. Este modelo tem a intenção de fornecer uma nova lente para interpretarmos e analisarmos as dinâmicas de redes sociais como o Colab. Exploraremos em detalhes como esse modelo pode ser utilizado para calcular a opinião média dos usuários, levando em consideração tanto o sentimento expresso em suas postagens quanto as personas associadas a cada um. Além disso, discutiremos como a combinação de técnicas de processamento de linguagem natural e Modelagem Baseada em Agentes pode proporcionar insights mais profundos sobre a formação da opinião pública e a dinâmica de interação entre diferentes grupos de usuários. Através dessa abordagem, buscamos não apenas entender, mas também prever tendências e padrões de comportamento dentro da plataforma, permitindo uma resposta mais eficaz e informada a diferentes cenários e desafios.

\chapter{Barômetro Social Hiperlocal}
\label{chapter:07_hyperlocalbarometer}
No capítulo anterior, exploramos a análise de sentimento das postagens de eventos de zeladoria pública e a metodologia dos modelos de classificação baseado em aprendizagem de máquina; categorizamos os usuários em duas personas: \textit{helper} e \textit{complainers}. Personas com postagens predominantemente \textit{helper} são proativos, transmitindo mensagens construtivas e buscando colaborar para o bem-estar da comunidade incentivando uma coletividade. Em contraste, os \textit{complainers} são mais críticos, frequentemente apontando falhas e expressando insatisfações, muitas vezes motivados por questões individualistas. Além disso, capturamos o score de sentimento expresso nas postagens dos usuários em um modelo regressivo que varia de -1 para postagens com sentimento negativo e 1 para postagens com sentimento positivo. Neste capítulo, exploraremos como essas informações podem ser utilizadas para calcular a pressão social hiperlocal, com a intenção entender como as preocupações dos cidadãos se disseminam na rede e permitindo uma análise contextualizada dos discursos.

Cada postagem no Colab é mais do que uma simples expressão de sentimentos; ela está diretamente vinculada a eventos específicos de zeladoria pública, com tipos de eventos pré-definidos, que dão contexto ao conteúdo compartilhado. Estes tipos de evento de zeladoria podem variar desde preocupações com segurança, como 'Ponto de tráfico de drogas', até perturbações como 'Ponto recorrente de Poluição Sonora' ou questões ambientais como 'Poda de Árvore'. 

A combinação desses tipos de eventos, o score de sentimento das postagens e as personas classificadas e a localização do evento, oferece uma visão holística das interações dos cidadãos na plataforma, permitindo uma compreensão mais aprofundada das preocupações e necessidades da comunidade. Nesse sentido, podemos considerar o Colab como uma ferramenta inovadora para monitoramento e análise de sentimentos em tempo real, capturando a essência das preocupações e sentimentos dos cidadãos em reação a mudanças que ocorrem no ambiente urbano. Ao categorizar postagens por eventos específicos, a plataforma fornece insights valiosos para tomadores de decisão, pesquisadores e outros stakeholders.

Portanto, consideramos que o Colab além de ser um aplicativo móvel, uma rede social de cidadania e uma empresa de GovTech, pode ser entendido também como o que viemos a chamar de 'Barômetro Social Hiperlocal'. Este conceito sugere que a plataforma não apenas capta as opiniões e sentimentos dos cidadãos, mas o faz levando em consideração as particularidades de diferentes localidades e temáticas. Assim, cada feedback fornecido pelos usuários pode ser interpretado como uma medida da 'pressão' social de uma área ou tema específico. 

O termo 'barômetro social' geralmente se refere à capacidade de medir a opinião pública. No entanto, ao adicionar 'hiperlocal', estamos destacando a especificidade geográfica e temática das opiniões. O Colab exemplifica esse conceito, permitindo uma análise contextualizada dos discursos tanto em diferentes localidades quanto em variados assuntos. Autoridades, organizações e cidadãos podem se beneficiar dessa ferramenta, obtendo uma compreensão mais profunda das preocupações locais e adaptando suas ações e políticas de acordo. 

Aplicativos como o Colab não apenas fornecem uma plataforma para participação cidadã, mas também se estabelecem como um instrumento vital para a tomada de decisões informadas em uma sociedade cada vez mais conectada e dinâmica. Os usuários do Colab não apenas expressam seus sentimentos e opiniões sobre eventos de zeladoria pública, mas também o fazem de forma geolocalizada e categorizada por tipo de evento. Esta característica única da plataforma a diferencia de outras redes sociais, como o Twitter, onde a opinião média sobre um tópico específico pode ser mais difícil de discernir devido à falta de categorização e geolocalização.

Durante a classificação manual das postagens para análise de sentimento e personas conduzida no capítulo anterior, identificamos padrões relacionados a tipos de eventos e sentimentos dos usuários. Por exemplo, postagens relacionadas à gentrificação mostram preocupações com a valorização imobiliária. Alguns usuários tendem a criar eventos de zeladoria no Colab destacando uma preocupação com a valorização ou desvalorização dos seus imóveis. Outras postagens refletem tensões sobre a presença de pessoas em situação de rua, enquanto algumas destacam preocupações com poluição sonora ou gestão da vegetação urbana. Além disso, postagens sobre gestão de resíduos e qualidade do ar indicam uma crescente conscientização ambiental. Alguns usuários também tendem a incorporar discursos políticos ou fiscais reverberando debates sobre governança e prestação de contas que acontecem fora das redes. 

Essa diversidade de tópicos revela a riqueza de insights que o Colab oferece sobre a dinâmica social e as prioridades locais. Nossa análise revelou que as preocupações dos cidadãos são variadas e contextualmente dependentes. O Colab reflete e amplifica as vozes dos cidadãos em questões locais. A geolocalização das postagens é fundamental para entender as preocupações específicas de cada comunidade, reforçando o caráter único do Colab como um barômetro social hiperlocal.

A singularidade do Colab está concentrada, principalmente, na produção e consumo de conteúdo em eventos de zeladoria pública. Estas postagens, categorizadas por tipos específicos de eventos e enriquecidas com informações geolocalizadas, fotos, comentários e \textit{likes} já fornecem aos stakeholders, especialmente às agências governamentais, uma perspectiva clara das necessidades e preocupações das comunidades. No entanto, ao adicionar uma camada de análise de sentimento, essa perspectiva pode se tornar ainda mais valiosa. Por exemplo, em um cenário em que uma mudança significativa é implementada, como a troca de empresas de coleta de lixo em uma cidade. A partir da análise dos dados produzido pelas interações do Colab, os tomadores de decisão podem monitorar em tempo real se a polaridade das postagens relacionadas a coleta de lixo piorou ou melhorou após essa mudança, servindo como um indicador da eficácia da decisão, ou, pelo menos, um indicador reação dos munícipes após essa mudança.

Assim, nossa hipotese é que aplicativos como o Colab não são apenas plataformas de interação, mas também podem ser entendidos como 'Barômetros Sociais Hiperlocais'. Esta perspectiva pode refinar os dados brutos da rede em um instrumento de \textit{feedback-loop}, não apenas para a expressão cidadã, eficácia de determinadas políticas públicas de acordo com o sentimento dos usuários. Ao compreender e responder às demandas expressas no Colab, as agências governamentais têm a oportunidade de aprimorar suas ações e políticas, garantindo uma gestão mais alinhada às necessidades e sentimentos das comunidades urbanas.

Entender o Colab como um 'Barômetro Social Hiperlocal' é informada e inspirada pelo conceito de Homogeneidade de Opiniões, conforme descrito por \citeonline{2023_Atiqi_BOOK}. A ideia de medir a pressão social de determinados assuntos ou temas nas comunidades da rede surgiu a partir da busca pela quantificação da homogeneidade de opiniões dos usuários.

\begin{citacao}
	'A more general concept defines echo chamber as the lack of information diversity a person is exposed with. The opinion does not necessarily have to be agreed by the person, but the type of opinions surrounding them should be homogeneous. A non-political example as suggested by Pentland is in the case of online trading (...) The lack of opinion diversity in this case is also considered as an echo chamber'. \cite[p. 17]{2023_Atiqi_BOOK}.
\end{citacao}

A homogeneidade de opiniões, conforme definido pelo autor, é um indicador-chave de câmaras de eco. Ao analisar padrões nas postagens e nos sentimentos expressos pelos usuários, é possível discernir se uma opinião ou perspectiva está sendo predominantemente reforçada. Também é possível entender a distribuição de opiniões entre os relacionamentos dos usuários na rede, verificando se a opinião é compartilhada por um grupo de amigos ou se é amplamente aceita por toda a rede. Essas heurísticas são cruciais para identificar câmaras de eco e podem ser aplicadas para detectar e mitigar a polarização de opiniões.

Câmaras de eco têm o potencial de distorcer a percepção da realidade e intensificar a polarização de opiniões, o que pode influenciar decisões políticas e a percepção das necessidades da comunidade. O Colab, ao ser interpretado como um 'Barômetro Social Hiperlocal', oferece uma fonte de dados para uma análise profunda das opiniões dos usuários, levando em conta sentimentos e personas. Esta análise pode revelar a homogeneidade de opiniões na rede, indicando se determinadas comunidades estão operando como câmaras de eco. A partir dessa perspectiva, os stakeholders têm a oportunidade de promover diálogos mais diversificados e formular políticas públicas mais alinhadas com as demandas da população.

A perspectiva do barômetro social no Colab não apenas destaca a polaridade e o sentimento das postagens, mas também revela padrões nas interações dos usuários. Por exemplo, ao avaliar postagens sobre 'poda de árvores' em uma comunidade, é possível discernir se a maioria dos usuários são \textit{helper} ou \textit{complainers} e qual é o sentimento predominante. Além disso, ao entender como as opiniões se agrupam e se propagam, podemos identificar pontos de influência na rede, locais onde intervenções podem ser mais impactantes para dissipar câmaras de eco e promover uma diversidade de opiniões.

Baseado nesse paradigma do Colab como um barômetro social, emergem heurísticas específicas que podem ser utilizadas tanto para quantificar a pressão social em comunidades hiperlocais quanto para detectar câmaras de eco. A pressão social, por sua vez, pode oferecer insights sobre a homogeneidade de opiniões na rede. Estas heurísticas, que serão detalhadas na próxima seção, fornecem uma estrutura robusta para entender a dinâmica das opiniões e sentimentos dos usuários, oferecendo insights valiosos para a análise e intervenção em contextos urbanos.

\section{Polarização e Participação Cidadã no Ciberespaço}

A polarização, um fenômeno profundamente enraizado na psicologia social e política, encontrou um terreno fértil e amplificado na era digital. Estudos em análise de redes sociais têm destacado essa tendência. \citeonline{2020_Cossard}, por exemplo, delineia os grupos 'pro-vacina' e 'anti-vacina', enquanto \citeonline{2014_Colleoni} evidencia a divisão entre 'Democratas' e 'Republicanos' nas redes sociais. De forma intrigante, \citeonline{2018_Jasny} explora a polarização em contextos off-line, examinando o discurso de políticos sobre mudanças climáticas. Embora a pesquisa de \citeonline{2018_Jasny} não se concentre diretamente em ambientes online, suas observações sobre a polarização entre as abordagens 'Binding Commitment', que propõem medidas mais conservadoras, e o 'Clean Power Plan', que enfatiza a transição para energias limpas, ressoam com as dinâmicas observadas nas plataformas digitais.

\begin{figure}[htbp]
	\centering
	\begin{subfigure}{0.3\textwidth}
		\includegraphics[width=\linewidth]{images/echo_chamber_graph_a.jpg}
		\caption{Gráfico demonstra câmara de eco entre usuários que seguem páginas de Ciência vs. Conspiraçao no Facebook e como nichos se afunilam promovendo isolamento entre usuários.}
		\fdireta{2019_Brugnoli}
		\label{fig:echo_chamber_graph_a}
	\end{subfigure}
	\hfill
	\begin{subfigure}{0.3\textwidth}
		\includegraphics[width=\linewidth]{images/echo_chamber_graph_b.jpg}
		\caption{Gráfico demonstra a câmara de eco entre Céticos vs. Defensores da vacinação na Itália e como a polarização pode contribuir para a desinformação.}
		\fdireta{2020_Cossard}
		\label{fig:echo_chamber_graph_b}
	\end{subfigure}
	\hfill
	\begin{subfigure}{0.3\textwidth}
		\includegraphics[width=\linewidth]{images/echo_chamber_graph_c.jpg}
		\caption{Grafico demonstra câmara de eco entre organicações no debate de políticas públicas de meio ambiente e mudanças climáticas dos EUA em 2016, especificamente sobre investir ou não em um plano de energia limpa.}
		\fdireta{2018_Jasny}
		\label{fig:echo_chamber_graph_c}
	\end{subfigure}
	\caption{Gráficos de polarização em estudos de análise de câmaras de eco.}
	\label{fig:echo_chamber_graph}
\end{figure}

Essa formação de grupos ideológicos não é meramente um reflexo da natureza humana, mas é intensificada pela arquitetura e algoritmos das plataformas digitais. Na busca por otimizar a experiência do usuário, essas plataformas frequentemente reforçam crenças preexistentes, gerando 'bolhas de filtro', conforme observado por \citeonline{2016_Flaxman}. Essas bolhas, embora possam servir como escudos contra informações perturbadoras, também restringem a exposição a uma gama diversificada de perspectivas. No entanto, a era digital não se resume apenas a câmaras de eco. \citeonline{2019_Brugnoli} destaca que em ambientes online, mecanismos cognitivos, como a evitação de desafio, viés de confirmação, dissonância cognitiva e a busca por validação, são intensificados, com a validação frequentemente a um clique de distância. Essa noção, reforça papel das mídias sociais na formação e reforço dessas fenômenos no ciberespaço.

\begin{citacao}
	'Eu defino o ciberespaço como o espaço de comunicação aberto pela interconexão mundial dos computadores e das memórias dos computadores. Essa definição inclui o conjunto dos sistemas de comunicação eletrônicos (aí incluídos os conjuntos de redes hertzianas e telefônicas clássicas), na medida em que transmitem informações provenientes de fontes digitais ou destinadas à digitalização. Insisto na codificação digital, pois ela condiciona o caráter plástico, fluido, calculável com precisão e tratável em tempo real, hipertextual, interativo e, resumindo, virtual da informação que é, parece-me, a marca distintiva do ciberespaço' \cite[p. 102]{2010_Levy_BOOK}.
\end{citacao}

\citeonline{{2010_Levy_BOOK}}, por sua vez, argumenta que as dinâmicas 'entrelaçadas' do ciberespaço refletem uma confluência de atores, projetos e interpretações, muitas vezes em oposição. O autor salienta que, apesar das tendências dominantes da era digital, a manifestação dessas tendências na vida cotidiana se dá por vários aspectos. A diversidade de interesses e perspectivas é emblemática da natureza fluida do ciberespaço. Enquanto alguns enxergam o ciberespaço como um domínio de comunicação livre e comunitária, outros o veem como um mercado global expansivo. Essas visões frequentemente colidem, ilustrando a complexidade e diversidade de vozes no ciberespaço. O autor também enfatiza que a representatividade cultural no ciberespaço é proporcional ao engajamento ativo e à qualidade das contribuições de seus participantes. 

Embora existam obstáculos à expressão da diversidade cultural, eles são menos proeminentes no ciberespaço do que em outros meios. Isso sugere que o ciberespaço, ao conectar indivíduos de diferentes origens, amplifica a diversidade de perspectivas. Em resumo, \citeonline{{2010_Levy_BOOK}} oferece uma perspectiva equilibrada e otimista sobre polarização e diversidade no ciberespaço. Ele reconhece os desafios da coexistência de múltiplas perspectivas, mas também vê o ciberespaço como um meio de expressão da diversidade cultural e colaboração. Isso sugere que, embora a polarização seja uma realidade, o ciberespaço também oferece oportunidades para diálogo e colaboração.

No contexto das plataformas digitais, a perspectiva de Lévy sobre a cibercultura é essencial para entender a dinâmica da polarização. Enquanto plataformas como o Colab podem enfrentar desafios de 'bolhas de filtro' que limitam a diversidade de perspectivas, a natureza interconectada da cibercultura, conforme descrito por Lévy, também apresenta oportunidades. Essa interconexão pode facilitar diálogos construtivos e a negociação de diferentes pontos de vista. Portanto, ao reconhecer e valorizar essa diversidade, o Colab tem o potencial de se tornar um espaço inclusivo para o diálogo cidadão.

A abordagem de Lévy sobre a cibercultura ressalta a natureza interconectada e a valorização da diversidade de perspectivas em discursos online evocam a utilização de heurísticas analíticas inovadoras que possam extrair informações estratégicas a partir dos dados de redes sociais. O Colab, além de ser um aplicativo e uma rede social de cidadania, pode ser classificado como um 'barômetro social hiperlocal'. Isso significa que, ao analisar postagens do ponto de vista de sentimentos e personas, é possível inferir a pressão social sobre determinados assuntos, relacionados a tipos específicos de eventos, de comunidades específicas em locais específicos. Assim, o 'barômetro social hiperlocal' não é uma ferramenta separada, mas sim uma caracterização do próprio Colab e de sua capacidade de mediar e refletir as nuances das opiniões e sentimentos da comunidade.

Na análise sentimento e personas das postagens do usuários do Colab, uma tendência interessante se destaca: a maioria dos usuários foi classificado como \textit{helper}, com apenas alguns nichos dominados por \textit{complainers}. Esta classificação de personas proporciona entender de forma mais matizada da comunidade, destacando áreas de colaboração positiva e pontos de tensão. O conceito de 'barômetro social hiperlocal', aliado à análise de sentimentos das postagens, adiciona uma dimensão adicional ao nosso entendimento. Ao incorporar o tipo de evento como uma variável, podemos discernir nuances na pressão social manifestada pelos usuários. Enquanto, em média, as postagens tendem a adotar um tom mais neutro, a análise focada em eventos específicos revela áreas de intensa polarização, permitindo-nos identificar e abordar tópicos de particular tensão dentro da comunidade. Com as abordagens e ferramentas certas, como aquelas inspiradas por Lévy e implementadas no Colab, há um potencial significativo para promover a participação cidadã, o engajamento e a construção de comunidades mais informadas e coesas.

\section{Dinâmicas de Pressão Social}

A análise das interações no Colab revela uma rica dinâmica de participação cidadã em eventos de zeladoria pública. Com um total de 132.858 eventos registrados, observa-se uma participação ativa de 4.569 usuários únicos, indicando uma média de aproximadamente 29 eventos por usuário. Esta média sugere um engajamento considerável dos cidadãos na plataforma, refletindo sua preocupação e envolvimento ativo em questões de zeladoria em suas comunidades.

Ao explorar a estrutura de relacionamentos entre os usuários, identificamos 25.785 conexões, ou arestas, que delineiam a rede de interações no Colab. Estas arestas representam as conexões estabelecidas entre os 6.904 nós, ou entidades, que compõem a rede. Estes nós, em sua maioria, representam os usuários e suas características demográficas e geográficas.

Em relação à distribuição geográfica, Niterói emerge como a cidade com a maior representação, contabilizando 4.246 usuários. Em seguida, temos Santo André com 1.942 usuários e Mesquita com 716. No entanto, ao analisar a distribuição de eventos por cidade, observamos uma dinâmica interessante. Mesquita, apesar de ter o menor número de usuários, lidera em termos de eventos registrados, com um total de 63.927. Niterói, com o maior número de usuários, registra 42.191 eventos, enquanto Santo André contabiliza 26.740 eventos. Esta discrepância entre o número de usuários e o número de eventos sugere variações no nível de atividade e engajamento dos usuários em diferentes cidades.

A análise dos tipos de eventos reportados nas três cidades - Mesquita, Niterói e Santo André - revela padrões distintos de preocupações e demandas dos cidadãos em cada localidade, refletindo as particularidades e desafios urbanos enfrentados por cada comunidade.

\begin{figure}[htb]
	\centering
	\includegraphics[width=0.7\textwidth]{images/pie_event_distribution.png}
	\caption{Distribuição dos tipos de evento mais comuns em Mesquita, Niterói e Santo André.}
	\label{fig:pie_event_distribution}
\end{figure}

Em Mesquita, o evento mais reportado, com uma expressiva quantidade de 45.235 registros, é 'Entulho na calçada/via pública'. Este número elevado sugere que a gestão de resíduos e a limpeza urbana são desafios significativos para a cidade. A presença massiva de entulho nas vias pode indicar problemas na coleta regular de lixo ou na conscientização da população sobre o descarte adequado. Além disso, eventos como 'Bueiro entupido' e 'Esgoto a céu aberto' também figuram no top 10. reforçando a ideia de que a infraestrutura urbana e os serviços de saneamento são áreas de preocupação para os cidadãos de Mesquita.

Por outro lado, em Niterói, a principal preocupação está relacionada à iluminação pública, com 'Lâmpada apagada à noite' liderando a lista. Este tipo de evento, além de estar relacionado à segurança pública, também pode afetar a qualidade de vida dos cidadãos, uma vez que áreas mal iluminadas podem desencorajar atividades noturnas e a circulação de pessoas. Adicionalmente, 'Buraco nas vias' e 'Fiação irregular' também são frequentemente reportados, indicando possíveis desafios na manutenção das vias públicas e na infraestrutura elétrica da cidade.

Em Santo André, 'Buraco nas vias' lidera as preocupações, seguido por eventos relacionados à gestão de resíduos e manutenção de áreas verdes, como 'Entulho na calçada/via pública' e 'Poda de árvore'. Estes dados sugerem que, embora haja preocupações com a infraestrutura viária, também existe uma demanda significativa por espaços urbanos mais verdes e bem cuidados.

\begin{figure}[htb]
	\centering
	\caption{10 Principais Tipos de Eventos mais criados por Cidade}\label{fig:city-events}
	\begin{subfigure}[b]{0.317\textwidth}
		\includegraphics[width=\textwidth]{images/pie_event_distribution_niteroi.png}
		\caption{Niterói}
		\label{fig:niteroi-pie}
		\subcaption*{Em Niterói, o alto número de relatos sobre problemas como \textit{Lâmpada apagada à noite} e \textit{Buraco nas vias} pode indicar uma preocupação com a segurança noturna e a qualidade das estradas. A alta frequência desses eventos sugere a necessidade de intervenções específicas.}
	\end{subfigure} ~
	\begin{subfigure}[b]{0.317\textwidth}
		\includegraphics[width=\textwidth]{images/pie_event_distribution_sa.png}
		\caption{Santo André}
		\label{fig:santo-andre-pie}
		\subcaption*{Santo André destaca-se pela quantidade significativa de relatos sobre \textit{Buraco nas vias} e \textit{Entulho na calçada/via pública}. Esses problemas podem impactar a mobilidade urbana e a limpeza das áreas públicas. Talvez medidas de manutenção e limpeza sejam necessárias.}
	\end{subfigure} ~
	\begin{subfigure}[b]{0.317\textwidth}
		\includegraphics[width=\textwidth]{images/pie_event_distribution_mesquita.png}
		\caption{Mesquita}
		\label{fig:mesquita-pie}
		\subcaption*{Mesquita é caracterizada por um grande número de relatos sobre \textit{Entulho na calçada/via pública}, sugerindo uma preocupação com a limpeza em espaços públicos. A alta incidência desse problema pode indicar a necessidade de iniciativas de limpeza e conscientização.}
	\end{subfigure}
\end{figure}


Estas variações nas principais preocupações reportadas em cada cidade refletem a diversidade de desafios urbanos enfrentados por diferentes comunidades. A pressão social, como medida pela frequência e tipo de eventos reportados, serve como um indicativo das áreas que requerem atenção prioritária das autoridades locais. Ao mesmo tempo, a capacidade dos cidadãos de reportar e categorizar eventos em plataformas como o Colab permite uma compreensão mais aprofundada das dinâmicas locais, oferecendo uma ferramenta valiosa para a tomada de decisões informadas.

Além disso, a análise desses eventos também pode fornecer insights sobre a eficácia das políticas públicas em vigor. Por exemplo, um aumento súbito no número de eventos relacionados a 'Buraco nas vias' após uma temporada de chuvas pode indicar a necessidade de melhorias na infraestrutura viária. Da mesma forma, um número elevado de reportagens sobre 'Entulho na calçada/via pública' pode sinalizar a necessidade de campanhas de conscientização sobre descarte adequado ou de melhorias nos serviços de coleta de resíduos.

\section{Heurísticas para cálculo da Pressão Social Hiperlocal}

Nesta seção, abordaremos as heurísticas desenvolvidas para quantificar a pressão ou polaridade dos discursos nas postagens de eventos de zeladoria pública no Colab. Estas heurísticas, originadas de uma combinação de literatura existente e insights práticos, se tornaram cruciais para entender a opinião média de um grupo de usuários sobre eventos específicos de zeladoria pública. Elas são essenciais para medir a pressão social em comunidades hiperlocais, fornecendo insights valiosos para tomadores de decisão, pesquisadores e outros stakeholders interessados em compreender as complexas dinâmicas urbanas.

A relevância dessas heurísticas reside na sua capacidade de capturar a essência das opiniões dos cidadãos em contextos urbanos específicos. Ao analisar os tipos de eventos e as dinâmicas de participação cidadã na plataforma, podemos identificar tendências, preocupações e áreas de interesse, auxiliando na tomada de decisões informadas.

O conjunto de dados utilizado para esta análise foi meticulosamente compilado. Identificamos todos os usuários que fazem parte das comunidades da rede, conforme detalhado no \autoref{chapter:05_exploratory}, das cidades de Niterói, Santo André e Mesquita na rede Colab. Posteriormente, todas as postagens disponíveis desses usuários em eventos de zeladoria pública foram analisadas. Utilizamos o modelo de classificação descrito no \autoref{chapter:06_sentiment} para prever a persona do usuário e atribuir um score de sentimento a cada postagem. Esta metodologia nos permitiu obter insights sobre o comportamento dos usuários, como a expressão de suas opiniões e sentimentos. Adicionalmente, categorizamos os tipos de eventos associados a cada postagem.

\begin{table}[htbp]
	\centering
	\caption{Modelo de Dados para Análise de Pressão Social Hiperlocal}
	\begin{tabular}{ll}
		\toprule
		\textbf{Campo}    & \textbf{Descrição}                                  \\
		\midrule
		event\_id         & Identificador único do evento                       \\
		colab\_user\_id   & Identificador único do usuário do Colab             \\
		score             & Score de sentimento atribuído à postagem do usuário \\
		persona\_value    & Persona prevista para o usuário que fez a postagem  \\
		event\_type\_id   & Identificador único do tipo de evento               \\
		event\_type\_name & Nome descritivo do tipo de evento                   \\
		\bottomrule
	\end{tabular}
	\label{tab:modelo-dados-barometro}
\end{table}

Na \autoref{tab:modelo-dados-barometro}, cada linha representa uma postagem no Colab e inclui informações como o ID do evento, o ID do usuário do Colab, a pontuação de sentimento associada à postagem, a persona atribuída ao usuário que a fez, o ID do tipo de evento e o nome do tipo de evento relacionado. Esses dados formam a base essencial para nossas análises, permitindo-nos calcular a pressão social hiperlocal e compreender as dinâmicas das preocupações urbanas nessas comunidades específicas.

Com base neste modelo de dados, iniciamos o processo de filtragem e agregação. Primeiro, focamos nos membros ativos das comunidades identificadas na análise exploratória conduzida no \autoref{chapter:05_exploratory}. Em seguida, selecionamos tipos de eventos específicos, agrupados por tema, para nossa análise. Esta seleção foi guiada pela intenção de avaliar a pressão social hiperlocal em relação a preocupações específicas de zeladoria pública. Após a filtragem, calculamos duas métricas-chave para cada tipo de evento: a pontuação média de sentimento e a persona média. A primeira reflete o sentimento médio das postagens, enquanto a segunda representa a distribuição média das personas dos usuários. Essas métricas são calculadas para cada tipo de evento, permitindo-nos comparar e analisar as diferenças entre eles.

A visualização da pressão social hiperlocal é apresentada através de um gráfico de radar, uma representação gráfica que permite analisar e comparar diversas variáveis em relação a um ponto central. O gráfico possui dois planos distintos: o plano de score e o plano de persona.

\begin{quadro}[htb]
	\centering
	\includegraphics[width=0.7\textwidth]{images/social_barometer_plot.png}
	\caption{Gráfico de Radar para Análise de Pressão Social Hiperlocal. Os segmentos representam tipos de eventos, enquanto o eixo radial exibe valores médios de scores de sentimentos (plano vermelho) e personas (plano azul) atribuídos às postagens dos usuários relacionadas a cada tipo de evento.}
	\label{fig:social_barometer_plot}
\end{quadro}

No plano de score, cada segmento do gráfico de radar representa um tipo específico de evento do Colab, onde cada evento é associado a um ângulo theta. O eixo radial, representado pelo parâmetro 'R', indica o valor médio dos scores de sentimentos atribuídos às postagens dos usuários em relação a um determinado tipo de evento. Quanto mais distante do centro estiver um segmento, maior será o valor médio do score e, consequentemente, mais positivo será o sentimento expresso pelos usuários em relação a esse evento. Por outro lado, segmentos mais próximos do centro indicam scores médios mais baixos, refletindo sentimentos mais negativos.

No plano de persona, novamente, cada segmento do gráfico representa um tipo de evento, O eixo radial 'R' neste caso indica o valor médio das personas previstas dos usuários em relação ao tipo de evento. Quanto mais próximo de 0 estiver um segmento, mais os usuários tendem a assumir uma persona de \textit{helper}, caracterizada por atitudes positivas e colaborativas em relação ao evento. À medida que o valor de 'R' se aproxima de 1, os usuários tendem a adotar uma persona de \textit{complainer}, indicando uma postura mais crítica e insatisfeita. Essa representação gráfica única proporciona uma visão abrangente e comparativa das opiniões e personas dos usuários do Colab em relação a diferentes tipos de eventos, permitindo uma análise mais profunda das dinâmicas das comunidades hiperlocais.

Após a definição do modelo de dados e a coleta das postagens no Colab, começamos a formulação das heurísticas que conduziriam à análise da pressão social hiperlocal. Esta etapa foi crucial, pois a vastidão e variedade dos dados requeriam um direcionamento para captar efetivamente as nuances das dinâmicas urbanas. O primeiro passo foi a identificação dos tópicos de interesse. Escolhemos tópicos que são comumente discutidos em comunidades urbanas e têm um impacto direto na qualidade de vida dos cidadãos. Um exemplo elucidativo dessa seleção é o tópico 'Tarifa de Transporte Público'. 

A escolha desse tema não se deu apenas pela sua manifesta relevância em discussões urbanas e pelo impacto direto que exerce no cotidiano financeiro e rotineiro dos cidadãos, mas também por sua ressonância histórica. Na história recente do Brasil, podemos remeter a um período de turbulência política, cujo estopim foi justamente o descontentamento popular em relação ao aumento das tarifas de transporte. Em junho de 2013, uma série de protestos, inicialmente convocados contra o aumento das passagens, ganhou magnitude e se espalhou por diversas cidades do país. Rapidamente, as manifestações incorporaram uma variedade de pautas e descontentamentos, culminando em uma das maiores mobilizações populares das últimas décadas no Brasil. Esse evento histórico ilustra a capacidade do tema 'Tarifa de Transporte Público' de catalisar discussões mais amplas e mobilizar grandes contingentes da população em torno de demandas comuns.

Para cada tópico escolhido, definimos um conjunto de palavras-chave. Estas palavras-chave são termos ou expressões frequentemente associados ao tópico em questão. No caso do tópico 'Tarifa de Transporte Público', palavras como 'lotado', 'ônibus', 'metrô', 'tarifa' e 'aumento da tarifa' foram consideradas. Essas palavras-chave funcionam como um filtro inicial, permitindo-nos identificar postagens no Colab que possam estar relacionadas ao tópico em análise. Com as palavras-chave definidas, realizamos uma busca nas postagens para identificar os tipos de eventos associados a elas. Esta busca retorna uma variedade de eventos, que podem ou não estar diretamente relacionados ao tópico de interesse.

Por exemplo, uma postagem que menciona 'passagem está cara' em um contexto de 'ônibus danificado' sugere uma intersecção do tópico de 'Tarifa de Transporte Público' com um evento relacionado. Isso indica que o usuário está manifestando sua insatisfação com o serviço, relacionando o valor pago à qualidade recebida. Para refinar ainda mais nossa análise, criamos uma lista de eventos não relevantes, que são excluídos da análise final. Esta lista foi elaborada com base em nossa compreensão do tópico e na intuição de quais eventos poderiam desviar o foco da pressão social que queríamos captar. No exemplo anterior, os eventos como 'Ponto de infração de trânsito recorrente' e 'Rampa de acessibilidade irregular ou inexistente' foram considerados na análise, pois, mesmo não sendo diretamente sobre tarifas, são eventos que afetam a experiência do usuário no transporte público.

Após a filtragem e seleção, calculamos métricas para os eventos restantes, como pontuação média de sentimento e persona média. A combinação dessas métricas, associadas aos eventos filtrados, nos fornece um panorama da pressão social em relação ao tópico analisado. Essa abordagem, que combina a seleção de tópicos, identificação por palavras-chave e filtragem de eventos, permite-nos isolar e analisar os sentimentos e opiniões dos usuários em relação a questões urbanas específicas, proporcionando insights valiosos sobre as dinâmicas das comunidades urbanas.

\section{Tópicos de Pressão Social}

O Colab é uma plataforma de participação cidadã que proporciona um espaço virtual para os cidadãos expressarem suas preocupações, compartilharem experiências e debaterem questões urbanas relevantes. Neste ambiente, emergem tópicos de pressão social que refletem as preocupações específicas dos cidadãos. A escolha desses tópicos foi baseada não apenas na sua frequência de aparição na plataforma, mas também na sua relevância para as políticas públicas urbanas e na amplitude de impacto que podem ter nas comunidades. Ao escolher esses tópicos, procuramos abordar tanto questões mais generalizadas, como saúde e segurança, quanto questões emergentes e altamente debatidas, como gentrificação e higienismo social. Esses tópicos desempenham um papel fundamental na compreensão dos desafios enfrentados nas cidades.

Em uma plataforma como o Colab, as preocupações dos cidadãos podem ser expressas de maneira apaixonada e, por vezes, polarizada. Através de uma busca por palavras-chave específicas, identificamos e categorizamos diversos desses tópicos de pressão social que surgem nas discussões e associamos eventos de zeladoria pública a esses tópicos. Agora, nosso objetivo é analisar esses tópicos sob a perspectiva do Colab como um 'Barômetro Social Hiperlocal', investigando como eles impactam a dinâmica da plataforma e fornecem informações valiosas para decisores, pesquisadores e comunidades e munícipes. As métricas de pressão social organizadas por tipo de evento e contendo número de eventos criados, score médio de sentimento e persona média identificada estão disponíveis no \autoref{chapter:tables_social_pressure}.

\subsection{Mobilidade Urbana}
\label{section:mobilidade-urbana}

\begin{figure}[htb]
	\centering
	\includegraphics[width=0.7\textwidth]{images/colab_posts_mobility.png}
	\caption{Usuário do Colab expressando insatisfação com a qualidade da ciclofaixa em comparação com a rua de automóveis.}
	\label{fig:colab_posts_mobility}
\end{figure}

A mobilidade urbana é um tópico essencial nas cidades modernas, especialmente com o crescimento das populações urbanas e os desafios enfrentados pelos sistemas de transporte. Na plataforma colaborativa Colab, cidadãos discutem intensamente sobre este tema, refletindo a era das 'smart cities' e 'connected citizens'. As opiniões se polarizam entre preferências por transporte pessoal, transporte público, ciclovias e caminhadas. Questões ambientais, como a redução das emissões de carbono, e temas de acessibilidade e equidade também dividem opiniões, influenciando debates sobre planejamento urbano e alocação de recursos.

As métricas de pressão sociais disponíveis na \autoref{tab:eventos_populares_mobility} destacam uma série de eventos relacionados à mobilidade urbana, cada um com um número significativo de ocorrências e variações notáveis nos scores de sentimento e personas. Entre os eventos mais frequentes, 'Bloqueio na via' e 'Ponto de infração de trânsito recorrente' aparecem 16 vezes cada, ambos com scores negativos de -0.2272 e -0.2458, respectivamente, indicando uma percepção predominantemente negativa dos usuários. A persona associada a esses eventos é alta, sugerindo que esses eventos são frequentemente reportados por indivíduos que se sentem fortemente inclinados a expressar sua insatisfação.

Por outro lado, 'Entulho na calçada/via pública' possui um score relativamente menos negativo de -0.0303, refletindo uma reação menos intensa dos usuários. Eventos como 'Bueiro sem tampa' e 'Manutenção de ciclovia/ciclofaixa' apresentam os scores de sentimento mais negativos, -0.3101 e -0.2643, respectivamente, indicando preocupações significativas com a infraestrutura urbana. A análise das personas mostra que eventos como 'Ponto de transporte clandestino' têm uma persona muito alta (0.9333), indicando uma forte tendência dos usuários de reportar esses problemas específicos. No geral as métricas de pressão social revelam uma preocupação comum com a manutenção e segurança das vias públicas, refletindo uma pressão social significativa sobre esses tópicos.

\begin{figure}[htb]
	\centering
	\includegraphics[width=0.7\textwidth]{images/wordcloud_mobility.png}
	\caption{Wordcloud com palavras mais frequentes em postagens sobre mobilidade urbana}
	\label{fig:wordcloud_mobility}
\end{figure}

\begin{quadro}[htb]
	\centering
	\includegraphics[width=0.7\textwidth]{images/social_barometer_mobility.png}
	\caption{Gráfico de Radar ilustrando a pressão social em relação à mobilidade urbana. O eixo radial mostra os scores de sentimentos (plano vermelho) e personas (plano azul), enquanto os segmentos descrevem diversos eventos urbanos.}
	\label{fig:social_barometer_mobility}
\end{quadro}

\subsection{Infrações de Trânsito}
\label{sec:eventos_populares_traffic}

\begin{figure}[htb]
	\centering
	\includegraphics[width=0.7\textwidth]{images/colab_posts_traffic.png}
	\caption{Usuário do Colab expressando sua opinião sobre infrações de trânsito.}
	\label{fig:colab_posts_traffic}
\end{figure}

No tópico de infrações de trânsito, a análise de dados mostra um padrão complexo na percepção dos cidadãos sobre as ações das autoridades de trânsito. As métricas disponíveis na \autoref{tab:eventos_populares_traffic} demonstram uma clara dicotomia: por um lado, muitos acham que as autoridades poderiam fazer mais para resolver as infrações; por outro, alguns veem as autoridades como excessivamente punitivas, criando uma 'indústria da multa'. Palavras-chave como 'multa', 'infração', 'detran' e 'IPVA' são frequentemente mencionadas, indicando preocupações com penalidades e questões burocráticas.

\begin{figure}[htb]
	\centering
	\includegraphics[width=0.7\textwidth]{images/wordcloud_traffic.png}
	\caption{Wordcloud com palavras mais frequentes em postagens sobre infrações de trânsito}
	\label{fig:wordcloud_traffic}
\end{figure}

Ao analisar sentimentos e perspectivas sobre infrações, nota-se uma diversidade de reações. Alguns tópicos como 'Conservação (via pública)' e 'Via de terra com desnível' recebem sentimentos positivos, sugerindo satisfação com medidas adotadas. Em contraste, questões como 'Manutenção de faixa de pedestre' e 'Ônibus/trem/metrô danificado' têm sentimentos negativos, indicando insatisfação. Além disso, problemas como 'Rampa de acessibilidade irregular' e 'Bueiro sem tampa' tendem a gerar críticas, enquanto questões como 'Retirada de árvore' e 'Conservação (via pública)' veem posturas mais colaborativas.

\begin{quadro}[htb]
	\centering
	\includegraphics[width=0.7\textwidth]{images/social_barometer_traffic.png}
	\caption{Gráfico de Radar ilustrando a pressão social em relação ao tópico de Infrações de Trânsito.}
	\label{fig:social_barometer_traffic}
\end{quadro}

Esses dados mostram que as preocupações dos cidadãos com infrações de trânsito vão além das multas, abrangendo a eficiência das autoridades e a qualidade de vida urbana. O alto engajamento em tópicos como multas e zonas de infração recorrente revela um interesse significativo da comunidade nesses assuntos. A pressão social reflete uma mistura de críticas construtivas e insatisfação, destacando a diversidade de opiniões e a complexidade das dinâmicas sociais relacionadas às infrações de trânsito.

\subsection{Tarifa de Transporte Público}
\label{sec:eventos_populares_busfare}

\begin{figure}[htb]
	\centering
	\includegraphics[width=0.7\textwidth]{images/colab_posts_busfare.png}
	\caption{Exemplo de postagem no Colab relacionadas ao tópico de Tarifa de Transporte Público}
	\label{fig:colab_posts_busfare}
\end{figure}

No contexto da tarifa de transporte público, os dados da \autoref{tab:eventos_populares_busfare} mostram que este é um tópico de grande preocupação para os cidadãos. Palavras-chave como 'passagem', 'ônibus', 'terminal' e 'cobertura' indicam foco nas questões de custo e acessibilidade do transporte público. Muitas das postagens relacionadas a este tópico expressam insatisfação, como é evidenciado pelas menções frequentes a 'Ônibus superlotado' e 'Ônibus fora do horário/rota', com sentimentos majoritariamente negativos e uma tendência dos usuários a se expressarem como críticos (perfil de \textit{complainer}).

\begin{figure}[htb]
	\centering
	\includegraphics[width=0.6\textwidth]{images/wordcloud_busfare.png}
	\caption{Wordcloud com palavras mais frequentes em postagens sobre infrações de trânsito}
	\label{fig:wordcloud_busfare}
\end{figure}

Por outro lado, há sinais de disposição para colaborar em resolver problemas relacionados ao transporte público, especialmente em tópicos como 'Conservação (via pública)' e 'Faixa de pedestre apagada', onde os sentimentos são menos negativos e a tendência é para um perfil mais colaborativo (perfil de \textit{helper}).

\begin{quadro}[htb]
	\centering
	\includegraphics[width=0.7\textwidth]{images/social_barometer_bus_fare.png}
	\caption{Gráfico de Radar ilustrando a pressão social em relação ao tópico de Tarifa de Transporte Público}
	\label{fig:social_barometer_bus_fare}
\end{quadro}

A frequência do termo 'passagem' sugere uma preocupação especial com os custos do transporte, enquanto a prevalência do termo 'ônibus' em comparação com 'metrô' indica que os problemas com ônibus são mais relatados ou percebidos como mais críticos. Nos casos de eventos descritos como 'Agentes e Operadores de trânsito' ou 'Metrô/trem danificado', a postura é predominantemente crítica, com sentimentos negativos, apontando para áreas que podem necessitar de atenção imediata por parte das autoridades. Em contraste, questões como 'Manutenção de pintura da via' e 'Banco danificado' geram reações mais colaborativas, mesmo que menos frequentes, indicando que em certas situações, os cidadãos estão dispostos a oferecer feedback construtivo.

Em resumo, a discussão sobre tarifas de transporte público não se limita a ser um tópico de debate intenso; ela também abre caminho para uma colaboração significativa entre os cidadãos e as autoridades. Isso implica um potencial valioso para uma parceria construtiva na busca de melhorias no serviço de transporte público.

Para a administração das cidades, isso significa que há uma oportunidade de aproveitar esse engajamento cívico para melhorar os serviços de transporte. Os dados coletados das discussões dos cidadãos fornecem insights específicos sobre os problemas enfrentados pelos usuários do transporte público, como superlotação, inconsistências de horários e condições das infraestruturas. Ao analisar esses dados, as autoridades podem identificar as áreas mais críticas que necessitam de atenção imediata.

Por exemplo, questões frequentemente mencionadas, como a superlotação dos ônibus, podem indicar a necessidade de aumentar a frequência das viagens ou adicionar mais veículos nas rotas mais movimentadas. Da mesma forma, reclamações sobre o estado de conservação dos veículos ou infraestruturas (como terminais e estações) podem guiar os esforços de manutenção e investimento.

Utilizar esses dados também pode auxiliar no planejamento de políticas de transporte mais inclusivas e sustentáveis. Por exemplo, entender as preocupações sobre tarifas pode ajudar a modelar estruturas de preços mais equitativas e considerar alternativas como subsídios para grupos vulneráveis. Portanto, essse tópico social não é somente termômetro das percepções e experiências dos cidadãos, mas também uma bússola para direcionar políticas públicas e investimentos de maneira mais informada e responsiva às necessidades da comunidade.

\subsection{Saúde Pública}
\label{sec:eventos_populares_public_health}

\begin{figure}[htb]
	\centering
	\includegraphics[width=0.7\textwidth]{images/colab_posts_health.png}
	\caption{Exemplo de postagem no Colab relacionada ao tópico de Saúde Pública}
	\label{fig:colab_posts_health}
\end{figure}

A saúde pública é um tópico crucial para as comunidades, como demonstram os dados da \autoref{tab:eventos_populares_public_health}. Palavras-chave como 'samu', 'ambulância', 'médico' e 'vacina' são frequentemente mencionadas, destacando a importância dos serviços de saúde. Termos como 'dengue' e 'zika' apontam para preocupações com doenças emergentes. Questões ambientais, como 'Foco de mosquito da dengue/zika' e 'Descarte irregular de lixo', também são proeminentes, indicando problemas de saúde relacionados ao meio ambiente.

\begin{figure}[htb]
	\centering
	\includegraphics[width=0.7\textwidth]{images/wordcloud_public_health.png}
	\caption{Wordcloud com palavras mais frequentes em postagens sobre Saúde Pública}
	\label{fig:wordcloud_public_health}
\end{figure}

A análise mostra sentimentos majoritariamente negativos em relação a problemas como poluição sonora e descarte irregular de lixo, embora alguns eventos, como 'Atendimento na Clínica da Família', apresentem uma perspectiva mais positiva. Este contraste reflete uma diversidade de opiniões e preocupações na comunidade.

\begin{quadro}[htb]
	\centering
	\includegraphics[width=0.7\textwidth]{images/social_barometer_public_health.png}
	\caption{Gráfico de Radar ilustrando a pressão social em relação ao tópico de Saúde Pública.}
	\label{fig:social_barometer_public_health}
\end{quadro}

Na administração urbana, a análise da saúde publica sob uma perspectiva de pressão social pode ser fundamental para aprimorar os serviços oferecidos à população. Primeiramente, o entendimento das preocupações específicas dos cidadãos, evidenciado pela análise de dados, permite que as autoridades foquem na melhoria dos serviços mais criticados. Por exemplo, a eficiência do Serviço de Atendimento Móvel de Urgência (SAMU) e a disponibilidade de vacinas são aspectos que podem ser significativamente aprimorados com base no feedback coletado. Esse foco direcionado não apenas melhora a qualidade do atendimento em saúde, mas também aumenta a confiança da população nos serviços públicos.

Além disso, a identificação frequente de doenças endêmicas, como dengue e zika, nos dados sugere a necessidade urgente de campanhas eficazes de prevenção e controle de vetores. O planejamento dessas campanhas pode ser otimizado ao se considerar as áreas e questões mais mencionadas pelos cidadãos, garantindo uma ação direcionada e mais eficiente. A conscientização e educação da população acerca dessas doenças também se torna mais focada e impactante.

A análise detalhada dos sentimentos e personas revela as atitudes variadas dos cidadãos, fornecendo informações valiosas para moldar respostas e políticas públicas alinhadas com as expectativas da comunidade. Essa abordagem baseada em dados permite uma resposta mais sensível e adaptada às necessidades específicas da população, resultando em uma administração mais eficaz e empática.

Por fim, o engajamento cívico, incentivado através de plataformas de diálogo entre cidadãos e autoridades, fortalece a participação comunitária na saúde pública. Esta abordagem colaborativa não só contribui para a resolução de problemas, mas também promove uma maior transparência e responsabilidade nas decisões públicas. Além disso, o uso de plataformas digitais para captar a voz da população se destaca como uma ferramenta crucial, permitindo uma gestão pública mais inclusiva e alinhada às necessidades reais da comunidade. Essa interação dinâmica entre cidadãos e gestores é essencial para construir cidades mais resilientes e inclusivas, onde a saúde pública é uma prioridade compartilhada e ativamente perseguida por todos os envolvidos.

\subsection{Distanciamento Social}
\label{sec:eventos_populares_social_distancing}

\begin{figure}[htb]
	\centering
	\includegraphics[width=0.7\textwidth]{images/colab_posts_social_distancing.png}
	\caption{Exemplo de postagem no Colab denunciando aglomeração de pessoas na pandemia.}
	\label{fig:colab_posts_social_distancing}
\end{figure}

A análise dos dados coletados no Colab sobre distanciamento social durante a pandemia de COVID-19 oferece insights valiosos para a administração pública. As discussões refletem a sensibilidade das comunidades à pandemia e a importância da conscientização sobre medidas de saúde, como aglomerações e uso de máscaras. Essas informações são cruciais para a implementação de estratégias de vigilância participativa e detecção digital de doenças. As métricas de pressão sociais foram resumidas na \autoref{tab:eventos_populares_social_distancing}

\begin{figure}[htb]
	\centering
	\includegraphics[width=0.7\textwidth]{images/wordcloud_social_distancing.png}
	\caption{Wordcloud com palavras mais frequentes em postagens sobre Distanciamento Social}
	\label{fig:wordcloud_social_distancing}
\end{figure}

A vigilância participativa, um conceito onde cidadãos contribuem ativamente para o monitoramento de questões de saúde pública, é evidenciada na maneira como os usuários do Colab relatam e reagem a situações relacionadas ao distanciamento social. A predominância de personas do tipo \textit{helper} em eventos como superlotação no transporte público mostra que os cidadãos não apenas identificam problemas, mas também estão dispostos a colaborar na busca de soluções ou demonstram uma certa empatia principalmente no que diz respeito ao distanciamento social em espaços de trânsito em massa. Este tipo de engajamento oferece às autoridades de saúde pública uma fonte valiosa de informações em tempo real, permitindo uma resposta mais rápida e eficiente a situações emergentes.

\begin{quadro}[htb]
	\centering
	\includegraphics[width=0.7\textwidth]{images/social_barometer_social_distancing.png}
	\caption{Gráfico de Radar ilustrando a pressão social em relação ao tópico de Distanciamento Social.}
	\label{fig:social_barometer_social_distancing}
\end{quadro}

Por outro lado, a forte reação negativa em eventos como superlotação de ônibus e a predominância de personas do tipo \textit{complainer} indicam uma alta pressão social e insatisfação com a gestão atual desses problemas. Este feedback direto dos cidadãos é essencial para que as autoridades compreendam as áreas críticas que necessitam de atenção imediata e ajustem suas políticas e práticas em conformidade.

Além disso, a detecção digital de doenças, que envolve a utilização de dados digitais para monitorar e identificar tendências de saúde, é reforçada pela análise desses dados. A expressão de preocupações com aglomerações e condições sanitárias irregulares em estabelecimentos fornece às autoridades um panorama das preocupações de saúde em tempo real. Isso possibilita uma abordagem proativa na gestão de saúde pública, onde medidas preventivas e campanhas de conscientização podem ser direcionadas para áreas e temas específicos identificados através da plataforma.

Ao interpretar esses eventos sob a perspectiva de pressão social oferecemos uma nova dimensão para stakeholders, com o potencial aprimorar as estratégias de saúde pública, mas também capacitar uma resposta mais eficaz às preocupações emergentes, encorajando uma gestão colaborativa em crises de saúde. A interação ativa entre cidadãos e a análise crítica dessas informações estabelecem uma base para uma vigilância participativa eficiente e oferece oportunidades para estratégias de detecção digital de doenças. Essa abordagem não só reforça as medidas de saúde pública, mas também fortalece o bem-estar e a segurança da comunidade, demonstrando o poder da tecnologia e da participação cidadã na melhoria da saúde pública.

\subsection{Mudança Climática}
\label{sec:eventos_populares_weather}

\begin{figure}[htb]
	\centering
	\includegraphics[width=0.7\textwidth]{images/colab_posts_social_clima.png}
	\caption{Exemplo de postagem no Colab denunciando os efeitos do desmatamento na mudança climática.}
	\label{fig:colab_posts_social_clima}
\end{figure}

Ao analisar os eventos do Colab, podemos filtrar por tipos de eventos e palavras-chave específicas ao tópico de mudanças climáticas, revelando como as preocupações cotidianas dos usuários em relação a problemas urbanos refletem uma consciência mais ampla sobre mudança climática. Esta abordagem permite identificar tendências e padrões específicos na percepção pública, ilustrando como as questões locais se entrelaçam com o debate global sobre o clima. Os dados refletem uma ampla variedade de reações a problemas urbanos e ambientais, desde a manutenção de espaços verdes até a gestão da infraestrutura urbana e questões de saúde pública. As métricas apresentadas na \autoref{tab:eventos_populares_weather} destacam a diversidade de eventos e a complexidade das opiniões dos cidadãos em relação às mudanças climáticas e meio ambiente, o que demonstra uma crescente sensibilidade a crises ambientais e a necessidade de ações proativas.

\begin{figure}[htb]
	\centering
	\includegraphics[width=0.7\textwidth]{images/wordcloud_weather.png}
	\caption{Wordcloud com palavras mais frequentes em postagens sobre Mudança Climática}
	\label{fig:wordcloud_weather}
\end{figure}

O interesse significativo da comunidade em eventos proativos, como iniciativas de arborização, apesar de uma participação ativa limitada, sugere um reconhecimento geral da importância de tais ações para a sustentabilidade ambiental. Por outro lado, a forte reação negativa e a postura crítica em relação a problemas como desmatamento irregular e ocupação de áreas públicas destacam uma preocupação profunda com a perda de espaços naturais e a preservação ambiental.

\begin{quadro}[htb]
	\centering
	\includegraphics[width=0.7\textwidth]{images/social_barometer_weather.png}
	\caption{Gráfico de Radar ilustrando a pressão social em relação ao tópico de Mudança Climática.}
	\label{fig:social_barometer_weather}
\end{quadro}

A análise revela uma polarização nas atitudes em relação às mudanças climáticas. Alguns usuários criam eventos que diretamente vinculam as mudanças climáticas como a causa de problemas urbanos, enquanto outros abordam esses problemas como questões de gestão pública ou falta de ação por parte dos agentes governamentais. Essa polarização reflete as diferentes perspectivas dos usuários, onde alguns veem as mudanças climáticas como uma motivação subjacente para os problemas, enquanto outros focam nas questões de governança e gestão pública. Essa diversidade de abordagens pode ser vista como um reflexo das opiniões variadas dos usuários em relação às mudanças climáticas e à forma como elas se manifestam em suas comunidades.

Os resultados apontam para a necessidade de estratégias de comunicação e engajamento mais eficazes para abordar essa divisão. A compreensão dessas dinâmicas é crucial para os formuladores de políticas e gestores públicos, pois permite o desenvolvimento de abordagens mais inclusivas e abrangentes que considerem as diversas perspectivas e preocupações dos cidadãos. Esta análise destaca a importância de incorporar a voz da comunidade no planejamento e na implementação de políticas de mudança climática e gestão ambiental.

\subsection{Paisagismo}
\label{sec:eventos_populares_landscape}

\begin{figure}[htb]
	\centering
	\includegraphics[width=0.7\textwidth]{images/colab_posts_paisagismo.png}
	\caption{Exemplo de postagens no Colab sobre Paisagismo}
	\label{fig:colab_posts_paisagismo}
\end{figure}

O paisagismo e como cidades tratam tópicos como arborização e o gerenciamento de parques e praças não se trata somente de uma preocupação estética; é um elemento intrínseco à qualidade de vida nas cidades, com repercussões significativas em áreas como meio ambiente, mudanças climáticas e no bem-estar da população. A análise dos dados de pressão social hiperlocal revela uma variedade de perspectivas e preocupações dos cidadãos em relação a eventos de paisagismo em suas comunidades. As métricas foram resumidas na \autoref{tab:eventos_populares_landscape} e demonstram que além de se preocuparem com o meio ambiente em um nível macro como alagamentos e ondas de calor, os usuários também estão preocupados com o seu ecossistema local, representado principalmente por demandas de paisagismo.

\begin{figure}[htb]
	\centering
	\includegraphics[width=0.7\textwidth]{images/wordcloud_landscape.png}
	\caption{Wordcloud com palavras mais frequentes em postagens sobre Paisagismo}
	\label{fig:wordcloud_landscape}
\end{figure}

A análise mostra que eventos como a poda de árvores e a retirada de árvores frequentemente geram um debate considerável. A persona associada a esses eventos varia, refletindo uma diversidade de opiniões. Enquanto alguns cidadãos expressam apoio às práticas de manejo da vegetação urbana, destacando a importância da segurança e estética, outros manifestam preocupações sobre o impacto na preservação das áreas verdes e sombra nas cidades. Isso demonstra a complexidade das questões relacionadas à vegetação urbana e destaca a necessidade de políticas e práticas que considerem essa diversidade de perspectivas.

\begin{quadro}[htb]
	\centering
	\includegraphics[width=0.7\textwidth]{images/social_barometer_landscape.png}
	\caption{Gráfico de Radar ilustrando a pressão social em relação ao tópico de Paisagismo}
	\label{fig:social_barometer_landscape}
\end{quadro}

Além disso, eventos como desmatamento irregular e equipamento público danificado recebem críticas significativas. Isso sugere que os cidadãos estão atentos à conservação das áreas verdes urbanas e à manutenção adequada de espaços públicos. Essas preocupações estão intrinsecamente ligadas à qualidade de vida nas cidades, pois afetam a acessibilidade, segurança e o uso desses espaços. Outros eventos, como a ocupação irregular de áreas públicas, demonstram uma persona predominantemente \textit{complainer}, indicando insatisfação com o uso inadequado de espaços urbanos. A falta de calçadas e faixas de pedestres em boas condições também gera críticas, evidenciando a importância da infraestrutura urbana para a mobilidade e segurança dos cidadãos.

A conexão entre paisagismo e questões ambientais, como mudanças climáticas, é evidente em eventos como a poda de árvores e a retirada de árvores. O manejo inadequado da vegetação urbana pode afetar a regulação da temperatura local e a absorção de poluentes, destacando a necessidade de práticas sustentáveis de paisagismo. No entanto, é importante notar que as árvores também podem ter um impacto no fornecimento de energia elétrica. Eventos relacionados à falta de energia e fiação irregular indicam a preocupação dos cidadãos com a confiabilidade da infraestrutura elétrica em espaços públicos. Árvores próximas a fios de energia podem representar um risco, evidenciando a necessidade de equilibrar a preservação da vegetação com a segurança energética.

A análise da pressão social do paisagismo urbano destaca a complexidade das questões envolvidas e a diversidade de perspectivas dos cidadãos. Isso ressalta a importância de uma abordagem aberta e inclusiva na gestão do paisagismo, considerando as expectativas e necessidades da comunidade. Além disso, a conexão entre paisagismo, meio ambiente, mudanças climáticas e energia elétrica destaca a necessidade de políticas e práticas que promovam a sustentabilidade e o bem-estar nas cidades.

\subsection{Meio Ambiente}
\label{sec:eventos_populares_environment}

\begin{figure}[htb]
	\centering
	\includegraphics[width=0.7\textwidth]{images/colab_posts_social_environment.png}
	\caption{Postagem do colab evidenciando o engajamento cívico em questões ambientais.}
	\label{fig:colab_posts_social_environment}
\end{figure}

O meio ambiente é um tópico de extrema importância para as comunidades urbanas em todo o mundo, uma vez que as questões ambientais têm um impacto direto na qualidade de vida dos cidadãos nas cidades. Isso abrange aspectos como saúde pública, acesso a áreas verdes, qualidade do ar e da água, biodiversidade urbana e clima local. A qualidade do ambiente urbano influencia a saúde física e mental dos habitantes das cidades, sendo a poluição do ar, a contaminação da água e a falta de áreas verdes causadoras de problemas de saúde, como doenças respiratórias, alergias, doenças cardiovasculares e estresse. Além disso, a degradação ambiental pode afetar negativamente a economia das cidades, com desvalorização imobiliária e deslocamento de comunidades de baixa renda de áreas afetadas. A preservação de espaços verdes urbanos desempenha um papel importante na regulação do clima e na melhoria da qualidade do ar, e a sustentabilidade a longo prazo das cidades depende de práticas ambientalmente responsáveis, como reciclagem e eficiência energética.

\begin{figure}[htb]
	\centering
	\includegraphics[width=0.7\textwidth]{images/wordcloud_environment.png}
	\caption{Wordcloud com palavras mais frequentes em postagens sobre Meio Ambiente}
	\label{fig:wordcloud_environment}
\end{figure}

\begin{quadro}[htb]
	\centering
	\includegraphics[width=0.7\textwidth]{images/social_barometer_environment.png}
	\caption{Gráfico de Radar ilustrando a pressão social em relação ao tópico de Meio Ambiente}
	\label{fig:social_barometer_environment}
\end{quadro}

Ao analisar as métricas de pressão social resumidas na \autoref{tab:eventos_populares_environment}, observamos uma intersecção de tipos de eventos comuns a crise climática e paisagismo. Em alguns tópicos, como proteção animal, a maioria dos usuários adota uma abordagem colaborativa e positiva, demonstrando preocupação com a fauna e o meio ambiente. No entanto, em eventos relacionados a questões de saneamento básico e conservação da água, os usuários tendem a expressar insatisfação e críticas, refletindo preocupações com problemas como coleta inadequada de lixo e esgoto a céu aberto. A participação ativa em discussões sobre questões de paisagismo e arborização indica um desejo de melhorar as áreas verdes urbanas. Em relação às mudanças climáticas, eventos como incêndios florestais e desmatamento ilegal recebem críticas, indicando preocupações com a degradação ambiental. Além disso, a popularidade de eventos relacionados ao saneamento básico e áreas verdes urbanas sugere que esses problemas são altamente relevantes para a qualidade de vida nas cidades e despertam o interesse dos usuários. No entanto, é importante observar que essas discussões frequentemente refletem insatisfação com os serviços públicos relacionados a esses eventos, destacando a necessidade de melhorias.

Os dados também podem ser utilizados pelos administradores das cidades para promover ações que melhorem a qualidade de vida dos habitantes urbanos e a sustentabilidade ambiental. Primeiramente, a compreensão da diversidade de perspectivas dos usuários em relação a questões ambientais permite que os administradores considerem uma ampla gama de opiniões ao tomar decisões relacionadas ao meio ambiente. Isso pode ajudar a evitar soluções unilaterais e a garantir que as políticas e iniciativas sejam mais abrangentes e eficazes. Por exemplo, ao lidar com problemas de saneamento básico, os administradores podem levar em consideração as críticas e insatisfações expressas pelos usuários e buscar soluções que atendam às suas preocupações.

Além disso, a popularidade de eventos relacionados a questões ambientais indica quais problemas são mais relevantes para a população urbana. Os administradores podem usar essas informações para priorizar recursos e esforços em áreas que afetam significativamente a qualidade de vida das pessoas. Por exemplo, se eventos relacionados ao descarte irregular de lixo são amplamente discutidos, isso pode sinalizar a necessidade de melhorias na coleta de resíduos e na conscientização ambiental. A análise das personas médias dos usuários em diferentes tópicos também é informativa. Por exemplo, em eventos relacionados à proteção animal, a predominância da persona \textit{helper} indica um desejo de colaboração e apoio a iniciativas de preservação da fauna. Os administradores podem aproveitar esse espírito colaborativo para envolver a comunidade em esforços de conservação da vida selvagem e educação ambiental.

Por outro lado, em eventos relacionados a problemas de saneamento básico, onde a persona \textit{complainer} é mais comum, os administradores podem reconhecer a necessidade de abordar preocupações específicas e melhorar os serviços relacionados. Isso pode incluir a implementação de sistemas de coleta de lixo mais eficientes ou ações para evitar o esgoto a céu aberto. Os administradores podem usar essas métricas para identificar oportunidades de melhoria em áreas como paisagismo urbano e mitigação das mudanças climáticas. Por exemplo, ao reconhecer a disposição dos usuários em participar de iniciativas de plantio de árvores, os administradores podem promover programas de arborização urbana que não apenas tornam a cidade mais verde, mas também envolvem a comunidade.

Em resumo, dessas métricas não se limita apenas a fornecer um panorama das opiniões dos usuários, mas também oferece orientações valiosas para os administradores das cidades. Esses dados podem ser usados para tomar decisões informadas, priorizar áreas de atuação e envolver a comunidade de maneira mais eficaz na promoção de um ambiente urbano saudável e sustentável. Portanto, ao compreender as perspectivas e preocupações dos cidadãos, os administradores podem implementar políticas e iniciativas que atendam melhor às necessidades da população e ao meio ambiente.

\subsection{Gentrificação}
\label{sec:eventos_populares_social_gentrification}

\begin{figure}[htb]
	\centering
	\includegraphics[width=0.7\textwidth]{images/colab_posts_social_gentrification.png}
	\caption{Usuário reclama da desvalorização de imóveis causada por mato alto.}
	\label{fig:colab_posts_social_gentrification}
\end{figure}

A gentrificação tem se consolidado como um dos tópicos mais debatidos nas discussões sobre desenvolvimento urbano, desencadeando paixões e preocupações de diferentes partes da população. Dada a sua natureza multidimensional, é imperativo entender as nuances de como a gentrificação é percebida e quais as principais áreas de preocupação. Através da análise de dados de postagens de eventos de zeladoria pública no Colab, podemos criar um tópico de pressão social que nos informa sobre a opinião e sentimento dos usuários apresentado na \autoref{tab:eventos_populares_social_gentrification}.

\begin{figure}[htb]
	\centering
	\includegraphics[width=0.7\textwidth]{images/wordcloud_gentrification.png}
	\caption{Wordcloud com palavras mais frequentes em postagens sobre Gentrificação}
	\label{fig:wordcloud_gentrification}
\end{figure}

\begin{quadro}[htb]
	\centering
	\includegraphics[width=0.7\textwidth]{images/social_barometer_gentrification.png}
	\caption{Gráfico de Radar ilustrando a pressão social em relação ao tópico de Gentrificação}
	\label{fig:social_barometer_gentrification}
\end{quadro}

Os dados revelam uma polarização entre dois grupos de usuários: aqueles que têm interesses econômicos na área, como proprietários de imóveis, que reportam eventos relacionados à desvalorização da propriedade; e aqueles preocupados com os impactos adversos da gentrificação, como despejos, que reportam eventos indicando uma degradação das condições de vida e do acesso a espaços públicos.

Os sentimentos expressos nas postagens também variam, com eventos relacionados à poluição sonora e desmatamento tendo sentimentos predominantemente negativos, enquanto eventos sobre parques e praças recebem avaliações mais positivas. A persona média tende a se inclinar para a categoria \textit{complainer} em muitos tópicos, indicando insatisfação e demanda por melhorias.

\begin{figure}[htb]
	\centering
	\includegraphics[width=0.7\textwidth]{images/colab_posts_social_favelizacao.png}
	\caption{Usuário denuncia ocupaçao irregular de área pública e aluz a favelização.}
	\label{fig:colab_posts_social_favelizacao}
\end{figure}

Os eventos mais populares, como o descarte irregular de lixo e imóveis abandonados, refletem a preocupação com a degradação e falta de zeladoria, sugerindo que a gentrificação não se limita à chegada de novos moradores, mas também envolve negligência e abandono de áreas valorizadas anteriormente.

Os dados revelam tendências importantes, como a crescente preocupação com a insegurança e deterioração das áreas afetadas pela gentrificação, bem como a expulsão de moradores de longa data devido à especulação imobiliária. Além disso, há consciência sobre os fatores econômicos por trás da gentrificação, como especulação imobiliária e privatização.

Essa análise pode ser valiosa para os administradores das cidades ao informar políticas públicas relacionadas à gentrificação. Os dados destacam a necessidade de abordagens equilibradas que considerem tanto o desenvolvimento econômico quanto o bem-estar das comunidades. Políticas que promovam a inclusão, a preservação do patrimônio cultural e a melhoria da infraestrutura podem ajudar a minimizar os efeitos negativos da gentrificação e promover um desenvolvimento urbano mais sustentável.

\subsection{Higienismo Urbano}
\label{sec:eventos_populares_homeland}

\begin{figure}[htb]
	\centering
	\includegraphics[width=0.7\textwidth]{images/colab_posts_higienismo.png}
	\caption{Exemplo de postagens sobre Higienismo Urbano no Colab}
	\label{fig:colab_posts_higienismo}
\end{figure}

Higienismo urbano é um tópico complexo que se relaciona com a maneira como as cidades são projetadas e gerenciadas em busca de uma aparência 'limpa' e 'ordenada'. Em um exame mais profundo do termo, o higienismo urbano é entrelaçado com questões de exclusão e marginalização social. Historicamente, o higienismo foi uma abordagem nas cidades que buscava combater doenças e promover a salubridade por meio da construção de infraestruturas de saneamento e reordenamento urbano. No entanto, em tempos modernos, essa abordagem se estendeu além de questões puramente sanitárias e evoluiu para a promoção de cidades esteticamente agradáveis, muitas vezes à custa de deslocar ou invisibilizar populações vulneráveis.

O higienismo urbano refere-se a práticas e políticas que buscam 'limpar' ou 'embelezar' espaços urbanos através da remoção ou ocultação de populações e condições consideradas 'indesejadas'. Um exemplo notório dessa prática ocorreu durante os preparativos para a Copa do Mundo de 2014 no Brasil. No Rio de Janeiro, para apresentar uma imagem mais 'limpa' aos visitantes internacionais, algumas favelas localizadas em pontos estratégicos como às margens das vias expressas, foram ocultadas por tapumes. Esta tentativa de mascarar a realidade socioeconômica da cidade atraiu críticas significativas, pois, em vez de resolver as questões subjacentes, a medida apenas escondia o problema. Paralelamente, uma das manifestações mais tangíveis do higienismo urbano é o que se convencionou chamar de 'design hostil'. Esta prática arquitetônica visa tornar os espaços públicos intencionalmente desconfortáveis para deter certos grupos, especialmente os sem-teto. Bancos com divisórias que impedem o repouso e picos no chão para desencorajar que se deitem são exemplos comuns dessa abordagem. Ambas as práticas, seja o ocultamento das realidades urbanas ou o design hostil, refletem uma abordagem que prioriza a estética e a ordem em detrimento do bem-estar e dos direitos dos cidadãos.

\begin{figure}[htb]
	\centering
	\includegraphics[width=0.7\textwidth]{images/social_barometer_homeland.png}
	\caption{Gráfico de Radar ilustrando a pressão social em relação ao tópico de Higienismo Urbano}
	\label{fig:social_barometer_homeland}
\end{figure}

O debate sobre o higienismo urbano revela uma polarização marcante. Por um lado, existem aqueles que apoiam ações de 'limpeza' e 'ordem' nas cidades, muitas vezes com uma perspectiva higienista. Por outro lado, há vozes que buscam abordagens mais humanas e inclusivas, procurando soluções reais para problemas sociais em vez de escondê-los ou afastá-los.

Os dados apresentados na \autoref{tab:eventos_populares_homeland} revelam essa divisão. Alguns eventos são percebidos como problemas por aqueles que têm uma atitude mais crítica, enquanto outros são vistos de forma positiva, refletindo o valor atribuído a espaços públicos bem conservados e serviços eficientes.

A polarização também se reflete na linguagem usada pelos usuários. Enquanto alguns expressam empatia ao mencionar 'moradores de rua', outros usam termos pejorativos, indicando a dicotomia na percepção pública. No entanto, há uma preocupação genuína com as populações vulneráveis, visto que problemas como 'Calçada inexistente' e 'Iluminação pública irregular' afetam a segurança e a acessibilidade de todos os cidadãos, especialmente os mais vulneráveis.

\begin{figure}[htb]
	\centering
	\includegraphics[width=0.7\textwidth]{images/wordcloud_homepand.png}
	\caption{Wordcloud com palavras mais frequentes em postagens sobre Higienismo Urbano}
	\label{fig:wordcloud_homepand}
\end{figure}

A dicotomia nos resultados demonstra um desafio para as cidades modernas: equilibrar o desejo por ordem e estética com a necessidade de justiça social e inclusão. O higienismo urbano e o design hostil são duas faces dessa moeda, e o conteúdo gerado por usuários de aplicativos como o Colab oferecem um vislumbre das opiniões e preocupações do público a respeito dessas questões.

Esses dados podem informar políticas públicas ao destacar a necessidade de um equilíbrio entre a busca por uma cidade mais ordenada e esteticamente agradável e a garantia de justiça social e inclusão. Administradores de cidades podem usar essas informações para desenvolver estratégias que abordem as preocupações legítimas dos cidadãos, ao mesmo tempo em que promovem a igualdade e o bem-estar de todos.

\subsection{Segurança Pública}
\label{sec:eventos_populares_security}

\begin{figure}[htb]
	\centering
	\includegraphics[width=0.7\textwidth]{images/colab_posts_security.png}
	\caption{Exemplo de um evento do Colab denunciando as inseguranças sentidas pelo usuário.}
	\label{fig:colab_posts_security}
\end{figure}

O cenário urbano brasileiro passou por transformações nas últimas décadas, especialmente na área de segurança pública, um tópico sensível e debatido. Nesse contexto, surgiram polarizações, principalmente entre defensores de abordagens repressivas e apoiadores de políticas sociais. Essas divisões muitas vezes refletem as inclinações políticas dos indivíduos e são amplificadas por líderes políticos e mídia.

\begin{figure}[htb]
	\centering
	\includegraphics[width=0.7\textwidth]{images/wordcloud_security.png}
	\caption{Wordcloud com palavras mais frequentes em postagens sobre Segurança Pública}
	\label{fig:wordcloud_security}
\end{figure}

A alta incidência de assaltos e roubos, juntamente com um tom predominantemente negativo nas discussões, reflete uma preocupação significativa com a criminalidade. No entanto, a presença de ajudantes é menos notável, indicando que esses problemas podem ser vistos como desafiadores demais para ações comunitárias, exigindo intervenção governamental.

\begin{quadro}[htb]
	\centering
	\includegraphics[width=0.7\textwidth]{images/social_barometer_security.png}
	\caption{Gráfico de Radar ilustrando a pressão social em relação ao tópico de Segurança Pública.}
	\label{fig:social_barometer_security}
\end{quadro}

A análise dos sentimentos e personas revela padrões interessantes. A \autoref{tab:eventos_populares_security} demonstra que eventos relacionados à poluição sonora e maus tratos a animais geram sentimentos negativos e são associados principalmente à persona \textit{complainer}. Por outro lado, eventos como vandalismo em bicicletários recebem reações positivas, sugerindo um engajamento positivo da comunidade.

No que diz respeito ao patrimônio público, há uma forte preocupação com a preservação do patrimônio histórico, refletida por uma alta incidência de personas \textit{complainer}. Isso destaca a valorização cultural e histórica da comunidade. No entanto, problemas estéticos, como pintura, geram reações menos intensas.

A análise revela uma polarização significativa na discussão sobre segurança pública, com predominância de reclamações e críticas. No entanto, também mostra um grupo ativo disposto a colaborar na identificação de soluções, principalmente para crimes contra a estrutura pública e a vida.

Esses dados indicam uma oportunidade para as autoridades locais capitalizarem o engajamento ativo da comunidade, direcionando-o para iniciativas colaborativas de segurança e melhorias urbanas. No entanto, também representam um desafio em termos de gestão e resposta às expectativas dos cidadãos, exigindo uma abordagem equilibrada que inclua todas as vozes da comunidade em um diálogo construtivo. Essas informações podem ser valiosas para informar políticas públicas relacionadas à segurança pública nas cidades.

A conclusão que emerge desses dados é a de uma comunidade ativamente engajada nas questões de segurança pública, mas cujo engajamento é predominantemente orientado para a expressão de preocupações e demandas por ação. Isso sinaliza uma oportunidade para os tomadores de decisão e as autoridades locais de capitalizar sobre essa participação ativa, canalizando-a para iniciativas colaborativas de segurança e melhorias urbanas. Ao mesmo tempo, revela um desafio significativo em termos de gestão e resposta às expectativas dos cidadãos, exigindo uma abordagem que equilibre a resposta imediata às preocupações com a promoção de um diálogo construtivo que inclua todas as vozes da comunidade.

\subsection{Política de Drogas}
\label{sec:eventos_populares_drugs}

\begin{figure}[htb]
	\centering
	\includegraphics[width=0.7\textwidth]{images/colab_posts_drugs.png}
	\caption{Exemplo de postagem de um usuário denunciando um terreno abandonado usado para consumo de drogas.}
	\label{fig:colab_posts_drugs}
\end{figure}

A questão das drogas nas cidades do Brasil é complexa e abrange diversos aspectos, como saúde pública, segurança e questões sociais, econômicas e políticas. A política de drogas gera polarização, com diferentes perspectivas, desde abordagens mais focadas na prevenção e tratamento até políticas mais rígidas de repressão. As métricas de pressão social sobre o tema disponíveis na \autoref{tab:eventos_populares_drugs} não possui tantos eventos quantos outros tópicos, mas ainda podem iluminar o discurso sobre drogas no Colab principalmente ao analisarmos o sentimento das postagens.

\begin{figure}[htb]
	\centering
	\includegraphics[width=0.7\textwidth]{images/wordcloud_drugs.png}
	\caption{Wordcloud com palavras mais frequentes em postagens sobre Política de Drogas}
	\label{fig:wordcloud_drugs}
\end{figure}

Ao analisar como os usuários do aplicativo Colab percebem o problema das drogas, observamos essa polarização. Alguns veêm o uso de drogas como um problema de saúde pública, destacando a necessidade de tratamento e apoio aos dependentes químicos. Outros o enxergam mais como uma questão de segurança pública, associando-o à criminalidade e degradação urbana. A forma como os usuários do Colab se referem aos usuários de drogas também revela essa polarização, com palavras negativas e estigmatizantes sendo usadas. Isso pode refletir uma tendência de responsabilizar o indivíduo pelo uso de drogas, em vez de considerar o contexto social e econômico mais amplo.

\begin{quadro}[htb]
	\centering
	\includegraphics[width=0.7\textwidth]{images/social_barometer_drugs.png}
	\caption{Gráfico de Radar ilustrando a pressão social em relação ao tópico de Política de Drogas.}
	\label{fig:social_barometer_drugs}
\end{quadro}

Há também uma associação entre problemas de zeladoria pública e menções a drogas, sugerindo que áreas abandonadas ou negligenciadas podem se tornar locais propícios para o consumo e tráfico de drogas. Essa dinâmica destaca a necessidade de uma abordagem holística na gestão urbana, que não apenas aborde o problema das drogas, mas também as condições sociais e urbanas que o favorecem.

Para os administradores das cidades, esses dados podem ser valiosos na formulação de políticas públicas relacionadas às drogas. Eles devem considerar a necessidade de abordagens que abrangem tanto questões de saúde pública quanto de segurança, bem como investimentos em infraestrutura e revitalização de áreas degradadas. Uma abordagem integrada que leve em conta as percepções e experiências dos cidadãos, juntamente com os fatores estruturais e sociais é essencial para enfrentar o problema das drogas nas cidades.

\subsection{Pânico Moral}
\label{sec:eventos_populares_moral_panic}

\begin{figure}[htb]
	\centering
	\includegraphics[width=0.7\textwidth]{images/colab_posts_karen.png}
	\caption{Exemplo de postagem de um usuário denunciando um evento de poluição sonora e associando-o a um gênero musical.}
	\label{fig:colab_posts_karen}
\end{figure}

O conceito de 'pânico moral', conforme delineado por Stanley Cohen em sua obra seminal 'Folk Devils and Moral Panics', fornece um quadro interpretativo valioso para compreender como certas questões sociais e comportamentos são amplificados e distorcidos dentro do discurso público, levando à criação de 'bodes expiatórios' ou 'demônios sociais'. No contexto brasileiro contemporâneo, e especialmente através de plataformas interativas como o Colab, podemos observar a manifestação desse fenômeno de forma peculiar. Cohen descreve o pânico moral como uma reação da sociedade a um grupo ou subcultura percebida como ameaçadora aos valores e interesses sociais estabelecidos \cite{2002_Cohen_BOOK}. Essa dinâmica é frequentemente caracterizada por uma distorção e exagero na percepção da ameaça. No Colab, essas expressões de pânico moral podem ser particularmente reveladoras, proporcionando uma janela para entender como certas preocupações são moldadas e amplificadas no espaço digital. Os dados da \autoref{tab:eventos_populares_moral_panic} apesar de limitados em quantidade, podem iluminar um pouco sobre o discurso dos usuários principalmente relacionados a certos tipos de demandas. 

Por exemplo, reclamações sobre poluição sonora, especialmente relacionadas a gêneros musicais como o funk e o pagode, podem refletir não apenas preocupações legítimas com o ruído, mas também preconceitos culturais e sociais. Isso se alinha com a noção de Cohen sobre a criação de 'bodes expiatórios', onde determinados estilos musicais e seus apreciadores são estigmatizados, representando mais do que uma mera perturbação sonora, mas sim uma ameaça à ordem e aos valores sociais.

\begin{figure}[htb]
	\centering
	\includegraphics[width=0.7\textwidth]{images/wordcloud_moral.png}
	\caption{Wordcloud com palavras mais frequentes em postagens sobre Pânico Moral}
	\label{fig:wordcloud_moral}
\end{figure}

Além disso, o uso de termos pejorativos e discriminatórios em postagens relativas a eventos de zeladoria pública ou questões LGBTQIA+ pode sinalizar a presença de um pânico moral. Nesse contexto, grupos ou comportamentos são retratados de maneira distorcida, contribuindo para a disseminação de estereótipos e a marginalização de minorias. A presença de tais termos no Colab sugere que essas atitudes não estão restritas a esferas privadas ou conversas informais, mas se infiltram em discussões públicas sobre a gestão urbana e social.

A análise desses padrões de comunicação no Colab pode fornecer insights valiosos sobre as tensões subjacentes na sociedade brasileira. Permite aos pesquisadores e formuladores de políticas entender melhor não apenas as preocupações práticas dos cidadãos, mas também as dinâmicas psicossociais e culturais que moldam a percepção pública de certos grupos e questões. Ao identificar e analisar manifestações de pânico moral na plataforma, é possível adotar uma abordagem mais informada e matizada para enfrentar tanto as questões práticas da vida urbana quanto os desafios mais amplos de integração social e tolerância.

Ao analisar os resultados de pressão social, encontramos evidências claras de uma polarização relacionada à música, especialmente ao pagode e ao funk. Essa polarização é evidenciada pelos eventos de 'Ponto recorrente de poluição sonora' e 'Emissão de fumaça preta'. Ambos os eventos têm uma persona média predominantemente \textit{complainer}. Isso sugere que os usuários tendem a adotar uma abordagem crítica e insatisfeita em relação a eventos de poluição sonora, possivelmente devido ao impacto negativo do barulho, que frequentemente está associado a festas de pagode e funk. Essa conexão entre música e eventos de zeladoria pública demonstra um claro preconceito musical e uma polarização em relação a esses gêneros, que são percebidos negativamente por alguns membros da comunidade.

\begin{figure}[htb]
	\centering
	\includegraphics[width=0.7\textwidth]{images/colab_posts_moral_panic.png}
	\caption{Usuário denuncia ocupação irregular de área pública como degeneração urbana.}
	\label{fig:colab_posts_moral_panic}
\end{figure}

Além disso, as palavras-chave identificadas, como 'narguile', 'tabacaria' e 'grafite', sugerem que os usuários também estão associando comportamentos como fumar narguilé e grafites a eventos de zeladoria pública. Essas associações parecem ser negativas, pois esses eventos têm persona média predominantemente \textit{complainer}, refletindo a preocupação ou descontentamento dos usuários em relação a esses comportamentos. Essa conexão entre eventos culturais e preocupações com a zeladoria pública indica que a plataforma está sendo usada para expressar opiniões e críticas sobre questões culturais e comportamentais, além das questões de infraestrutura tradicionais.

Outro aspecto importante a ser destacado é a associação entre eventos de zeladoria pública, como 'Fiação irregular', 'Ocupação irregular de área pública' e 'Construção irregular', e os eventos culturais mencionados anteriormente, como bailes de funk e pagode. Os usuários parecem se incomodar com o barulho e a informalidade desses eventos e, como resultado, criam eventos de zeladoria pública para relatar irregularidades e tentar fechar esses estabelecimentos. Essa estratégia sugere uma tentativa de utilizar as ferramentas disponíveis na plataforma para influenciar o fechamento desses locais, refletindo uma tentativa de impor normas sociais e urbanas.

\begin{quadro}[htb]
	\centering
	\includegraphics[width=0.7\textwidth]{images/social_barometer_moral.png}
	\caption{Gráfico de Radar ilustrando a pressão social em relação ao tópico de Pânico Moral.}
	\label{fig:social_barometer_moral}
\end{quadro}

A associação entre eventos culturais e eventos de zeladoria pública também é evidente nas métricas de pressão social. Os eventos de 'Comércio aberto irregularmente', 'Aglomeração de pessoas' e 'Evento Irregular' têm persona média equilibrada entre \textit{helper} e \textit{complainers}. Isso indica uma diversidade de opiniões em relação a esses eventos, possivelmente refletindo a complexidade do tópico e as diferentes perspectivas dos usuários. Essa diversidade sugere que, enquanto alguns usuários estão preocupados com a infração das regras e o impacto negativo na comunidade, outros podem ver esses eventos como oportunidades culturais ou econômicas.

Além disso, a questão da informalidade na organização desses eventos culturais, como mencionado com bailes de funk e pagode, é uma preocupação subjacente. Os eventos de 'Imóvel ou terreno abandonado' e 'Patrimônio histórico em risco' também refletem uma persona média predominantemente \textit{complainer}. Isso sugere que os usuários frequentemente expressam insatisfação em relação à falta de preservação do patrimônio histórico e à presença de imóveis ou terrenos abandonados, possivelmente relacionando essas questões à informalidade e ao descuido urbanos.

No entanto, é importante notar que, apesar da polarização evidente em relação a eventos culturais e comportamentais, outros eventos, como 'Lixeira quebrada', 'Descarte irregular de lixo' e 'Poda de árvore', têm persona média mais equilibrada, sugerindo que os usuários tendem a adotar uma postura mais colaborativa e positiva em relação a essas questões de infraestrutura.

Em relação à questão da polarização, os resultados indicam que a música, especialmente os gêneros de pagode e funk, parece ser um ponto sensível que gera opiniões polarizadas. Isso pode ser atribuído a uma série de fatores, incluindo preconceito musical, percepções culturais e experiências pessoais. Além disso, a associação de comportamentos como fumar narguilé e grafites a eventos de zeladoria pública demonstra como as preocupações urbanas podem ser interligadas a aspectos culturais e comportamentais.

A questão da informalidade na organização desses eventos culturais também é notável. Os usuários parecem preocupados com a falta de regulamentação e controle em torno desses eventos, o que os leva a relatar problemas de infraestrutura e irregularidades para tentar influenciar o fechamento desses estabelecimentos. Isso pode refletir uma tentativa de impor normas sociais e urbanas por meio da plataforma Colab. Além disso, é importante destacar a relevância do evento 'Ocupação irregular de área pública', que possui uma persona média predominantemente \textit{helper}. Isso sugere que os usuários estão ativamente envolvidos em relatar ocupações irregulares de áreas públicas e trabalhar para resolver esse problema. Essa é uma indicação positiva de participação cidadã na plataforma.

A análise das métricas de pressão social no Colab revela não apenas a polarização significativa em relação a eventos culturais e comportamentos associados, mas também lança luz sobre a existência de um fenômeno que pode ser caracterizado como 'pânico moral'. O pânico moral ocorre quando a sociedade reage de forma exagerada e moralista a determinados comportamentos ou eventos, muitas vezes atribuindo a eles uma ameaça à ordem social e aos valores tradicionais. Nesse contexto, a polarização em torno de eventos culturais como festas de pagode e funk, bem como comportamentos como fumar narguilé e grafites, reflete não apenas as preferências individuais, mas também preconceitos musicais e percepções culturais profundamente enraizadas. Os usuários do Colab estão expressando suas opiniões de forma polarizada, com alguns adotando uma postura crítica e outros buscando colaborar na resolução dessas questões.

A análise das personas médias e scores médios é fundamental para chegarmos a essa conclusão. A persona média nos ajuda a entender como os usuários se comportam e se relacionam com os eventos de zeladoria pública, enquanto o score médio de sentimento nos fornece uma medida objetiva das opiniões expressas. A combinação dessas métricas permite uma compreensão mais completa das dinâmicas sociais e das percepções dos usuários. Essas informações têm implicações importantes para stakeholders, como governos municipais e organizações da sociedade civil. Primeiramente, eles podem usar essas análises para compreender as preocupações e perspectivas dos cidadãos em relação a eventos de zeladoria pública específicos. Isso pode ajudar na formulação de políticas e estratégias mais alinhadas com as necessidades da comunidade.

A identificação do pânico moral em torno de certos tópicos culturais pode levar a esforços educacionais e de conscientização. Os stakeholders podem trabalhar para promover um diálogo mais informado e inclusivo sobre esses temas, reduzindo a polarização e fomentando uma compreensão mútua. Em conclusão, a análise das métricas de pressão social no Colab não apenas oferece insights valiosos sobre as preocupações urbanas e sociais, mas também sugere oportunidades para a construção de comunidades mais informadas e colaborativas, onde as divergências são tratadas com empatia e compreensão.

\subsection{Política Tributária}
\label{sec:eventos_populares_taxes}

\begin{figure}[htb]
	\centering
	\includegraphics[width=0.7\textwidth]{images/colab_posts_taxes.png}
	\caption{Usuário do Colab denunciando problema de esgoto e fazendo referência a impostos.}
	\label{fig:colab_posts_taxes}
\end{figure}

No contexto das discussões sobre questões urbanas, observamos a presença recorrente de argumentos que abordam a questão dos altos impostos municipais e estaduais, assim como os custos associados aos serviços públicos oferecidos à população. Esses argumentos frequentemente destacam a relação intrínseca entre a carga tributária e a qualidade dos serviços públicos. Além disso, os usuários também levantam questões relacionadas à corrupção e à má gestão dos recursos públicos, que podem ser diretamente associadas à política tributária. A análise dos eventos relacionados a essa temática resumidos na \autoref{tab:eventos_populares_taxes}, revela uma tendência marcante: a maioria deles exibe scores médios negativos. Essa tendência sugere que os usuários tendem a expressar sentimentos predominantemente negativos em relação a eventos vinculados a impostos, como o IPTU e IPVA. Essa predominância de sentimentos negativos pode indicar um profundo descontentamento em relação à carga tributária vigente ou à forma como os impostos são administrados.

É importante notar que a conexão direta entre o aumento de impostos e a piora dos serviços públicos pode ser simplista. A relação entre carga tributária e qualidade dos serviços é complexa e influenciada por vários fatores, como gestão pública, alocação de recursos e eficiência administrativa. Portanto, é fundamental adotar uma abordagem mais abrangente para compreender essa interação.

\begin{figure}[htb]
	\centering
	\includegraphics[width=0.7\textwidth]{images/wordcloud_taxes.png}
	\caption{Wordcloud com palavras mais frequentes em postagens sobre Política Tributária}
	\label{fig:wordcloud_taxes}
\end{figure}

Nesse contexto, investigar as dinâmicas de pressão social relacionadas à política tributária e às questões urbanas, a fim de compreender melhor como os usuários do Colab se engajam nesse debate e como suas opiniões e sentimentos se manifestam. Ao explorar os eventos selecionados e as métricas de pressão social associadas a eles, buscamos identificar tendências, padrões e possíveis polarizações nas discussões sobre impostos e serviços públicos. Notavelmente, constatamos que, quando a persona dos usuários é identificada como \textit{complainer}, denotando uma tendência a expressar reclamações ou insatisfações individualistas, e o score de sentimento da postagem é majoritariamente negativo muitos usuários fazem referência a termos tributários, como IPTU, impostos e IPVA, em suas postagens. Essa observação sugere que, em grande parte, os usuários utilizam os impostos como justificativa para expressar seu descontentamento em relação à qualidade dos serviços públicos, argumentando que, considerando a quantidade de impostos pagos, os serviços deveriam ser de melhor qualidade. Essa análise crítica permitirá uma visão mais informada sobre as preocupações dos cidadãos em relação à política tributária e contribuirá para um diálogo mais substancial e equilibrado sobre esse tema de relevância pública.

\begin{quadro}[htb]
	\centering
	\includegraphics[width=0.7\textwidth]{images/social_barometer_taxes.png}
	\caption{Gráfico de Radar ilustrando a pressão social em relação ao tópico de Política Tributária.}
	\label{fig:social_barometer_taxes}
\end{quadro}

Para uma análise mais detalhada dos resultados do Barômetro de Pressão Social relacionados à política tributária, procedemos à categorização dos eventos em seis áreas distintas: transporte público, infraestrutura e patrimônio público, meio ambiente, saneamento básico, iluminação e energia, e saúde pública. Cada uma dessas categorias abrange eventos que foram mencionados pelos usuários e apresenta uma perspectiva única sobre como a política tributária é percebida em relação a diferentes serviços públicos e questões urbanas. As avaliações de persona e score foram realizadas para cada categoria, permitindo uma compreensão mais profunda das preocupações e sentimentos dos usuários em relação a esses tópicos específicos. A seguir, exploraremos os resultados dessas categorias e suas implicações em relação à percepção pública sobre impostos e serviços públicos.

No que diz respeito ao transporte público, os eventos relacionados a problemas como 'ônibus superlotado' 'ponto de ônibus danificado' e 'ônibus fora do horário/rota' exibem persona média próxima \textit{complainer} e scores médios negativos. Isso sugere que os usuários tendem a reclamar dessas questões, expressando sentimentos predominantemente negativos. Essas reclamações podem estar relacionadas aos altos impostos pagos pelos cidadãos. Os usuários podem argumentar que, dado o montante de impostos que pagam, esperam um serviço de transporte público de maior qualidade. Portanto, há uma correlação entre reclamações sobre transporte público e insatisfação com a política tributária, na medida em que os impostos podem ser vistos como financiadores dos serviços de transporte.

Quando examinamos eventos relacionados à infraestrutura e patrimônio público, como 'buraco nas vias' 'estação de ônibus/trem/metrô danificada' 'fiação irregular' e 'patrimônio histórico em risco' observamos uma persona média próxima \textit{complainer} e scores médios negativos. Isso indica que os usuários tendem a reclamar dessas questões e expressar sentimentos predominantemente negativos. É interessante notar que, nesse contexto, a reclamação não está diretamente relacionada aos impostos, mas sim à qualidade dos serviços públicos e à conservação da infraestrutura. No entanto, há uma conexão indireta com a política tributária, uma vez que os cidadãos podem questionar a alocação de recursos e a eficácia do gasto público, considerando os impostos que pagam.

Ao examinar eventos de Meio Ambiente, como 'Vazamento de água' e 'Mato alto', apresentam scores médios negativos, indicando uma tendência de sentimentos negativos entre os usuários. No entanto, a persona média é próxima de 1, o que sugere uma inclinação mais forte para o perfil \textit{complainer}. Nesse contexto, os usuários podem estar insatisfeitos com a gestão dos recursos públicos e, possivelmente, associando essa insatisfação aos impostos pagos.

Os dados podem ser úteis para administradores de cidades ao informar políticas públicas relacionadas ao tema. Eles devem considerar não apenas a política tributária, mas também outros fatores que afetam a prestação de serviços públicos de qualidade. Além disso, é crucial evitar generalizações simplistas que levem a discursos polarizados e à disseminação de informações imprecisas. Ao analisar criticamente as questões urbanas e a política tributária, os usuários podem contribuir de maneira mais construtiva para o debate público e para a formulação de políticas que busquem a melhoria da qualidade de vida nas cidades.

\subsection{Corrupção}
\label{sec:eventos_populares_corruption}

\begin{figure}[htb]
	\centering
	\includegraphics[width=0.7\textwidth]{images/colab_posts_corruption.png}
	\caption{Exemplo de postagem de um usuário levantando suspeitas de corrupção em um evento de zeladoria pública.}
	\label{fig:colab_posts_corruption}
\end{figure}

A corrupção é um tema recorrente nas discussões urbanas no Brasil, ligada a problemas como infraestrutura precária e serviços públicos deficientes. No Colab, observa-se uma dinâmica polarizada entre aqueles que denunciam a corrupção como causa desses problemas. A análise das métricas de pressão social disponíveis na \autoref{tab:eventos_populares_corruption} revela nuances importantes nessa dinâmica.

\begin{figure}[htb]
	\centering
	\includegraphics[width=0.7\textwidth]{images/wordcloud_corruption.png}
	\caption{Wordcloud com palavras mais frequentes em postagens sobre Corrupção}
	\label{fig:wordcloud_corruption}
\end{figure}

Usuários frequentemente mencionam palavras-chave como 'corrupção', 'propina' e 'fraude' ao discutir questões urbanas. Isso mostra que a corrupção está no centro das preocupações dos cidadãos. Eles relacionam casos de corrupção a problemas como obras de infraestrutura malfeitas e serviços públicos deficientes.

\begin{quadro}[htb]
	\centering
	\includegraphics[width=0.7\textwidth]{images/social_barometer_corruption.png}
	\caption{Gráfico de Radar ilustrando a pressão social em relação ao tópico de Corrupção.}
	\label{fig:social_barometer_corruption}
\end{quadro}

Ao analisar eventos urbanos específicos, vemos como os usuários percebem a corrupção em seu contexto. Alguns eventos têm uma persona predominantemente \textit{complainer}, como o 'Ponto de infração de trânsito recorrente', onde os usuários denunciam infrações ligadas à corrupção no sistema de fiscalização. Outros eventos, como 'Foco de mosquito da dengue/zika' e 'Comércio aberto irregularmente', também têm personas \textit{complainer}, mas com scores médios positivos, indicando que os usuários estão dispostos a oferecer soluções construtivas.

Por outro lado, eventos como 'Bloqueio na via' e 'Via de terra com desnível' têm personas \textit{complainer} com scores médios negativos, indicando alta insatisfação e falta de confiança nas autoridades em relação a essas questões. Os usuários acreditam que a corrupção contribui para a má gestão da infraestrutura.

Em resumo, a análise dos eventos relacionados à corrupção destaca a polarização desse tópico no Colab. Para os administradores das cidades, esses dados podem informar políticas públicas de diversas maneiras. Eles podem considerar a necessidade de promover transparência e responsabilidade governamental para atender às preocupações dos cidadãos. Além disso, podem buscar envolver os usuários na busca por soluções construtivas e na fiscalização das questões urbanas. Uma abordagem abrangente que leve em conta as diferentes perspectivas dos cidadãos é essencial para lidar com a corrupção e seus efeitos na vida urbana.

\subsection{Eleições e Políticos}
\label{sec:eventos_populares_polititians}

\begin{figure}[htb]
	\centering
	\includegraphics[width=0.7\textwidth]{images/colab_posts_polititians.png}
	\caption{Exemplo de postagem ironizando políticos ao fazer referência a um evento de zeladoria pública.}
	\label{fig:colab_posts_polititians}
\end{figure}

As discussões sobre eleições e políticos no contexto dos problemas urbanos no Brasil refletem uma interseção significativa entre a política e as questões que afetam diretamente a vida dos cidadãos. No Colab, um espaço digital onde os cidadãos identificam e discutem problemas nas cidades, essa conexão se torna evidente. Os usuários não apenas abordam questões relacionadas à infraestrutura e serviços públicos, mas também vinculam essas questões ao cenário político do país. Esse ambiente propicia debates intensos e frequentemente polarizados.

\begin{figure}[htb]
	\centering
	\includegraphics[width=0.7\textwidth]{images/wordcloud_polititians.png}
	\caption{Wordcloud com palavras mais frequentes em postagens sobre Eleições e Políticos}
	\label{fig:wordcloud_polititians}
\end{figure}

As métricas de pressão social da \autoref{tab:eventos_populares_polititians} demonstra alguns pontos específicos onde os usuários parecem atribuir demandas da cidade a políticos ou partidos específicos. Enquanto alguns expressam insatisfação com impostos elevados e tarifas onerosas, outros conectam os problemas urbanos às promessas e ações de políticos, destacando a ineficiência e questões políticas subjacentes. Essa polarização reflete as divisões políticas na sociedade brasileira e contribui para debates intensos sobre como a política afeta a qualidade de vida nas cidades.

Os usuários frequentemente mencionam políticos de diferentes partidos, refletindo a diversidade política da cidade. As discussões em torno de políticos e partidos podem ser calorosas e estão diretamente ligadas a questões urbanas, como infraestrutura e serviços públicos. Durante períodos eleitorais, como as eleições para prefeito e vereador, a pressão social em torno desse tópico aumenta significativamente. Os eleitores discutem candidatos, suas propostas e o histórico de mandatos anteriores. Eles também compartilham informações sobre como registrar seus votos, destacando a importância da participação cívica. A pressão social relacionada a políticos e eleições é frequentemente acompanhada de debates sobre o desempenho dos vereadores e prefeitos atuais, destacando a centralidade desses representantes municipais para os cidadãos.

\begin{quadro}[htb]
	\centering
	\includegraphics[width=0.7\textwidth]{images/social_barometer_polititians.png}
	\caption{Gráfico de Radar ilustrando a pressão social em relação ao tópico de Eleições e Políticos.}
	\label{fig:social_barometer_polititians}
\end{quadro}

A análise da pressão social relacionada ao tópico revela uma dinâmica complexa e polarizada. Alguns eventos atraem uma postura mais positiva e colaborativa (\textit{helper}), enquanto outros geram críticas intensas e insatisfação (\textit{complainer}). Esses resultados destacam que os usuários têm uma postura diversificada em relação aos problemas urbanos e que a polarização é uma característica proeminente dessas discussões.

\begin{figure}[htb]
	\centering
	\includegraphics[width=0.7\textwidth]{images/colab_posts_polititians_2.png}
	\caption{Exemplo de postagem de usuário criticando políticos.}
	\label{fig:colab_posts_polititians_2}
\end{figure}

Para administradores das cidades, esses dados podem ser valiosos para informar políticas públicas relacionadas a problemas urbanos e políticos. É importante considerar a diversidade de perspectivas e a polarização ao desenvolver estratégias para lidar com questões como infraestrutura, serviços públicos e representação política. Além disso, durante os períodos eleitorais, é fundamental promover o engajamento cívico e a transparência para fortalecer a confiança dos cidadãos nas instituições políticas. Uma abordagem equilibrada e inclusiva é essencial para encontrar soluções eficazes para os desafios urbanos.

\section{Mapeando a Opinião Pública}

O experimento de análise das discussões no Colab nos trouxe valiosos insights sobre a diversidade de perspectivas e abordagens dos usuários em relação a uma variedade de tópicos urbanos. Ao examinar os valores de persona média e score médio em eventos relacionados a pressão social, como paisagismo, meio ambiente, gentrificação, higienismo social, segurança pública, drogas, pânico moral, política tributária, corrupção e políticos, pudemos identificar tendências interessantes que lançam luz sobre a dinâmica das discussões na plataforma.

Primeiramente, ficou evidente que não existe um único tipo de evento com uma persona predominante \textit{helper} ou \textit{complainer}. A distribuição de personas e scores varia amplamente entre os eventos, refletindo a complexidade e a diversidade das questões urbanas. Isso nos leva a concluir que a comunidade do Colab aborda diferentes problemas com uma ampla gama de atitudes, desde a busca ativa por soluções até a expressão de críticas construtivas ou negativas.

Além disso, a análise revelou que mesmo nas discussões em que predominam as personas \textit{helper} ainda podem surgir críticas construtivas, e nas discussões com predominância de personas \textit{complainer} ainda podem existir elementos construtivos. Isso sugere que os usuários estão dispostos a considerar diferentes perspectivas e contribuir para melhorias, independentemente de sua atitude inicial em relação ao problema.

Quanto à polarização no Colab, observamos que, embora as discussões possam ser críticas e até mesmo negativas em alguns casos, a maioria dos usuários parece estar comprometida em abordar e resolver os desafios urbanos. A polarização pode estar presente em debates políticos e em tópicos sensíveis, como corrupção e gentrificação, mas a presença significativa de personas \textit{helper} indica uma disposição para encontrar soluções construtivas, mesmo em meio a divergências.

\begin{quadro}[htb]
	\centering
	\includegraphics[width=0.7\textwidth]{images/network_niteroi_personas_plot.png}
	\caption{Mobilidade Urbana - Helpers vs. Complainers em Niterói.}
	\label{fig:network_niteroi_personas_plot}
\end{quadro}

É interessante observar como uma abordagem de engenharia de software pode ser aplicada a um problema social complexo como a análise de pressão social hiperlocal no Colab. As heurísticas desenvolvidas no início do capítulo desempenham um papel crucial na criação de métricas quantitativas que permitem medir a opinião média e as personas dos usuários em relação a diferentes tipos de eventos de zeladoria pública. Além disso, essas heurísticas fornecem uma base sólida para a criação de representações visuais, como o gráfico de radar, que ajudam a visualizar as dinâmicas sociais de forma acessível.

A coleta de dados estruturados, através do desenvolvimento de um modelo de dados bem definido, é essencial para organizar informações relevantes, proporcionando uma compreensão clara dos elementos essenciais para a análise. Além disso, a aplicação de algoritmos de classificação e processamento de linguagem natural, como o uso de aprendizado de máquina para classificar usuários em personas e atribuir scores de sentimento às postagens, automatiza análises complexas de texto, economizando tempo e recursos.

A criação de representações visuais, como o gráfico de radar, desempenha um papel crucial na comunicação de insights complexos de maneira acessível. Essas visualizações destacam padrões e tendências, facilitando a tomada de decisões informadas. A abordagem de engenharia de software também permite a iteração e o aprendizado contínuo, possibilitando melhorias constantes no sistema com base em resultados e em evoluções nas dinâmicas sociais.

A integração eficaz das análises e insights com os tomadores de decisão, como agências governamentais e organizações da sociedade civil, é fundamental para garantir que os resultados sejam aplicados para informar políticas e ações práticas. Essa abordagem colaborativa promove uma governança mais informada e eficaz, aproveitando a tecnologia para compreender e resolver problemas sociais complexos.

A aplicação de abordagens de engenharia de software à análise de problemas sociais complexos é uma maneira eficaz de aproveitar a tecnologia para compreender melhor as dinâmicas sociais, identificar áreas de preocupação e promover o envolvimento cidadão. Além disso, a abordagem iterativa permite que o sistema evolua e se adapte às necessidades em constante mudança das comunidades urbanas, contribuindo para uma governança mais informada e eficaz.

O experimento nos ensinou que a plataforma Colab é um reflexo da complexidade das questões urbanas e das opiniões diversificadas de seus usuários. Podemos aprender que a diversidade de perspectivas é uma força, pois permite a consideração de uma ampla gama de soluções para os problemas urbanos. A análise também nos lembra da importância de um diálogo construtivo e da busca de soluções colaborativas para promover uma cidade mais eficiente, segura e inclusiva.

Em um mundo onde as divisões são cada vez mais comuns, o Colab nos mostra que, apesar das diferenças, as comunidades podem se unir em busca de um objetivo comum: tornar as cidades melhores lugares para se viver. Portanto, ao continuar a incentivar o diálogo e a colaboração, o Colab pode desempenhar um papel importante na melhoria das condições urbanas e na promoção do bem-estar de todos os cidadãos.

\section{Aspectos hiperlocais}

Ao explorarmos a dinâmica da pressão social e sua manifestação por meio de discussões e relatos de problemas urbanos no Colab, entendemos como os usuários da plataforma expressam preocupações, sentimentos e opiniões sobre uma variedade de tópicos, destacando as questões mais relevantes e polarizadoras que afetam suas comunidades. Agora, avançaremos na análise para ressaltar o aspecto hiperlocal, concentrando-nos na comparação entre as três cidades analisadas.

Considerar aspectos hiperlocais, é reconhecer que diferentes localidades podem reagir de maneiras diversas aos mesmos problemas urbanos. Niterói e Mesquita, situadas no estado do Rio de Janeiro, e Santo André, localizada em São Paulo, compartilham desafios comuns enfrentados por muitas cidades brasileiras, como infraestrutura, segurança pública e qualidade de vida. No entanto, a percepção e a priorização desses desafios podem variar significativamente entre essas duas cidades. Da mesma forma, diferentes bairros da mesma cidade podem ter preocupações e necessidades distintas criando um padrão de criação de demandas diferente entre as regiões. É importante notar também que a percepção de certos tipos de problemas pode ser palpável para alguns grupos sociais e invisível para outros. O importante é que o aplicativo Colab mantém todas essas demandas criadas através de eventos de zeladoria geo-referenciadas, o que permite uma análise hiperlocal.

Para entender melhor essas nuances, analisamos os tipos de eventos mais criados em Niterói, Mesquita e Santo André. Embora ambos os municípios enfrentem problemas relacionados a buracos nas vias e lâmpadas apagadas à noite, a classificação dos tipos de eventos revela diferenças marcantes. Niterói apresenta uma alta incidência de eventos relacionados a problemas nas vias, como 'Buraco nas vias' 'Calçada irregular' e 'Ponto de infração de trânsito recorrente'. Esses eventos estão diretamente ligados à mobilidade urbana e à infraestrutura viária, indicando uma preocupação significativa dos cidadãos em relação a essa questão. Portanto, um tópico de pressão social relevante para Niterói é a 'Mobilidade Urbana'. Mesquita tem um grande número de eventos relacionados à natureza e ao meio ambiente, como 'Poda de árvore' e 'Mato alto'. Esses eventos indicam uma preocupação com a preservação ambiental e o paisagismo urbano. Portanto, um tópico de pressão social relevante para Santo André é o 'Meio Ambiente'. Santo André enfrenta desafios relacionados à limpeza e à infraestrutura urbana, como 'Entulho na calçada/via pública' e 'Esgoto a céu aberto'. Esses problemas podem afetar diretamente a saúde pública dos cidadãos. Portanto, um tópico de pressão social relevante para Santo André é a 'Saúde Pública'.

Embora esses tópicos sejam relevantes para as três cidades, a análise revela que cada uma delas tem uma percepção e uma priorização únicas dos problemas urbanos. Essas diferenças podem ser atribuídas a uma variedade de fatores, como a infraestrutura urbana, a composição demográfica e a cultura local. A análise também destaca a importância de uma abordagem hiperlocal para entender melhor as necessidades e os desafios de cada comunidade. Ao considerar essas nuances, os gestores públicos podem tomar decisões mais informadas e eficazes, contribuindo para uma governança mais inclusiva e participativa. Essa análise inicial destaca como as diferentes cidades reagem e percebem os problemas urbanos de maneira única, mesmo quando enfrentam desafios semelhantes. A partir desses dados, podemos aprofundar nossa investigação para entender melhor os fatores locais que moldam essas percepções e prioridades, contribuindo para uma compreensão mais abrangente da pressão social hiperlocal e suas implicações na gestão urbana.

A partir da definição dos tópicos a serem analisados, desenvolvemos uma metodologia para criar visualizações de pressão social hiperlocal utilizando dados de geolocalização de eventos postados em mídias sociais. A metodologia emprega bibliotecas como Folium e Bokeh para a geração dos mapas interativos e gráficos de rede. Primeiramente, os dados são filtrados para uma região de interesse, no caso, a cidade de Santo André. Em seguida, são selecionadas palavras-chave relevantes relacionadas à saúde pública, tais como 'hospital' 'pronto socorro' 'vacinação' entre outras. Eventos que não estão diretamente relacionados a essas palavras-chave são excluídos do conjunto de dados. A visualização inclui dois componentes principais: mapas de calor (heatmaps) e um gráfico de rede (network).

Para os heatmaps, a ponderação é ajustada com base em uma métrica de sentimento, e um mapa base é criado, centrado na média das coordenadas geográficas dos eventos selecionados. Os dados ponderados são utilizados para gerar o mapa de calor, com uma escala de cores que indica o sentimento dos eventos.

No gráfico de rede, as conexões entre os eventos e os usuários que os postaram são identificadas e representadas. A biblioteca NetworkX é utilizada para criar o gráfico, e os nós são posicionados com base nas coordenadas geográficas dos eventos. Uma adição importante à visualização é a coloração dos nós com base nas personas identificadas: azul para os \textit{helpers}  e vermelho para os \textit{complainers}. Além disso, o polígono da cidade é utilizado como referência para centralizar tanto o mapa de rede quanto os pontos de eventos. Isso contribui significativamente para destacar a localização dos eventos em relação à área geográfica da cidade, possibilitando uma análise mais detalhada das dinâmicas de saúde pública em nível hiperlocal.

Essas visualizações permitem a identificação de padrões de atividade relacionados aos tópicos de pressão social a nível hiperlocal. Os heatmaps revelam áreas de maior atividade e sentimentos associados aos eventos, permitindo uma compreensão mais granularizada das preocupações da comunidade em áreas específicas da cidade. O gráfico de rede, por sua vez, mostra como os eventos e usuários estão interconectados, possibilitando a identificação de influenciadores ou grupos de interesse em questões de saúde pública. A combinação de heatmaps e gráficos de rede fornece insights valiosos para formuladores de políticas públicas, pesquisadores e profissionais de saúde na tomada de decisões e no planejamento de intervenções em saúde pública a nível hiperlocal.

\subsection{Mobilidade Urbana em Niterói}

A análise dos dados de pressão social sobre mobilidade urbana em Niterói da \autoref{tab:eventos_populares_mobility_niteroi} revela as percepções dos cidadãos em relação a esse tema. Podemos notar uma divisão entre "complainers" e "helpers" em relação a diferentes eventos, refletindo as atitudes da comunidade.

\begin{quadro}[htb]
	\centering
	\includegraphics[width=0.7\textwidth]{images/heatmap_niteroi.PNG}
	\caption{Heatmap de Pressão Social para Mobilidade Urbana em Niterói}
	\label{fig:heatmap_niteroi}
\end{quadro}

\begin{figure}[htb]
	\centering
	\includegraphics[width=0.7\textwidth]{images/wordcloud_niteroi.png}
	\caption{Palavras mais frequentes nas postagens de Niterói}
	\label{fig:wordcloud_niteroi}
\end{figure}

Eventos com uma persona predominantemente \textit{complainer} indicam insatisfações, como o caso do evento relacionado à condição sanitária irregular de estabelecimentos na cidade. Por outro lado, eventos com uma persona predominantemente \textit{helper}, como aqueles relacionados a infrações de trânsito, demonstram a disposição da comunidade em colaborar para melhorar a situação.

\begin{quadro}[htb]
	\centering
	\includegraphics[width=0.7\textwidth]{images/social_barometer_niteroi.png}
	\caption{social barometer niteroi}
	\label{fig:social_barometer_niteroi}
\end{quadro}

Os eventos mais populares, como buracos nas vias, refletem a importância da infraestrutura viária e segurança no trânsito para os cidadãos de Niterói. A presença de eventos relacionados à acessibilidade também destaca desafios significativos nessa área, chamando a atenção para a necessidade de políticas que melhorem a acessibilidade urbana.

Além disso, os dados revelam sentimentos positivos e negativos relacionados a diferentes eventos, com a comunidade expressando insatisfação em relação à poluição sonora e falta de energia, enquanto demonstra apoio a iniciativas de arborização.

Para os administradores da cidade, esses insights podem informar políticas públicas eficazes. Eles podem priorizar a regulamentação de estabelecimentos com condições sanitárias precárias, promover campanhas de conscientização sobre as leis de trânsito e investir na manutenção viária. Além disso, medidas para melhorar a acessibilidade e reduzir a poluição sonora podem ser implementadas com base nos dados coletados.

Em resumo, essas métricas de pressão social podem orientar as autoridades locais na criação de políticas públicas que promovam uma mobilidade urbana mais segura, acessível e eficiente em Niterói, atendendo às necessidades e preocupações da comunidade.

\subsection{Meio Ambiente em Mesquita}
A análise dos dados sobre o meio ambiente em Mesquita revela informações cruciais sobre as preocupações e percepções da comunidade local. Conforme apresentado na \autoref{tab:eventos_populares_mesquita}, diferentes eventos, como descarte irregular de lixo, poluição do ar e desmatamento, foram analisados para identificar tendências e opiniões da comunidade. Alguns eventos são relatados de maneira crítica e negativa, enquanto outros são vistos de forma mais positiva.

\begin{quadro}[htb]
	\centering
	\includegraphics[width=0.7\textwidth]{images/heatmap_mesquita.PNG}
	\caption{Heatmap de Pressão Social para Meio Ambiente em Mesquita}
	\label{fig:heatmap_mesquita}
\end{quadro}

Eventos como 'Descarte irregular de lixo' e 'Ponto de queimada irregular recorrente' têm uma perspectiva predominantemente negativa e pontuações negativas. Por outro lado, eventos como 'Limpeza de Canais' e 'Poda de árvore' são relatados de maneira mais positiva, com pontuações positivas. A análise dos eventos mais frequentes mostra que 'Descarte irregular de lixo' e 'Ponto de queimada irregular recorrente' são as principais fontes de insatisfação na comunidade.

\begin{quadro}[htb]
	\centering
	\includegraphics[width=0.7\textwidth]{images/social_barometer_mesquita.png}
	\caption{social barometer mesquita}
	\label{fig:social_barometer_mesquita}
\end{quadro}

\begin{figure}[htb]
	\centering
	\includegraphics[width=0.7\textwidth]{images/wordcloud_mesquita.png}
	\caption{Wordcloud com palavras mais frequentes nas postagens de Mesquita}
	\label{fig:wordcloud_mesquita}
\end{figure}

Esses dados podem ser usados pelos administradores da cidade para informar políticas públicas relacionadas ao meio ambiente. Eles destacam a necessidade de abordar o descarte inadequado de resíduos sólidos e a ocorrência de queimadas irregulares como prioridades. Além disso, a comunidade demonstra apoio a ações como a poda de árvores e a conservação da via pública, indicando áreas onde investimentos e esforços positivos podem ser direcionados para melhorar a qualidade de vida e a sustentabilidade ambiental na região. A diversidade de perspectivas ressalta a importância de envolver ativamente a comunidade na proteção e preservação ambiental.

\subsection{Saúde Pública em Santo André}

A análise da pressão social relacionada à saúde pública em Santo André conforme a \autoref{tab:eventos_populares_sa} resume alguma das preocupações específicas dos usuários sobre esse tema. Os eventos relatados, desde infraestrutura inadequada até problemas de higiene e segurança, refletem a complexidade dessas preocupações.

\begin{quadro}[htb]
	\centering
	\includegraphics[width=0.7\textwidth]{images/heatmap_santo_andre.PNG}
	\caption{Heatmap de Pressão Social para Saúde Pública em Santo André}
	\label{fig:heatmap_santo_andre}
\end{quadro}

Alguns eventos, como 'Coleta de container' e 'Limpeza de área pública', estão associados a preocupações legítimas da comunidade em relação à falta de serviços e manutenção adequados na cidade. Por outro lado, eventos como 'Falta de água' são percebidos como situações que requerem intervenção positiva para resolver o problema. No entanto, também existem eventos com críticas significativas em relação à infraestrutura de saúde e à conformidade com regulamentos, indicando uma insatisfação com essas áreas.

\begin{quadro}[htb]
	\centering
	\includegraphics[width=0.7\textwidth]{images/social_barometer_santo_andre.png}
	\caption{social barometer santo andré}
	\label{fig:social_barometer_santo_andre}
\end{quadro}

No contexto da pandemia de COVID-19, os eventos mais populares, como 'Lixo' e 'Aglomeração de pessoas', ganham ainda mais relevância e urgência. 'Aglomeração de pessoas' reflete a disposição da comunidade para colaborar, mas ainda há preocupações substanciais em relação à aglomeração. 'Lixo' apresenta uma mistura de percepções positivas e críticas, indicando a necessidade de melhorias na gestão de resíduos.

\begin{quadro}[htb]
	\centering
	\includegraphics[width=0.7\textwidth]{images/network_santo_andre_personas_map.png}
	\caption{Rede de eventos populares relacionados a Saúde Pública em Santo André}
	\label{fig:network_santo_andre_personas_map}
\end{quadro}

\begin{figure}[htb]
	\centering
	\includegraphics[width=0.7\textwidth]{images/wordcloud_santo_andre.png}
	\caption{Palavras mais frequentes nas postagens de Santo André}
	\label{fig:wordcloud_santo_andre}
\end{figure}

Em resumo, os dados refletem a complexidade da saúde pública em Santo André, com a comunidade demonstrando uma consciência aguçada das questões que afetam seu bem-estar. As reclamações estão concentradas em áreas críticas, e o diálogo positivo em alguns eventos indica um desejo de cooperação para resolver esses problemas. Para os administradores das cidades, esses dados podem ser usados para informar políticas públicas relacionadas à saúde pública, identificando áreas de foco e necessidades específicas da comunidade.

\section{Trabalhos Futuros}

Em um contexto acadêmico, a abordagem central deste estudo reside na análise das câmaras de eco e na concepção do Colab como um barômetro social hiperlocal. Essa perspectiva oferece uma visão singular e abrangente das interações e percepções das comunidades locais em relação a tópicos polarizadores. No entanto, vale ressaltar que essa metodologia não se limita a seu emprego na pesquisa de câmaras de eco, sendo passível de extensão para diversas outras áreas de estudo.

A observação do Colab como um barômetro social hiperlocal implica em uma compreensão detalhada das opiniões, discussões e manifestações das comunidades locais em relação a questões específicas. Nesse contexto, a polarização emerge como um tema fundamental, visto que a plataforma proporciona uma visão inédita das dinâmicas de polarização no nível mais próximo da comunidade, indo além das análises tradicionais de polarização em escala mais ampla.

Essa abordagem também sugere que a pressão social hiperlocal proproem uma dimensão rica e relevante que pode ser investigada em outros contextos de pesquisa. Embora este estudo se concentre nas câmaras de eco e nas dinâmicas sociais específicas do Colab, é possível extrapolar essas heurísticas e metodologias para áreas de estudo diversas. Assim, abre-se a perspectiva para futuros trabalhos que explorem como a pressão social hiperlocal pode ser aplicada e analisada em campos como políticas públicas, meio ambiente, saúde pública, desenvolvimento urbano, engajamento cívico e conflitos sociais, entre outros.

Em pesquisas futuras, há um vasto campo a ser explorado no aprimoramento das heurísticas e métodos de análise da pressão social hiperlocal, especialmente no contexto de plataformas como o Colab. Uma das direções promissoras seria a incorporação de dimensões temporais e geolocalização de maneira mais robusta. Isso poderia ser alcançado através do desenvolvimento de algoritmos de análise de séries temporais para monitorar a evolução das discussões e do sentimento em torno de tópicos específicos ao longo do tempo. Compreender como as percepções e prioridades mudam sazonalmente ou em resposta a eventos específicos seria fundamental para uma análise mais holística. Além disso, a geolocalização desempenha um papel crítico na pressão social hiperlocal. A capacidade de mapear a localização precisa de eventos e discussões poderia permitir uma análise mais granular das dinâmicas das comunidades urbanas. Seria interessante explorar técnicas avançadas de geolocalização, como a análise de dados geoespaciais, para identificar padrões de pressão social em áreas urbanas específicas. Isso poderia ajudar as autoridades locais a direcionar recursos de forma mais eficaz e tomar decisões informadas sobre políticas públicas.

Também, vale a pena considerar a extensibilidade dessas heurísticas para outras redes sociais e plataformas além do Colab. A pressão social hiperlocal não é exclusiva de uma única plataforma, e compreender como ela se manifesta em diferentes contextos online poderia fornecer insights valiosos sobre o comportamento cívico e a participação pública em ambientes digitais variados. Isso abriria a porta para a criação de um framework mais amplo e aplicável a diversas situações. Outra área de pesquisa interessante seria a avaliação das implicações políticas e de governança da pressão social hiperlocal. Como as percepções locais podem influenciar as decisões políticas e a prestação de serviços públicos? Investigar como as vozes hiperlocais impactam as políticas urbanas e como as autoridades respondem a essas pressões seria um campo de estudo crítico para aprimorar a governança democrática.

A compreensão das dinâmicas da pressão social hiperlocal pode ser aplicada para informar a tomada de decisões políticas e a prestação de serviços públicos em nível local. Os governos municipais podem usar essa análise para identificar as principais preocupações dos cidadãos e a eficácia de suas políticas, adaptando-as de acordo com as demandas locais. Em estudos relacionados ao meio ambiente e desastres naturais, a pressão social hiperlocal pode ser usada para rastrear as preocupações e percepções das comunidades locais. Isso permite uma resposta mais eficaz a eventos climáticos extremos, poluição e outras questões ambientais. A análise da pressão social hiperlocal pode ser aplicada ao monitoramento de questões de saúde pública, como surtos de doenças infecciosas. A detecção rápida de preocupações da comunidade em relação à saúde pode ajudar as autoridades de saúde a tomar medidas preventivas e educativas. No contexto do desenvolvimento urbano, a pressão social hiperlocal pode ser usada para avaliar o impacto de projetos de infraestrutura, como construção de estradas, expansão de transporte público e desenvolvimento imobiliário. Isso permite uma análise mais detalhada das preocupações e necessidades das comunidades afetadas. Em estudos sobre conflitos e tensões sociais, a pressão social hiperlocal pode ser usada para identificar pontos de atrito em comunidades específicas. Isso pode ajudar na prevenção de conflitos e na promoção da coesão social.

Por fim, considerando a crescente importância da inteligência artificial e da análise de big data, explorar como essas tecnologias podem ser aplicadas de forma ética e eficaz na detecção e análise da pressão social hiperlocal é uma área promissora de pesquisa. O desenvolvimento de ferramentas automatizadas para identificar tópicos de pressão e monitorar o sentimento em tempo real poderia revolucionar a forma como as cidades respondem às demandas de seus cidadãos. Há uma série de oportunidades empolgantes para aprimorar nossas abordagens de análise da pressão social hiperlocal, incorporando dimensões temporais, geolocalização e estendendo essas heurísticas para diferentes plataformas e contextos. Isso não apenas enriqueceria nosso entendimento das comunidades urbanas, mas também teria implicações práticas significativas para a governança local e o envolvimento cívico.

\section{Conclusões}

No decorrer deste capítulo, exploramos as dinâmicas da opinião pública no Colab, uma plataforma que abriga discussões sobre tópicos urbanos diversos. Nossas análises revelaram uma diversidade de perspectivas e abordagens por parte dos usuários em relação a questões urbanas. Observamos que não há uma única persona predominante, variando amplamente entre eventos, refletindo a complexidade das questões urbanas. Isso indica que a comunidade do Colab lida com diferentes problemas de maneiras variadas, desde a busca por soluções até a expressão de críticas construtivas ou negativas.

Surpreendentemente, mesmo nas discussões em que predominam personas \textit{helper}, ainda podem surgir críticas construtivas, e vice-versa. Isso demonstra a disposição dos usuários em considerar diferentes perspectivas e contribuir para melhorias, independentemente de sua atitude inicial em relação ao problema. Além disso, apesar de algumas polarizações em debates políticos e tópicos sensíveis, a maioria dos usuários parece comprometida em abordar e resolver os desafios urbanos. A presença de personas \textit{helper} indica uma disposição para encontrar soluções construtivas, mesmo em meio a divergências.

A aplicação de abordagens de engenharia de software foi essencial para nossa análise, permitindo-nos desenvolver heurísticas para medir a opinião média e as personas dos usuários em relação a diferentes eventos urbanos. Essas heurísticas serviram de base para representações visuais, como o gráfico de radar, que auxiliaram na visualização das dinâmicas sociais de forma acessível. Além disso, a coleta de dados estruturados por meio de um modelo bem definido foi fundamental para organizar as informações e facilitar a análise. Integrando a tecnologia de aprendizado de máquina na forma de algoritmos de classificação e processamento de linguagem natural, automatizamos a análise de texto, economizando tempo e recursos. Isso nos permitiu examinar um grande volume de dados em larga escala.

Aqui reside uma conexão profunda entre a fenomenologia e a engenharia de software. Assim como a fenomenologia busca compreender as experiências subjetivas de indivíduos, nossa abordagem de engenharia de software nos permitiu mergulhar profundamente nas experiências dos usuários do Colab, revelando as nuances de suas interações e perspectivas. Ao aplicar heurísticas para medir opiniões e personas, criamos uma lente através da qual podemos observar as complexidades das redes sociais em escala.

Além disso, reconhecemos a importância de integrar os resultados de nossa análise com os tomadores de decisão, como agências governamentais e organizações da sociedade civil. Isso garante que nossas descobertas sejam utilizadas para informar políticas e ações práticas, promovendo uma governança mais informada e eficaz.

A análise também nos levou a explorar o fenômeno da polarização, que é amplamente observado em ambientes digitais, como redes sociais. Identificamos como as plataformas digitais podem contribuir para a formação de câmaras de eco e bolhas de filtro, reforçando crenças preexistentes e restringindo a exposição a diferentes perspectivas. No entanto, também reconhecemos que o ciberespaço, como definido por Levy, é um ambiente multifacetado, onde diversas vozes e visões coexistem.

Levy nos lembra que a cibercultura é caracterizada pela diversidade de atores, projetos e interpretações, frequentemente em oposição. O ciberespaço, apesar de desafios, oferece oportunidades para o diálogo e a colaboração entre diferentes perspectivas culturais. Isso é particularmente relevante no contexto do Colab, onde a diversidade de perspectivas é uma força que permite a consideração de uma ampla gama de soluções para os problemas urbanos.

Ao incorporar a análise de sentimentos e personas dos usuários em nossa compreensão do Colab, estamos agregando uma dimensão adicional à nossa capacidade de medir a pressão social hiperlocal. Isso nos permite identificar áreas de polarização e tensão dentro da comunidade, bem como oportunidades para promover o diálogo e a colaboração. A abordagem de Levy à cibercultura destaca a interconexão e a valorização da diversidade de perspectivas, e o Colab tem o potencial de se tornar um espaço inclusivo para o diálogo cidadão e a construção de comunidades informadas e coesas.

Em última análise, o Colab, como um 'barômetro social hiperlocal' e uma plataforma de cidadania, tem o potencial de desempenhar um papel crucial na promoção da participação cidadã, no engajamento construtivo e na melhoria das condições urbanas. À medida que continuamos a valorizar a diversidade de perspectivas e a buscar soluções colaborativas, podemos avançar na construção de cidades mais eficientes, seguras e inclusivas, em um ciberespaço que amplifica tanto os desafios quanto as oportunidades da opinião pública. Este capítulo nos conduziu à compreensão de que o Colab é um espaço de convergência, onde perspectivas culturais variadas têm a oportunidade de dialogar e colaborar. A diversidade de vozes é a essência da cibercultura, e o Colab é um exemplo poderoso disso.

No próximo capítulo, continuaremos a explorar a complexidade das redes sociais, mas com um foco específico na detecção de câmaras de eco. Essa pesquisa se conecta diretamente ao nosso experimento atual, onde desenvolvemos heurísticas para medir a opinião média e as personas dos usuários em relação a diferentes tipos de eventos. As heurísticas e os insights obtidos até agora servirão como base fundamental para a próxima etapa de nossa investigação.

Além disso, nosso conjunto de dados abrange não apenas os eventos criados pelos usuários, classificados por score e persona, mas também inclui informações detalhadas sobre os relacionamentos de rede social entre os usuários das três cidades, revelando os usuários mais influentes, cada seguidor e quantos usuários cada pessoa segue além dos tipos de eventos com os quais interagem. Cada evento também é enriquecido com comentários e likes, proporcionando uma visão mais completa das interações dos usuários. Com base nesses dados e nas heurísticas criadas a partir deles, seremos capazes de inferir não apenas a opinião média dos usuários em relação a diversos tipos de evento, mas também avaliar a homogeneidade de opiniões com base no tipo de evento com o qual cada usuário interage com mais frequência e compará-lo com o tipo de evento mais interagido pelos membros de sua rede social. Com cada evento, comentário ou interação, os usuários não apenas expressam uma opinião, mas também se engajam em um processo de co-criação e interpretação de significado no ciberespaço. Portanto, as ideias de barômetro social hiperlocal, pressão social e detecção de câmaras de eco estão intrinsecamente conectadas porque elas refletem as dinâmicas e nuances das relações humanas que se desdobram no ciberespaço. O conceito de 'barômetro social hiperlocal' captura a essência de como as interações individuais se somam para criar um panorama mais amplo da opinião pública em uma região específica. Ao mesmo tempo, a pressão social revela como os usuários influentes podem moldar ou direcionar as opiniões e interações dentro de suas redes. Finalmente, a detecção de câmaras de eco é uma consequência direta dessas interações, mostrando como grupos de usuários podem se tornar insulares em suas opiniões, reforçando crenças e perspectivas semelhantes.

Dentro dessa complexa teia de relacionamentos e interações, um usuário não é apenas um ponto de dados, mas uma entidade viva que busca conexão, expressão e entendimento. Os dados que coletamos e as heurísticas que aplicamos, portanto, não são meros exercícios técnicos, contudo, é fundamental lembrar que por trás desses dados, existem indivíduos com experiências reais e tangíveis. O pensamento de Heidegger sobre a 'técnificação das mãos' sugere uma preocupação com a maneira pela qual a tecnologia pode distanciar o ser humano de uma vivência autêntica do mundo. Quando um usuário do Colab cria um evento ou interage na plataforma, essa 'técnificação' pode estar presente, influenciando como ele percebe e se relaciona com o espaço digital. Nesse sentido, a própria criação de heurísticas e o uso da engenharia de software podem ser vistos como uma extensão desse processo de 'técnificação'. Por outro lado, Merleau-Ponty, com sua ênfase na corporeidade e percepção, serve como um lembrete de que as interações no Colab são manifestações de experiências corpóreas e perceptivas dos usuários.

A polarização, as câmaras de eco e até mesmo as análises de sentimentos não são meros produtos técnicos, mas emergem das experiências vividas e sentidas das pessoas. Ao avançarmos na integração entre engenharia de software e fenomenologia, é crucial lembrar que cada clique, postagem ou interação no Colab tem sua origem na experiência humana. Deste modo, nossas análises, embora apoiadas por ferramentas tecnológicas sofisticadas, devem sempre manter a centralidade do ser humano e suas experiências. Assim, mesmo quando exploramos as complexidades técnicas e analíticas, reconhecemos que elas servem para nos aproximar, e não distanciar, das nuances e riquezas da experiência humana no ciberespaço e na vida real.

\chapter{Detectando câmaras de eco}
\label{chapter:08_echochamberdetection}
Neste capítulo, exploramos a hipótese de que a formação de câmaras de eco em uma rede social pode ser matematicamente representada através de uma combinação de várias estatísticas descritivas, cada uma correspondendo a um critério específico para a detecção de câmaras de eco. Esses critérios incluem a densidade da comunidade, a homogeneidade das opiniões, as conexões externas e o efeito de influenciadores, todos fundamentais para entender a dinâmica das redes sociais e, mais especificamente, a formação de câmaras de eco.

Através da análise detalhada de cada critério e da introdução de parâmetros adicionais como o \sigla{$ECC$}{Coeficiente Individual de Câmara de Eco} e o \sigla{$GEC$}{Coeficiente Global de Câmara de Eco}, buscamos desenvolver um modelo matemático que possa quantificar a probabilidade de formação de câmaras de eco em uma rede social.

Este modelo é posteriormente implementado computacionalmente através da classe Python EchoChamberDetector, que utiliza algoritmos tradicionais para identificação de comunidades e análise de redes sociais combinado com técnicas de análise de conteúdo e sentimento. A implementação permite não apenas a identificação de potenciais câmaras de eco, mas também oferece insights sobre a influência de diferentes fatores na sua formação, proporcionando uma ferramenta robusta e adaptável para análise de redes sociais no contexto da detecção de câmaras de eco.

\section{Heurísticas para Detecção}

Nossa hipótese é que a probabilidade de formação de câmaras de eco em uma rede social pode ser expressa matematicamente como uma combinação de várias estatísticas descritivas, cada uma correspondendo a um critério específico para a detecção de câmaras de eco:

\begin{itemize}
	\item \textbf{Densidade da comunidade}: A densidade da comunidade é uma medida do grau de interconexão entre os membros de uma comunidade. Em uma rede social, podemos calcular a densidade da comunidade como a proporção de conexões possíveis que realmente existem entre os membros da comunidade. Uma comunidade densa é aquela em que os membros estão altamente interconectados. Em termos de redes sociais, isso pode ser medido pelo número de conexões que cada indivíduo tem dentro de sua comunidade. Uma comunidade densa é mais propensa a formar uma câmara de eco, pois a informação circula predominantemente dentro do grupo, reforçando as opiniões existentes \cite{2015_Recuero_BOOK}.

	\item \textbf{Homogeneidade das opiniões}: A homogeneidade das opiniões refere-se ao grau de concordância ou semelhança nas opiniões dos membros de uma comunidade. Em uma rede social, isso pode ser medido pela frequência com que certas opiniões ou pontos de vista são compartilhados ou endossados pelos membros da comunidade. Quanto mais homogêneas forem as opiniões dentro de uma comunidade, maior será a probabilidade de uma câmara de eco se formar \cite{2016_WanYu}.

	\item \textbf{Conexões externas}: As conexões externas referem-se às interações entre os membros de uma comunidade e indivíduos ou grupos fora da comunidade. Em uma rede social, isso pode ser medido pelo número de conexões que cada indivíduo tem fora de sua comunidade. Quanto menor o número de conexões externas, maior a probabilidade de câmaras de eco se formarem, pois a informação é menos provável de ser desafiada por opiniões divergentes \cite{ref3}.

	\item \textbf{Efeito de influenciadores}: Os influenciadores são indivíduos ou grupos que têm um impacto significativo sobre as opiniões e comportamentos dos outros em uma rede social. Em uma rede social, isso pode ser medido pela centralidade do nó, ou seja, o número de conexões que um indivíduo tem, ou pelo número de vezes que o conteúdo de um indivíduo é compartilhado ou endossado por outros. A presença de influenciadores na rede pode influenciar a formação de câmaras de eco, pois eles podem reforçar opiniões existentes e promover a homogeneidade dentro da comunidade \cite{2014_Gilbuena}.
\end{itemize}

Antes de aprofundarmos a formulação matemática, é crucial entender e definir claramente o que constitui uma "câmara de eco". Em contextos de redes sociais, uma câmara de eco é frequentemente vista como um ambiente em que os indivíduos são expostos predominantemente a informações que reforçam suas crenças e opiniões preexistentes, limitando assim a exposição a perspectivas divergentes. Este fenômeno pode resultar em polarização de opiniões e uma percepção distorcida da realidade. 

Além disso, ao considerar a metodologia proposta por \citeonline{2023_Atiqi_BOOK}, é essencial destacar a relevância dos parâmetros $GEC$ e $ECC$. Enquanto o $GEC$ fornece uma visão macro da tendência da rede inteira em formar câmaras de eco, o $ECC$ se concentra na propensão individual de cada membro da rede em se isolar em neses ambientes. A inclusão desses parâmetros oferece uma abordagem mais holística, permitindo uma análise tanto no nível da rede como um todo quanto no nível individual.

No contexto da plataforma Colab, uma 'câmara de eco' pode ser definida como \textit{um subconjunto de usuários que, ao interagir predominantemente com tópicos específicos de zeladoria pública, demonstram uma homogeneidade notável em suas opiniões e perspectivas}. Em essência, uma câmara de eco no Colab indica uma tendência de agrupamento de usuários que compartilham e reforçam visões semelhantes, limitando assim a diversidade de perspectivas a que são expostos.

Com base nessas hipóteses e definições, podemos construir um modelo que inclua essas estatísticas descritivas. O modelo será usado para estimar os parâmetros que quantificam a influência dessas estatísticas na probabilidade de formação de câmaras de eco. Podemos expressar a probabilidade de formação de câmaras de eco como uma função de várias estatísticas descritivas:

\begin{equation}
	\begin{split}
		P(\text{{Câmara de Eco}}) = \exp(&\beta_1 \cdot \text{{Densidade}} + \\
		&\beta_2 \cdot \text{{Homogeneidade}} + \\
		&\beta_3 \cdot \text{{Conexões Externas}} + \\
		&\beta_4 \cdot \text{{Influenciadores}} + \\
		&\beta_5 \cdot \text{{Exposição Média}} + \\
		&\beta_6 \cdot \text{{$GEC$}} + \\
		&\beta_7 \cdot \text{{$ECC$}})
	\end{split}
\end{equation}

em que $\beta_1$, $\beta_2$, $\beta_3$, $\beta_4$, $\beta_5$, $\beta_6$ e $\beta_7$ são os parâmetros estimados para cada estatística descritiva e exp é a função exponencial. Essa equação fornece uma maneira de quantificar matematicamente a probabilidade de formação de câmaras de eco em uma rede social, com base nos critérios estabelecidos. Os parâmetros $\beta$ são comumente usados para quantificar a influência de várias estatísticas descritivas na probabilidade de formação de eventos complexos. Os valores dos parâmetros $\beta$ são geralmente estimados a partir dos dados. No entanto, a escolha dos valores iniciais para esses parâmetros pode ter um impacto significativo na convergência do modelo. Portanto, é comum iniciar os parâmetros beta com valores simples, como 1, e ajustá-los iterativamente para melhorar o ajuste do modelo.

Os parâmetros beta podem ser interpretados em termos de suas implicações para a probabilidade de formação de câmaras de eco. Por exemplo, um valor positivo para o parâmetro beta associado à densidade da comunidade sugeriria que comunidades mais densas têm maior probabilidade de formar câmaras de eco. Analogamente, um valor positivo para o parâmetro beta associado à homogeneidade das opiniões indicaria que comunidades com opiniões mais homogêneas têm maior probabilidade de formar câmaras de eco. Os aspectos de escala a serem considerados dependem dos critérios específicos empregados para detectar câmaras de eco. Por exemplo, ao considerar a densidade da comunidade como um critério, a escala pode ser considerada em termos do número de conexões dentro da comunidade em relação ao número total de conexões possíveis. Se a homogeneidade das opiniões for um critério, a escala pode ser considerada em termos da variação das opiniões dentro da comunidade.

\begin{figure}[htb]
	\centering
	\includegraphics[width=0.7\textwidth]{images/echo_chamber_strength_by_homo.png}
	\caption{Distribuição da $FCE$ por Homogeneidade de Opiniões em uma rede simulada}
	\label{fig:echo_chamber_strength_by_homo}
\end{figure}

Essa métrica derivativa, que chamaremos de \sigla{$FCE$}{Força da Câmara de Eco} representa um \textit{snapshot}, ou um retrato do estado atual da comunidade em termos de polarização, isolamento e homogeneidade de opiniões.

\section{Implementação Computacional}

Para aplicar nosso teorema de probabilidade de câmaras de eco na prática e derivarmos a Força da Câmara de Eco de cada comunidade ($FCE$), desenvolvemos a classe Python \texttt{EchoChamberDetector}. Essa classe permite analisar redes sociais e identificar potenciais câmaras de eco com base nos critérios estabelecidos em nosso teorema. Um componente importante do modelo é identificar comunidades dentro da rede. Utilizamos dois algoritmos diferentes: Louvain e também o \texttt{SignificanceVertexPartition}, uma função de classificação de comunidades disponivel no pacote \texttt{leidenalg}. Esses algoritmos são usados como etapas preliminares para identificar as comunidades da rede. Uma vez que as comunidades foram identificadas usando um dos algoritmos, prosseguimos com a análise das câmaras de eco. Para cada comunidade detectada, realizamos cálculos estabelecendo relações entre a comunidade da rede com o resto da rede como um todo:

\begin{itemize}
	\item Fator de densidade da comunidade: calculado contando a proporção de arestas existentes em relação ao número máximo possível de arestas na comunidade. Quanto maior a densidade, maior a probabilidade de formação de uma câmara de eco.
	\item Homogeneidade das opiniões: calculada a partir do desvio padrão das pontuações de opiniões dos membros da comunidade. Quanto menor o desvio padrão, maior a homogeneidade das opiniões e maior a probabilidade de formação de uma câmara de eco.
	\item Fator de conexões externas: calculado a partir da proporção de conexões da comunidade que são externas, ou seja, que se conectam a nós fora da comunidade. Quanto menor o número de conexões externas, maior a probabilidade de formação de uma câmara de eco.
	\item Fator de influenciadores: calculado a partir da proporção de de usuários com alta \textit{eigencentrality} presentes na comunidade em relação ao tamanho total da comunidade. Quanto maior a proporção de influenciadores, maior a probabilidade de formação de uma câmara de eco.
\end{itemize}

Além dessas estatísticas, também incorporamos duas heurísticas adicionais baseadas no trabalho de \citeonline{2023_Atiqi_BOOK}: Parâmetro Global de Câmara de Eco e Exposição média. A metodologia de cálculo do $GEC$ é baseada na abordagem apresentada por \citeonline{2023_Atiqi_BOOK}. O $GEC$ é uma medida da tendência geral de uma rede social para formar câmaras de eco. Ele é calculado como a soma do produto dos sinais das opiniões de todos os pares de usuários conectados na rede.

Para calcular o $GEC$, primeiro precisamos definir a opinião de um usuário. No contexto de redes sociais, a opinião de um usuário é geralmente representada por um score que reflete a polaridade de suas postagens ou interações. Valores mais próximos de 1 representam postagens mais positivas, enquanto valores mais próximos de 0 representam sentimentos mais negativos.

Dado um par de usuários conectados $(u, v)$, o produto dos sinais de suas opiniões é dado por $\text{sign}(o_u) \cdot \text{sign}(o_v)$, onde $o_u$ e $o_v$ são as opiniões dos usuários $u$ e $v$, respectivamente, e $\text{sign}$ é a função sinal que retorna -1 para números negativos, 0 para zero e 1 para números positivos.

O $GEC$ é então calculado somando o produto dos sinais das opiniões de todos os pares de usuários conectados na rede:

\begin{equation}
	GEC = \sum_{(u, v) \in E} \text{sign}(o_u) \cdot \text{sign}(o_v)
\end{equation}

em que $E$ é o conjunto de todas as arestas na rede.

Um valor positivo de $GEC$ indica uma tendência para a formação de câmaras de eco, pois sugere que os usuários tendem a se conectar com outros usuários que compartilham opiniões semelhantes. Por outro lado, um valor negativo de $GEC$ indica uma tendência para a diversidade de opiniões, pois sugere que os usuários tendem a se conectar com outros usuários que têm opiniões diferentes.

É importante notar que o $GEC$ é uma medida global que reflete a tendência geral da rede para formar câmaras de eco. Ele não fornece informações sobre a formação de câmaras de eco em nível de comunidade ou individual. Para obter essas informações, precisamos de outras heurísticas, como as discutidas na seção anterior.

A exposição média é outra heurística importante que \citeonline{2023_Atiqi_BOOK} introduz em sua metodologia. Definida como a soma das diferenças entre a opinião do usuário e o sentimento da notícia exposta, média sobre todos os usuários. Matematicamente, isso pode ser expresso como:

\begin{equation}
	\text{{Exposição Média}} = \frac{1}{N} \sum_{i=1}^{N} |o_i - s_j|
\end{equation}

em que $o_i$ é a opinião do usuário $i$, $s_j$ é o sentimento da notícia $j$ exposta ao usuário $i$, e $N$ é o número total de usuários na rede.

A exposição média é uma medida de quão expostos os usuários estão a opiniões que diferem das suas. Uma alta exposição média indica que os usuários estão sendo expostos a uma variedade de opiniões, o que pode reduzir a probabilidade de formação de câmaras de eco. Por outro lado, uma baixa exposição média sugere que os usuários estão principalmente expostos a opiniões que são semelhantes às suas, aumentando a probabilidade de formação de câmaras de eco.

No contexto do Colab, a exposição média pode ser calculada aproveitando os tipos de postagens com os quais os usuários mais interagem. Cada tipo de evento, seja ele relacionado a infraestrutura, segurança pública, entre outros, pode ser representado como um número único. Isso cria um vetor de características para cada usuário que reflete os tipos de eventos com os quais ele interage. A exposição média de um usuário pode então ser calculada como a diferença entre o vetor de características do usuário e o vetor médio de características de todos os eventos na rede.

Matematicamente, isso pode ser expresso da seguinte maneira:

\begin{equation}
	\text{{Exposição Média}} = \frac{1}{N} \sum_{i=1}^{N} ||v_i - \bar{v}||
\end{equation}

em que $v_i$ é o vetor de características do usuário $i$, $\bar{v}$ é o vetor médio de características de todos os eventos na rede, e $N$ é o número total de usuários na rede. A norma $||.||$ pode ser a norma euclidiana, que mede a distância geométrica entre os dois vetores, ou outra norma apropriada.

Nesse contexto, um usuário que interage com uma variedade de tipos de eventos terá uma exposição média menor, pois seu vetor de características será mais semelhante ao vetor médio de características de todos os eventos. Por outro lado, um usuário que interage principalmente com um tipo específico de evento terá uma exposição média maior, pois seu vetor de características será mais diferente do vetor médio.

Essa abordagem permite uma análise mais granular da exposição dos usuários a diferentes tipos de eventos e pode ajudar a identificar usuários ou comunidades que estão potencialmente isolados em câmaras de eco.

\begin{itemize}
	\item \textbf{Exposição média}: A exposição média é definida como a soma das diferenças entre a opinião do usuário e o sentimento das notícias expostas, média sobre todos os usuários. Quanto maior a exposição média, maior a probabilidade de formação de uma câmara de eco, pois indica que os usuários estão sendo expostos a notícias que reforçam suas opiniões existentes.
	\item \textbf{Parâmetro Global de Câmara de Eco ($GEC$)}: O $GEC$ é definido como a soma do produto dos sinais das opiniões de todos os pares de usuários conectados na rede. Um valor positivo de $GEC$ indica uma tendência para a formação de câmaras de eco, pois sugere que os usuários tendem a se conectar com outros usuários que compartilham opiniões semelhantes.
	\item \textbf{Coeficiente Individual de Câmara de Eco ($ECC$)}: O $ECC$ é definido como a métrica que avalia a polarização de opiniões baseado na exposição a diferentes perspectivas. O cálculo do $ECC$ considera a contribuição de cada membro da comunidade, considerando o score de sentimento atribuída aos eventos que eles criaram, bem como as personas predominantes na comunidade. Ao somar as contribuições individuais de cada membro e normalizá-las pelo tamanho da comunidade, o $ECC$ fornece uma medida quantitativa da polarização, onde valores mais altos indicam uma maior polarização em direção às perspectivas predominantes da comunidade.
\end{itemize}

Com base nessas estatísticas, a probabilidade de formação de câmaras de eco pode ser calculada para cada comunidade. Para representar essa probabilidade, utilizamos a $FCE$ que é calculada como uma função exponencial das estatísticas descritivas, ponderadas pelos parâmetros beta correspondentes. Para interpretar as métricas obtidas e chegar a uma heuristica que determine quais comunidades dentro de uma rede provavelmente são câmaras de eco, realizamos uma análise para identificar os componentes principais.

\begin{figure}[htb]
	\centering
	\includegraphics[width=0.7\textwidth]{images/echo_chamber_strength_pairplot.png}
	\caption{Distribuição da $FCE$ e das métricas utilizadas para o cálculo da $FCE$ em uma rede simulada}
	\label{fig:echo_chamber_strength_pairplot}
\end{figure}

\subsection{Análise de Componentes Principais}

O \sigla{$PCA$}{Principal Component Analysis} é uma técnica estatística de redução de dimensionalidade amplamente utilizada para identificar padrões em dados de alta dimensão, revelando as correlações subjacentes. No contexto das câmaras de eco, o $PCA$ pode ajudar a desvendar quais métricas têm maior importância ou influência na determinação da presença de câmaras de eco em uma rede. Isso pode ser útil para identificar quais métricas são mais eficazes na detecção de câmaras de eco e quais métricas podem ser descartadas para simplificar o modelo. A analise revelou quatro componentes principais:

\begin{itemize}
	\item PC1 parece capturar principalmente a $FCE$, a densidade da comunidade e a exposição média. Isso indica que comunidades com maior densidade e usuários com menor exposição média têm uma forte correlação com a presença de câmaras de eco.
	\item PC2 destaca a homogeneidade das opiniões e as conexões externas, porém em direções opostas. Comunidades com homogeneidade elevada e baixas conexões externas estão associadas à formação de câmaras de eco.
	\item PC3 ressalta as conexões externas e o $ECC$ em direções opostas. Indicando que comunidades com baixas conexões externas, mas um alto $ECC$, podem ter propensão a ser câmaras de eco.
	\item PC4 tem forte peso na homogeneidade das opiniões e no $ECC$, ambos em direções opostas. Isso sugere que a presença de influenciadores e uma homogeneidade alta na opinião podem ser fatores determinantes para a formação de câmaras de eco.
\end{itemize}

\begin{figure}[htb]
	\centering
	\includegraphics[width=0.7\textwidth]{images/mesquita_pca.png}
	\caption{Visualização $PCA$ das comunidades da rede de mesquita destacando os componentes principais para formação de câmaras de eco.}
	\label{fig:mesquita_pca}
\end{figure}

Na figura \autoref{fig:mesquita_pca} podemos observar um scatterplot dos componentes ou métricas analisadas e a sua distribuição nos eixos dos dois componentes principais. O eixo x (PC1) captura a $FCE$, densidade da comunidade e exposição média dos usuários. O eixo y (PC2) destaca a homogeneidade das opiniões e conexões externas em direções opostas. Cada ponto azul representa uma comunidade, evidenciando correlações entre as características mencionadas. Baseado nesses resultados, podemos propor a seguinte heurística para identificar comunidades com maior probabilidade de serem câmaras de eco:

\begin{itemize}
	\item 1. Calcular os valores para todas as métricas para cada comunidade.
	\item 2. Ponderar cada métrica pelo peso correspondente ao $PCA$.
	\item 3. Ordenar as comunidades com base nos valores ponderados.
\end{itemize}

Comunidades com valores altos nesta classificação terão maior probabilidade de ser câmaras de eco. Isso ocorre porque, conforme indicado pelo $PCA$, estas comunidades teriam alta densidade, alta $FCE$ e baixa exposição média, todos fatores que contribuem para a formação de câmaras de eco.

Concluindo, ao incorporar os insights do $PCA$ com as métricas fornecidas, podemos desenvolver uma abordagem mais informada e precisa para identificar câmaras de eco em redes sociais. Esta heurística pode ser usada como uma ferramenta valiosa para moderadores e administradores de redes sociais, permitindo-lhes identificar e abordar proativamente comunidades que têm propensão a se tornarem câmaras de eco.

Essa abordagem ilustra a importância de combinar medidas globais e locais na detecção de câmaras de eco. Enquanto medidas globais como o $GEC$ fornecem uma visão geral da tendência da rede para formar câmaras de eco, medidas locais são necessárias para identificar câmaras de eco específicas e entender a dinâmica dentro dessas comunidades.

\subsection{Derivando parâmetros beta}

Na implementação da classe EchoChamberDetector, a escolha dos valores dos parâmetros beta é um aspecto crucial para a eficácia do modelo. Esses parâmetros, que quantificam a influência de várias estatísticas descritivas na probabilidade de formação de câmaras de eco, podem ser derivados de várias maneiras, dependendo do contexto específico e das características da rede. No caso da rede social Colab, os valores dos parâmetros beta podem ser informados pelas estatísticas da rede, conforme apresentado na \autoref{tab:colab_gephi_statistics}.

O parâmetro $\beta_1$, que representa a densidade da comunidade, pode ser informado pelo coeficiente de agrupamento médio da rede. Uma rede com um coeficiente de agrupamento médio alto tende a ter comunidades densas.

O parâmetro $\beta_2$, que representa a homogeneidade das opiniões, pode ser informado pelo número de comunidades na rede. Uma rede com muitas comunidades tende a ter opiniões mais homogêneas dentro de cada comunidade.

O parâmetro $\beta_3$, que representa as conexões externas, pode ser informado pelo número de componentes fracamente conectados na rede. Uma rede com muitos componentes fracamente conectados tende a ter menos conexões externas.

O parâmetro $\beta_4$, que representa o efeito dos influenciadores, pode ser informado pela mudança da soma da centralidade de eigenvector na rede. Uma rede com uma alta mudança da soma da centralidade de eigenvector tende a ter influenciadores mais influentes.

O parâmetro $\beta_5$, que representa a exposição média, pode ser informado pelo comprimento médio do caminho na rede. Uma rede com um comprimento médio de caminho longo tende a ter uma exposição média mais alta.

O parâmetro $\beta_6$, que representa o $GEC$, pode ser informado pela modularidade da rede. Uma rede com alta modularidade tende a ter um $GEC$ mais alto.

Finalmente, o parametro $\beta_7$ que representa o $ECC$, que é uma medida da contribuição individual do usuário em comunidades com tendência para formar câmaras de eco, que é inversamente proporciaonal ao número de nós em uma rede, visto que quanto mais diversificada for a rede, menor será a contribuição individual de cada usuário. Dessa forma, o parâmetro $\beta_7$ pode ser informado pelo número de nós na rede.

Os valores betas derivados desse racional podem ser aplicados a rede do Colab como um todo, mas precisam ser adaptados para diferentes topologias de rede. Por exemplo o valor para numero de comunidades e o valor para numero de componentes fracamente conectados mudam drasticamente ao comprar cidades como Caruarú, Recife e Niterói, por exemplo. Dessa forma, incorporamos o cálculo desses racionais na implementação da classe \texttt{EchoChamberDetector} que calcula esses valores iniciais com base nas estatísticas da rede dado um grafo G. Isso permite uma abordagem mais informada e adaptativa para a detecção de câmaras de eco, que pode ser ajustada para diferentes redes e contextos. Para testar a classe, criamos um modelo aleatório que gera um grafo de rede social com pelo menos uma câmara de eco.

\subsection{Modelo de classificação}

\tikzstyle{startstop} = [rectangle, rounded corners, minimum width=3cm, minimum height=1cm, text centered, draw=black, fill=red!30]
\tikzstyle{process} = [rectangle, minimum width=3cm, minimum height=1cm, text centered, text width=3cm, draw=black, fill=orange!30]
\tikzstyle{decision} = [diamond, minimum width=3cm, minimum height=1cm, text centered, draw=black, fill=green!30]
\tikzstyle{arrow} = [thick,->,>=stealth]

\begin{tikzpicture}[node distance=2cm]

	\node (start) [startstop] {Derivar Parâmetros Beta};
	\node (partition) [process, below of=start] {Particionamento do Grafo em Comunidades};
	\node (compute) [process, below of=partition] {Computar $FCE$ para cada comunidade};
	\node (density) [process, below of=compute, xshift=-2cm] {Calcular Densidade};
	\node (homogeneity) [process, below of=compute, xshift=2cm] {Calcular Homogeneidade de Opiniões};
	\node (external) [process, below of=density] {Calcular Conexões Externas};
	\node (influencers) [process, below of=homogeneity] {Calcular Influenciadores};
	\node (ecc) [process, below of=external] {Calcular $ECC$};
	\node (exposure) [process, below of=influencers] {Calcular Exposição Média};
	\node (force) [process, below of=ecc] {Calcular $FCE$};
	\node (classify) [process, below of=force] {Classificar câmaras de eco utilizando aprendizagem de máquina};
	\node (end) [startstop, below of=classify] {Retornar comunidades identificadas como câmaras de eco};

	\draw [arrow] (start) -- (partition);
	\draw [arrow] (partition) -- (compute);
	\draw [arrow] (compute) -- (density);
	\draw [arrow] (compute) -- (homogeneity);
	\draw [arrow] (compute) -- (external);
	\draw [arrow] (compute) -- (influencers);
	\draw [arrow] (compute) -- (ecc);
	\draw [arrow] (compute) -- (exposure);
	\draw [arrow] (compute) -- (force);
	\draw [arrow] (force) -- (classify);
	\draw [arrow] (classify) -- (end);

	\draw [arrow] (density) -- ++(0,-1.2) -| (compute);
	\draw [arrow] (homogeneity) -- ++(0,-1.2) -| (compute);
	\draw [arrow] (external) -- ++(0,-1.2) -| (compute);
	\draw [arrow] (influencers) -- ++(0,-1.2) -| (compute);
	\draw [arrow] (ecc) -- ++(0,-1.2) -| (compute);
	\draw [arrow] (exposure) -- ++(0,-1.2) -| (compute);

\end{tikzpicture}

A detecção de câmaras de eco em redes sociais é uma tarefa complexa e muitas vezes não-determística. Ao refletir a modelagem de software para uma \textit{pipeline} de detecção, compartimentalizamos o processo em duas etapas: A primeira etapa envolve a análise das métricas da topologia da rede, incluindo a identificação de comunidades e o cálculo de heurísticas específicas para cada comunidade. Isso nos permite derivar o componente $FCE$. Quanto mais elevado esse componente em uma comunidade, maior a probabilidade de que ela seja uma câmara de eco, como evidenciado por nossa análise de Componentes Principais ($PCA$). Com base nessa análise, comunidades que apresentam valores elevados nesse componente tendem a exibir alta homogeneidade entre seus membros, densidade moderada e baixa exposição média. Além disso, a presença de influenciadores e conexões externas pode influenciar positivamente esse componente.

No entanto, essa métrica sozinha não é suficiente para identificar eficazmente as câmaras de eco, uma vez que a distribuição dos valores da métrica pode se sobrepor entre comunidades que são e não são câmaras de eco. Baseado em observações empíricas, comunidades que são câmaras de eco tendem a apresentar picos pronunciados na métrica da $FCE$ em comparação com outras comunidades que não são câmaras de eco, porém é difícil definir um limiar para essa métrica que separe efetivamente as duas classes sem considerar informações sobre o tamanho da rede, densidade, etc. Além disso, essa métrica não é adequada para identificar se uma rede não possui nenhuma câmara de eco. Inicialmente, exploramos várias heurísticas, como definir limiares, desvios-padrão, clusterização, k-means e análise de picos, mas nenhuma delas demonstrou um desempenho satisfatório, medido pelo F1 Score. Alguns modelos performavam bem com \textit{true-positives} mas tinham alta taxa de \textit{false-negatives}, enquanto outros tinham o oposto. Isso indica que a abordagem de modelagem precisa ser adaptativa e considerar tanto as características de topologia da rede quanto a complexidade das interações entre os membros da comunidade mesmo quando comunidades têm um alto valor em $FCE$.

Para mitigar esses problemas, a segunda etapa do processo envolve a aplicação de um modelo de classificação baseado em aprendizagem de máquina, para determinar se uma comunidade é uma câmara de eco ou não. O algoritmo do modelo pode ser ajustado utilizando o design pattern \textit{strategy} e para a classificação, utilizamos um modelo baseado em \texttt{VotingEnsamble}, mas primariamente utilizando o classificador de florestas aleatórias. Esse algoritmo foi escolhido devido à sua capacidade de lidar com características complexas e interações entre variáveis. Utilizamos um conjunto de recursos, como tamanho do grafo, $FCE$, Densidade, Homogeneidade, Conexões Externas, $ECC$, $GEC$, Exposição Média e Número de Membros, para treinar o modelo. A decisão é tomada com base na probabilidade calculada pelo modelo, onde valores próximos a 1 indicam uma alta probabilidade de a comunidade ser uma câmara de eco.

Portanto, ao combinar a análise das métricas derivadas da topologia da rede com um modelo de classificação de aprendizado de máquina, conseguimos identificar de maneira mais precisa as comunidades que são predominantemente câmaras de eco na rede. Essa abordagem se mostrou mais consistente que outras heurísticas testadas, com o objetivo de produzir uma detecção mais confiável e eficaz.

\section{Simulações de Câmaras de Eco com Modelos Aleatórios}

A análise de redes sociais é um campo que se beneficia significativamente da aplicação de modelos estatísticos, e um dos mais influentes é o modelo Markoviano. Nomeado em homenagem ao matemático russo Andrey Markov, este modelo tem sido uma ferramenta fundamental em várias disciplinas desde o início do século XX. A característica definidora de um modelo Markoviano é a sua propriedade de "sem memória", onde a probabilidade de transição para um estado futuro depende exclusivamente do estado presente, independentemente de como o sistema chegou ao seu estado atual.

Esta propriedade de "sem memória" simplifica a análise e a computação de sistemas complexos, tornando os modelos Markovianos uma escolha atraente para uma variedade de aplicações. No entanto, a aplicação dos modelos Markovianos na análise de redes sociais é particularmente interessante. Neste contexto, os nós da rede representam indivíduos ou entidades, e as arestas representam relações ou interações entre eles. Os modelos Markovianos podem ser usados para modelar a evolução dessas redes ao longo do tempo, levando em conta a dependência entre as interações.

Uma extensão desses modelos, conhecida como \sigla{$ERGM$}{Modelos Exponenciais de Grafos Aleatórios}, permite a modelagem de dependências mais complexas entre as arestas, capturando assim a estrutura de interação global da rede. Estes modelos representam uma abordagem inovadora para a análise de redes sociais, oferecendo vantagens significativas em relação aos métodos tradicionais de análise de grafos. Enquanto as técnicas tradicionais tendem a se concentrar em propriedades individuais dos nós ou arestas, os $ERGMs$ permitem a modelagem de dependências complexas entre as arestas, capturando assim a estrutura de interação global da rede.

Os $ERGMs$ são particularmente úteis para modelar fenômenos sociais complexos, como a formação de câmaras de eco. A análise das redes da Colab, até o presente momento, tem sido conduzida utilizando técnicas convencionais de análise de redes, tais como medidas de centralidade e modularidade, aplicadas a capturas estáticas da rede. Essas capturas representam o estado da rede em pontos específicos no tempo, fornecendo uma visão instantânea das conexões entre os usuários. O modelo de dados disponibilizado pelo Colab, apresentado em detalhes na \autoref{sec:colab_data_analysis}, consiste em uma lista de arestas que representam as conexões entre os usuários, juntamente com informações temporais que indicam quando essas conexões foram estabelecidas e possivelmente removidas. Além disso, cada usuário é caracterizado por uma série de atributos, como descrito na \autoref{tab:user_model}. A estrutura desses dados, que capturam tanto a topologia da rede quanto as características dos usuários, torna a análise baseada em $ERGMs$ particularmente apropriada, pois permite a modelagem de dependências complexas entre as arestas, proporcionando uma representação mais precisa da estrutura de interação global da rede.

Nesse contexto, os , uma extensão dos modelos Markovianos, oferecem uma abordagem metodológica robusta. Os $ERGMs$ permitem a modelagem de dependências complexas entre as arestas, proporcionando uma representação mais precisa da estrutura de interação global da rede, mesmo em uma visão estática. Esta capacidade é particularmente relevante para a detecção de câmaras de eco, fenômenos caracterizados por comunidades altamente interconectadas dentro de uma rede, onde a informação circula predominantemente dentro do grupo. A literatura recente fornece vários exemplos de como os $ERGMs$ podem ser usados para modelar a formação de câmaras de eco. Por exemplo, \citeonline{2022_Sun} usaram o $ERGM$ para calcular o efeito da câmara de eco e o efeito de três mecanismos de interação na câmara de eco. Eles descobriram que o comportamento de imitação e interação entre grupos está positivamente relacionado ao efeito da câmara de eco.

\subsection{Modelando a rede do Colab com \textit{ERGM}}

O propósito de utilizar simulações $ERGM$ neste estudo é alcançar a capacidade de criar redes sociais sintéticas que se assemelhem à rede Colab afim de testar e otimizar as heurísticas de detecção de câmaras de eco. Para atingir esse objetivo, implementamos um modelo $ERGM$ para descrever a rede social Colab, permitindo assim a identificação e o estudo detalhado das câmaras de eco presentes nessa rede.  O modelo é especificado usando a notação formulaica do pacote $ERGM$, onde a fórmula define as estatísticas descritivas que descrevem a estrutura da rede e as preferências dos usuários. A seguir apresentamos a fórmula do modelo $ERGM$ proposto para a rede social Colab:

Seja $G = (V, E)$ o grafo da rede social, onde $V$ é o conjunto de nós (usuários) e $E$ é o conjunto de arestas (conexões). Queremos modelar a probabilidade de ocorrência desse grafo com base em diferentes características e atributos dos nós e das arestas. A equação do modelo é definida como:

\[
	P(G) = \frac{e^{\sum_i \beta_i s_i(G)}}{C(\boldsymbol{\beta})}
\]

em que $P(G)$ é a probabilidade de ocorrência do grafo $G$, $\beta_i$ são os parâmetros do modelo, $s_i(G)$ são as estatísticas do modelo e $C(\boldsymbol{\beta})$ é uma constante de normalização.

As estatísticas do modelo capturam diferentes aspectos da rede social e são representadas por $s_i(G)$. Neste modelo específico, as estatísticas incluem:

\begin{itemize}
	\item $s_{\text{edges}}(G)$: O número de arestas presentes no grafo $G$. Essa estatística captura a formação de conexões entre os usuários.

	\item $s_{\text{nodematch("location")}}(G)$: O número de pares de nós que compartilham a mesma localização. Essa estatística reflete a tendência dos usuários em seguir outros usuários da mesma cidade.

	\item $s_{\text{istar(1)}}(G)$: O número de estrelas com um nó central em cada nó. Essa estatística modela a formação de conexões em torno de um usuário central.

	\item $s_{\text{ostar(1)}}(G)$: O número de estrelas com um nó central apontando para cada nó. Essa estatística captura a formação de conexões direcionadas para um usuário central.

	\item $s_{\text{nodematch("age")}}(G)$: O número de pares de nós que compartilham a mesma faixa etária. Essa estatística reflete a preferência dos usuários em seguir outros usuários da mesma idade.

	\item $s_{\text{gwidegree}}(G)$: A distribuição dos graus de entrada ponderados dos nós na rede. Essa estatística modela a propagação de popularidade entre os usuários.

	\item $s_{\text{gwodegree}}(G)$: A distribuição dos graus de saída ponderados dos nós na rede. Essa estatística captura a propagação de atividade entre os usuários.

	\item $s_{\text{nodefactor("age")}}(G)$: A distribuição dos fatores dos nós relacionados à idade. Essa estatística reflete a preferência dos usuários com base na idade.

	\item $s_{\text{gwdsp}}(G)$: A distribuição das distâncias geodésicas entre os nós na rede. Essa estatística considera a proximidade entre os usuários na formação de conexões.

	\item $s_{\text{mutual}}(G)$: O número de conexões mútuas presentes no grafo $G$. Essa estatística captura a existência de conexões bidirecionais entre os usuários.
\end{itemize}

O modelo proposto busca capturar características específicas do Colab e sua dinâmica de rede social colaborativa.

A inclusão da estatística "edges" (arestas) no modelo reflete a importância das conexões entre os usuários. No Colab, os usuários interagem uns com os outros por meio de conexões, seguindo e sendo seguidos. Essas conexões representam o fluxo de informações e colaboração na plataforma. Portanto, é essencial considerar o número de arestas presentes no grafo para modelar a formação dessas conexões.

A estatística "nodematch("location")" (correspondência de nós por localização) foi incluída para capturar a tendência dos usuários em seguir outros usuários da mesma cidade. No Colab, os usuários tendem a se conectar e colaborar com pessoas que estão geograficamente próximas a eles. Isso ocorre porque a colaboração local pode facilitar encontros pessoais, discussões presenciais e o desenvolvimento de projetos conjuntos. Portanto, ao considerar a localização como um fator de correspondência entre os nós, o modelo leva em conta essa preferência dos usuários por conexões locais.

As estatísticas "istar(1)" e "ostar(1)" foram adicionadas para modelar a formação de conexões em torno de um usuário central e a existência de conexões direcionadas para um usuário central, respectivamente. No contexto do Colab, certos usuários podem desempenhar um papel central na rede, sendo altamente conectados e influentes. Esses usuários centrais são frequentemente seguidos por outros usuários e podem ser fontes de informações, ideias e colaborações. Portanto, é relevante considerar a presença de conexões centradas em nós e conexões direcionadas para nós centrais no modelo.

A inclusão da estatística "nodematch("age")" (correspondência de nós por idade) se baseia na observação de que os usuários do Colab podem ter preferências em seguir outros usuários da mesma faixa etária. Isso pode ocorrer devido a interesses comuns, experiências compartilhadas ou abordagens semelhantes para a colaboração. Ao considerar a idade como um fator de correspondência entre os nós, o modelo leva em conta essa preferência dos usuários por conexões com pessoas da mesma faixa etária.

As estatísticas "gwidegree" e "gwodegree" foram incluídas para modelar a propagação de popularidade e atividade na rede do Colab, respectivamente. No contexto da plataforma, certos usuários podem ser mais populares e ativos do que outros. A popularidade pode ser medida pelo número de seguidores de um usuário, enquanto a atividade pode ser quantificada pelo número de pessoas que um usuário segue. Essas estatísticas capturam a disseminação da popularidade e da atividade entre os usuários, refletindo a dinâmica de influência e engajamento social no Colab.

A estatística "nodefactor("age")" foi adicionada para capturar a distribuição dos fatores relacionados à idade dos usuários. No Colab, a idade pode ser um fator importante que influencia as preferências e comportamentos dos usuários. Ao considerar os fatores relacionados à idade dos usuários como uma variável categórica, o modelo pode levar em conta possíveis diferenças nas interações e padrões de conexão com base na idade dos usuários.

Por fim, a estatística "gwdsp" (distância geodésica ponderada) foi incluída para modelar a influência do caminho geodésico na formação de conexões no Colab que representa a menor distância entre dois nós na rede. Ao considerar a distância geodésica ponderada, o modelo pode capturar o efeito da proximidade e acessibilidade na formação de conexões. Isso é relevante, pois usuários mais próximos podem estar mais propensos a interagir e colaborar uns com os outros.

Ao combinar todas essas estatísticas em um único modelo, buscamos capturar as características específicas da rede social colaborativa do Colab. Essas escolhas foram baseadas em observações e conhecimentos prévios sobre a dinâmica da plataforma, levando em consideração a importância das conexões, localização, usuários centrais, preferências relacionadas à idade, popularidade, atividade, fatores categóricos e distâncias geodésicas. O modelo visa fornecer uma representação adequada e abrangente da rede do Colab, permitindo análises mais detalhadas sobre sua estrutura e dinâmica.

\subsubsection*{Implementação de $ERGM$}

Criamos um conjunto de algoritmos em R descrevevendo a implementaçao do modelo $ERGM$ criado a partir de um sub-sampling do modelo de dados original do Colab contendo 10.000 vértices. Esse grafo original foi utilizado para simulação de grafos de rede com a intenção de converger em um grafo com estrutura similar ao grafo original. O treinamento do modelo funciona por meio de iterações. Cada iteração é uma etapa em que o modelo atualiza os parâmetros com base nos dados observados da rede e compara com as redes simuladas. O objetivo é encontrar um conjunto de parâmetros que melhor explique a estrutura da rede observada.

No início do processo, definimos a fórmula do modelo, que especifica quais características da rede estamos considerando e como elas influenciam a probabilidade dessas características ocorrerem. Durante as iterações, o modelo avalia quão bem a rede observada se ajusta às redes simuladas com base na fórmula do modelo. Ele ajusta os parâmetros para melhorar a correspondência entre a rede observada e as redes simuladas. O processo continua até que o modelo atinja a convergência, ou seja, os parâmetros alcancem um estado estável em que iterações adicionais não resultem em mudanças significativas nos valores dos parâmetros. Ao final das iterações, obtemos as estimativas finais dos parâmetros, que refletem a influência de cada termo na fórmula do modelo na probabilidade de ocorrerem determinadas características na rede. Essas estimativas nos fornecem insights sobre os processos subjacentes que moldam a estrutura da rede, como a formação de conexões entre os usuários com base na localização ou a preferência por seguir outros usuários da mesma faixa etária.

Após desenvolver o modelo baseado em $ERGM$, realizamos uma série de experimentos para avaliar sua adequação e desempenho. Primeiro, executamos o modelo em nossa rede social colaborativa do Colab, usando o conjunto de dados completo com 10.000 vértices. Em seguida, executamos o modelo em uma série de redes simuladas, geradas a partir do conjunto de dados original. Por fim, comparamos os resultados do modelo com os dados observados e as redes simuladas, avaliando a adequação do modelo e a qualidade de ajuste. A Figura \ref{fig:ergm_diagnostics} mostra os resultados do diagnóstico do modelo, que inclui a comparação entre os dados observados e as redes simuladas, bem como a distribuição dos parâmetros estimados. A comparação entre os dados observados e as redes simuladas mostra que o modelo se ajusta bem aos dados, pois a rede observada está dentro do intervalo de confiança de 85\% das redes simuladas. Além disso, a distribuição dos parâmetros estimados mostra que todos os parâmetros são significativos, pois seus valores estão fora do intervalo de confiança dos parâmetros simulados. Esses resultados indicam que o modelo é adequado para representar a estrutura da rede social do Colab.

\begin{quadro}[!htb]
	\caption{Diagnóstico do modelo $ERGM$}
	\label{fig:ergm_diagnostics}
	\centering
	\includegraphics[scale=0.5]{images/ergm_diagnostics.png}
	\fautor
\end{quadro}

Após a convergência do modelo, adotamos uma abordagem baseada em simulação para avaliar a capacidade do modelo $ERGM$ em gerar redes que mimetizassem a estrutura da rede social do Colab. Usando o código em R, geramos vários "edgelists" de redes aleatórias. Estas redes simuladas não apenas representaram conexões entre usuários, mas também incorporaram eventos aleatórios, como curtidas e votos, indicadores de engajamento; e comentários, indicadores de diálogo; em diferentes tipos de eventos relacionados de zeladoria pública. Essa simulação foi essencial para entender o comportamento dinâmico dos usuários na plataforma. Por exemplo, é comum em redes sociais que usuários interajam mais frequentemente com postagens que são relevantes para suas próprias crenças ou interesses em sua proximidade geográfica. Ao introduzir eventos aleatórios, buscamos replicar esse comportamento com o objetivo de que as redes simuladas se assemelhassem, tanto quanto possível, à rede real do Colab.

\subsection{Visualizando Câmaras de Eco em redes simuladas}

De volta ao Python, podemos utilizar a classe \texttt{NetworkPlotter} para exibir a rede com os nós colorizados baseado nas comunidades identificadas. A \autoref{fig:ergm_random_network} ilustra o plot de uma das redes geradas. Ao observar a imagem da rede simulada, é possível identificar diversas características que podem ser correlacionadas com o comportamento observado nas redes do Colab:

\begin{itemize}
	\item \textbf{Clusters Distintos:} A presença de grupos coloridos distintos na visualização sugere a formação de comunidades. Alguma das comunidades podem ser consideradas câmaras de eco, especialmente quando indivíduos com interesses ou opiniões semelhantes tendem a interagir mais entre si do que com outros membros da rede.
	\item \textbf{Conexões Inter-Comunidades:} Ainda que existam clusters distintos, é notável que há diversas conexões entre eles. Isto pode indicar a presença de usuários "ponte", que interagem com múltiplas comunidades e podem ser essenciais na difusão de informações entre grupos diferentes.
	\item \textbf{Densidade Variável:} Algumas comunidades parecem ser mais densamente conectadas do que outras. Isto pode ser um indicativo de grupos mais ativos ou com maior engajamento na plataforma. No contexto do Colab, isso poderia representar comunidades mais engajadas em discussões ou atividades de zeladoria pública.
	\item \textbf{Centralidade Variável}: A \textit{eigencentrality}, representada pelo tamanho dos nós, indica que existem alguns usuários que são mais centrais do que outros. Isto pode ser um indicativo de usuários com maior influência na rede, que podem ser considerados líderes de opinião ou influenciadores.
\end{itemize}

\begin{figure}[!htb]
	\caption{Plot de um rede aleatória gerada pelo modelo $ERGM$ baseado na topologia da rede do Colab}
	\label{fig:ergm_random_network}
	\centering
	\includegraphics[width=0.9\textwidth]{images/ergm_random_network.png}
	\fautor
\end{figure}

\begin{figure}[!htb]
	\caption{Plot da topologia da rede de usuários Colab no Rio de Janeiro}
	\label{fig:network_plot_rio_de_janeiro}
	\centering
	\includegraphics[width=0.9\textwidth]{images/network_plot_rio_de_janeiro.png}
	\fautor
\end{figure}

Estas observações reforçam a importância de considerar não apenas as conexões diretas entre os usuários, mas também suas interações com conteúdos e eventos. Para plataformas como o Colab, entender estas dinâmicas pode ser crucial para promover maior engajamento, identificar líderes de opinião ou detectar áreas de interesse emergente. Para uma validação mais empírica do modelo $ERGM$, podemos comparar a topologia da rede simulada com a topologia da rede real. A \autoref{fig:network_plot_rio_de_janeiro} ilustra a topologia da rede do Colab no Rio de Janeiro, enquanto a \autoref{fig:ergm_random_network} ilustra a topologia de uma rede simulada. Ao comparar as duas imagens, é possível observar que a rede simulada apresenta algumas características semelhantes à rede real:

\begin{itemize}
	\item \textbf{Concentração de Interações:} Em ambas as imagens, pode-se observar que há uma concentração significativa de nós (usuários) no centro da rede, cercados por conexões densas. Isso sugere que há usuários ou postagens específicas que são particularmente populares ou influentes na plataforma, atraindo uma quantidade desproporcional de interações. Estes poderiam ser chamados de "hubs" ou influenciadores dentro da rede.
	\item \textbf{Comunidades:} Em ambas as imagens também se observa um padrão semelhante na distribuição da comunidade, tanto em sua densidade, número de nós e nos aspectos de centralidade, destacando poucos usuários com grande influência na rede.
	\item \textbf{Hubs vs. Periferia:} Enquanto há uma densa concentração de interações no centro, a periferia da rede apresenta nós menos conectados. Estes nós periféricos podem representar novos usuários, usuários menos ativos ou postagens menos populares.
\end{itemize}

A rede simulada baseada no modelo $ERGM$ parece replicar com sucesso muitas das características observadas na rede real do Colab. A capacidade de simular redes sociais, especialmente aquelas com características complexas e interações multifacetadas, é fundamental para avançar na pesquisa e análise de redes. A rede simulada, com base no modelo $ERGM$, ilustra a potência desta abordagem, ao conseguir reproduzir muitas das características intrínsecas observadas na rede real do Colab. Esta correspondência não é apenas uma validação da precisão do modelo $ERGM$, mas também destaca sua relevância prática na modelagem de redes sociais.

O emprego de simulações, como aquelas produzidas pelo modelo $ERGM$, oferece vários benefícios para os analistas de dados. Primeiramente, permite testar e validar heurísticas em ambientes controlados, reduzindo os riscos associados ao trabalho com dados reais, que podem ser ruidosos, incompletos ou distorcidos. Além disso, ao trabalhar com simulações, os analistas podem se distanciar de potenciais vieses inerentes aos dados reais, garantindo uma análise mais objetiva e imparcial. Este distanciamento das simulações dos dados reais é particularmente importante em contextos onde as interações e os padrões de comportamento dos usuários podem ser sensíveis ou privados. Ao utilizar redes simuladas, os pesquisadores podem conduzir experimentos e análises sem comprometer a privacidade ou a integridade dos usuários reais da aplicação.

A capacidade de simular redes que se assemelham às reais destaca a importante sinergia entre engenharia de software e análise de dados. Ao criar pipelines que facilitam a validação de heurísticas sociais complexas, a engenharia de software desempenha um papel crucial em potencializar a análise em contextos de democracia digital. Essas simulações proporcionam aos analistas de dados uma ferramenta robusta para coletar métricas relevantes a partir de redes simuladas, como a detecção de câmaras de eco, sem se envolver diretamente com dados reais. Isso, por sua vez, aprimora a confiabilidade das análises e otimiza os processos de tomada de decisão baseados em dados.

A utilização do modelo $ERGM$ é um exemplo desta interseção entre engenharia, simulação e análise. A eficácia em replicar características observadas em redes reais, como a do Colab, enfatiza a importância de ferramentas e técnicas multidisciplinares para enriquecer e refinar a pesquisa e prática em análise de redes sociais.

\begin{figure}[!htb]
	\caption{Plot da rede $ERGM$ com câmaras de eco identificadas}
	\label{fig:ergm_random_network_echo_chambers}
	\centering
	\includegraphics[width=0.9\textwidth]{images/ergm_random_network_echo_chambers.png}
	\fautor
\end{figure}

Para visualizar as câmaras de eco identificadas, criamos a visualização com diferenciação cromática clara, onde cada cor distinta representa uma câmara de eco específica e os nós em branco correspondem a atores ou entidades fora dessas câmaras. Esta representação gráfica foi concebida para permitir uma rápida identificação e análise das estruturas subjacentes destas câmaras de eco, realçando as suas relações, centralidade e densidade dentro da rede global. Na simulação da rede representada pela \autoref{fig:ergm_random_network_echo_chambers}, podemos identificar três câmaras de eco distintas, representadas pelas cores vermelha, azul e lilás.

Ao focarmos no grupo vermelho, observamos que esta câmara de eco, sendo artificialmente inserida, situa-se isolada dos demais grupos, indicando uma certa polarização e falta de interconexão com outros atores da rede. O seu posicionamento, juntamente com a proximidade entre os nós, sugere uma alta densidade interna, denotando uma possível homogeneidade de opiniões ou interações frequentes entre os membros.

Em contraste, a câmara de eco azul exibe uma estrutura mais dispersa e um posicionamento mais central na rede, o que pode indicar uma influência ou conexão mais ampla com outros atores, mesmo aqueles fora de sua câmara. Além disso, a variação no tamanho dos nós dentro dessa comunidade sugere uma hierarquia ou diferenças na centralidade de eigenvector entre os seus membros.

A câmara de eco lilás, embora menor em comparação com a azul, possui uma distribuição interessante, estendendo-se em uma direção específica e contendo nós com centralidades variadas. Isso pode indicar uma subcomunidade em formação ou um grupo com opiniões mais diversificadas.

Em geral, enquanto a câmara de eco artificial vermelha apresenta um isolamento claro, as câmaras de eco identificadas pelo modelo mostram-se mais integradas à rede. Esta visualização destaca a complexidade e a interconexão das câmaras de eco no cenário digital, enfatizando a necessidade de abordagens analíticas refinadas para compreender plenamente as dinâmicas e implicações dessas estruturas em redes sociais.

A implementação dos modelos $ERGM$ permitiu gerar redes aleatórias que espelham com um certo grau de aleatoriedade, a rede de usuários do aplicativo Colab. Esta abordagem demonstrou ser fundamental, pois através dessas simulações foi possível validar com precisão e amadurecer o modelo de detecção de câmaras de eco previamente construído. A confiabilidade desse método se reflete na sua capacidade de discernir quais comunidades no Colab são potencialmente câmaras de eco, oferecendo assim uma ferramenta valiosa para aqueles que desejam investigar essas estruturas.

A adição do modelo de visualização, por sua vez, amplia o alcance desse método. Essa representação gráfica, interpretável empiricamente, não só beneficia os stakeholders do aplicativo Colab, como também serve como uma ferramenta instrutiva para agentes governamentais e analistas que buscam compreender a dinâmica dessas redes.

\section{Validação do Modelo de Detecção}

A partir das redes simuladas introduzidas no tópico anterior, realizamos uma série de intervenções controladas para emular as características de câmaras de eco observadas em cenários reais. Inicialmente, identificamos empiricamente, avaliando as métricas de densidade, centralidade e conexões externas, comunidades dentro da simulação que apresentavam maior predisposição a se tornarem câmaras de eco, com base em atributos estruturais e dinâmicas de interação. Em seguida, manipulamos de forma direcionada os parâmetros de centralidade, densidade, conexões externas e homogeneidade de opiniões dessas comunidades, com o objetivo de fortalecer artificialmente as características inerentes às câmaras de eco.

O propósito destas manipulações foi duplo. Primeiro, queríamos assegurar que todas as redes simuladas contivessem pelo menos uma representação clara de câmara de eco, servindo como um padrão de referência. Segundo, essa configuração controlada proporcionou um cenário ideal para avaliar a eficácia do nosso algoritmo de detecção. Com a presença de uma "câmara de eco artificial" claramente definida, poderíamos quantificar o sucesso do modelo em identificar tal estrutura dentro da rede.

A eficácia do modelo de detecção foi, portanto, avaliada com base em sua capacidade de identificar e classificar corretamente estas comunidades modificadas como câmaras de eco. A abordagem empregada assemelha-se à metodologia usada em treinamentos de modelos de classificação de aprendizado de máquina, onde a presença de um padrão conhecido é fundamental para avaliar a precisão do algoritmo. Por meio deste procedimento, buscamos não apenas validar nosso modelo de detecção, mas também estabelecer um protocolo replicável para futuras investigações no campo da detecção de câmaras de eco em redes sociais simuladas.

Das dez redes simuladas examinadas, o modelo demonstrou competência ao identificar, consistentemente, a câmara de eco artificial em todas as instâncias analisadas. Nota-se, além disso, que os parâmetros $GEC$, $ECC$, densidade, homogeneidade de opiniões e centralidade para as demais câmaras de eco identificadas se alinham adequadamente com os resultados antecipados em nossa proposição inicial.

Interessantemente, uma em carácter experimental, ao alterar o algoritmo de detecção de comunidades precedente à identificação de câmaras de eco, os resultados obtidos para as comunidades identificadas apresentam variações consideráveis. Quando adotado o algoritmo padrão, que utiliza uma função de qualidade ancorada em significância para determinar as partições das comunidades, os resultados tendem a ser mais consistentes, o que faz sentido considerando a natureza direcionada do grafo da rede.

No entanto, ao empregar o algoritmo de Louvain, que possui uma propensão a consolidar comunidades menores em agrupamentos maiores, usuários pertencentes as câmaras de eco artificialmente inseridos foram alocados em comunidades diferentes da sua comunidade original, de forma que, as métricas associadas às câmaras de eco para os usuários manipulados elevaram-se a tal ponto que o sistema identificou comunidades com dimensões até o dobro das originais como alta probabilidade de serem câmaras de eco.

Curiosamente, esse fenômeno amplificado parece estar correlacionado à centralidade dos usuários nas respectivas comunidades. Tal observação sugere que a centralidade pode desempenhar um papel crítico na forma como as câmaras de eco são percebidas e detectadas. Por outro lado, ao adotar o algoritmo de Girvan-Newman, que se caracteriza por sua capacidade de identificar comunidades sobrepostas e, no entanto, demanda um tempo computacional substancialmente mais extenso, observou-se uma granularidade reduzida nas comunidades. Contudo, devido à otimização do algoritmo para identificar essas sobreposições, a comunidade associada à câmara de eco original foi replicada e corretamente identificada pelo modelo de detecção.

Essas descobertas destacam a importância da escolha do algoritmo na detecção precisa de câmaras de eco em contextos simulados, portanto, torna-se imperativo ressaltar que o algoritmo de detecção foi essencialmente considerado "calibrado" para funcionar de forma otimizada em situações em que o grafo é direcionado e as comunidades são identificadas pelo algoritmo \texttt{SignificanceVertexPartition}. O algoritmo é singular por sua capacidade de quantificar a significância estatística das partições, proporcionando assim uma compreensão detalhada e robusta sobre a estrutura intrínseca das comunidades. Essa técnica avalia a qualidade das partições, comparando a densidade de arestas dentro das comunidades com aquela esperada sob um modelo nulo.

Não obstante, cabe ressaltar que a escolha pelo algoritmo de significância não foi predeterminada na concepção original do modelo, mas sim foi fruto de um processo experimental. Ao longo da pesquisa, vários algoritmos de detecção de comunidades foram avaliados, e o de significância emergiu como o mais apropriado, especialmente devido à sua conveniência e eficácia para grafos direcionados. Anteriormente, enfatizamos que muitos algoritmos de detecção de comunidades não são intrinsecamente otimizados para lidar com grafos direcionados, uma característica que pode comprometer a acurácia da detecção de câmaras de eco.

As simulações, portanto, revelaram que, enquanto o modelo de detecção de câmaras de eco consegue consistentemente identificar a câmara de eco artificialmente inserida quando empregando o algoritmo de significância de partição de comunidades, essa consistência pode variar consideravelmente com a adoção de outros algoritmos de detecção de comunidades. A íntima relação entre as heurísticas de detecção de comunidades e detecção de câmaras de eco não é surpreendente, visto que muitas das métricas empregadas para identificar câmaras de eco derivam de metodologias clássicas de detecção de comunidades.

Dessa forma, entendemos que, ao abordar a complexa tarefa de detecção de câmaras de eco, é indispensável ponderar sobre os aspectos topológicos da rede em análise. Escolher um algoritmo de partição de comunidades alinhado com essas características topológicas pode não apenas potencializar a precisão da detecção, mas também fornecer insights mais profundos sobre as dinâmicas subjacentes das redes sociais em estudo.

\subsection{Heurísticas de Validação}

Para avaliarmos o desempenho do modelo de detecção de câmaras de eco, empregamos algumas análises com os resultados da detecção nas redes simuladas. Ao todo foram realizadas 1000 simulações com gráficos aleatórios e uma quantidade aleatória de comunidades com alta $FCE$ foram inseridas artificialmente e conectadas aleatoriamente a um ou mais nós do grafo principal. A partir dos resultados das execuções, utilizamos métricas comuns de análise de F-Score para determinar a eficiência e eficácia do modelo. Essas métricas são fundamentais para a análise crítica e a compreensão da capacidade do modelo em distinguir corretamente entre comunidades que são câmaras de eco e aquelas que não são, tentando avaliar inclusive o desempenho em grafos que não possuem câmaras de eco artificialmente inseridas. A seguir, descrevemos as métricas utilizadas no contexto de detecção da câmaras de eco:

\begin{itemize}
	\title{Heurísticas de Validação}
	\item \textbf{True Positives (Verdadeiros Positivos):} Representa o número de casos em que o modelo classificou corretamente uma comunidade como uma câmara de eco quando ela realmente era. No contexto da detecção de câmaras de eco, um verdadeiro positivo ocorre quando o modelo identifica com precisão uma comunidade como uma câmara de eco e essa comunidade é realmente uma câmara de eco.
	\item \textbf{False Positives (Falsos Positivos):} Indica o número de casos em que o modelo erroneamente classificou uma comunidade como uma câmara de eco quando não era. Isso significa que o modelo detectou uma câmara de eco onde ela não existia.
	\item \textbf{False Negatives (Falsos Negativos):} Representa a quantidade de vezes em que o modelo falhou em reconhecer uma câmara de eco real. Em outras palavras, o modelo não detectou uma câmara de eco que estava presente nos dados.
	\item \textbf{True Negatives (Verdadeiros Negativos):} Indica o número de casos em que o modelo acertadamente não classificou uma comunidade como uma câmara de eco, quando de fato não era uma câmara de eco.
\end{itemize}

\begin{itemize}
	\title{Métricas de Desempenho}
	\item \textbf{Precision (Precisão):} É a proporção de verdadeiros positivos em relação ao total de positivos previstos pelo modelo. Isso mede a precisão das previsões positivas. Uma alta precisão significa que o modelo comete menos erros ao classificar comunidades como câmaras de eco.
	\item \textbf{Recall (Recall):} Também chamado de Sensibilidade, é a proporção de verdadeiros positivos em relação ao total de casos reais de câmaras de eco. Isso mede a capacidade do modelo de detectar todas as câmaras de eco presentes nos dados.
	\item \textbf{F1 Score:} É a média harmônica da precisão e do recall. É uma métrica útil quando se deseja equilibrar precisão e recall. Um valor alto de F1 Score indica um bom equilíbrio entre a capacidade do modelo de fazer previsões precisas e de identificar todas as câmaras de eco.
	\item \textbf{Specificity (Especificidade):} Também conhecida como Taxa de Verdadeiros Negativos, é a proporção de verdadeiros negativos em relação ao total de negativos previstos pelo modelo. Ou seja, mede a capacidade do modelo de identificar corretamente comunidades que não são câmaras de eco.
\end{itemize}

\begin{table}[!ht]
	\centering
	\caption{Resultados da Validação do Modelo de Detecção}
	\begin{tabular}{|l|l|}
		\hline
		Execuções                                       & 1000   \\ \hline
		Tamanho Mínimo dos Grafos                       & 100    \\ \hline
		Tamanho Máximo dos Grafos                       & 2000   \\ \hline
		Quantidade Média de Membros por Comunidade      & 8      \\ \hline
		Quantidade Mínima de Câmaras de Eco Detectáveis & 0      \\ \hline
		Quantidade Máxima de Câmaras de Eco Detectáveis & 13     \\ \hline
		Média de Câmaras de Eco por Execução            & 5      \\ \hline
		Quantidade Mínima de Membros das Câmaras de Eco & 3      \\ \hline
		Quantidade Máxima de Membros das Câmaras de Eco & 16     \\ \hline
		Verdadeiros Positivos                           & 490    \\ \hline
		Verdadeiros Negativos                           & 490    \\ \hline
		Falsos Positivos                                & 10     \\ \hline
		Falsos Negativos                                & 10     \\ \hline
		Precisão                                        & 0.98   \\ \hline
		Recall                                          & 0.98   \\ \hline
		Score F1                                        & 0.98   \\ \hline
		Specificidade                                   & 0.8334 \\ \hline
	\end{tabular}
\end{table}

\begin{figure}[htb]
	\centering
	\includegraphics[width=0.7\textwidth]{images/echo_chamber_model_validation.png}
	\caption{Resultados da validação do modelo de detecção de câmaras de eco}
	\label{fig:echo_chamber_model_validation}
\end{figure}

Os resultados do modelo demonstram um desempenho sólido na detecção de câmaras de eco. Com um alto número de verdadeiros positivos baixa taxa de falsos positivos, a precisão é bastante elevada. Além disso, o recall também é igualmente alto, indicando que o modelo é capaz de identificar a grande maioria das câmaras de eco presentes nas comunidades. O F1 Score, que é uma métrica que combina precisão e recall, também é notável, com um valor de 0.98. Isso reflete a capacidade equilibrada do modelo em lidar com ambas as classes, câmaras de eco e não câmaras de eco. A especificidade, que avalia a capacidade do modelo de identificar corretamente comunidades que não são câmaras de eco, apresenta um valor de 0.83. Isso indica que o modelo possui uma boa capacidade de evitar falsos positivos, embora haja um ligeiro espaço para melhoria nessa métrica. Em geral, os resultados sugerem que o modelo é eficaz na identificação de câmaras de eco, com um desempenho sólido em termos de precisão, recall e F1 Score. No entanto, como em qualquer modelo, é importante continuar aprimorando-o para abordar qualquer espaço para melhorias, especialmente na especificidade. A \autoref{fig:echo_chamber_model_validation} ilustra os resultados da validação modelo. A partir desses resultados, considerando a origem da rede em simulações baseadas nas redes do Colab, a modelagem de comportamentos baseada em agentes e a inserção artificial de comunidade altamente polarizadas, é possivel que haja um certo nível de \textit{overfitting}, porém nos sentimos confiantes para aplicar o modelo de detecção de câmaras de eco nas redes do Colab.

Contudo, é crucial entender a natureza quantitativa e orientativa do modelo. Enquanto o modelo pode indicar comunidades mais polarizadas com valores que desviam muito do padrão do grafo da rede como um todo, a intenção não é rotular definitivamente uma comunidade como sendo uma câmara de eco. Se trata de uma ferramenta para guiar a análise, e não para definir conclusões. A justificativa principal para esse racional remete ao experimento de trocar o método de partição de comunidades. Quando o modelo Girvan-Newman foi usado, usuários foram incluidos em comunidades que não foram identificadas como câmaras de eco pelo modelo de significância de partição de comunidades. Isso sugere que o modelo de detecção de câmaras de eco é sensível ao método de partição de comunidades, e que a escolha de um método de partição de comunidades deve ser feita com cuidado. Por tanto, considerando essas incertezas, entendemos que a decisão final sobre a natureza de uma comunidade deve ser feita por um analista humano, que pode levar em consideração outros fatores, como a natureza dos comentários e a atividade dos usuários.

Ao realizar essas análises mais definitivas, por mais elucidativa que seja a perspectiva visual, é essencial reconhecer suas limitações. A representação gráfica, embora intuitiva, não projeta todas as métricas derivadas durante a análise das comunidades e da detecção de câmaras de eco. A análise dos aspectos de centralidade, densidade, conexões externas, $ECC$ e $GEC$, quando comparada entre todas as comundiades da rede, pode oferecer insights muitas vezes sutis e profundamente informativos, portanto são indispensáveis para um entendimento mais robusto das origens e comportamentos das câmaras de eco em redes sociais. No próximo tópico, aprofundaremos a análise dessas métricas, dando destaque à sua relevância na compreensão das câmaras de eco nas redes de Niterói, Santo André e Mesquita.

\section{Estudo de caso das redes de Niterói, Santo André e Mesquita}

A complexidade e a dinâmica das redes sociais tornam essencial a utilização de ferramentas analíticas e representações gráficas para entender suas estruturas e comportamentos. As câmaras de eco, como fenômeno emergente nesses ambientes, desafiam os analistas a encontrar maneiras eficazes de identificá-las e compreendê-las. No entanto, a mera visualização dessas estruturas, embora poderosa, não é suficiente para capturar a totalidade do fenômeno. Como veremos nesta seção, uma investigação mais aprofundada das métricas específicas, como centralidade, densidade, conexões externas, $ECC$ e $GEC$, proporciona uma compreensão mais rica e nuanceada das câmaras de eco e sua manifestação em redes específicas. 

Neste estudo de caso, direcionamos nossa atenção para as redes de Niterói, Santo André e Mesquita, explorando as particularidades e insights revelados por essas métricas. A metodologia adotada consiste em passos discretos resumidos a seguir:

\begin{itemize}
	\item \textbf{Carregamento das Redes:} As redes das respectivas cidades, previamente isoladas no Gephi, foram carregadas no formato de dataframe edgelist, sendo posteriormente convertidas em grafos direcionados utilizando a biblioteca NetworkX.
	\item \textbf{Filtragem de Dados:} Conjuntos de dados contendo postagens sobre eventos de zeladoria pública, juntamente com seus respectivos comentários, likes e votações (positivas e negativas) foram incorporados. Uma filtragem subsequente assegurou que cada cidade possuísse eventos e interações pertinentes somente aos usuários de sua rede.
	\item \textbf{Particionamento de Comunidades:} Adotou-se o modelo SignificanceVertexPartition para discernir e segmentar as comunidades de cada cidade. Através deste processo, foi possível determinar o número de comunidades, bem como os tamanhos máximo, mínimo e médio dessas comunidades.
	\item \textbf{Derivação de Parâmetros:} Com base nas características topológicas do grafo, foram derivados os parâmetros beta, essenciais para a etapa subsequente de identificação.
	\item \textbf{Detecção de Câmaras de Eco:} Através do modelo de detecção, foram identificadas comunidades com elevada probabilidade de serem câmaras de eco. Este julgamento baseou-se em métricas de análise de rede, polaridade das postagens, uniformidade dos tipos de eventos, engajamento representado por likes e votações, bem como interações como comentários e postagens.
	\item \textbf{Visualização:} Por fim, as comunidades diagnosticadas como potenciais câmaras de eco foram destacadas no gráfico. Este destaque foi baseado tanto em coloração diferenciada quanto no dimensionamento dos nós, considerando fatores de centralidade.
\end{itemize}

\subsection{Santo André}

\begin{table}[ht]
	\centering
	\caption{Métricas de Detecção de Câmaras de Eco da Rede da Cidade de Santo André}
	\label{tab:echo-chamber-metrics-santo-andre}
	\begin{tabular}{l|l}
		\toprule
		\textbf{Topologia da Rede}          & \textbf{Valor}                   \\
		\midrule
		Nós                                 & 2102                             \\
		Arestas                             & 6430                             \\
		Nº de comunidades                   & 243                              \\
		\toprule
		\textbf{Conteúdo}                   & \textbf{Valor}                   \\
		\midrule
		Tamanho da maior comunidade         & 44                               \\
		Tamanho da menor comunidade         & 3                                \\
		Tamanho médio das comunidades       & 6.56                             \\
		Nº de comentários                   & 49912                            \\
		Nº de likes                         & 165563                           \\
		Nº de votos                         & 11222                            \\
		Modularidade do grafo               & 0.5438                           \\
		\midrule
		\textbf{Polarização e Homofilia}    &                                  \\
		\midrule
		Coeficiente Global de Câmara de Eco & 0.04619                          \\
		$FCE$ no Grafo                      & 1.0743                           \\
		Nº de Câmaras de Eco Detectadas     & 4                                \\
		\midrule
		\textbf{Comunidade}                 & \textbf{Índice de Câmara de Eco} \\
		\midrule
		Comunidade 233                      & 2.0471                           \\
		Comunidade 225                      & 2.0119                           \\
		Comunidade 160                      & 1.9558                           \\
		Comunidade 207                      & 1.9112                           \\
		\bottomrule
	\end{tabular}
\end{table}

A rede de Santo André, composta por 2.102 nós e 6.430 arestas, contém 243 comunidades distintas, indicando uma estrutura que promove tanto interconectividade quanto fragmentação. A alta modularidade da rede sugere a presença de grupos isolados de discussão, facilitando a formação de câmaras de eco onde as opiniões circulam sem contestação externa. A análise das métricas de conteúdo, incluindo 49.912 comentários, 165.563 curtidas e 11.222 votos, revela um alto nível de engajamento, com uma predominância de expressões de concordância passiva (curtidas) sobre envolvimento ativo (comentários). Este comportamento pode contribuir para a manutenção das câmaras de eco, onde a concordância é mais visível do que o debate.

Ao examinar a estrutura da rede, observamos que a densidade e a homogeneidade das comunidades desempenham papéis cruciais na formação de câmaras de eco. A densidade elevada de comunidades, refletida pelo parâmetro $\beta_1$, facilita a circulação e reforço de opiniões compartilhadas, enquanto a homogeneidade de opiniões, representada pelo parâmetro $\beta_2$, indica polarização potencial. A presença limitada de conexões externas ($\beta_3$) sugere um isolamento que pode agravar a formação de câmaras de eco. Influenciadores ($\beta_4$) têm uma influência moderada, não desafiando significativamente as opiniões predominantes. A exposição média elevada ($\beta_5$) reforça continuamente as opiniões existentes, dificultando a consideração de novas informações. O Coeficiente Global de Câmara de Eco ($GEC$) e a Força de Câmara de Eco ($FCE$) indicam uma tendência moderada da rede como um todo para formar câmaras de eco, com variações significativas a nível individual.

\subsubsection*{Comunidades identificadas como potenciais câmaras de eco}

\begin{table}[ht]
	\centering
	\caption{Resumo das Métricas de Câmaras de Eco das Comunidades em Santo André}
	\label{tab:community-metrics-santo-andre}
	\resizebox{\textwidth}{!}{%
		\begin{tabular}{lcccccc}
			\toprule
			\textbf{Comunidade} & \textbf{$FCE$} & \textbf{Densidade} & \textbf{Homogeneidade} & \textbf{Conexões Externas} & \textbf{$ECC$} & \textbf{Exposição Média} \\
			\midrule
			233                 & 1.2281         & 0.1667             & 0.2700                 & 0.3750                     & 0.2700         & 0.2188                   \\
			225                 & 1.2342         & 0.1429             & 0.6626                 & 0.5714                     & 0.6626         & 0.1741                   \\
			160                 & 1.2093         & 0.1905             & 0.3244                 & 0.3846                     & 0.3244         & 0.1810                   \\
			207                 & 1.1850         & 0.1429             & 0.6565                 & 0.2500                     & 0.6565         & 0.1648                   \\
			\bottomrule
		\end{tabular}
	}
\end{table}

O modelo de detecção de câmaras de eco identificou quatro comunidades como potenciais câmaras de eco. A análise dessas comunidades revela importantes nuances sobre o comportamento dos grupos dentro da rede. As métricas geradas a partir da aplicação dos parâmetros beta fornecem uma visão quantitativa que permite a identificação de padrões de interação e homogeneidade de opiniões que caracterizam as câmaras de eco.

A comunidade 233 apresenta um índice de câmara de eco elevado, indicando uma forte tendência à formação de um eco. Embora não seja particularmente densa, esta comunidade mostra uma moderada homogeneidade de opiniões, sugerindo que os membros tendem a compartilhar e reforçar visões similares. A baixa densidade pode indicar que, mesmo em redes menos interconectadas, a uniformidade de perspectivas pode prevalecer e contribuir para a formação de câmaras de eco. A proporção de conexões externas é moderada, sugerindo alguma abertura à influência externa, mas não suficiente para prevenir a formação de um eco isolado.

A comunidade 225, com um índice de câmara de eco ligeiramente menor que a comunidade 233, exibe uma homogeneidade de opiniões significativamente mais acentuada. Isso sugere uma forte tendência dos membros em reforçarem opiniões similares entre si. A densidade dessa comunidade não é alta, o que indica que a uniformidade de perspectivas não depende apenas da densidade da rede. Apesar da presença relativamente maior de conexões externas em comparação com a comunidade 233, estas ainda não são suficientes para evitar a formação de uma câmara de eco.

A análise revela que a Força de Câmara de Eco ($FCE$) não é diretamente proporcional à densidade ou homogeneidade das opiniões, mas resulta de uma combinação complexa dessas e outras variáveis. Compreender essas dinâmicas é crucial para avaliar a polarização da rede e a pressão social exercida sobre os membros, tanto pelos \textit{helpers} quanto pelos \textit{complainers}. O equilíbrio entre a exposição a opiniões divergentes e a tendência de permanecer dentro de uma câmara de eco homogênea é delicado, refletindo o potencial de polarização da rede como um todo.

\begin{table}[ht]
	\centering
	\caption{Resumo das Métricas de Câmaras de Eco das Comunidades em Santo André}
	\label{tab:community-barometer-metrics-santo-andre}
	\resizebox{\textwidth}{!}{%
		\begin{tabular}{lcccc}
			\toprule
			\textbf{Comunidade} & \textbf{Qtd. Usuários} & \textbf{Qtd. Eventos Criados} & \textbf{Score Médio} & \textbf{Persona Média} \\
			\midrule
			233                 & 3                      & 2                             & 0.0405               & 0.0                    \\
			225                 & 3                      & 4                             & -0.3004              & 0.0                    \\
			160                 & 4                      & 3                             & -0.3171              & 0.0                    \\
			207                 & 3                      & 8                             & -0.0747              & 1.0                    \\
			\bottomrule
		\end{tabular}
	}
\end{table}

Ao analisar as comunidades sob a perspectiva da pressão social, observa-se uma paisagem diversificada de opiniões e comportamentos cívicos, como refletido pelas métricas de persona e score de sentimento associados a eventos de zeladoria pública. Esta análise revela não apenas as preocupações prementes dos cidadãos, mas também as disposições emocionais e atitudes que estes possuem em relação a variados aspectos da gestão urbana. Através da interpretação dessas métricas, pode-se desenhar um panorama das dinâmicas comunitárias e a polarização existente nos tópicos abordados.

\begin{quadro}[htb]
	\centering
	\includegraphics[width=0.7\textwidth]{images/echo_chambers_santo_andre_social_pressure.png}
	\caption{Pressão Social nas Comunidades identificadas como câmara de eco em Santo André}
	\label{fig:echo_chambers_santo_andre_social_pressure}
\end{quadro}

A Comunidade 233, a qual manifesta uma inclinação para a persona \textit{helper}, como visto na atividades relacionadas à poda de árvores, com um score de sentimento ligeiramente positivo, reflete um envolvimento comunitário construtivo. A poda de árvores, embora possa parecer trivial, desempenha um papel crucial na manutenção da infraestrutura urbana e na prevenção de danos potenciais, sugerindo que os usuários dessa comunidade estão ativamente engajados em manter a ordem e estética de seu ambiente, além de prevenir problemas maiores.

A Comunidade 225, por outro lado, apresenta um cenário mais crítico. Com eventos tais como esgoto a céu aberto e fiscalização de obras particulares recebendo scores de sentimento fortemente negativos e uma persona consistente de \textit{helper}, revela uma comunidade profundamente insatisfeita e preocupada com questões de saúde pública e regulamentação urbana. O esgoto a céu aberto, por exemplo, não só representa um risco à saúde como também uma falha grave na infraestrutura urbana, o que ressalta a gravidade com que tais questões são vistas pelos cidadãos.

A Comunidade 160, emerge um mosaico mais complexo de reações cívicas. Aqui, encontramos desde a preocupação com a falta de sinalização e infraestrutura básica, como exemplificado pela placa de sinalização quebrada/inexistente, até questões ambientais e de saúde pública como focos de mosquito da dengue/zika. A persona média é igual para todos os eventos criados indicando uma postura mais de \textit{helper}. Interessantemente, eventos como mato alto apresentam um score de sentimento positivo, indicando que os usuários veem tais eventos como oportunidades de melhoria, ao invés de apenas problemas a serem denunciados. Em contraste, eventos como área com risco de deslizamento, que recebe um dos scores de sentimento mais negativos, ressaltam uma sensação de urgência e um apelo por ação imediata, embora pareça que a comunidade não se sente capaz de contribuir de maneira significativa para sua resolução, por exemplo, muitas postagens referenciam órgãos públicos e autoridades.

A Comunidade 207 também demonstra uma preocupação com a ordem social e saúde pública, como visto na reação a eventos como comércio aberto irregularmente durante a pandemia e aglomeração de pessoas, ambos com scores de sentimento negativos e uma persona unânime de \textit{complainer}. Tais dados sugerem uma vigilância ativa e uma disposição para denunciar comportamentos considerados irresponsáveis ou perigosos durante um período de crise de saúde pública.

\begin{quadro}[htb]
	\centering
	\includegraphics[width=0.7\textwidth]{images/echo_chamber_santo_andre_heatmap.PNG}
	\caption{Heatmap de Pressão Social nas Comunidades identificadas como câmara de eco em Santo André}
	\label{fig:echo_chamber_santo_andre_heatmap}
\end{quadro}

Estes insights revelam que, enquanto alguns tópicos suscitam uma abordagem colaborativa e proativa por parte dos usuários, outros provocam uma resposta crítica e demandam por mudanças mais estruturais, as quais podem estar além do escopo de ação direta dos cidadãos. A presença de personas predominantemente de \textit{complainer} em eventos que refletem falhas graves de infraestrutura ou riscos à saúde pública aponta para uma comunidade que não apenas reconhece estes problemas, mas também os vê como inaceitáveis e passíveis de crítica aberta.

\subsubsection*{Visualizando Câmaras de Eco em Santo André}

\begin{figure}[htb]
	\centering
	\includegraphics[width=0.95\textwidth]{images/network_community_santo_andre.png}
	\caption{Visualização das Comunidades da Rede de Santo André}
	\label{fig:network_community_santo_andre}
\end{figure}

\begin{figure}[htb]
	\centering
	\includegraphics[width=0.95\textwidth]{images/echo_chambers_santo_andre.png}
	\caption{Visualização das Câmaras de Eco da Rede de Santo André}
	\label{fig:echo_chambers_santo_andre}
\end{figure}

Ao analisar o gráfico das câmaras de eco na rede social de Santo André, observa-se uma estrutura complexa com nós centrais que atuam como pontos de convergência e disseminação de informações. O principal hub, com a maior centralidade de grau, é um influenciador chave dentro da rede, potencialmente afetando e sendo afetado por diversas interações sociais. A comunidade 233 está proximamente vinculada a este hub central, possivelmente funcionando como intermediário de ideias e influências, enquanto a comunidade 225 está mais periférica, desenvolvendo diálogos internos com menor interferência externa, o que pode levar à consolidação de perspectivas compartilhadas e reforçar o fenômeno das câmaras de eco.

A formação de cliques dentro dessas comunidades reflete uma tendência de comunicação insular que pode exacerbar a homogeneidade de opiniões e diminuir a exposição a perspectivas divergentes. Esses cliques, ou grupos altamente conectados, ilustram microcosmos onde o reforço mútuo de crenças é provável, intensificando a polarização ideológica. Compreender essas dinâmicas é crucial para desenvolver estratégias de comunicação e intervenção que promovam a diversidade de pensamento e o diálogo entre diferentes grupos sociais. Autoridades locais e organizações da sociedade civil podem usar essa análise para identificar pontos de intervenção, visando fomentar maior integração comunitária e mitigar a formação de câmaras de eco que contribuem para a divisão social em Santo André.

\subsection{Mesquita}

\begin{table}[ht]
	\centering
	\caption{Métricas de Detecção de Câmaras de Eco da Rede da Cidade de Mesquita}
	\label{tab:echo-chamber-metrics-mesquita}
	\begin{tabular}{l|l}
		\toprule
		\textbf{Topologia da Rede}          & \textbf{Valor}                   \\
		\midrule
		Nós                                 & 933                              \\
		Arestas                             & 3578                             \\
		Nº de comunidades                   & 79                               \\
		\toprule
		\textbf{Conteúdo}                   & \textbf{Valor}                   \\
		\midrule
		Tamanho da maior comunidade         & 32                               \\
		Tamanho da menor comunidade         & 3                                \\
		Tamanho médio das comunidades       & 5.78                             \\
		Nº de comentários                   & 40924                            \\
		Nº de likes                         & 156865                           \\
		Nº de votos                         & 10355                            \\
		\midrule
		\textbf{Parâmetro Beta}             & \textbf{Valor}                   \\
		\midrule
		Beta 1                              & 0.3088                           \\
		Beta 2                              & 0.0476                           \\
		Beta 3                              & 0.5                              \\
		Beta 4                              & 0.0157                           \\
		Beta 5                              & 1                                \\
		Beta 6                              & 0.4906                           \\
		Beta 7                              & 0.0011                           \\
		\midrule
		\textbf{Polarização e Homofilia}    &                                  \\
		\midrule
		Coeficiente Global de Câmara de Eco & 0.03213                          \\
		$FCE$ no Grafo                      & 1.0982                           \\
		Nº de Câmaras de Eco Detectadas     & 8                                \\
		\midrule
		\textbf{Comunidade}                 & \textbf{Índice de Câmara de Eco} \\
		\midrule
		Comunidade 65                       & 2.3062                           \\
		Comunidade 52                       & 2.1801                           \\
		Comunidade 53                       & 2.0651                           \\
		Comunidade 54                       & 2.0548                           \\
		Comunidade 71                       & 1.9880                           \\
		Comunidade 66                       & 1.9604                           \\
		Comunidade 75                       & 1.9221                           \\
		Comunidade 61                       & 1.9030                           \\
		\bottomrule
	\end{tabular}
\end{table}

A rede de Mesquita, composta por 933 nós e 3.578 arestas, revela uma dinâmica complexa de interações sociais com 79 comunidades distintas. A modularidade sugere uma estrutura comunitária bem definida, e as métricas $GEC$ e $FCE$ indicam a presença de câmaras de eco, com o $FCE$ de 1.0982 refletindo a força dessas câmaras. Oito comunidades foram identificadas como potenciais câmaras de eco, apresentando um Índice de Câmara de Eco entre 1.903 e 2.306, evidenciando uma homogeneidade de opiniões e interações mais limitadas aos membros da própria rede.

\subsubsection*{Comunidades identificadas como potenciais câmaras de eco}

\begin{table}[ht]
	\centering
	\caption{Resumo das Métricas de câmaras de eco das Comunidades}
	\label{tab:community-metrics-mesquita}
	\resizebox{\textwidth}{!}{%
		\begin{tabular}{l|cccccc}
			\toprule
			\textbf{Comunidade} & \textbf{$FCE$} & \textbf{Densidade} & \textbf{Homogeneidade} & \textbf{Conexões Externas} & \textbf{$ECC$} & \textbf{Exposição Média} \\
			\midrule
			65                  & 2.305          & 1                  & 0.3782                 & 0.4545                     & 0.3782         & 0.5152                   \\
			52                  & 2.0757         & 0.8333             & 0.7117                 & 0.375                      & 0.7117         & 0.4444                   \\
			53                  & 1.9353         & 0.8333             & 0.5619                 & 0.1667                     & 0.5619         & 0.5333                   \\
			54                  & 2.1601         & 0.5                & 0.5242                 & 0.5714                     & 0.5242         & 0.5595                   \\
			71                  & 2.1159         & 0.3333             & 0.375                  & 0.6667                     & 0.375          & 0.5444                   \\
			66                  & 2.0464         & 0.6667             & 0.3291                 & 0.4286                     & 0.3291         & 0.5152                   \\
			75                  & 1.7704         & 0.3333             & 0.312                  & 0.5                        & 0.312          & 0.359                    \\
			61                  & 1.9738         & 0.3333             & 0.267                  & 0.6667                     & 0.267          & 0.4167                   \\
			\bottomrule
		\end{tabular}
	}
\end{table}

Ao analisar mais detalhadamente essas comunidades, observamos várias tendências:

\begin{itemize}
	\item Comunidade 65: Essa comunidade tem uma densidade 1, indicando uma interconexão completa entre seus membros. No entanto, a homogeneidade de 0.3782 e uma média de exposição de 0.5152 mostram que há um grau significativo de concordância dentro da comunidade e uma tendência para os membros serem expostos a opiniões semelhantes.
	\item Comunidade 52: apresenta uma densidade mais baixa de 0.8333, mas uma alta homogeneidade de 0.7117, refletindo uma forte concordância nas opiniões. Além disso, a baixa taxa de conexões externas indica que os membros desta comunidade têm interações limitadas fora dela, potencialmente restringindo a diversidade de informações.
	\item Comunidade 71: Com uma densidade mais baixa de 0.3333 e conexões externas de 0.6667, mostram uma interação mais diversificada, mas ainda assim apresentam características de câmara de eco devido à sua homogeneidade e média de exposição.
\end{itemize}

\begin{table}[ht]
	\centering
	\caption{Resumo das Métricas de Câmaras de Eco das Comunidades em Mesquita}
	\label{tab:community-barometer-metrics-mesquita}
	\resizebox{\textwidth}{!}{%
		\begin{tabular}{lcccc}
			\toprule
			\textbf{Comunidade} & \textbf{Qtd. Usuários} & \textbf{Qtd. Eventos Criados} & \textbf{Score Médio} & \textbf{Persona Média} \\
			\midrule
			65                  & 3                      & 28                            & -0.3303              & 0.5357                 \\
			52                  & 3                      & 20                            & -0.4542              & 0.75                   \\
			53                  & 3                      & 3                             & 0.4763               & 0.0                    \\
			54                  & 3                      & 41                            & 0.2266               & 0.3902                 \\
			71                  & 3                      & 12                            & -0.0717              & 0.0833                 \\
			66                  & 3                      & 25                            & -0.0486              & 0.28                   \\
			75                  & 3                      & 10                            & -0.0639              & 0.2                    \\
			61                  & 3                      & 51                            & -0.0563              & 0.2941                 \\
			\bottomrule
		\end{tabular}
	}
\end{table}

As métricas de pressão social apresentadas na \autoref{tab:community-barometer-metrics-mesquita} fornecem uma visão quantitativa dos padrões de comportamento e das atitudes predominantes dentro de cada comunidade identificada na cidade de Mesquita. Vamos discorrer sobre as razões pelas quais essas comunidades podem ter sido identificadas como câmaras de eco em potencial.

A Comunidade 65 apresenta um score médio negativo e uma persona média inclinada para \textit{complainer}, o que sugere um ambiente onde predominam visões críticas e talvez insatisfação. A quantidade significativa de eventos criados aponta para um alto nível de atividade e engajamento entre os usuários. A combinação de sentimentos negativos e uma tendência para reclamações pode facilitar a formação de uma câmara de eco, onde perspectivas similares são reforçadas e visões divergentes podem ser menos expressas ou aceitas.

Múltiplas reclamações sobre carros estacionados obstruindo o tráfego e pedestres sugerem uma preocupação coletiva com a ordem pública e a funcionalidade urbana. A repetição dessas queixas pode indicar uma frustração crescente e uma vontade de ver mudanças significativas. Relatos sobre vazamento de água e falta de iluminação pública refletem problemas mais profundos de infraestrutura. A constância desses problemas e a percepção de inação podem alimentar um sentimento de negligência e uma demanda por atenção e solução das autoridades. Eventos relacionados à saúde pública, como bueiros entupidos, ressaltam a importância da manutenção urbana e dos serviços públicos. A gravidade com que tais questões são tratadas pela comunidade pode fomentar um consenso em torno da necessidade de intervenção e melhorias.

Esses padrões de comportamento e atitudes, quando combinados com a tendência para scores negativos e uma persona média inclinada para \textit{complainer}, criam um terreno fértil para o desenvolvimento de câmaras de eco. Os membros da comunidade podem se encontrar em um ciclo de reforço mútuo, onde a prevalência de opiniões negativas e críticas é amplificada, limitando a exposição a pontos de vista alternativos e potencialmente exacerbando a polarização.

A Comunidade 52, com um score médio ainda mais negativo e uma alta persona média de \textit{complainer}, indica um grupo fortemente inclinado a expressar descontentamento. A densidade de interações negativas pode contribuir para um sentimento compartilhado de insatisfação, solidificando ainda mais a comunidade como uma potencial câmara de eco. Por exemplo, as repetidas menções um mercado da cidade e os carros estacionados indevidamente destacam uma questão persistente que claramente preocupa a comunidade. A continuidade dessas queixas e a percepção de inação das autoridades podem fortalecer uma narrativa comum de negligência, alimentando um sentimento compartilhado de injustiça. Outras questões, como o barulho e as consequências da feira na rua Fortunato, também contribuem para uma atmosfera de frustração. As referências aos impactos negativos na qualidade de vida e ao desrespeito às normas de convivência social podem amplificar o eco dessas vozes descontentes. Ademais, os relatos de abuso por parte dos motoristas e a falta de fiscalização adequada da prefeitura são temas recorrentes que podem contribuir para um sentimento de unidade baseado na adversidade compartilhada. A insistência em problemas de saúde pública, como o descarte irregular de lixo, e as preocupações com os riscos que representam para a comunidade, reforçam as bases para um diálogo interno que ressoa dentro dos limites dessa câmara de eco virtual.

A combinação de sentimentos negativos persistentes, um alto grau de concordância nas posturas de reclamação e a atividade significativa dos membros na criação de eventos são fatores que podem explicar por que a Comunidade 52 foi identificada como uma câmara de eco em potencial. A falta de diversidade de perspectivas e a prevalência de uma narrativa comum amplificam as vozes existentes e restringem a exposição a opiniões divergentes, potencializando a polarização comunitária.

Por outro lado, a Comunidade 53 tem um score médio positivo e uma persona média que sugere uma predominância de "helpers". Esta comunidade parece ser um exemplo de um grupo mais equilibrado, onde as opiniões positivas têm espaço e a dinâmica de câmara de eco pode não ser tão pronunciada.

A partir dos eventos reportados, é possível observar um comportamento proativo em relação aos problemas urbanos: Solicitação de Serviços de Limpeza: A comunidade ativamente solicita a retirada de entulhos e a limpeza de áreas públicas, como é o caso do pedido para retirada de entulho e terra deixada pela lama. Isso reflete uma preocupação com a manutenção da ordem e a limpeza em Mesquita, apontando para uma atitude de cuidado e responsabilidade com o ambiente compartilhado; Preocupação com Infraestrutura e Prevenção: O pedido de drenagem do rio próximo às casas dos moradores revela uma conscientização sobre os riscos de enchentes e uma busca por medidas preventivas. A menção de que o serviço não é feito há anos sugere um anseio por atenção contínua e gestão adequada dos recursos naturais e infraestruturais.

A Comunidade 53, apesar de seus esforços proativos e da natureza positiva de suas ações, pode ser considerada uma câmara de eco em potencial precisamente por causa dessa homogeneidade e isolamento. O baixo valor em conexões externas sugere que, embora a comunidade esteja alinhada em torno de objetivos comuns e mostre uma atitude positiva, há pouco influxo de novas ideias ou influências de fora. Esse isolamento pode restringir a diversidade de pensamento e a troca com outras comunidades, criando um ambiente onde as perspectivas predominantes são continuamente reforçadas.

Nesse sentido, a Comunidade 53 pode estar propensa a desenvolver uma visão unilateral das questões que enfrenta, onde as soluções e as narrativas são cíclicas e auto-sustentadas dentro do grupo. Mesmo que as ações sejam positivas e os resultados sejam benéficos, a falta de diversidade de opiniões e a baixa interação com outras comunidades podem limitar a capacidade de inovação e a adequação às mudanças de circunstâncias ou novos desafios. Assim, enquanto a Comunidade 53 pode exemplificar um tipo de câmara de eco onde a ressonância é de natureza positiva, é importante para a saúde do diálogo cívico e para o desenvolvimento da comunidade que haja um esforço consciente para engajar com perspectivas diversas e integrar feedback externo, para evitar o isolamento e promover um ecossistema de ideias mais robusto e resiliente.

A Comunidade 54 em Mesquita mostra um conjunto misto de comportamentos, conforme refletido nos eventos reportados. A "Persona Média" dessa comunidade está em um valor intermediário, indicando uma proporção de membros ativos tanto na assistência quanto nas reclamações. O "Score Médio" positivo sugere uma atitude geral construtiva em relação às interações da comunidade. Essa mistura de personas de \textit{helper} e \textit{complainer} e os scores positivos podem, em uma análise superficial, parecer contradizer a noção de câmara de eco. No entanto, a identificação da Comunidade 54 como uma câmara de eco em potencial pode ser devido a outros fatores dinâmicos dentro da comunidade:

\begin{itemize}
	\item Solicitações de Manutenção e Serviços: Os membros da Comunidade 54 se mostram preocupados com questões de manutenção urbana, como bueiros entupidos e iluminação pública inadequada. O ativo envio de solicitações para abordar esses problemas reflete uma tentativa de promover o bem-estar coletivo.
	\item Expressão de Frustrações: As queixas sobre a falta de ação em relação a problemas recorrentes, como esgoto a céu aberto e estacionamento irregular, indicam uma disposição para compartilhar e amplificar insatisfações comuns.
	\item Repetição de Problemas: A ocorrência de problemas persistentes, mesmo após repetidas notificações, pode criar um sentimento de coesão na comunidade, onde as pessoas se unem em torno de frustrações compartilhadas.
\end{itemize}

Apesar das ações positivas, a potencial formação de câmaras de eco pode derivar da intensa repetição e reforço de preocupações específicas dentro da comunidade. Se essas preocupações são predominantemente endossadas e reiteradas sem a incorporação de novas soluções ou perspectivas, a comunidade pode ficar isolada em seu discurso e ação.

O isolamento da Comunidade 54 pode ser exacerbado por um baixo nível de conexões externas, sugerindo que, embora haja uma atitude geral positiva, a comunidade pode ser insular e resistente a influências externas. Este fechamento pode limitar a exposição a diferentes abordagens ou soluções para os problemas que enfrentam, mantendo a comunidade em um ciclo de ressonância de sentimentos e ações semelhantes.

Portanto, a Comunidade 54 pode ser considerada uma câmara de eco em potencial, não apenas pela possibilidade de reforço de atitudes negativas, mas pela repetição de ações positivas em um contexto isolado, o que pode inadvertidamente restringir a diversidade de pensamento e impulsionar a uniformidade de opiniões e abordagens dentro da comunidade.

A Comunidade 71 em Mesquita revela, através dos eventos relatados, um padrão de preocupações com a infraestrutura urbana e o bem-estar público. Os membros desta comunidade, com um "Score Médio" ligeiramente negativo e uma "Persona Média" baixa, indicam um engajamento predominantemente positivo, com uma leve tendência a expressar reclamações.

O perfil da Comunidade 71 pode sugerir uma câmara de eco em potencial, mas de uma natureza diferente. A disposição para o \textit{helper} é evidente, mas o score médio negativo aponta para a presença de frustrações que, quando compartilhadas e reforçadas dentro da comunidade, podem criar um loop de retroalimentação. As questões relatadas frequentemente incluem:

\begin{itemize}
	\item Infraestrutura e Manutenção Urbana: A comunidade está ativa na reportagem de problemas como esgoto a céu aberto e calçadas danificadas. Esses problemas não só afetam a qualidade de vida mas também indicam um risco potencial para a segurança dos moradores.
	\item Bloqueio na Via e Acessibilidade: A repetição de queixas sobre carros estacionados indevidamente nas calçadas demonstra uma preocupação contínua com a acessibilidade e a segurança dos pedestres.
	\item Solicitações de Intervenção: Embora a comunidade esteja solicitando ações para resolver problemas, a persistência desses problemas sugere que as soluções não estão sendo efetivas ou suficientes, o que pode alimentar um sentimento de negligência por parte das autoridades.
\end{itemize}

A identificação dessa comunidade como uma potencial câmara de eco pode estar ligada à maneira como essas preocupações são comunicadas e reforçadas entre os membros. A natureza repetitiva dos problemas e a falta de soluções eficazes podem criar um ambiente onde a insatisfação é amplificada, a despeito da intenção positiva das interações.

O isolamento da comunidade, possivelmente refletido em um baixo nível de conexões externas, pode agravar esse fenômeno, mantendo o grupo focado em suas próprias experiências e soluções, limitando a exposição a novas ideias e abordagens. Esse fechamento pode não somente impedir a entrada de perspectivas alternativas, mas também promover um sentimento de solidariedade baseado em frustrações comuns, solidificando a comunidade como uma câmara de eco onde as mesmas preocupações e opiniões são continuamente compartilhadas e validadas.

A Comunidade 66, demonstra um envolvimento ativo na identificação e resolução de problemas de infraestrutura e saúde pública. A "Persona Média" de 0.28 sugere uma inclinação para a persona de \textit{helper}, embora o "Score Médio" ligeiramente negativo indique que, apesar dos esforços positivos, ainda há uma presença significativa de frustrações e descontentamento. Os membros desta comunidade frequentemente relatam questões como esgoto a céu aberto, entulho nas calçadas, bueiros sem tampa, e estacionamentos que bloqueiam passagens, o que indica uma conscientização e preocupação com o ambiente local. Eles buscam ativamente a atenção e intervenção da prefeitura para resolver essas questões, o que reflete um desejo de melhoria das condições urbanas.

A Comunidade 66 pode ter sido identificada como uma câmara de eco em potencial devido à repetição dessas preocupações e à aparente falta de respostas eficazes ou mudanças duradouras. As solicitações de manutenção e serviços podem ser atendidas de maneira temporária ou insatisfatória, o que pode levar a uma sensação de ciclo contínuo de problemas e soluções temporárias. Quando as vozes da comunidade ressoam com reclamações semelhantes e buscam apoio entre si, isso pode criar um ambiente onde essas opiniões e experiências são reforçadas, fortalecendo a percepção interna de negligência e a necessidade de ação comunal.

Adicionalmente, o baixo nível de conexões externas pode contribuir para o isolamento da comunidade, fazendo com que ela dependa fortemente das soluções e opiniões internas, e potencialmente limitando a entrada de novas ideias ou estratégias de resolução de problemas. A consequência é um ciclo de reforço em que a comunidade pode se tornar mais coesa em suas visões e abordagens, característica típica de uma câmara de eco. Mesmo que haja um esforço genuíno para melhorar o ambiente local, a falta de interações externas e a persistência de problemas podem perpetuar um estado de isolamento e umidade de opiniões e ações.

A Comunidade 75 apresenta uma $FCE$ de 1.7704, o que é comparativamente mais baixo do que algumas outras comunidades. Isso poderia indicar uma menor tendência para a formação de uma câmara de eco. No entanto, a "Densidade" e a "Homogeneidade" moderadas, combinadas com um nível razoável de "Conexões Externas" e um $ECC$ de 0.312, sugerem que há um certo nível de interação interna consistente e algum grau de isolamento.

Os eventos relatados pela Comunidade 75 e o "Score Médio" levemente negativo com uma "Persona Média" indicam uma inclinação para comportamentos de \textit{helper}, mas com um reconhecimento de problemas persistentes, como buracos nas vias e iluminação pública ineficaz.

A descrição dos problemas sugere que a comunidade está ativamente envolvida em reportar questões de infraestrutura, um sinal de engajamento cívico com o ambiente urbano. Apesar de alguns relatos apresentarem um "score" positivo, indicando que há esforços para resolver os problemas, a presença de reclamações frequentes sobre os mesmos problemas pode indicar uma situação onde as soluções não são duradouras ou eficazes.

Essa dinâmica pode levar à formação de uma câmara de eco em potencial, onde a comunidade continua a discutir e reforçar as mesmas preocupações, enquanto luta para ser ouvida ou para alcançar mudanças significativas. o fato de os membros da comunidade estarem relatando esses problemas de forma consistente e, às vezes, sem ver uma solução efetiva, pode contribuir para a formação de uma câmara de eco. Já as "Conexões Externas" não são tão baixas sugere que a comunidade não é completamente isolada; no entanto, o reforço de preocupações internas e a busca por soluções dentro da própria comunidade podem limitar a diversidade de perspectivas e a adoção de novas abordagens.

Portanto, mesmo que a Comunidade 75 não mostre uma tendência forte para a formação de uma câmara de eco, os padrões de comportamento e comunicação interna podem ainda promover uma ressonância de ideias e soluções que se alinham estreitamente com as experiências e percepções compartilhadas, potencialmente diminuindo a receptividade a influências externas e inovadoras.

A Comunidade 61 possui uma $FCE$ de 1.9738, com uma densidade de 0.3333 e uma homogeneidade de 0.267, sugerindo que seus membros tendem a ser moderadamente conectados e compartilham um nível de concordância em suas interações. A comunidade apresenta um valor relativamente alto de "Conexões Externas", o que indica que, apesar de haver uma tendência à formação de uma câmara de eco, existe uma abertura para influências externas, possivelmente mitigando o efeito isolacionista.

Com uma "Persona Média" de 0.2941 e um "Score Médio" de -0.0563, os membros da comunidade parecem ter uma leve tendência a comportamentos de \textit{complainer}, mas não de forma extrema. As reclamações frequentes e a busca por soluções para problemas de infraestrutura e serviços públicos, como esgoto a céu aberto, entulho na via pública e bueiros entupidos, indicam um engajamento ativo na melhoria das condições locais. Entretanto, a persistência desses problemas, mesmo com o envolvimento da comunidade, pode levar a uma ressonância de insatisfação que caracteriza a comunidade como uma câmara de eco em potencial.

Essa dinâmica aponta para uma comunidade que, embora esteja envolvida em questões locais e propensa a expressar descontentamento, mantém-se relativamente aberta a inputs externos, graças à maior taxa de conexões externas em comparação com outras comunidades analisadas. Isso sugere que a Comunidade 61 pode beneficiar-se de um diálogo contínuo com outras comunidades e stakeholders para alcançar soluções mais eficazes para os problemas relatados.

As comunidades de Mesquita que foram identificadas como potenciais câmaras de eco apresentam índices que superam o ponto de corte estabelecido pelo Coeficiente Global de Câmara de Eco ($GEC$), que neste caso é 1.9. Esse valor de corte é derivado da análise do grafo da rede como um todo, considerando que um $GEC$ acima de 1 normalmente indica uma tendência à formação de câmaras de eco, o ponto de corte geralmente é um valor com um intervalo comum entre 1.8 a 2.0 dependendo da escala e complexidade da rede em questão.

O fato de as comunidades 71, 66, 75 e 61 possuírem um "Echo Chamber Probability Index" inferior a 2, mas ainda acima do ponto de corte de 1.9, as coloca em uma zona crítica onde elementos de alta polarização e formação de câmaras de eco podem estar em desenvolvimento. Isso sugere que, enquanto há sinais de polarização de opiniões e uma tendência ao isolamento informacional, pode haver ainda espaço para diálogo e introdução de perspectivas diversas.

Ao considerar os eventos reportados pelos usuários dessas comunidades, observamos uma mistura de comportamentos reativos e proativos, com ações que variam entre reclamações e tentativas de resolução de problemas. Isso indica que, apesar da tendência à formação de câmaras de eco, há ainda um engajamento ativo com questões comunitárias que pode ser crucial para mitigar a polarização e promover um ambiente mais aberto e menos suscetível à formação de câmaras de eco rígidas.

\subsubsection*{Visualizando Câmaras de Eco em Mesquita}

Ao analisar o plot das câmaras de eco na rede da cidade de Mesquita, observa-se uma teia de interações que ressalta a complexidade do tecido social digital. Dentro deste contexto, identificam-se comunidades distintas que, através de laços de comunicação frequentes e reforçados, podem funcionar como potenciais câmaras de eco. Estas estruturas sociais são fundamentais para entender como informações e ideologias circulam e se fortalecem dentro de grupos específicos.

\begin{figure}[htb]
	\centering
	\includegraphics[width=0.95\textwidth]{images/network_communities_mesquita.png}
	\caption{Visualização das Comunidades da Rede de Mesquita}
	\label{fig:network_communities_mesquita}
\end{figure}

Neste cenário, verifica-se que alguns membros da comunidade da câmara de eco estão situados em posições estratégicas, a apenas um passo de distância do hub principal da rede. Este fato é de suma importância, pois indica não apenas uma proximidade física no espaço da rede, mas também um potencial para influência direta e troca de informações. Estes indivíduos, possivelmente atores chave na disseminação de conteúdo, ocupam posições de gatekeepers ou pontes, filtrando e direcionando o fluxo de informações.

\begin{figure}[htb]
	\centering
	\includegraphics[width=0.95\textwidth]{images/echo_chambers_mesquita.png}
	\caption{Visualização das Câmaras de Eco da Rede de Mesquita}
	\label{fig:echo_chambers_mesquita}
\end{figure}

Além disso, a identificação de cliques inteiramente compostos por membros da comunidade da câmara de eco ilumina a presença de subgrupos altamente coesos. Tais subgrupos são caracterizados por uma forte tendência à uniformidade de opinião e podem exacerbar o fenômeno de polarização. O diálogo dentro desses cliques tende a reforçar crenças pré-existentes, minimizando a exposição a perspectivas alternativas e potencialmente criando um solo fértil para a disseminação de desinformação.

A presença dessas estruturas fechadas e auto-referenciais é particularmente relevante em contextos onde a informação é um vetor de poder e influência. Dentro das câmaras de eco, o intercâmbio de ideias tende a seguir um padrão homogêneo, onde a informação é reciclada e reverberada, amplificando crenças e opiniões com pouca ou nenhuma contestação. Este processo pode levar a uma distorção da realidade percebida, afetando a tomada de decisões e a formação de consensos em um nível mais amplo. A análise destas dinâmicas é crucial para o entendimento de como as comunidades digitais se formam e operam, bem como para o desenvolvimento de estratégias que promovam o diálogo construtivo e a diversidade de pensamento.

\subsection{Niterói}

\begin{table}[ht]
	\centering
	\caption{Métricas de Detecção de Câmaras de Eco da Rede da Cidade de Niterói}
	\label{tab:echo-chamber-metrics-niteroi}
	\begin{tabular}{l|l}
		\toprule
		\textbf{Topologia da Rede}          & \textbf{Valor}                   \\
		\midrule
		Nós                                 & 4274                             \\
		Arestas                             & 14175                            \\
		Nº de comunidades                   & 542                              \\
		\toprule
		\textbf{Conteúdo}                   & \textbf{Valor}                   \\
		\midrule
		Tamanho da maior comunidade         & 159                              \\
		Tamanho da menor comunidade         & 3                                \\
		Tamanho médio das comunidades       & 7.36                             \\
		Nº de comentários                   & 80731                            \\
		Nº de likes                         & 199668                           \\
		Nº de votos                         & 11640                            \\
		\midrule
		\textbf{Parâmetro Beta}             & \textbf{Valor}                   \\
		\midrule
		Beta 1                              & 0.0789                           \\
		Beta 2                              & 0.01886                          \\
		Beta 3                              & 0.11117                          \\
		Beta 4                              & 0.00144                          \\
		Beta 5                              & 1                                \\
		Beta 6                              & 0.51574                          \\
		Beta 7                              & 0.00023                          \\
		\midrule
		\textbf{Polarização e Homofilia}    &                                  \\
		\midrule
		Coeficiente Global de Câmara de Eco & 0.0213                           \\
		$FCE$ no Grafo                      & 1.05116                          \\
		Nº de Câmaras de Eco Detectadas     & 1                                \\
		\midrule
		\textbf{Comunidade}                 & \textbf{Índice de Câmara de Eco} \\
		\midrule
		Comunidade 515                      & 2.066647                         \\
		\bottomrule
	\end{tabular}
\end{table}

Ao analisar as métricas de detecção de câmaras de eco geradas para a rede da cidade de Niterói, encontramos diversos pontos de interesse que lançam luz sobre a dinâmica da rede e a possibilidade de câmaras de eco em seu interior. A topologia da rede é notável com um grande volume de nós e arestas, indicando uma rede social relativamente densa. No entanto, a divisão em 542 comunidades revela uma fragmentação significativa, com uma média de apenas cerca de 7.36 membros por comunidade. A maior comunidade contém 159 membros, enquanto a menor tem apenas 3. A análise da modularidade revela uma estrutura razoavelmente dividida em subgrupos, com um valor de 0.515. Quando examinamos o conteúdo da rede, observamos um grande volume de atividade, com 80731 comentários, 199668 likes e 11640 votos. Esses números indicam uma rede social ativa e engajada, onde os usuários interagem regularmente.

A análise dos parâmetros Beta oferece insights adicionais. Os valores desses parâmetros variam, sugerindo que diferentes aspectos da rede podem influenciar a formação de câmaras de eco. Por exemplo, Beta 3, com um valor de 0.11117, indica que a presença de conexões externas pode ser um fator importante na prevenção da formação de câmaras de eco, enquanto Beta 6, com um valor de 0.51574, sugere que o coeficiente de câmara de eco global ($GEC$) desempenha um papel significativo.

No que diz respeito à polarização e homofilia, o coeficiente global de câmara de eco ($GEC$) é relativamente baixo, com um valor de 0.0213. Isso pode indicar que a rede como um todo não está fortemente polarizada, o que é uma boa notícia em termos de diversidade de perspectivas. No entanto, a $FCE$ no grafo, com um valor de 0.95116, sugere que, quando as câmaras de eco estão presentes, elas têm uma influência significativa.

A detecção identificou apenas uma comunidade como uma potencial câmara de eco, com um índice de câmara de eco de 2.066647. Isso é intrigante, pois sugere que, embora a rede como um todo não seja fortemente polarizada, existe pelo menos uma comunidade que demonstra uma notável homogeneidade em suas opiniões e perspectivas. A existência de apenas uma comunidade identificada como potencial câmara de eco pode ser atribuída a vários fatores.

Primeiro, pode ser devido à forma como as comunidades são definidas ou particionadas na rede. Se a detecção de comunidades não for sensível o suficiente para identificar subdivisões dentro das comunidades maiores, isso poderia resultar na identificação de apenas uma câmara de eco.

Segundo, as características específicas da rede de Niterói podem influenciar esse resultado. Por exemplo, a natureza das interações entre os membros da comunidade, a distribuição de opiniões e a presença de influenciadores podem variar amplamente de uma rede para outra. Essas peculiaridades podem fazer com que algumas redes sejam mais propensas a formar câmaras de eco do que outras.

Por fim, é importante mencionar que o ajuste de parâmetros nas heurísticas de partição de comunidades também pode afetar os resultados. Experimentos anteriores indicam que mesmo pequenas alterações nos parâmetros podem levar a diferentes resultados na detecção de câmaras de eco. No entanto, ao repetir a detecção utilizando outras heurísticas de partição de comunidades e parâmetros beta, observamos que a presença de certos usuários na comunidade 515 pode influenciar negativamente a dinâmica da comunidade, aumentando a probabilidade de polarização ao longo do tempo. Em outras palavras, a centralidade e as características desses usuários tendem a direcionar a comunidade em direção à formação de uma câmara de eco, dependendo da sua influência na comunidade.

Em resumo, a análise das métricas de detecção de câmaras de eco para a rede da cidade de Niterói revela uma rede social ativa e fragmentada, com uma comunidade identificada como potencial câmara de eco. Embora a rede como um todo não pareça fortemente polarizada, essa descoberta destaca a importância de compreender as nuances da dinâmica da rede e as características específicas que podem influenciar a formação de câmaras de eco em contextos sociais.

\subsubsection*{Comunidade 515}

\begin{table}[ht]
	\centering
	\caption{Métricas de Detecção de Câmaras de Eco da Comunidade 515 de Niterói}
	\label{tab:echo-chamber-metrics-community515}
	\begin{tabular}{l|l}
		\toprule
		\textbf{Métrica}  & \textbf{Valor} \\
		\midrule
		$FCE$             & 1.8582         \\
		Densidade         & 0.3333         \\
		Homogeneidade     & 0.271          \\
		Conexões Externas & 0.6            \\
		$ECC$             & 0.571          \\
		$GEC$             & 0.0321         \\
		Exposição Média   & 0.3704         \\
		\bottomrule
	\end{tabular}
\end{table}

Essas particularidades nos convidam a examinar as métricas de detecção de câmaras de eco da Comunidade 515 de Niterói, conforme apresentadas na Tabela \ref{tab:echo-chamber-metrics-community515}. Essas métricas oferecem insights valiosos sobre a dinâmica e o potencial de formação de câmaras de eco dentro desta comunidade específica.

Ao observar a $FCE$, que apresenta um valor de 1.8582, podemos inferir que essa comunidade possui uma tendência significativa a formar câmaras de eco. Este valor, que é maior que 1, indica que os membros desta comunidade têm uma exposição predominantemente alta a informações que reforçam suas crenças e opiniões preexistentes. Isso sugere que os indivíduos na Comunidade 515 estão mais propensos a interagir com conteúdo e opiniões que estão alinhados com suas próprias visões, o que pode contribuir para uma polarização de opiniões dentro da comunidade.

A densidade, com um valor de 0.3333, significa que os membros desta comunidade estão moderadamente interconectados. Embora não seja uma densidade extremamente alta, essa interconexão ainda indica que os membros têm uma rede de comunicação relativamente sólida dentro da comunidade, o que pode facilitar a formação de câmaras de eco. A homogeneidade das opiniões, representada por um valor de 0.571, também é relevante. Esse número sugere que existe uma concordância ou semelhança mediana nas opiniões dos membros da Comunidade 515. Portanto, essa métrica indica que as opiniões compartilhadas dentro desta comunidade têm uma certa uniformidade, o que é um fator que contribui para a formação de câmaras de eco. As "Conexões Externas" da comunidade, avaliada em 0.6, mostra que os membros da têm uma quantidade considerável de interações com indivíduos ou grupos fora da comunidade. Essas conexões externas podem atenuar, em certa medida, a formação de câmaras de eco, já que a exposição a perspectivas divergentes é mais provável quando há conexões externas significativas.

O $ECC$ tem um valor de 0.271, que está em concordância com a homogeneidade das opiniões. Isso significa que a probabilidade de um membro desta comunidade se isolar em uma câmara de eco é relativamente alta. Por outro lado, o "$GEC$" apresenta um valor de 0.0321. Esse valor mais baixo indica que, considerando a rede como um todo, a tendência à formação de câmaras de eco não é tão proeminente. No entanto, as métricas individuais dentro da Comunidade 515 revelam que essa comunidade específica tem uma dinâmica interna que favorece a formação de câmaras de eco. Por fim, a "Exposição Média" na comunidade, com um valor de 0.3704, indica que, em média, os membros são frequentemente expostos a opiniões e conteúdo que estão alinhados com suas próprias visões, contribuindo para o fenômeno da câmara de eco.

\begin{table}[ht]
	\centering
	\caption{Resumo das Métricas Pressão Social das Comunidades Identificadas como Câmara de Eco em Niterói}
	\label{tab:community-metrics-niteroi}
	\resizebox{\textwidth}{!}{%
		\begin{tabular}{lcccc}
			\toprule
			\textbf{Comunidade} & \textbf{Qtd. Usuários} & \textbf{Qtd. Eventos Criados} & \textbf{Score Médio} & \textbf{Persona Média} \\
			\midrule
			155                 & 4                      & 22                            & -0.3013              & 0.7273                 \\
			\bottomrule
		\end{tabular}
	}
\end{table}

Ao análisar a comunidade 515 de Niterói com as heurísticas de pressão social descritas no capítulo anterior, algumas particularidades interessantes são reveladas, que podem ajudar a entender por que essa foi a única câmara de eco identificada. A comunidade consiste em um grupo relativamente pequeno, com apenas 8 membros. Essa pequena dimensão pode contribuir para a formação de uma câmara de eco, pois a interação limitada com perspectivas externas pode levar a uma maior homogeneidade de opiniões.

\begin{quadro}[htb]
	\centering
	\includegraphics[width=0.7\textwidth]{images/niteroi_community_515.png}
	\caption{Gráfico de radar ilustrando as métricas de pressão social da Comunidade 515 de Niterói}
	\label{fig:niteroi_community_515}
\end{quadro}

Considerando aspectos de conteúdo e criação de eventos no Colab, os membros da comunidade criaram um total de 180 eventos, o quedemonstra um alto nível de engajamento e atividade dentro da comunidade. Além disso, os membros da comunidade deram um total de 9.975 likes. Esse número é notavelmente alto em relação ao número de usuários na comunidade, sugerindo que eles têm uma tendência a apoiar e endossar as opiniões e eventos na rede. Foram realizados 256 votos up/down em eventos de outros usuários. A comunidade fez 1.202 comentários em eventos de outros usuários. Esses indicadores de engajamento mostram que os membros da comunidade estão ativamente envolvidos na rede e estão dispostos a se envolver em discussões com membros externos à comunidade.

Os tipos de eventos mais comuns na comunidade 515 estão relacionados a problemas e reclamações em infraestrutura urbana, como lâmpadas apagadas à noite, pontos de infração de trânsito recorrentes, buracos nas vias e calçadas irregulares. Esses eventos podem gerar discussões e interações específicas dentro da comunidade, contribuindo para a formação de uma câmara de eco em torno desses tópicos. A análise da persona dos usuários mostra que a maioria deles tende a ser \textit{complainer}. Isso indica uma predisposição para expressar opiniões negativas e reclamações, o que pode amplificar a polarização e a formação de câmaras de eco. A análise do sentimento das postagens revela que a maioria dos eventos tem um sentimento negativo, polarizando ainda mais as opiniões dentro da comunidade.

Embora haja um número significativo de eventos criados, eles estão concentrados em alguns tópicos específicos, como problemas de infraestrutura urbana e trânsito. A falta de diversidade de tópicos pode contribuir para a formação de uma câmara de eco, pois os membros da comunidade estão focados em um conjunto limitado de preocupações. Ao analisar os eventos criados pelos membros das comunidades, observamos que um tópico recorrente de discussão parece envolver as dinâmicas de trânsito e estacionamento no centro da cidade. Os usuários demonstram com scores majoritariamente negativos, descontentamento sobre a falta de vagas de estacionamento bem como multas que receberam e consideram injustas, principalmente porque muitas das postagens também denunciam infranções cometidas por outros motoristas e parecem cobrar uma multa do poder público. Além disso, os usuários parecem reclamar da falta de sinalização e de fiscalização por parte da prefeitura, evidenciado pelos eventos criados no tópico Manutenção/implantação de placa de sinalização, com 10 eventos criados.

\begin{quadro}[htb]
	\centering
	\includegraphics[width=0.7\textwidth]{images/echo_chamber_niteroi_heatmap.PNG}
	\caption{Heatmap ilustrando a distribuição de tópicos na comunidade 515 no mapa de Niterói}
	\label{fig:echo_chamber_niteroi_heatmap}
\end{quadro}

A comunidade 515 de Niterói se destaca por sua intensa atividade relacionada a problemas e reclamações na infraestrutura urbana, abordando questões como iluminação noturna deficiente, pontos de infração de trânsito recorrentes e condições precárias de vias e calçadas. Essa concentração em tópicos específicos pode ser um fator-chave na identificação dessa comunidade como uma câmara de eco. Além disso, a análise da persona dos usuários revela uma tendência significativa para expressar opiniões negativas e reclamações, o que contribui para a amplificação da polarização. A predominância de postagens com sentimentos negativos dentro da comunidade também intensifica essa polarização. Esses fatores, combinados, contribuem para a identificação da comunidade 515 como uma câmara de eco, onde opiniões similares são amplificadas, resultando em uma polarização acentuada em torno desses problemas específicos.

\subsubsection*{Visualizando Câmaras de Eco em Niterói}

\begin{figure}[htb]
	\centering
	\includegraphics[width=0.95\textwidth]{images/network_communities_niteroi.png}
	\caption{Visualização das Comunidades da Rede de Niterói}
	\label{fig:network_communities_niteroi}
\end{figure}

Ao analisar o gráfico das câmaras de eco na rede social da cidade de Niterói, percebe-se que a centralidade de conexões é dominada pelo hub principal, identificado pelo nó 113450. Este nó se destaca como uma entidade significativa dentro da rede, provavelmente desempenhando um papel crucial na difusão de informações. A comunidade indicada como alta probabilidade de ser câmara de eco, não apresentou membros notoriamente próximos desse hub central. No entanto, a existência de cliques fortes na comunidade, onde todos os usuários seguem e são seguidos pelos outros usuários, evidencia a presença de comunicações intensas e circuitos fechados de informação entre esses indivíduos. A comunidade, que possui 5 usuários, tem 2 cliques, de 3 e 2 usuários respectivamente, além de serem interconectados por um usuário que faz parte dos dois cliques. Estes subgrupos altamente interconectados são característicos de câmaras de eco, onde o isolamento de membros de influências externas pode levar a uma polarização e a uma visão de mundo uniformizada. O entendimento dessas microestruturas é fundamental para decifrar as dinâmicas de informação e influência na cidade, oferecendo pistas sobre como ideias e crenças podem ser fortalecidas dentro de comunidades fechadas e como intervenções podem ser desenhadas para promover um discurso mais aberto e diversificado.

\begin{figure}[htb]
	\centering
	\includegraphics[width=0.95\textwidth]{images/echo_chambers_niteroi.png}
	\caption{Visualização das Câmaras de Eco da Rede de Niterói}
	\label{fig:echo_chambers_niteroi}
\end{figure}

Sob a perspectiva de análise de redes sociais, Niterói destaca-se como um espaço digital intrigante onde a densidade e modularidade moderada coexistem com sentimentos negativos e polarizados, especialmente em temas de mobilidade urbana, gerando uma alta pressão social sobre este tópico. Apesar desta polarização, a cidade demonstrou um engajamento diversificado em diferentes tópicos de pressão social e uma alta porcentagem de usuários com persona \textit{helper}, contribunindo para a heteogngnei de opiniões e a interações com outras perspectivas, melhorando a exposição média dos usuários, o que ajuda a equilibrar a polaridade da rede como um todo. No entanto, uma métrica $GEC$ superior a 1 apontou a possibilidade de câmaras de eco na rede, ou que comunidades mais polarizadas possam estar em um processo de formação de câmara de eco. Dentre estas, a heurística de detecção de câmaras de eco identificou a Comunidade 515 como o único ambiente com parâmetros suficientes para ser considerado um espaço de alta polarização e homogeneidade de opiniões.

A análise detalhada da Comunidade 515 confirmou os aspectos de alta polarização e homogeneidade de opiniões, especialmente relacionadas a questões de trânsito no centro da cidade, caracterizando-a como uma câmara de eco. O plot de heatmap da pressão social exercida pela Comunidade 515 no mapa de Niterói corroborou que a maioria dos usuários desta comunidade atua em eventos de trânsito no centro, exercendo uma forte pressão social nesta localidade. Este estudo revela a importância de analisar as redes sociais em um nível granular para entender a dinâmica de comunidades específicas que podem estar em desacordo com a dinâmica geral da rede. A identificação e análise de tais comunidades polarizadas podem fornecer insights valiosos para as autoridades locais e administradores de plataformas, permitindo uma resposta mais informada e possivelmente, a criação de estratégias para promover o diálogo e a diversidade de opiniões, mitigando a formação de câmaras de eco e polarização excessiva em questões críticas de infraestrutura urbana.

\section{Discussão dos Resultados}

O fenômeno das câmaras de eco nas redes sociais emerge como um desafio central na era da informação digital, onde os diálogos e opiniões estão cada vez mais polarizados e isolados. Este capítulo buscou identificar e analisar as câmaras de eco nas redes sociais das cidades de Santo André, Niterói e Mesquita, utilizando um teorema matemático e um modelo de detecção baseado em métricas de pressão social, homogeneidade de opiniões e topologia de rede. A relevância deste estudo se ancora na necessidade de entender como as estruturas sociais digitais podem influenciar a formação de opiniões e a disseminação de informações, com potenciais implicações para o tecido social e a governança democrática.

A metodologia implementada neste estudo representa uma abordagem integrada, combinando simulação de redes sociais, análise heurística e visualização avançada. Inicialmente, empregamos o Modelo de Redes Gráficas Exponenciais ($ERGM$) para modelar a rede de interações sociais dos usuários do Colab. $ERGM$ é particularmente adequado para este propósito, pois permite incorporar a estrutura e os processos sociais nas simulações de rede, refletindo as tendências e os padrões observados em dados de redes sociais reais.

Para enriquecer nossa simulação e aproximar nossos modelos da dinâmica realista do comportamento do usuário, integramos o conceito de modelagem baseada em agentes. Criamos agentes autônomos, ou "bots", programados para gerar conteúdo e interagir com outros usuários de maneira coerente com as personas identificadas na plataforma. Cada bot foi configurado com um conjunto de regras de interação e perfis de comportamento que refletem as heurísticas derivadas da análise de sentimentos e da classificação de personas, como \textit{Helpers} e \textit{Complainers}.

Essa fusão de $ERGM$ com a modelagem baseada em agentes permite não apenas simular a estrutura da rede, mas também animar a rede com comportamentos individuais que podem influenciar a evolução da rede ao longo do tempo. A interação entre os bots é projetada para imitar as ações e reações dos usuários humanos, proporcionando insights sobre como a pressão social e a influência podem se manifestar e se espalhar através das conexões sociais. Com essa abordagem híbrida, visamos capturar a intersecção entre a topologia da rede e os processos dinâmicos que ocorrem dentro dela, oferecendo uma nova perspectiva sobre a pressão social nas redes hiperlocais do Colab.

Através das simulações, conseguimos replicar as estruturas sociais do ambiente digital com um alto grau de precisão. As heurísticas de detecção de câmaras de eco foram então aplicadas para identificar subgrupos altamente interconectados e homogêneos, potencialmente suscetíveis a efeitos de ressonância ideológica. Para complementar a análise quantitativa, técnicas de visualização foram aprimoradas, proporcionando uma representação intuitiva e acessível das estruturas da rede, destacando as potenciais comunidades em câmaras de eco.

Topologicamente, as câmaras de eco identificadas nas redes das três cidades mostraram características distintas. Santo André teve quatro comunidades identificadas, com variáveis graus de interconexão. Niterói destacou-se com uma única comunidade, que após análise do conteúdo, constatou-se através doss níveis de polaridade e baixa conexões externas como uma câmara de eco sob o tópico de pressão social  de "Infrações de Trânsito". Mesquita, por outro lado, revelou oito comunidades, com índices de câmara de eco, obtidos através de análise $PCA$, significativamente mais altos do que os encontrados nas outras cidades, indicando uma tendência para a homogeneidade de opiniões e a forte pressão social.

Essas métricas de pressão social, como frequência de interações e criação de conteúdo dentro de grupos homogêneos, interagindo os mesmos tipos de evento, revelaram muito sobre os diálogos digitais das cidades. Em Mesquita, onde as métricas indicaram uma pressão social mais intensa, os diálogos tendiam a ser mais uniformes e menos diversificados, sugerindo uma rede mais suscetível à formação de câmaras de eco. Em contraste, Niterói apresentou uma variedade mais ampla de interações, o que pode ser sintomático de uma maior exposição a perspectivas diversas e um debate mais saudável.

Ao analisar a performance do modelo de detecção de câmaras de eco, principalmente ao analisar as métricas de pressão social e a polaridade das postagens criadas pelos membros da comunidade, revela uma capacidade de identificar, de maneira consistente, comunidades com alta probabilidade de serem câmaras de eco.

Em Santo André, as quatro comunidades detectadas indicam uma rede moderadamente fragmentada, enquanto em Niterói, a presença de uma única comunidade ampla sugere uma centralização menor, porém, os tópicos dos eventos e os scores de sentimento prendominantemente negativos, sugerem que essa comundiade específica está descontente com os órgãos governamentais nas questões de trânsito no centro e as suas opiniões negativas exercem uma pressão social moderada, especialmente no centro da cidade. Em Mesquita, o alto número de comunidades com fortes indicadores de câmaras de eco sugere um ambiente digital mais isolado e polarizado. Os resultados dos experimentos forneceram evidências substanciais que corroboram nosso teorema matemático. A presença de comunidades com altos índices de câmara de eco em Mesquita, em particular, alinha-se com as previsões do teorema, destacando a influência de conexões densas e homogeneidade ideológica na formação desses subgrupos.

As comunidades identificadas como potenciais câmaras de eco oferecem insights valiosos sobre as redes das respectivas cidades. Em Santo André, a diversidade das comunidades sugere um espectro de diálogos e potenciais pontos de vista. Em Niterói, a rede centralizada pode facilitar o acesso a uma gama mais ampla de opiniões, embora isso não isente completamente o risco de polarização. Em Mesquita, a presença de múltiplas comunidades isoladas indica um ambiente propício à formação de opiniões homogêneas, reforçando a necessidade de intervenções para promover a diversidade e o diálogo.

\begin{quadro}[htb]
	\centering
	\includegraphics[width=0.7\textwidth]{images/echo_chamber_strength_by_size.png}
	\caption{Relação entre o tamanho das comunidades e a $FCE$}
	\label{fig:echo_chamber_strength_by_size}
\end{quadro}

É importante notar que devido ao tamanho das comunidades identificadas como câmaras de eco em potencial, bem como a quantidade limitada de postagens disponíveis para análise cujo o autor faz parte dessas comundiades é bem menor em comparação com algumas comunidades maiores com usuários mais antigos da rede. Essa limitação pode ter afetado a precisão das métricas de pressão social e polaridade, especialmente em Mesquita, onde a maioria das comunidades identificadas era pequena, entre 3 a 5 membros. Porém, é importante notar que, o tamanho médio das comunidades em Mesquita é de 5.78 membros, portanto a média de membro das comunidades identificadas como câmaras de eco está dentro da média de comunidades de todas as cidades, exceto Niterói. No entanto, a análise de sentimentos e a classificação de personas foram realizadas em toda a rede, o que nos permite inferir com segurança que as comunidades não identificadas como câmaras de eco não apresentam características que as tornem suscetíveis a esse fenômeno, mesmo com uma grande quantidade de dados disponíveis para essas comunidades.

Além disso, um aspecto crucial da metodologia utilizada é a capacidade de discernir câmaras de eco genuínas de comunidades com alta atividade que, de outra forma, poderiam ser equivocadamente classificadas como tal devido ao volume substancial de postagens. O algoritmo empregado demonstrou precisão, evitando falsos positivos mesmo em comunidades com numerosas postagens disponíveis para análise, o que poderia distorcer as métricas de pressão social e polaridade. Portanto, as comunidades identificadas como câmaras de eco refletem uma configuração específica de interações e não são simplesmente artefatos de um volume maior ou menor de dados.

A relação entre o modelo proposto e as técnicas de detecção de comunidades merece uma análise cuidadosa. Por exemplo, ao utilizar o algoritmo de Leiden para a identificação de comunidades, observou-se que a maior comunidade em Niterói possui 159 membros. Portanto, ao aplicar o modelo de detecção de câmaras de eco a esta comunidade, tratou-se o coletivo como uma entidade única, analisando a polaridade, homogeneidade de opiniões, conexões eternas dos 159 membros da comunidade para o grafo e as outras comunidades identificadas. Essa abordagem pode não capturar a formação de subgrupos internos. É concebível que dentro de comunidades de grande escala, subgrupos menores operem como câmaras de eco independentes. Contudo, a análise de tais subgrupos não se enquadra no escopo do presente estudo, uma vez que as heurísticas utilizadas são aplicadas ao grafo na sua totalidade e não a componentes isolados.

Essencialmente, ao considerar uma comunidade ampla como um único grafo, informações cruciais sobre conexões externas — um componente vital para a precisão do nosso modelo — podem ser obscurecidas. A análise de comunidades mais extensas como um todo, sem reconhecer a existência potencial de subgrupos, poderia levar a uma interpretação errônea das métricas de detecção de câmaras de eco. Portanto, enquanto o modelo demonstra robustez na identificação de câmaras de eco em uma escala global, ele não é otimizado para o reconhecimento de dinâmicas de câmaras de eco que possam existir dentro de subestruturas de comunidades maiores.

Os resultados deste estudo sugerem que a integração da análise de sentimentos com métricas derivadas de análise de redes sociais pode oferecer uma abordagem valiosa para a investigação da polarização. Essa combinação de metodologias fornece uma base para explorar as complexidades na dinâmica das opiniões, o que pode ser instrumental para um entendimento mais profundo da estrutura e influência das câmaras de eco.

Adicionalmente, a incorporação da classificação de personas, do score de sentimentos das postagens e dos tipos de eventos mais frequentemente engajados pelos usuários, fornece um arcabouço robusto para derivar métricas de homogeneidade de opiniões. Esta análise é aprofundada pelo fator de exposição externa a opiniões divergentes, avaliando o contraste entre os eventos que um usuário posta versus aqueles com os quais interage mais na rede. Essas métricas são particularmente pertinentes para a análise de polarização dentro de redes sociais digitais.

No contexto do Colab, uma plataforma social dedicada à cidadania, a incorporação da geolocalização das postagens indicando o local do evento reportado adiciona uma nova dimensão a análise. A combinação desses dados com o modelo de pressão social hiperlocal permite não apenas a identificação, mas também a localização de áreas com alta concentração de reclamações ou elogios, tipos específicos de eventos e o sentimento predominante relacionado a eles. Esta dimensão geográfica fornece uma perspectiva valiosa sobre onde os cidadãos estão mais vocalmente insatisfeitos ou satisfeitos, oferecendo uma ferramenta poderosa para órgãos governamentais e políticos na alocação de recursos e na resposta às preocupações dos cidadãos de maneira direcionada e eficiente.

Este estudo se posiciona na intersecção da engenharia de software e das ciências sociais, buscando trazer contribuições quantitativas para a análise de fenômenos sociais complexos, como as câmaras de eco em redes digitais. Com base em uma extensa revisão da literatura em análise de redes e a integração de métricas derivativas, assim como a adoção da modelagem baseada em agentes de Atiqi e as heurísticas de $GEC$ e $ECC$, foi possível desenvolver um teorema matemático para a análise quantitativa da polaridade em comunidades online.

Através da implementação de um modelo validado com o uso de $ERGM$ e simulações de redes que espelham as características observadas no Colab, o estudo avançou na aplicação prática de teorias matemáticas e computacionais. A incorporação de dados de análise de sentimentos e a classificação manual de postagens, refinada posteriormente por um classificador de aprendizado de máquina, forneceu os insumos essenciais para o mecanismo de detecção de câmaras de eco. Este mecanismo identificou variações significativas na constituição e dinâmica das comunidades nas três cidades estudadas — uma comunidade em Niterói, oito em Mesquita e quatro em Santo André.

A partir desses experimentos, aprendemos que as técnicas de engenharia de software podem ser aplicadas com sucesso para elucidar aspectos da dinâmica social. A análise quantitativa, apoiada por modelos matemáticos e computacionais, pipelines de pre-processamento e a classificação dos \textit{outputs} gerados oferece uma nova lente para examinar a formação e a influência das câmaras de eco. Além disso, os resultados sublinham a importância de considerar a singularidade de cada comunidade ao aplicar esses modelos, destacando como variáveis locais, como a geografia social e o comportamento dos usuários, podem afetar a manifestação de polarização e a pressão social.

\chapter{Conclusão}
\label{chapter:09_conclusion}
\input{tex/09_conclusion}

% ---
% Finaliza a parte no bookmark do PDF, para que se inicie o bookmark na raiz
% ---
\bookmarksetup{startatroot}% 
% ---

% ----------------------------------------------------------
% ELEMENTOS PÓS-TEXTUAIS
% ----------------------------------------------------------
\postextual

% ----------------------------------------------------------
% Referências bibliográficas
% ----------------------------------------------------------
\bibliography{references}

% ---------------------------------------------------------------------
% GLOSSÁRIO
% ---------------------------------------------------------------------

% Arquivo que contém as definições que vão aparecer no glossário
\newword{Colab}{O Colab é um aplicativo brasileiro que promove a participação cidadã na melhoria das cidades. Permite que os usuários relatem problemas e sugiram ideias diretamente para as autoridades, buscando soluções colaborativas para questões urbanas. O objetivo é fortalecer a conexão entre cidadãos e poder público, visando a transparência e a construção de cidades mais sustentáveis e inclusivas.}
\newword{Eigenvector}{Em análise de redes, um eigenvector de uma matriz de adjacência de um grafo representa um vetor próprio associado a um valor próprio específico. Em termos simples, um eigenvector é um vetor no qual a importância relativa de um nó em um grafo é determinada pela sua conectividade com outros nós. Os eigenvectors são amplamente utilizados em medidas de centralidade, como a centralidade de eigenvector, que identifica nós importantes com base na sua influência sobre a rede. O cálculo dos eigenvectors é fundamental para entender a estrutura e a dinâmica de redes complexas.}
\newword{Aresta}{Em análise de redes, uma aresta é a conexão entre dois nós em um grafo. Ela representa uma relação ou interação entre os nós e pode ser direcionada ou não direcionada, dependendo da presença ou ausência de uma direção específica. As arestas são essenciais para compreender a estrutura e a dinâmica das redes, bem como para analisar propriedades como conectividade e fluxo de informações.}
\newword{Nó}{Em análise de redes, um nó é um elemento fundamental em um grafo que representa uma entidade individual. Também conhecido como vértice, ele pode representar uma pessoa, um local, um objeto ou qualquer outra entidade relevante para o estudo em questão. Os nós são conectados por arestas, que representam as relações ou interações entre eles. Eles desempenham um papel crucial na análise de redes, permitindo a investigação de propriedades como conectividade, centralidade e fluxo de informações.}
\newword{Gephi}{O Gephi é uma ferramenta de software para visualização e análise de redes. Ele possui recursos para importar dados de redes, organizar nós e arestas, e realizar análises, como medir centralidade e identificar comunidades. É amplamente utilizado em áreas como análise de redes sociais e biológicas para compreender as estruturas complexas das redes.}
\newword{Neo4j}{O Neo4j é um sistema de gerenciamento de banco de dados orientado a grafos. Ele é usado para armazenar e consultar dados conectados de forma eficiente, permitindo representar relacionamentos complexos entre entidades.}
% Comando para incluir todas as definições do arquivo glossario.tex
\glsaddall
% Impressão do glossário
\printglossaries

% ----------------------------------------------------------
% Apêndices
% ----------------------------------------------------------

% ---
% Inicia os apêndices
% ---
\begin{apendicesenv}
	\chapter{Código Fonte}
	\label{chapter:source_code}
	\begin{codigo}[caption={Exemplo de classe Python para deteção de câmaras de eco}, label={codigo:echochamberdetector}, language=Python, breaklines=true]
    class EchoChamberDetector:
    def __init__(self, nodes_df, edges_df, events_df, colab_comments_df, colab_likes_df, colab_updown_df, strategy, classifier=None, communities=None):
        self.nodes = nodes_df
        self.edges = edges_df
        self.events = events_df
        self.colab_comments = colab_comments_df
        self.colab_likes = colab_likes_df
        self.colab_updown = colab_updown_df
        self.G = nx.from_pandas_edgelist(self.edges, 'source', 'target', create_using=nx.DiGraph())
        self.strategy = strategy
        self.classifier = classifier
        self.communities = communities
        self.min_community_size = 3
        self.echo_chambers = []

    def get_unique_event_types_for_user_from_df(self, user_id, dataframe):
        return dataframe[dataframe['colab_user_id_from'] == user_id]['event_type_id'].unique()

    def get_all_unique_event_types_for_user(self, user_id):
        event_types_from_events = self.events[self.events['colab_user_id']
                                              == user_id]['event_type_id'].unique()
        event_types_from_comments = self.get_unique_event_types_for_user_from_df(
            user_id, self.colab_comments)
        event_types_from_likes = self.get_unique_event_types_for_user_from_df(
            user_id, self.colab_likes)
        event_types_from_updown = self.get_unique_event_types_for_user_from_df(
            user_id, self.colab_updown)
        all_event_types = np.concatenate([event_types_from_events, event_types_from_comments,
                                          event_types_from_likes, event_types_from_updown])
        # Convert all elements to strings
        all_event_types_str = [str(event_type) for event_type in all_event_types]

        return np.unique(all_event_types_str)

    def get_all_scores_for_user(self, user_id):
        scores_from_events = self.events[self.events['colab_user_id']
                                         == user_id]['score'].values
        updown_votes = self.colab_updown[self.colab_updown['colab_user_id_from'] == user_id]
        #scores_from_updown = np.where(updown_votes['vote'] == 'down', -1, 1)
        scores_from_updown = []
        all_scores = np.concatenate([scores_from_events, scores_from_updown])
        return all_scores

    def get_persona_for_user(self, user_id):
        user_events = self.events[self.events['colab_user_id'] == user_id]
        if not user_events.empty:
            return user_events['persona'].iloc[0]
        else:
            return None

    def get_events_created_by_user(self, user_id):
        return self.events[self.events['colab_user_id'] == user_id]

    def calculate_community_density(self, community):
        edges = [(n1, n2) for n1 in community for n2 in community if n1 !=
                 n2 and self.G.has_edge(n1, n2)]
        density = len(edges) / (len(community) * (len(community) - 1))
        return density

    def calculate_modularity(self, G):
        """
        Calcula a modularidade do grafo.
        :param G: O grafo.
        :return: O valor da modularidade.
        """
        ug = G.to_undirected()
        # Detecção de comunidades usando o algoritmo de Louvain
        partition = community_louvain.best_partition(ug)

        # Calculando a modularidade
        modularity = community_louvain.modularity(partition, ug)
        #print(f"graph modularity {modularity}")
        return modularity

    def calculate_global_gec(self):
        total_gec = 0
        total_weight = 0

        for community in self.communities:
            community_size = len(community)
            community_gec = self.calculate_community_gec(community)

            # Weighted sum of GEC, weighted by the size of each community
            total_gec += community_gec * community_size
            total_weight += community_size

        # Calculate the weighted average GEC
        global_gec = total_gec / total_weight if total_weight > 0 else 0
        #print(f"Global Echo Chamber Coefficient {global_gec}")
        return global_gec

    def calculate_average_external_connections(self, G):
        external_connections = {}  # A dictionary to store external connections for each node

        # Initialize external connections count to 0 for all nodes
        for node in G.nodes():
            external_connections[node] = 0

        # Calculate external connections for each node
        for edge in G.edges():
            source, target = edge
            if source not in G.nodes() or target not in G.nodes():
                # One of the nodes is external, increment external connections count
                if source not in G.nodes():
                    external_connections[target] += 1
                if target not in G.nodes():
                    external_connections[source] += 1

        # Calculate the total external connections count
        total_external_connections = sum(external_connections.values())

        # Calculate the average external connections
        average_external_connections = total_external_connections / len(G.nodes())

        return average_external_connections

    def calculate_external_connections(self, community):
        external_connections = 0
        possible_connections = 0
        for node in community:
            for neighbor in self.G.neighbors(node):
                if neighbor not in community:
                    external_connections += 1
                possible_connections += 1
        if possible_connections > 0:
            factor = 1.0 - (external_connections / possible_connections)
        else:
            factor = 1.0
        return factor

    def calculate_homogeneity_of_opinions(self,community):
        # Obter todas as pontuações de opinião dos membros da comunidade
        all_scores = [score for user in community for score in self.get_all_scores_for_user(user)]

        # Filtrar out NaN values
        all_scores = [score for score in all_scores if not np.isnan(score)]
        #print("ALL SCORES", all_scores)

        # Se não houver pontuações suficientes
        if len(all_scores) <= 1:
            return 0

        # Calculate weighted homogeneity
        total_weight = len(all_scores)
        weighted_std = np.std(all_scores) * total_weight

        # Normalize the weighted std to a value between 0 and 1
        normalized_weighted_std = 1 / (1 + weighted_std)

        return normalized_weighted_std


    def calculate_echo_chamber_metrics(self, community):
        print(f"calculate_echo_chamber_metrics {community}")
        # Calculate individual metrics
        density = self.calculate_community_density(community)
        homogeneity = self.calculate_homogeneity_of_opinions(community)
        external_connections = self.calculate_external_connections(community)
        influencers = self.calculate_community_influence(self.G, community)
        ecc = self.calculate_ecc_for_community(community)
        gec = self.global_gec  # Global Echo Chamber factor
        average_exposure = self.calculate_community_exposure(community)

        # Echo chamber strength calculation
        strength = np.exp(
            self.beta1 * density +
            self.beta2 * homogeneity +
            self.beta3 * external_connections +
            self.beta4 * influencers +
            self.beta5 * average_exposure  +
            self.beta6 * gec +
            self.beta7 * ecc
        )

        # Create a dictionary to store all metrics including strength
        echo_chamber_metrics = {
            'graph_size': len(self.G.nodes()),
            'echo_chamber_strength': round(strength,4),
            'density': round(density,4),
            'homogeneity': round(homogeneity,4),
            'external_connections': round(external_connections,4),
            'ecc': round(ecc,4),
            'gec': round(gec,4),
            'average_exposure': round(average_exposure,4),
            'size': len(community)
        }
        print(f"metrics {echo_chamber_metrics}")
        return echo_chamber_metrics


    def calculate_graph_echo_chamber_strength(self):
        #print("Calculating graph echo chamber strength...")
        # Extract the largest connected component as a subgraph
        largest_cc = max(nx.connected_components(self.G.to_undirected() ), key=len)
        subgraph = self.G.subgraph(largest_cc)

        # Filter nodes with a degree (in + out in a directed graph) of 3 or more
        filtered_nodes = [node for node, degree in subgraph.degree() if degree >= 3]

        # Create a subgraph with only those filtered nodes
        filtered_subgraph = subgraph.subgraph(filtered_nodes)
        community = list(filtered_subgraph.nodes)
        ##
        density = nx.density(self.G)
        #print(f"graph density: {density}")
        homogeneity = self.calculate_homogeneity_of_opinions(community)
        #print(f"graph homogeneity: {homogeneity}")
        try:
            eig_centralities = nx.eigenvector_centrality_numpy(G, max_iter=1000, tol=1e-03)
        except:
            #print("ARPACK did not converge, falling back to largest component or different method")
            # Handle the error by using the largest connected component, for example
            largest_cc = max(nx.connected_components(self.G.to_undirected()), key=len)
            subgraph = self.G.subgraph(largest_cc)
            eig_centralities = nx.eigenvector_centrality_numpy(subgraph, max_iter=1000, tol=1e-03)
        #
        influencers = median(eig_centralities.values())
        external_connections = self.calculate_average_external_connections(self.G)
        #print(f"graph external_connections: {external_connections}")
        ecc = self.calculate_community_ecc(community)
        #print(f"graph ecc: {ecc}")
        self.global_gec = self.calculate_global_gec()
        #print(f"graph gec: {gec}")
        average_exposure = self.calculate_community_exposure(community)
        #print(f"graph average_exposure: {average_exposure}")
        strength = np.exp(
            self.beta1 * density +
            self.beta2 * homogeneity +
            self.beta3 * external_connections +
            self.beta4 * influencers +
            self.beta5 * average_exposure +
            self.beta6 * self.global_gec+
            self.beta7 * ecc
        )
        #print(f"Graph Echo Chamber Strength {strength}")
        return strength

    def calculate_community_influence(self, G, community_nodes):
        try:
            eig_centralities = nx.eigenvector_centrality_numpy(G, max_iter=1000, tol=1e-03)
        except:
            #print("ARPACK did not converge, falling back to largest component or different method")
            # Handle the error by using the largest connected component, for example
            largest_cc = max(nx.connected_components(G), key=len)
            subgraph = G.subgraph(largest_cc)
            eig_centralities = nx.eigenvector_centrality_numpy(subgraph, max_iter=1000, tol=1e-03)

        # Calculate influence scores for nodes in the community
        community_influence_scores = {node: eig_centralities[node] for node in community_nodes}

        # Calculate the community-level influence score (e.g., average)
        community_influence_score = sum(community_influence_scores.values()) / len(community_nodes)

        # Normalize the influence score
        max_influence_score = max(eig_centralities.values())
        normalized_influence_factor = community_influence_score / max_influence_score

        return normalized_influence_factor

    def is_echo_chamber(self, community):
        if not hasattr(self, 'graph_strength'):
          self.graph_strength = self.calculate_graph_echo_chamber_strength()
        community_strength = self.calculate_echo_chamber_strength(community)
        return community_strength >= self.graph_strength

    def derive_betas(self, G):
        #print("Deriving Betas...")
        # Beta1 - Coeficiente de Agrupamento Médio
        avg_clustering_coeff = nx.average_clustering(G)
        beta1 = avg_clustering_coeff  # Ou um valor inicial de 0.171, conforme o contexto

        # Beta2 - Homogeneidade das Opiniões
        num_communities = len(list(nx.community.greedy_modularity_communities(G)))
        beta2 = 1  # Ou um valor inicial de 1/352

        # Beta3 - Conexões Externas
        num_weakly_connected_components = nx.number_weakly_connected_components(G)
        beta3 = 1 / num_weakly_connected_components  # Ou um valor inicial de 1/329

        # Bfeta4 - Efeito dos Influenciadores
        try:
            eig_centralities = nx.eigenvector_centrality_numpy(G, max_iter=1000, tol=1e-03)
        except:
            #print("ARPACK did not converge, falling back to largest component or different method")
            # Handle the error by using the largest connected component, for example
            largest_cc = max(nx.connected_components(G), key=len)
            subgraph = G.subgraph(largest_cc)
            eig_centralities = nx.eigenvector_centrality_numpy(subgraph, max_iter=1000, tol=1e-03)
        median_centrality = median(eig_centralities.values())
        beta4 = median_centrality
        # Beta5 - Exposição Média
        if nx.is_strongly_connected(G):
            avg_path_length = nx.average_shortest_path_length(G)
        else:
            largest_cc = max(nx.strongly_connected_components(G), key=len)
            subgraph = G.subgraph(largest_cc)
            avg_path_length = nx.average_shortest_path_length(subgraph)
        beta5 = avg_path_length / len(self.G.nodes)
        # Beta6 - GEC (Modularidade)
        # Nota: Modularidade pode ser calculada usando algoritmos de detecção de comunidades
        modularity = self.calculate_modularity(G)
        beta6 = modularity  # Ou um valor inicial de 0.683
        beta7 = 1 / len(self.G.nodes)

        return beta1, beta2, beta3, beta4, beta5, beta6, beta7

    def calculate_community_ecc(self, community):
        scores = [self.get_all_scores_for_user(member) for member in community]
        all_scores = [score for sublist in scores for score in sublist]
        if not all_scores or np.all(np.isnan(all_scores)):
            return 0
        ecc = np.nanstd(all_scores)
        #print(f"calculate_community_ecc {ecc}")
        return ecc

    def calculate_ecc_for_community(self, community):
        total_contribution = 0

        for member in community:
            user_events = self.get_events_created_by_user(member)
            #print(f"user_events {user_events}")
            user_contribution = 0

            for index, row in user_events.iterrows():
                # Avalie a contribuição do evento com base em score e persona
                event_contribution = self.evaluate_event_contribution(row, community)
                user_contribution += event_contribution

            total_contribution += user_contribution

        # ECC da comunidade como um todo
        ecc = total_contribution / len(community)
        #print(f"ecc {ecc}")
        return ecc

    def evaluate_event_contribution(self, event, community):
        scores = [self.get_all_scores_for_user(member) for member in community]
        all_scores = [score for sublist in scores for score in sublist]
        if not all_scores or np.all(np.isnan(all_scores)):
            return 0
        scores = all_scores
        personas = [self.get_persona_for_user(member) for member in community]
        #print(f"scores {scores}")
        #print(f"personas {personas}")
        event_score = event['score']  # Score do evento
        event_persona = event['persona']  # Persona do evento
        #print(f"event_score {event_score}")
        #print(f"event_persona {event_persona}")
        if(event_score > 0): event_score = 1
        else: event_score = -1

        # Assuming you have already defined 'scores' and 'personas' for the community
        positive_scores = [score for score in scores if not np.isnan(score) and score > 0]
        negative_scores = [score for score in scores if not np.isnan(score) and score < 0]

        # Calculate the predominant score (positive or negative)
        predominant_score = 1 if len(positive_scores) > len(negative_scores) else -1 if len(negative_scores) > len(positive_scores) else 0
        #print(f"predominant_score {predominant_score}")
        non_none_personas = [persona for persona in personas if persona is not None]

        if non_none_personas:
            count_ones = non_none_personas.count(1)
            count_zeros = non_none_personas.count(0)

            # Calculate the predominant persona (1 or 0)
            predominant_persona = 1 if count_ones > count_zeros else 0 if count_zeros > count_ones else None
        else:
            predominant_persona = None
        #print(f"predominant_persona {predominant_persona}")
        # Compare o evento com as características predominantes da comunidade
        if event_score == predominant_score and event_persona == predominant_persona:
            #print("event contributes to polarization")
            return 1  # Evento contribui positivamente
        else:
            #print("event DOES NOT contribute to polarization")
            return 0  # Evento não contribui

    def calculate_community_gec(self, community):
        #print("Calculating all scores...")
        scores = [self.get_all_scores_for_user(member) for member in community]

        #print("Flattening all scores...")
        # Flatten the list, ignoring empty arrays and replacing NaNs with 0
        all_scores = [np.nan_to_num(score) for sublist in scores for score in sublist if len(sublist) > 0]

        # Check if all_scores is empty
        if not all_scores:
            return 0

        #print("Pairwise product sum...")
        # Calculate the sum of products of scores for all pairs of users
        pairwise_product_sum = sum(a * b for a, b in combinations(all_scores, 2))
        #print(f"Pairwise product sum: {pairwise_product_sum}")

        # Normalize by the number of pairs
        number_of_pairs = len(list(combinations(all_scores, 2)))
        #print(f"Number of pairs: {number_of_pairs}")
        gec = pairwise_product_sum / number_of_pairs if number_of_pairs > 0 else 0
        #print(f"calculate_community_gec {gec}")
        return gec

    def calculate_community_exposure(self, community):
        all_community_event_types = set()
        for member in community:
            member_event_types = self.get_all_unique_event_types_for_user(
                member)
            all_community_event_types.update(member_event_types)
        total_community_event_types = len(all_community_event_types)
        if total_community_event_types == 0:
            return 0
        individual_exposures = []
        for member in community:
            member_event_types = self.get_all_unique_event_types_for_user(
                member)
            member_exposure = len(set(member_event_types)) / \
                total_community_event_types
            individual_exposures.append(member_exposure)
        average_exposure = sum(individual_exposures) / len(community)
        #print(f"calculate_community_exposure {average_exposure}")
        return average_exposure

    def analyse_metrics(self, metrics_list, t):
        data = [metrics['echo_chamber_strength'] for metrics in metrics_list]
        peaks, _ = find_peaks(data, height=1)
        print(f"Peaks: {peaks}")
        echo_chamber_indices = peaks.tolist()
        return echo_chamber_indices

    def identify_echo_chambers(self, beta1=1.0, beta2=1.0, beta3=1.0, beta4=1.0, beta5=1.0, beta6=1.0, beta7=1.0, metric='threshold', metric_param = 2):
        self.beta1 = beta1
        self.beta2 = beta2
        self.beta3 = beta3
        self.beta4 = beta4
        self.beta5 = beta5
        self.beta6 = beta6
        self.beta7 = beta7
        if self.communities is None:
          print("Calculating communities")
          self.communities = self.strategy.detect_communities(self.G)

        #print('Number of communities before filtering:', len(self.communities))

        # Filter communities based on the minimum size requirement
        self.communities = [community for community in self.communities if len(community) >= self.min_community_size]

        print('Number of communities after filtering:', len(self.communities))

        self.echo_chambers = []
        self.graph_strength = self.calculate_graph_echo_chamber_strength()
        #self.global_gec = 0
        print("Calculating community metrics...")
        all_communities_metrics = [self.calculate_echo_chamber_metrics(community) for community in self.communities]
        print(f"Community Metrics: {all_communities_metrics}")
        self.community_metrics = all_communities_metrics
        if self.classifier != None:
          b = self.classifier.analyse_metrics(all_communities_metrics)
        else :
          b = self.analyse_metrics(all_communities_metrics, metric_param)
        echo_chambers = [self.communities[index] for index in b]
        return echo_chambers
#
def detect_echo_chambers(df, n, e, comments_df, likes_df, votes_df, t=2.0, debug_g=None, validate_communities=None):
  selected_ids = set(df['colab_user_id'])
  # Direct filtering
  edgez = e[e['source'].apply(lambda x: x in selected_ids) | e['target'].apply(lambda x: x in selected_ids)]
  edgez
  # Crie um grafo a partir do DataFrame de arestas (edges_df)
  if debug_g != None:
    graph = debug_g
    #print("Using debug Graph")
  else:
    graph = nx.from_pandas_edgelist(edgez, 'source', 'target', create_using=nx.DiGraph())
  #
  print(graph)
  #print("Analysing Communities...")
  strategy = LeidenClusteringStrategy(algo=leidenalg.SignificanceVertexPartition, resolution=2, num_iterations=10)
  communities_summary = analyze_communities(graph, algorithm="leiden", resolution=2, algo=leidenalg.SignificanceVertexPartition)
  net_communities = communities_summary['communities']
  #print('number_of_communities:', communities_summary['number_of_communities'])
  #print('largest_community_size:', communities_summary['largest_community_size'])
  #print('smallest_community_size:', communities_summary['smallest_community_size'])
  #print('average_community_size:', communities_summary['average_community_size'])
  community_sizes = {}
  #
  if validate_communities != None:
    validate_communities(graph, net_communities)
  #
  filtered_comments, filtered_likes, filtered_updown = filter_dfs_based_on_edgelist(edgez,comments_df,likes_df,votes_df)
  # Print the number of rows for each DataFrame
  #print(f"Comments: {len(filtered_comments)}")
  #print(f"Likes: {len(filtered_likes)}")
  #print(f"Votes: {len(filtered_updown)}")
  detector = EchoChamberDetector(
    n, nx.to_pandas_edgelist(graph), df, filtered_comments, filtered_likes, filtered_updown,
    strategy=strategy,
    classifier=echo_chamber_metrics_classifier,
    communities=net_communities
  )
  beta1, beta2, beta3, beta4, beta5, beta6, beta7 = detector.derive_betas(graph)
  #print(f'beta1: {beta1}, beta2: {beta2}, beta3:{beta3}, beta4: {beta4}, beta5: {beta5}, beta6: {beta6}, beta7: {beta7}')
  echo_chambers = detector.identify_echo_chambers(beta1, beta2, beta3, beta4, beta5, beta6, beta7, metric_param=t)
  # Exiba as echo chambers encontradas
  print(f"Echo Chambers: {len(echo_chambers)}")
  for i, chamber in enumerate(echo_chambers, 1):
      print(f"Echo Chamber {i}: {len(chamber)}")
  return echo_chambers
\end{codigo}
% \input{tex/includes/networkplotter.tex}
% \newpage
% \input{tex/includes/mba.tex}
% \newpage
% \begin{codigo}[caption={Código de treinamento para análise de sentimento}, label={codigo:lex_train}, language=Python, breaklines=true]
import string
import pandas as pd
from dotenv import load_dotenv
import os
import csv
import spacy
import nltk
from nltk.tokenize import word_tokenize
from nltk.corpus import stopwords
from nltk.stem import RSLPStemmer
from nltk.stem import WordNetLemmatizer
from sklearn.feature_extraction.text import CountVectorizer
from wordcloud import WordCloud
import stanza
import deplacy
import graphviz
import matplotlib.pyplot as plt

load_dotenv()  # Carrega as variáveis de ambiente do arquivo .env

# Baixar os módulos do Spacy em português
!python -m spacy download pt_core_news_sm

# Baixar os recursos do NLTK
nltk.download('stopwords')
nltk.download('rslp')

# Carregar o modelo do Spacy em português
nlp = spacy.load("pt_core_news_sm")

# Inicializar os stemmers e lemmatizers
stemmer = RSLPStemmer()
lemmatizer = WordNetLemmatizer()

# Função para exibir o mapa sintático da frase utilizando Stanza e Deplacy
def exibirMapaSintatico(frase):
  nlp_stanza = stanza.Pipeline("pt")
  doc = nlp_stanza(frase)
  deplacy.render(doc)
  graphviz.Source(deplacy.dot(doc))

# Função para exibir a distribuição de palavras e nuvem de palavras
def visualizarDistribuicaoPalavras(collection, top):
  cv = CountVectorizer(stop_words=stopwords.words('portuguese'))
  words = cv.fit_transform(collection)
  sum_words = words.sum(axis=0)

  words_freq = [(word, sum_words[0, i]) for word, i in cv.vocabulary_.items()]
  words_freq = sorted(words_freq, key=lambda x: x[1], reverse=True)

  frequency = pd.DataFrame(words_freq, columns=['word', 'freq'])

  frequency.head(top).plot(x='word', y='freq', kind='bar', figsize=(20, 7), color='blue')
  wordcloud = WordCloud(background_color='white', width=1000, height=1000).generate_from_frequencies(dict(words_freq))
  plt.figure(figsize=(20, 8))
  plt.imshow(wordcloud)
  plt.title("WordCloud", fontsize=22)
  plt.show()

# Carregar o arquivo CSV com o dicionário Oplexicon
dict_oplexicon = {}
csvfile = pd.read_csv(os.getenv('URL_OPLEXICON'), low_memory=False)
for index, row in csvfile.iterrows():
  palavra = row[0]
  polaridade = row[2]
  dict_oplexicon[palavra] = polaridade

# Carregar o arquivo CSV com o dicionário Unilex
dict_unilex = {}
csvfile = pd.read_csv(os.getenv('URL_UNILEX'), low_memory=False)
for index, row in csvfile.iterrows():
  palavra = row[0]
  polaridade = row[1]
  dict_unilex[palavra] = polaridade

# Carregar o arquivo CSV com o dicionário WordNetAffectBr
dict_wordnetaffectbr = {}
csvfile = pd.read_csv("URL_WORDAFBR", low_memory=False)
for index, row in csvfile.iterrows():
  palavra = row[0]
  polaridade = row[1]
  dict_wordnetaffectbr[palavra] = polaridade

# Função de pré-processamento do texto
def preprocessamento(texto):
  texto = str(texto).lower()
  documento = nlp(texto)
  lista = []
  
  for token in documento:
    lista.append(lemmatizer.lemmatize(token.text))
  
  lista = [palavra for palavra in lista if palavra not in stopwords.words('portuguese') and palavra not in string.punctuation and not palavra.isdigit()]
  return lista

# Função para obter a polaridade da frase usando um dicionário
def obterPolaridade(frase, dicionario):
  frase_processada = preprocessamento(frase)
  frase_polaridade = [float(dicionario.get(palavra, 0)) for palavra in frase_processada]
  score = sum(frase_polaridade)
  return score

# Função para analisar o sentimento da frase usando diferentes dicionários
def analisarSentimento(frase, dicionarios):
  scores = [obterPolaridade(frase, dicionario) for dicionario in dicionarios]
  return scores

# Frase de teste
frase_teste = "Eu queria amar, mas tive medo"

# Comparando polaridades utilizando diferentes dicionários
dict_senticnet = {}  # Adicione o dicionário SenticNet
scores = analisarSentimento(frase_teste, [dict_oplexicon, dict_senticnet, dict_unilex, dict_wordnetaffectbr])
dicionarios = ["Oplexicon", "SenticNet", "Unilex", "WordNetAffectBr"]
for i in range(len(scores)):
  print("A polaridade da frase '", frase_teste, "' segundo o dicionário", dicionarios[i], "é:", scores[i])

# Carregar os dados do CSV de eventos do Colab
colab_events_url = os.getenv('URL_COLAB_EVENTS')
colab_events = pd.read_csv(colab_events_url, low_memory=False)

# Função para criar o conjunto de dados a partir dos eventos do Colab
def criarDataset(colab_events):
  dataset = colab_events.drop(columns=['status', 'created_at', 'event_type_id', 'event_type_name'])
  return dataset

# Ajustando o motor de sentimento

# Experimento de análise de sentimento com algumas frases de exemplo obtidas do CSV de eventos
# Adicionando palavras extras e corrigindo valores de palavras no dicionário lexicon
lexicon = dict(dict_oplexicon, **dict_senticnet)

palavras_lexicon = {
    "colabora": 1,
    "rua": 1,
    "iptu": -1,
    "prefeito": -1,
    "prefeitos": -1,
    "irregulares": -1,
    "indesejadas": -1,
    "bastasse": -1,
    "pública": 1,
    "horrorosas": -1,
    "indigência": -1,
    "inadequados": -1,
    "pt": -1,
    "psdb": -1,
    "pdt": -1,
    "corrupção": -1,
    "varias": -1,
    "árvore": 1,
    "urgente": 1,
    "número": 1,
    "frente": 1,
    "façam": -1,
    "asfaltem": -1,
    "prefeitura": 0.5,
    "sofreu": -1,
    "extremamente": -1,
    "mal": -1,
    "dejetos": -1,
    "vistoria": -1,
    "responsabilidade": -1,
    "secretaria": -1,
    "providência": -1,
    "gargalhadas": -1,
    "barulho": -1,
    "botequim": -1,
    "colab": -1,
    "algazarra": -1,
    "descaso": -1,
    "deveria": -1,
    "providências": -1,
    "reclamação": -1,
    "irrespirável": -1,
    "recorrentes": -1,
    "irregularidade": -1,
    "irregularidades": -1,
    "trabalhando": 1,
    "trabalhado": 1,
    "transparente": 1,
    "resolver": 0.5,
    "problemas": 0.5,
    "obstáculos": -1,
    "abismo": -1,
    "descumprimento": -1
}

for palavra, valor in palavras_lexicon.items():
  lexicon[palavra] = valor

# Testando a polaridade das frases de exemplo
frases_teste = [
    "Isso é culpa dos prefeitos que não ligam para a população",
    "Eu pago meu IPTU em dia é um absurdo isso acontecer",
    "Parabéns a prefeitura que tem trabalhado de uma forma transparente para resolver os problemas da cidade",
    "Fiação sem vergonha na Vila Olímpia. Fios caem até o chão, pondo em perigo a segurança do pedestre.",
    "Calçadão de Boa Viagem com várias depressões e buracos causados por infiltração de água",
    "O descumprimento da política nacional de mobilidade é evidente nessa intervenção feita na ponte Paulo Guerra. Não basta o resto da calçada da ponte estar esburacada e cheia de obstáculos, é preciso também abrir um abismo para o pedestre ter que ultrapassar. Essa alça construída para dar acesso ao Shopping Riomar é imoral e com certeza haverá atropelamentos, que chamarão de 'acidentes'. Não será acidente, será apenas o fruto de uma infraestrutura toda voltada para o fluxo de automóveis individuais em detrimento do pedestre em descumprimento a lei federal. Infelizmente o Ministério Público não intervém nesse caso. Ainda estamos muito longe de atingir a acessibilidade universal."
]

for frase in frases_teste:
  print("Polaridade da frase '", frase, "':", obterPolaridade(frase, lexicon))
  print("BREAKDOWN LEXICON")
  printLexiconPhrase(frase)
  print()

# Criar o dataset a partir dos eventos do Colab
dataset = criarDataset(colab_events)

# Calcular os scores das postagens usando o lexicon
scores = []
for index, row in dataset.iterrows():
  post = row[2]
  score = obterPolaridade(post, lexicon)
  scores.append(score)

dataset["score"] = scores

# Filtrar os piores e melhores scores
worst_scores = dataset.sort_values(by='score', ascending=True).head(1000)
best_scores = dataset.sort_values(by='score', ascending=False).head(1000)

# Visualizar a distribuição de palavras nos piores scores
visualizarDistribuicaoPalavras(worst_scores['description'], 10)

# Exibir o mapa sintático de uma postagem com pior score
exibirMapaSintatico(worst_scores['description'].values[0])

# Exibir os valores associados a cada palavra da postagem com pior score
printLexiconPhrase(worst_scores['description'].values[0])

# Visualizar a distribuição de palavras nos melhores scores
visualizarDistribuicaoPalavras(best_scores['description'], 10)

# Exibir o mapa sintático de uma postagem com melhor score
exibirMapaSintatico(best_scores['description'].values[0])

# Exibir os valores associados a cada palavra da postagem com melhor score
printLexiconPhrase(best_scores['description'].values[0])

# Concatenar os dataframes dos piores e melhores scores
frames = [worst_scores, best_scores]
result = pd.concat(frames).drop_duplicates().reset_index(drop=True).sort_values(by='score', ascending=True)

# Salvar o dataframe como CSV
result.to_csv('colab_sentiment_training.csv', encoding='utf-8-sig', index=False) 
\end{codigo}
% \newpage
% \begin{codigo}[caption={Modelo de classificação supervisionada de sentimentos}, label={codigo:sentiment_classifier}, language=Python, breaklines=true]
  import re
  import numpy as np
  import pandas as pd
  from sklearn.feature_selection import SelectKBest, chi2
  from sklearn.model_selection import train_test_split
  from sklearn.tree import DecisionTreeClassifier
  from sklearn.tree import export_graphviz
  from sklearn.metrics import accuracy_score, classification_report, confusion_matrix
  from sklearn.linear_model import LogisticRegression
  from sklearn.ensemble import RandomForestClassifier
  from sklearn.neighbors import KNeighborsClassifier
  from sklearn.preprocessing import MinMaxScaler
  from sklearn.metrics import roc_curve, auc, roc_auc_score
  import nltk
  from nltk.stem.porter import PorterStemmer
  from nltk.corpus import stopwords
  from sklearn.feature_extraction.text import CountVectorizer
  from wordcloud import WordCloud
  from gensim.models.doc2vec import LabeledSentence
  import matplotlib.pyplot as plt
  import seaborn as sns
  import pandas_profiling as pp
  import plotly.express as px
  
  nltk.download('stopwords')
  
  # Load data
  colab_events_url = os.getenv('URL_COLAB_EVENTS')
  colab_events_train = pd.read_csv(colab_events_url, low_memory=False)
  print(colab_events_train.shape)
  colab_events_train.head(2)
  
  # Data preprocessing
  colab_events_train['score_norm'] = colab_events_train['score'].apply(lambda x: 1 if x >= 1 else 0)
  colab_events_train['score_norm'].value_counts().plot.bar(color='pink', figsize=(6, 4))
  
  # Text preprocessing
  stopwords_set = set(stopwords.words('portuguese'))
  
  def sanitize_phrase(phrase):
      result = ""
      for word in phrase.split(" "):
          review = re.sub(r, ' ', word)
          review = review.lower()
          review = review.split()
          ps = PorterStemmer()
          # Stemming
          review = [ps.stem(word) for word in review if word not in stopwords_set]
          if review and len(review[0]) > 2:
              result += review[0] + " "
      return result.strip()
  
  # Example usage
  print(sanitize_phrase("O rato roeu a roupa do rei de roma joão josé"))
  
  # Vectorization
  cv = CountVectorizer(stop_words=stopwords.words('portuguese'))
  words = cv.fit_transform(colab_events_train.description)
  sum_words = words.sum(axis=0)
  words_freq = [(word, sum_words[0, i]) for word, i in cv.vocabulary_.items()]
  words_freq = sorted(words_freq, key=lambda x: x[1], reverse=True)
  frequency = pd.DataFrame(words_freq, columns=['word', 'freq'])
  frequency.head(50).plot(x='word', y='freq', kind='bar', figsize=(20, 7), color='blue')
  plt.title("Palavras mais utilizadas - Top 50")
  
  # WordCloud
  wordcloud = WordCloud(background_color='white', width=1000, height=1000).generate_from_frequencies(dict(words_freq))
  plt.figure(figsize=(20, 8))
  plt.imshow(wordcloud)
  plt.title("WordCloud", fontsize=22)
  
  # Model training
  cv = CountVectorizer(max_features=len(colab_events_train))
  x = cv.fit_transform(train_corpus).toarray()
  y = colab_events_train.iloc[:, 1]
  
  x_train, x_valid, y_train, y_valid = train_test_split(x, y, test_size=0.25, random_state=42)
  
  # Standardization
  sc = MinMaxScaler()
  x_train = sc.fit_transform(x_train)
  x_valid = sc.transform(x_valid)
  
  # RandomForestClassifier
  model = RandomForestClassifier()
  model.fit(x_train, y_train)
  y_pred = model.predict(x_valid)
  print("Training Accuracy:", model.score(x_train, y_train))
  print("Validation Accuracy:", model.score(x_valid, y_valid))
  print("F1 score:", f1_score(y_valid, y_pred, average="weighted"))
  cm = confusion_matrix(y_valid, y_pred)
  print(cm)
  
  # LogisticRegression
  model = LogisticRegression()
  model.fit(x_train, y_train)
  y_pred = model.predict(x_valid)
  print("Training Accuracy:", model.score(x_train, y_train))
  print("Validation Accuracy:", model.score(x_valid, y_valid))
  print("F1 score:", f1_score(y_valid, y_pred, average="weighted"))
  cm = confusion_matrix(y_valid, y_pred)
  print(cm)
  
  # DecisionTreeClassifier
  model = DecisionTreeClassifier()
  model.fit(x_train, y_train)
  y_pred = model.predict(x_valid)
  print("Training Accuracy:", model.score(x_train, y_train))
  print("Validation Accuracy:", model.score(x_valid, y_valid))
  print("F1 score:", f1_score(y_valid, y_pred, average="weighted"))
  cm = confusion_matrix(y_valid, y_pred)
  print(cm)
  
  # SVC
  model = SVC()
  model.fit(x_train, y_train)
  y_pred = model.predict(x_valid)
  print("Training Accuracy:", model.score(x_train, y_train))
  print("Validation Accuracy:", model.score(x_valid, y_valid))
  print("F1 score:", f1_score(y_valid, y_pred, average="weighted"))
  cm = confusion_matrix(y_valid, y_pred)
  print(cm)
  
  # XGBClassifier
  model = XGBClassifier()
  model.fit(x_train, y_train)
  y_pred = model.predict(x_valid)
  print("Training Accuracy:", model.score(x_train, y_train))
  print("Validation Accuracy:", model.score(x_valid, y_valid))
  print("F1 score:", f1_score(y_valid, y_pred, average="weighted"))
  cm = confusion_matrix(y_valid, y_pred)
  print(cm)
\end{codigo}
% \newpage
% \newpage
% \begin{codigo}[caption={Exemplo de script Python para criar um Grafo aleatório de uma rede social contendo pelo menos uma câmara de eco}, label={codigo:growing_random_models}, language=Python, breaklines=true]
# Crie um grafo vazio
G = nx.Graph()

# Adicione alguns nós altamente conectados
echo_chamber_1 = [str(uuid.uuid4()) for _ in range(8)]
G.add_edges_from([(n1, n2) for n1 in echo_chamber_1 for n2 in echo_chamber_1 if n1 != n2])

# Adicione outra câmara de eco
echo_chamber_2 = [str(uuid.uuid4()) for _ in range(5)]
G.add_edges_from([(n1, n2) for n1 in echo_chamber_2 for n2 in echo_chamber_2 if n1 != n2])

# Adicione alguns nós que não estão em câmaras de eco
non_echo_nodes = [str(uuid.uuid4()) for _ in range(8)]
for i in range(len(non_echo_nodes) - 1):
    G.add_edge(non_echo_nodes[i], non_echo_nodes[i+1])

# Adicione mais 50 nós que estão conectados de forma aleatória
additional_nodes = [str(uuid.uuid4()) for _ in range(50)]
for i in range(len(additional_nodes) - 1):
    G.add_edge(additional_nodes[i], additional_nodes[random.randint(0, len(additional_nodes) - 1)])

##
network_communities = community.greedy_modularity_communities(G)
plot_bokeh_network(G, network_communities, 'Network', Blues8)
\end{codigo}
% \newpage
% \input{tex/includes/ergm_model.tex}
% \newpage
% \begin{codigo}[caption={API para engine de classificação de sentimento baseado em NLP utilizando o Flask como middleware http}, label={codigo:sentiment_api}, language=Python, breaklines=true]
  import os
import re
import threading
import time
import numpy as np
import pandas as pd
from dotenv import load_dotenv
from flask import Flask, jsonify, request
from nltk.corpus import stopwords
from nltk.stem.porter import PorterStemmer
from sklearn.ensemble import (DecisionTreeClassifier, LogisticRegression,
                              RandomForestClassifier)
from sklearn.feature_extraction.text import CountVectorizer
from sklearn.preprocessing import MinMaxScaler
from sklearn.svm import SVC
from xgboost import XGBClassifier


#####################################################
# Classe de pré-processamento de texto
class TextPreprocessor:
    @staticmethod
    def sanitize_phrase(phrase):
        """
        Realiza o pré-processamento de uma frase, removendo stopwords,
        realizando stemming e limpeza.

        Args:
            phrase (str): A frase a ser pré-processada.

        Returns:
            str: A frase pré-processada.
        """
        result = ""
        for word in phrase.split(" "):
            review = re.sub(r, ' ', word)
            review = review.lower()
            review = review.split()
            ps = PorterStemmer()
            # Stemming
            review = [ps.stem(word)
                      for word in review if word not in stopwords_set]
            if review and len(review[0]) > 2:
                result += review[0] + " "
        return result.strip()
#
# Classe de carregamento de dados
class DataLoader:
    @staticmethod
    def load_data():
        """
        Carrega os dados a partir de uma fonte externa.

        Returns:
            pd.DataFrame: O DataFrame contendo os dados carregados.
        """
        colab_events_url = os.getenv('URL_COLAB_EVENTS')
        colab_events_train = pd.read_csv(colab_events_url, low_memory=False)
        return colab_events_train
#
# Classe de serviço do modelo
class SentimentClassifier:
    def __init__(self):
        self.cv = CountVectorizer(stop_words=stopwords.words('portuguese'))
        self.sc = MinMaxScaler()
        self.models = {
            'random_forest': RandomForestClassifier(),
            'logistic_regression': LogisticRegression(),
            'decision_tree': DecisionTreeClassifier(),
            'svc': SVC(),
            'xgboost': XGBClassifier()
        }

    def preprocess_text(self, text):
        """
        Realiza o pré-processamento do texto, incluindo a vetorização.

        Args:
            text (str): O texto a ser pré-processado.

        Returns:
            numpy.ndarray: O vetor pré-processado.
        """
        sanitized_phrase = TextPreprocessor.sanitize_phrase(text)
        vectorized_phrase = self.cv.transform([sanitized_phrase]).toarray()
        standardized_vector = self.sc.transform(vectorized_phrase)
        return standardized_vector

    def train_models(self, x, y):
        """
        Treina os modelos do classificador de sentimento.

        Args:
            x (numpy.ndarray): As características do conjunto de treinamento.
            y (numpy.ndarray): Os rótulos do conjunto de treinamento.
        """
        for model_name, model in self.models.items():
            model.fit(x, y)

    def predict_sentiment(self, text, model_name):
        """
        Realiza a previsão do sentimento do texto usando o modelo especificado.

        Args:
            text (str): O texto a ser classificado.
            model_name (str): O nome do modelo a ser utilizado.

        Returns:
            int: A classe de sentimento prevista.
        """
        standardized_vector = self.preprocess_text(text)
        model = self.models[model_name]
        prediction = model.predict(standardized_vector)
        return int(prediction)
#
# Classe de serviço de aplicação
class ModelService:
    def __init__(self):
        self.sentiment_classifier = None
        self.data_loader = None

    def bootstrap(self):
        """
        Realiza a inicialização do serviço de modelo, carregando os dados e treinando os modelos.
        """
        colab_events_train = self.data_loader.load_data()
        words = self.sentiment_classifier.cv.fit_transform(
            colab_events_train.description)
        x = self.sentiment_classifier.sc.fit_transform(words.toarray())
        y = colab_events_train.iloc[:, 1]
        self.sentiment_classifier.train_models(x, y)

    def predict_sentiment(self, phrase, model_name):
        """
        Realiza a previsão do sentimento da frase usando o modelo especificado.

        Args:
            phrase (str): A frase a ser classificada.
            model_name (str): O nome do modelo a ser utilizado.

        Returns:
            dict: O resultado da previsão de sentimento.
        """
        prediction = self.sentiment_classifier.predict_sentiment(
            phrase, model_name)
        return {'sentiment': prediction}
#####################################################
load_dotenv()  # Carregar variáveis de ambiente do arquivo .env
#
##
class App:
    """
    Classe responsável por encapsular as funcionalidades do Flask e fornecer rotas para previsão de sentimento.

    Attributes:
        app (Flask): Instância do aplicativo Flask.
        model_service (ModelService): Instância do serviço de modelo.
        is_training (bool): Variável para indicar se o treinamento está em andamento.

    Methods:
        predict(): Rota para previsão de sentimento.
        initialize_model_service(): Inicializa o serviço de modelo.
        get_status(): Rota para obter o status da aplicação.
        run(): Executa o aplicativo Flask.
    """

    def __init__(self):
        """
        Inicializa a classe App.

        Initializes:
            app (Flask): Instância do aplicativo Flask.
            model_service (ModelService): Instância do serviço de modelo.
            is_training (bool): Variável para indicar se o treinamento está em andamento.
        """
        self.app = Flask(__name__)
        self.model_service = ModelService()
        self.is_training = True

        self.app.route('/predict', methods=['POST'])(self.predict)
        self.app.route('/status', methods=['GET'])(self.get_status)
        self.app.before_first_request(self.initialize_model_service)    

    def initialize_model_service(self):
        """
        Inicializa o serviço de modelo.

        Este método é executado antes da primeira requisição ao aplicativo Flask.
        Carrega os dados e treina os modelos.
        """
        self.model_service.data_loader = DataLoader()
        self.model_service.sentiment_classifier = SentimentClassifier()

        # Inicia o treinamento em uma thread separada
        training_thread = Thread(target=self.train_models_thread)
        training_thread.start()

    def train_models_thread(self):
        """
        Método executado em uma thread separada para realizar o treinamento dos modelos.
        """
        self.is_training = True
        colab_events_train = self.model_service.data_loader.load_data()
        words = self.model_service.sentiment_classifier.cv.fit_transform(colab_events_train.description)
        x = self.model_service.sentiment_classifier.sc.fit_transform(words.toarray())
        y = colab_events_train.iloc[:, 1]
        self.model_service.sentiment_classifier.train_models(x, y)
        self.is_training = False

    def get_status(self):
        """
        Rota para obter o status da aplicação, incluindo a porcentagem de conclusão do treinamento.

        Returns:
            str: O status da aplicação com a porcentagem de conclusão do treinamento.
        """
        if self.is_training:
            progress = self.calculate_training_progress()
            status = {'status': 'Training in progress', 'progress': progress}
        else:
            status = {'status': 'Available'}

        return jsonify(status)

    def calculate_training_progress(self):
        """
        Calcula a porcentagem de conclusão do treinamento.

        Returns:
            float: A porcentagem de conclusão do treinamento.
        """
        total_steps = len(self.model_service.sentiment_classifier.models)
        completed_steps = total_steps - sum([model.n_jobs for model in self.model_service.sentiment_classifier.models.values()])
        progress = (completed_steps / total_steps) * 100
        return progress
##
if __name__ == '__main__':
    app = App()
    app.run()
\end{codigo}
	\chapter{Tabelas de Métricas de Pressão Social}
	\label{chapter:tables_social_pressure}
	
\begin{table}[htbp]
	\centering
	\caption{Métricas de pressão social do tópico de Mobilidade Urbana discutido na \autoref{section:mobilidade-urbana}}
	\label{tab:eventos_populares_mobility}
	\begin{tabular}{|l|c|c|c|c|}
		\hline
		\textbf{Tipo de Evento}                         & \textbf{Eventos} & \textbf{Score} & \textbf{Persona} \\
		\hline
		1725:Bloqueio na via                            & 16               & -0.2272        & 0.6535           \\
		\hline
		7:Ponto de infração de trânsito recorrente      & 16               & -0.2458        & 0.6454           \\
		\hline
		3:Buraco nas vias                               & 15               & -0.2305        & 0.1872           \\
		\hline
		3938:Entulho na calçada/via pública             & 13               & -0.0303        & 0.0022           \\
		\hline
		7561:Ponto de alagamento                        & 11               & -0.1329        & 0.3171           \\
		\hline
		7558:Ocupação irregular de área pública         & 11               & -0.2296        & 0.5789           \\
		\hline
		1727:Equipamento público danificado             & 11               & -0.2151        & 0.2941           \\
		\hline
		3917:Calçada irregular                          & 10               & -0.1119        & 0.1880           \\
		\hline
		1749:Manutenção de semáforo                     & 10               & -0.1963        & 0.4333           \\
		\hline
		3335:Lâmpada apagada à noite                    & 10               & -0.1377        & 0.1576           \\
		\hline
		1675:Bueiro sem tampa                           & 9                & -0.3101        & 0.0667           \\
		\hline
		1703:Ponto de travessia irregular               & 9                & -0.1792        & 0.2000           \\
		\hline
		1704:Calçada inexistente                        & 9                & -0.1193        & 0.3103           \\
		\hline
		1741:Manutenção de ciclovia/ciclofaixa          & 9                & -0.2643        & 0.3487           \\
		\hline
		1729:Ponto de transporte clandestino            & 8                & -0.1662        & 0.9333           \\
		\hline
		1711:Ponto de ônibus danificado                 & 8                & -0.1235        & 0.5263           \\
		\hline
		1734:Retirada de árvore                         & 8                & -0.2225        & 0.0847           \\
		\hline
		7574:Poda de árvore                             & 7                & -0.0959        & 0.0632           \\
		\hline
		10889:Placa de sinalização quebrada/inexistente & 7                & -0.1963        & 0.0843           \\
		\hline
		1733:Publicidade irregular em via pública       & 7                & -0.2081        & 0.6471           \\
		\hline
	\end{tabular}
\end{table}


\begin{table}[htbp]
	\centering
	\caption{Métricas de pressão social do tópico de Infrações de Trânsito discutido na \autoref{sec:eventos_populares_traffic}}
	\label{tab:eventos_populares_traffic}
	\begin{tabular}{|l|c|c|c|c|}
		\hline
		\textbf{Tipo de Evento}                           & \textbf{Eventos} & \textbf{Score} & \textbf{Persona} \\
		\hline
		7:Ponto de infração de trânsito recorrente        & 4                & -0.2781        & 0.7586           \\
		\hline
		1685:Estabelecimento com acessibilidade irregular & 4                & -0.3392        & 0.7500           \\
		\hline
		3:Buraco nas vias                                 & 3                & -0.2719        & 0.7273           \\
		\hline
		3938:Entulho na calçada/via pública               & 3                & -0.0405        & 0.2903           \\
		\hline
		1725:Bloqueio na via                              & 3                & -0.2907        & 0.8125           \\
		\hline
		1727:Equipamento público danificado               & 3                & -0.1544        & 1.0000           \\
		\hline
		7590:Bueiro entupido                              & 2                & -0.0036        & 0.0000           \\
		\hline
		7574:Poda de árvore                               & 2                & -0.5093        & 0.2500           \\
		\hline
		7558:Ocupação irregular de área pública           & 2                & -0.3412        & 0.9000           \\
		\hline
		3917:Calçada irregular                            & 2                & -0.4922        & 0.8000           \\
		\hline
		3335:Lâmpada apagada à noite                      & 2                & -0.0344        & 0.0000           \\
		\hline
		1749:Manutenção de semáforo                       & 2                & -0.2593        & 1.0000           \\
		\hline
		10889:Placa de sinalização quebrada/inexistente   & 2                & -0.0677        & 0.3333           \\
		\hline
		1703:Ponto de travessia irregular                 & 2                & -0.4231        & 0.5000           \\
		\hline
		1675:Bueiro sem tampa                             & 2                & -0.1105        & 1.0000           \\
		\hline
		1711:Ponto de ônibus danificado                   & 2                & -0.2792        & 0.6000           \\
		\hline
		1735:Via de terra com desnível                    & 1                & 0.5242         & 1.0000           \\
		\hline
		10869:Ônibus/trem/metrô danificado                & 1                & -0.7298        & 1.0000           \\
		\hline
		9978:Conservação (via pública)                    & 1                & 0.4875         & 0.0000           \\
		\hline
		9973:Má conduta de motorista ou cobrador          & 1                & -0.3405        & 0.3333           \\
		\hline
	\end{tabular}
\end{table}

\begin{table}[htbp]
	\centering
	\caption{Métricas de pressão social do tópico de Transporte Público discutido na \autoref{sec:eventos_populares_busfare}}
	\label{tab:eventos_populares_busfare}
	\begin{tabular}{|l|c|c|c|c|}
		\hline
		\textbf{Tipo de Evento}                              & \textbf{Eventos} & \textbf{Score} & \textbf{Persona} \\
		\hline
		1729:Ponto de transporte clandestino                 & 7                & -0.2516        & 0.7818           \\
		\hline
		3356:Ônibus fora do horário/rota                     & 7                & -0.2312        & 0.8861           \\
		\hline
		1727:Equipamento público danificado                  & 6                & -0.1464        & 0.4737           \\
		\hline
		9973:Má conduta de motorista ou cobrador             & 6                & -0.3900        & 0.7317           \\
		\hline
		10869:Ônibus/trem/metrô danificado                   & 5                & -0.2377        & 0.3855           \\
		\hline
		7:Ponto de infração de trânsito recorrente           & 5                & -0.2622        & 0.6483           \\
		\hline
		405:Ônibus superlotado                               & 5                & -0.2701        & 1.0000           \\
		\hline
		61:Ônibus/trem/metrô superlotado                     & 5                & -0.3126        & 0.8333           \\
		\hline
		59:Estação de ônibus/trem/metrô danificada           & 4                & -0.2304        & 0.3333           \\
		\hline
		1711:Ponto de ônibus danificado                      & 4                & -0.1144        & 0.3654           \\
		\hline
		1744:Ônibus danificado                               & 4                & -0.1378        & 0.7818           \\
		\hline
		1760:Estação de ônibus danificada                    & 4                & -0.2573        & 0.6000           \\
		\hline
		1683:Estabelecimento sem alvará                      & 3                & -0.3125        & 1.0000           \\
		\hline
		10889:Placa de sinalização quebrada/inexistente      & 2                & -0.1365        & 0.1818           \\
		\hline
		1685:Estabelecimento com acessibilidade irregular    & 2                & -0.1614        & 1.0000           \\
		\hline
		1704:Calçada inexistente                             & 2                & -0.2354        & 0.4000           \\
		\hline
		1710:Ponto de assalto/roubo                          & 2                & 0.0313         & 0.4348           \\
		\hline
		17:Rampa de acessibilidade irregular ou inexistente  & 2                & -0.4443        & 0.0000           \\
		\hline
		3956:Faixa de pedestre apagada                       & 2                & -0.1145        & 0.0000           \\
		\hline
		1761:Manutenção/implantação de infraestrutura viária & 2                & -0.2283        & 0.5455           \\
		\hline
	\end{tabular}
\end{table}

\begin{table}[htbp]
	\centering
	\caption{Métricas de pressão social do tópico de Saúde Pública discutido na \autoref{sec:eventos_populares_public_health}}
	\label{tab:eventos_populares_public_health}
	\begin{tabular}{|l|c|c|c|c|}
		\hline
		\textbf{Tipo de Evento}                               & \textbf{Eventos} & \textbf{Score} & \textbf{Persona} \\
		\hline
		1707:Foco de mosquito da dengue/zika                  & 7                & -0.1986        & 0.4800           \\
		\hline
		1694:Descarte irregular de lixo                       & 6                & -0.1959        & 0.4783           \\
		\hline
		1685:Estabelecimento com acessibilidade irregular     & 5                & -0.3009        & 0.7500           \\
		\hline
		7590:Bueiro entupido                                  & 5                & 0.0413         & 0.1333           \\
		\hline
		1681:Ponto recorrente de poluição sonora              & 4                & -0.3863        & 1.0000           \\
		\hline
		1706:Infestação animais perigosos                     & 4                & -0.3524        & 0.4667           \\
		\hline
		1742:Vazamento de esgoto                              & 4                & -0.2332        & 0.4286           \\
		\hline
		3353:Aglomeração de pessoas                           & 4                & -0.4755        & 0.8000           \\
		\hline
		1682:Estabelecimento com condição sanitária irregular & 3                & -0.2063        & 0.4444           \\
		\hline
		7561:Ponto de alagamento                              & 3                & -0.2729        & 0.2105           \\
		\hline
		1683:Estabelecimento sem alvará                       & 2                & -0.6291        & 1.0000           \\
		\hline
		7558:Ocupação irregular de área pública               & 2                & -0.1556        & 0.6667           \\
		\hline
		9980:Atendimento na Clínica da Família                & 2                & 0.4684         & 0.5000           \\
		\hline
		10901:Esgoto a céu aberto                             & 2                & -0.5841        & 0.4000           \\
		\hline
		1695:Praia suja                                       & 1                & -0.5235        & 0.0000           \\
		\hline
		1732:Área com risco de deslizamento                   & 1                & -0.4953        & 1.0000           \\
		\hline
	\end{tabular}
\end{table}

\begin{table}[htbp]
	\centering
	\caption{Métricas de pressão social do tópico de Distanciamento Social discutido na \autoref{sec:eventos_populares_social_distancing}}
	\label{tab:eventos_populares_social_distancing}
	\begin{tabular}{|l|c|c|c|c|}
		\hline
		\textbf{Tipo de Evento}                               & \textbf{Eventos} & \textbf{Score} & \textbf{Persona} \\
		\hline
		1681:Ponto recorrente de poluição sonora              & 4                & -0.3471        & 0.9867           \\
		\hline
		1682:Estabelecimento com condição sanitária irregular & 4                & -0.5293        & 0.8889           \\
		\hline
		7558:Ocupação irregular de área pública               & 4                & -0.2589        & 0.7857           \\
		\hline
		3353:Aglomeração de pessoas                           & 4                & -0.2041        & 0.9365           \\
		\hline
		3352:Comércio aberto irregularmente                   & 4                & -0.1758        & 0.9375           \\
		\hline
		1739:Evento Irregular                                 & 4                & -0.2484        & 1.0000           \\
		\hline
		61:Ônibus/trem/metrô superlotado                      & 3                & -0.1978        & 0.8333           \\
		\hline
		1749:Manutenção de semáforo                           & 3                & 0.2840         & 1.0000           \\
		\hline
		1683:Estabelecimento sem alvará                       & 3                & -0.1364        & 0.8333           \\
		\hline
		3938:Entulho na calçada/via pública                   & 3                & -0.3678        & 0.3333           \\
		\hline
		1706:Infestação animais perigosos                     & 3                & -0.1663        & 0.5000           \\
		\hline
		1725:Bloqueio na via                                  & 3                & -0.1948        & 0.6667           \\
		\hline
		405:Ônibus superlotado                                & 3                & -0.0226        & 1.0000           \\
		\hline
		7561:Ponto de alagamento                              & 2                & -0.3893        & 1.0000           \\
		\hline
		1732:Área com risco de deslizamento                   & 2                & -0.4953        & 1.0000           \\
		\hline
		1744:Ônibus danificado                                & 1                & -0.0703        & 1.0000           \\
		\hline
		1742:Vazamento de esgoto                              & 1                & -0.4113        & 1.0000           \\
		\hline
		1727:Equipamento público danificado                   & 1                & -0.0688        & 1.0000           \\
		\hline
		1708:Ponto de tráfico de drogas                       & 1                & -0.3938        & 1.0000           \\
		\hline
		3356:Ônibus fora do horário/rota                      & 1                & -0.4558        & 1.0000           \\
		\hline
	\end{tabular}
\end{table}

\begin{table}[htbp]
	\centering
	\caption{Métricas de pressão social do tópico de Mudança Climática discutido na \autoref{sec:eventos_populares_weather}}
	\label{tab:eventos_populares_weather}
	\begin{tabular}{|l|c|c|c|c|}
		\hline
		\textbf{Tipo de Evento}                         & \textbf{Eventos} & \textbf{Score} & \textbf{Persona} \\
		\hline
		7574:Poda de árvore                             & 13               & -0.1698        & 0.2177           \\
		\hline
		1734:Retirada de árvore                         & 10               & -0.1851        & 0.1681           \\
		\hline
		1727:Equipamento público danificado             & 9                & -0.2080        & 0.2727           \\
		\hline
		3938:Entulho na calçada/via pública             & 8                & -0.0117        & 0.2651           \\
		\hline
		3934:Fiação irregular                           & 7                & -0.4324        & 0.2800           \\
		\hline
		1696:Mato alto                                  & 7                & -0.0151        & 0.2411           \\
		\hline
		25:Vazamento de água                            & 6                & -0.0735        & 0.3103           \\
		\hline
		7590:Bueiro entupido                            & 6                & 0.0125         & 0.3368           \\
		\hline
		3917:Calçada irregular                          & 6                & -0.1955        & 0.2292           \\
		\hline
		1701:Desmatamento irregular                     & 6                & -0.3553        & 0.8095           \\
		\hline
		1707:Foco de mosquito da dengue/zika            & 5                & -0.2218        & 0.3659           \\
		\hline
		10889:Placa de sinalização quebrada/inexistente & 5                & -0.0963        & 0.4444           \\
		\hline
		10869:Ônibus/trem/metrô danificado              & 5                & -0.1651        & 0.8000           \\
		\hline
		7561:Ponto de alagamento                        & 5                & -0.1167        & 0.4471           \\
		\hline
		7558:Ocupação irregular de área pública         & 5                & -0.3763        & 0.5714           \\
		\hline
		1732:Área com risco de deslizamento             & 5                & -0.3487        & 0.4468           \\
		\hline
		1742:Vazamento de esgoto                        & 5                & -0.2076        & 0.5556           \\
		\hline
		1706:Infestação animais perigosos               & 5                & -0.0807        & 0.3077           \\
		\hline
		11656:Plantar uma árvore / Arborização          & 5                & -0.0657        & 0.0000           \\
		\hline
		1678:Falta de água                              & 4                & -0.3369        & 0.7778           \\
		\hline
	\end{tabular}
\end{table}

\begin{table}[htbp]
	\centering
	\caption{Métricas de pressão social do tópico de Paisagismo discutido na \autoref{sec:eventos_populares_landscape}}
	\label{tab:eventos_populares_landscape}
	\begin{tabular}{|l|c|c|c|c|}
		\hline
		\textbf{Tipo de Evento}                 & \textbf{Eventos} & \textbf{Score} & \textbf{Persona} \\
		\hline
		7574:Poda de árvore                     & 10               & -0.0504        & 0.1086           \\
		\hline
		1734:Retirada de árvore                 & 9                & -0.1876        & 0.0840           \\
		\hline
		1696:Mato alto                          & 7                & 0.0825         & 0.1143           \\
		\hline
		7590:Bueiro entupido                    & 6                & 0.0411         & 0.0000           \\
		\hline
		1701:Desmatamento irregular             & 6                & -0.3021        & 0.9000           \\
		\hline
		1727:Equipamento público danificado     & 6                & -0.2540        & 0.0625           \\
		\hline
		11656:Plantar uma árvore / Arborização  & 5                & -0.0235        & 0.0000           \\
		\hline
		1694:Descarte irregular de lixo         & 5                & -0.1186        & 0.2759           \\
		\hline
		1732:Área com risco de deslizamento     & 5                & -0.3202        & 0.3333           \\
		\hline
		1706:Infestação animais perigosos       & 4                & -0.1076        & 0.2857           \\
		\hline
		7558:Ocupação irregular de área pública & 4                & -0.3889        & 1.0000           \\
		\hline
		3934:Fiação irregular                   & 4                & -0.1428        & 0.3750           \\
		\hline
		25:Vazamento de água                    & 4                & -0.3128        & 0.3333           \\
		\hline
		1695:Praia suja                         & 4                & 0.4822         & 0.0000           \\
		\hline
		1704:Calçada inexistente                & 3                & -0.1514        & 0.0000           \\
		\hline
		3338:Iluminação pública irregular       & 3                & -0.0891        & 0.1176           \\
		\hline
		3352:Comércio aberto irregularmente     & 3                & 0.0877         & 0.0000           \\
		\hline
		1690:Lâmpada acesa de dia               & 3                & -0.1075        & 0.4000           \\
		\hline
		1688:Falta de energia                   & 3                & -0.0585        & 0.1429           \\
		\hline
		7561:Ponto de alagamento                & 3                & -0.5951        & 0.0000           \\
		\hline
	\end{tabular}
\end{table}

\begin{table}[htbp]
	\centering
	\caption{Métricas de pressão social do tópico de Meio Ambiente discutido na \autoref{sec:eventos_populares_environment}}
	\label{tab:eventos_populares_environment}
	\begin{tabular}{|l|c|c|c|c|}
		\hline
		\textbf{Tipo de Evento}                     & \textbf{Eventos} & \textbf{Score} & \textbf{Persona} \\
		\hline
		1694:Descarte irregular de lixo             & 9                & -0.2679        & 0.3349           \\
		\hline
		1701:Desmatamento irregular                 & 8                & -0.2849        & 0.5517           \\
		\hline
		7574:Poda de árvore                         & 8                & -0.3409        & 0.5574           \\
		\hline
		1700:Ponto de queimada irregular recorrente & 8                & -0.4036        & 0.4615           \\
		\hline
		1742:Vazamento de esgoto                    & 7                & -0.2459        & 0.3902           \\
		\hline
		1696:Mato alto                              & 7                & -0.1383        & 0.2885           \\
		\hline
		1728:Imóvel ou terreno abandonado           & 6                & -0.4314        & 0.3084           \\
		\hline
		7590:Bueiro entupido                        & 6                & 0.0412         & 0.1121           \\
		\hline
		1681:Ponto recorrente de poluição sonora    & 6                & -0.1772        & 0.7814           \\
		\hline
		1697:Maus tratos a animais                  & 5                & -0.4923        & 0.6667           \\
		\hline
		25:Vazamento de água                        & 5                & -0.3153        & 0.4286           \\
		\hline
		1686:Emissão de fumaça preta                & 5                & -0.2038        & 0.4211           \\
		\hline
		1704:Calçada inexistente                    & 4                & -0.3371        & 0.4706           \\
		\hline
		1727:Equipamento público danificado         & 4                & -0.3624        & 0.3409           \\
		\hline
		1695:Praia suja                             & 4                & -0.1338        & 0.4857           \\
		\hline
		1734:Retirada de árvore                     & 4                & -0.3201        & 0.2632           \\
		\hline
		1732:Área com risco de deslizamento         & 3                & -0.4890        & 1.0000           \\
		\hline
		1741:Manutenção de ciclovia/ciclofaixa      & 3                & -0.2437        & 0.6667           \\
		\hline
		1707:Foco de mosquito da dengue/zika        & 3                & -0.3356        & 0.3200           \\
		\hline
		7561:Ponto de alagamento                    & 3                & -0.1793        & 0.4651           \\
		\hline
	\end{tabular}
\end{table}

\begin{table}[htbp]
	\centering
	\caption{Métricas de pressão social do tópico de Gentrificação discutido na \autoref{sec:eventos_populares_social_gentrification}}
	\label{tab:eventos_populares_social_gentrification}
	\begin{tabular}{|l|c|c|c|c|}
		\hline
		\textbf{Tipo de Evento}                  & \textbf{Eventos} & \textbf{Score} & \textbf{Persona} \\
		\hline
		1694:Descarte irregular de lixo          & 11               & -0.2451        & 0.5593           \\
		\hline
		7558:Ocupação irregular de área pública  & 11               & -0.2313        & 0.6857           \\
		\hline
		1728:Imóvel ou terreno abandonado        & 11               & -0.2842        & 0.5366           \\
		\hline
		1681:Ponto recorrente de poluição sonora & 9                & -0.4871        & 1.0000           \\
		\hline
		1696:Mato alto                           & 8                & -0.0485        & 0.4173           \\
		\hline
		1727:Equipamento público danificado      & 7                & -0.1749        & 0.5672           \\
		\hline
		1742:Vazamento de esgoto                 & 6                & -0.1044        & 0.5714           \\
		\hline
		1701:Desmatamento irregular              & 5                & -0.2925        & 0.6875           \\
		\hline
		1726:Patrimônio histórico em risco       & 4                & -0.4648        & 0.7778           \\
		\hline
		10901:Esgoto a céu aberto                & 4                & -0.8154        & 0.7500           \\
		\hline
		7574:Poda de árvore                      & 4                & -0.1893        & 0.2619           \\
		\hline
		7561:Ponto de alagamento                 & 4                & -0.4481        & 0.6250           \\
		\hline
		3353:Aglomeração de pessoas              & 4                & -0.2428        & 1.0000           \\
		\hline
		1750:Varrição                            & 4                & -0.1550        & 0.4444           \\
		\hline
		1734:Retirada de árvore                  & 4                & -0.1920        & 0.5556           \\
		\hline
		1697:Maus tratos a animais               & 4                & -0.1677        & 0.6250           \\
		\hline
		1702:Aterro sanitário irregular          & 3                & -0.3067        & 0.0000           \\
		\hline
		3934:Fiação irregular                    & 3                & -0.0069        & 0.8000           \\
		\hline
		1695:Praia suja                          & 3                & -0.2380        & 0.5000           \\
		\hline
		1692:Lixeira quebrada                    & 3                & -0.0418        & 0.0000           \\
		\hline
	\end{tabular}
\end{table}

\begin{table}[htbp]
	\centering
	\caption{Métricas de pressão social do tópico de Higienismo Urbano discutido na \autoref{sec:eventos_populares_homeland}}
	\label{tab:eventos_populares_homeland}
	\begin{tabular}{|l|c|c|c|c|}
		\hline
		\textbf{Tipo de Evento}                               & \textbf{Eventos} & \textbf{Score} & \textbf{Persona} \\
		\hline
		7558:Ocupação irregular de área pública               & 11               & -0.2045        & 0.4783           \\
		\hline
		1715:Veículo abandonado                               & 8                & -0.5278        & 0.3704           \\
		\hline
		3938:Entulho na calçada/via pública                   & 8                & -0.1915        & 0.5930           \\
		\hline
		1708:Ponto de tráfico de drogas                       & 8                & -0.1438        & 1.0000           \\
		\hline
		3353:Aglomeração de pessoas                           & 7                & -0.4988        & 0.9167           \\
		\hline
		1696:Mato alto                                        & 7                & -0.2704        & 0.5385           \\
		\hline
		1710:Ponto de assalto/roubo                           & 7                & -0.2319        & 0.7273           \\
		\hline
		3335:Lâmpada apagada à noite                          & 5                & -0.0654        & 0.4375           \\
		\hline
		1728:Imóvel ou terreno abandonado                     & 5                & -0.4752        & 0.3636           \\
		\hline
		1725:Bloqueio na via                                  & 5                & -0.5444        & 1.0000           \\
		\hline
		1682:Estabelecimento com condição sanitária irregular & 4                & -0.2800        & 0.2000           \\
		\hline
		1726:Patrimônio histórico em risco                    & 3                & -0.4025        & 0.5000           \\
		\hline
		7574:Poda de árvore                                   & 3                & -0.1037        & 0.1667           \\
		\hline
		1704:Calçada inexistente                              & 3                & -0.3245        & 1.0000           \\
		\hline
		1706:Infestação animais perigosos                     & 2                & -0.4822        & 0.5000           \\
		\hline
		1700:Ponto de queimada irregular recorrente           & 2                & 0.4031         & 1.0000           \\
		\hline
		3917:Calçada irregular                                & 2                & -0.0305        & 1.0000           \\
		\hline
		1695:Praia suja                                       & 2                & -0.4259        & 1.0000           \\
		\hline
		3338:Iluminação pública irregular                     & 2                & 0.3730         & 0.0000           \\
		\hline
		9981:Atendimento no Posto de Saúde                    & 1                & 0.5002         & 0.0000           \\
		\hline
	\end{tabular}
\end{table}

\begin{table}[htbp]
	\centering
	\caption{Métricas de pressão social do tópico de Segurança Pública discutido na \autoref{sec:eventos_populares_security}}
	\label{tab:eventos_populares_security}
	\begin{tabular}{|l|c|c|c|c|}
		\hline
		\textbf{Tipo de Evento}                     & \textbf{Eventos} & \textbf{Score} & \textbf{Persona} \\
		\hline
		7558:Ocupação irregular de área pública     & 19               & -0.3704        & 0.8333           \\
		\hline
		1727:Equipamento público danificado         & 15               & -0.0478        & 0.3250           \\
		\hline
		1681:Ponto recorrente de poluição sonora    & 14               & -0.4606        & 0.9802           \\
		\hline
		1728:Imóvel ou terreno abandonado           & 12               & -0.2901        & 0.6154           \\
		\hline
		1696:Mato alto                              & 12               & -0.0493        & 0.2000           \\
		\hline
		1697:Maus tratos a animais                  & 7                & -0.3150        & 1.0000           \\
		\hline
		1701:Desmatamento irregular                 & 5                & -0.2063        & 0.6667           \\
		\hline
		9973:Má conduta de motorista ou cobrador    & 5                & -0.5674        & 1.0000           \\
		\hline
		1706:Infestação animais perigosos           & 5                & 0.1264         & 0.0000           \\
		\hline
		10869:Ônibus/trem/metrô danificado          & 5                & 0.4379         & 0.0000           \\
		\hline
		7572:Estabelecimento sem nota fiscal        & 4                & -0.0990        & 1.0000           \\
		\hline
		1740:Bicicletário/paraciclo danificado      & 4                & 0.4526         & 0.0000           \\
		\hline
		1729:Ponto de transporte clandestino        & 4                & -0.0816        & 1.0000           \\
		\hline
		121:Pintura                                 & 4                & 0.0247         & 0.0000           \\
		\hline
		1726:Patrimônio histórico em risco          & 3                & -0.6750        & 1.0000           \\
		\hline
		1703:Ponto de travessia irregular           & 3                & -0.3048        & 0.0000           \\
		\hline
		1700:Ponto de queimada irregular recorrente & 3                & 0.0365         & 1.0000           \\
		\hline
		1692:Lixeira quebrada                       & 3                & 0.0188         & 0.0000           \\
		\hline
		1686:Emissão de fumaça preta                & 2                & -0.6271        & 1.0000           \\
		\hline
		1709:Ponto de exploração sexual de menores  & 2                & -0.0083        & 1.0000           \\
		\hline
	\end{tabular}
\end{table}

\begin{table}[htbp]
	\centering
	\caption{Métricas de pressão social do tópico de Política de Drogas discutido na \autoref{sec:eventos_populares_drugs}}
	\label{tab:eventos_populares_drugs}
	\begin{tabular}{|l|c|c|c|c|}
		\hline
		\textbf{Tipo de Evento}                  & \textbf{Eventos} & \textbf{Score} & \textbf{Persona} \\
		\hline
		7558:Ocupação irregular de área pública  & 11               & -0.2017        & 0.6667           \\
		\hline
		1728:Imóvel ou terreno abandonado        & 10               & -0.3732        & 0.8462           \\
		\hline
		1727:Equipamento público danificado      & 9                & -0.3882        & 0.3333           \\
		\hline
		3353:Aglomeração de pessoas              & 6                & -0.2493        & 1.0000           \\
		\hline
		1694:Descarte irregular de lixo          & 5                & -0.2694        & 0.3333           \\
		\hline
		1696:Mato alto                           & 5                & 0.3668         & 0.0000           \\
		\hline
		1726:Patrimônio histórico em risco       & 5                & -0.8286        & 1.0000           \\
		\hline
		7574:Poda de árvore                      & 5                & -0.5005        & 0.5000           \\
		\hline
		1681:Ponto recorrente de poluição sonora & 3                & -0.4550        & 1.0000           \\
		\hline
		1707:Foco de mosquito da dengue/zika     & 3                & -0.4270        & 1.0000           \\
		\hline
		1688:Falta de energia                    & 2                & -0.1643        & 0.0000           \\
		\hline
		3338:Iluminação pública irregular        & 1                & -0.2125        & 0.0000           \\
		\hline
		1732:Área com risco de deslizamento      & 1                & -0.4953        & 1.0000           \\
		\hline
		1750:Varrição                            & 1                & 0.3936         & 1.0000           \\
		\hline
		1740:Bicicletário/paraciclo danificado   & 1                & 0.5062         & 1.0000           \\
		\hline
		1734:Retirada de árvore                  & 1                & 0.4690         & 1.0000           \\
		\hline
		9973:Má conduta de motorista ou cobrador & 1                & -0.0712        & 1.0000           \\
		\hline
	\end{tabular}
\end{table}

\begin{table}[htbp]
	\centering
	\caption{Métricas de pressão social do tópico de Pânico Moral discutido na \autoref{sec:eventos_populares_moral_panic}}
	\label{tab:eventos_populares_moral_panic}
	\begin{tabular}{|l|c|c|c|c|}
		\hline
		\textbf{Tipo de Evento}                    & \textbf{Eventos} & \textbf{Score} & \textbf{Persona} \\
		\hline
		1681:Ponto recorrente de poluição sonora   & 9                & -0.3983        & 1.0000           \\
		\hline
		3353:Aglomeração de pessoas                & 9                & -0.2718        & 0.9394           \\
		\hline
		7558:Ocupação irregular de área pública    & 8                & -0.3703        & 0.9091           \\
		\hline
		3352:Comércio aberto irregularmente        & 7                & -0.2571        & 0.9375           \\
		\hline
		1694:Descarte irregular de lixo            & 5                & -0.3687        & 0.7500           \\
		\hline
		1739:Evento Irregular                      & 5                & -0.5900        & 1.0000           \\
		\hline
		1728:Imóvel ou terreno abandonado          & 4                & -0.3551        & 1.0000           \\
		\hline
		7574:Poda de árvore                        & 3                & -0.0028        & 0.2500           \\
		\hline
		1727:Equipamento público danificado        & 3                & -0.1315        & 0.4167           \\
		\hline
		1726:Patrimônio histórico em risco         & 2                & -0.3454        & 1.0000           \\
		\hline
		1753:Construção irregular                  & 1                & -0.1647        & 1.0000           \\
		\hline
		3934:Fiação irregular                      & 1                & -0.1474        & 1.0000           \\
		\hline
		3338:Iluminação pública irregular          & 1                & -0.4396        & 0.0000           \\
		\hline
		1734:Retirada de árvore                    & 1                & -0.0372        & 0.5000           \\
		\hline
		1742:Vazamento de esgoto                   & 1                & 0.0068         & 1.0000           \\
		\hline
		1686:Emissão de fumaça preta               & 1                & -0.6271        & 1.0000           \\
		\hline
		1733:Publicidade irregular em via pública  & 1                & -0.6114        & 1.0000           \\
		\hline
		1711:Ponto de ônibus danificado            & 1                & -0.4399        & 0.0000           \\
		\hline
		1709:Ponto de exploração sexual de menores & 1                & -0.2906        & 0.5000           \\
		\hline
		1692:Lixeira quebrada                      & 1                & 0.0188         & 0.0000           \\
		\hline
	\end{tabular}
\end{table}

\begin{table}[htbp]
	\centering
	\caption{Métricas de pressão social do tópico de Política Tributária discutido na \autoref{sec:eventos_populares_taxes}}
	\label{tab:eventos_populares_taxes}
	\begin{tabular}{|l|c|c|c|c|}
		\hline
		\textbf{Tipo de Evento}                               & \textbf{Eventos} & \textbf{Score} & \textbf{Persona} \\
		\hline
		1694:Descarte irregular de lixo                       & 9                & -0.0427        & 0.8966           \\
		\hline
		3:Buraco nas vias                                     & 8                & -0.2060        & 0.9167           \\
		\hline
		3917:Calçada irregular                                & 7                & -0.1545        & 0.4444           \\
		\hline
		7:Ponto de infração de trânsito recorrente            & 6                & -0.5726        & 1.0000           \\
		\hline
		7574:Poda de árvore                                   & 6                & -0.3658        & 0.9231           \\
		\hline
		1696:Mato alto                                        & 6                & -0.0494        & 0.8276           \\
		\hline
		3938:Entulho na calçada/via pública                   & 6                & -0.0718        & 1.0000           \\
		\hline
		1678:Falta de água                                    & 6                & -0.2894        & 1.0000           \\
		\hline
		3335:Lâmpada apagada à noite                          & 6                & -0.0613        & 0.6897           \\
		\hline
		1715:Veículo abandonado                               & 5                & -0.2586        & 1.0000           \\
		\hline
		1727:Equipamento público danificado                   & 5                & -0.0481        & 0.8000           \\
		\hline
		1735:Via de terra com desnível                        & 5                & -0.1554        & 1.0000           \\
		\hline
		3338:Iluminação pública irregular                     & 5                & -0.0619        & 0.6667           \\
		\hline
		1683:Estabelecimento sem alvará                       & 5                & -0.3461        & 1.0000           \\
		\hline
		1682:Estabelecimento com condição sanitária irregular & 5                & -0.2227        & 1.0000           \\
		\hline
		7590:Bueiro entupido                                  & 5                & -0.2488        & 0.9474           \\
		\hline
		25:Vazamento de água                                  & 5                & -0.4253        & 1.0000           \\
		\hline
		1750:Varrição                                         & 4                & -0.2105        & 0.6667           \\
		\hline
		7558:Ocupação irregular de área pública               & 4                & -0.4740        & 1.0000           \\
		\hline
		1742:Vazamento de esgoto                              & 4                & -0.1369        & 0.8500           \\
		\hline
	\end{tabular}
\end{table}

\begin{table}[htbp]
	\centering
	\caption{Métricas de pressão social do tópico de Corrupção discutido na \autoref{sec:eventos_populares_corruption}}
	\label{tab:eventos_populares_corruption}
	\begin{tabular}{|l|c|c|c|c|}
		\hline
		\textbf{Tipo de Evento}                         & \textbf{Eventos} & \textbf{Score} & \textbf{Persona} \\
		\hline
		1681:Ponto recorrente de poluição sonora        & 4                & -0.3772        & 1.0000           \\
		\hline
		7:Ponto de infração de trânsito recorrente      & 3                & -0.2115        & 1.0000           \\
		\hline
		1707:Foco de mosquito da dengue/zika            & 2                & 0.2704         & 1.0000           \\
		\hline
		1708:Ponto de tráfico de drogas                 & 2                & -0.2053        & 1.0000           \\
		\hline
		7558:Ocupação irregular de área pública         & 2                & -0.2315        & 1.0000           \\
		\hline
		3352:Comércio aberto irregularmente             & 2                & 0.1643         & 1.0000           \\
		\hline
		1749:Manutenção de semáforo                     & 1                & -0.0244        & 1.0000           \\
		\hline
		7572:Estabelecimento sem nota fiscal            & 1                & -0.3197        & 1.0000           \\
		\hline
		3917:Calçada irregular                          & 1                & -0.1449        & 1.0000           \\
		\hline
		3338:Iluminação pública irregular               & 1                & -0.1885        & 1.0000           \\
		\hline
		1739:Evento Irregular                           & 1                & -0.5827        & 1.0000           \\
		\hline
		1744:Ônibus danificado                          & 1                & -0.2220        & 1.0000           \\
		\hline
		1735:Via de terra com desnível                  & 1                & -0.7028        & 1.0000           \\
		\hline
		1725:Bloqueio na via                            & 1                & -0.6911        & 1.0000           \\
		\hline
		1710:Ponto de assalto/roubo                     & 1                & -0.0995        & 1.0000           \\
		\hline
		1704:Calçada inexistente                        & 1                & -0.2403        & 1.0000           \\
		\hline
		10889:Placa de sinalização quebrada/inexistente & 1                & -0.5688        & 1.0000           \\
		\hline
	\end{tabular}
\end{table}

\begin{table}[htbp]
	\centering
	\caption{Métricas de pressão social do tópico de Eleições e Políticos discutidos na \autoref{sec:eventos_populares_polititians}}
	\label{tab:eventos_populares_polititians}
	\begin{tabular}{|l|c|c|c|c|}
		\hline
		\textbf{Tipo de Evento}                         & \textbf{Eventos} & \textbf{Score} & \textbf{Persona} \\
		\hline
		1681:Ponto recorrente de poluição sonora        & 4                & -0.3772        & 1.0000           \\
		\hline
		7:Ponto de infração de trânsito recorrente      & 3                & -0.2115        & 1.0000           \\
		\hline
		1707:Foco de mosquito da dengue/zika            & 2                & 0.2704         & 1.0000           \\
		\hline
		1708:Ponto de tráfico de drogas                 & 2                & -0.2053        & 1.0000           \\
		\hline
		7558:Ocupação irregular de área pública         & 2                & -0.2315        & 1.0000           \\
		\hline
		3352:Comércio aberto irregularmente             & 2                & 0.1643         & 1.0000           \\
		\hline
		1749:Manutenção de semáforo                     & 1                & -0.0244        & 1.0000           \\
		\hline
		7572:Estabelecimento sem nota fiscal            & 1                & -0.3197        & 1.0000           \\
		\hline
		3917:Calçada irregular                          & 1                & -0.1449        & 1.0000           \\
		\hline
		3338:Iluminação pública irregular               & 1                & -0.1885        & 1.0000           \\
		\hline
		1739:Evento Irregular                           & 1                & -0.5827        & 1.0000           \\
		\hline
		1744:Ônibus danificado                          & 1                & -0.2220        & 1.0000           \\
		\hline
		1735:Via de terra com desnível                  & 1                & -0.7028        & 1.0000           \\
		\hline
		1725:Bloqueio na via                            & 1                & -0.6911        & 1.0000           \\
		\hline
		1710:Ponto de assalto/roubo                     & 1                & -0.0995        & 1.0000           \\
		\hline
		1704:Calçada inexistente                        & 1                & -0.2403        & 1.0000           \\
		\hline
		10889:Placa de sinalização quebrada/inexistente & 1                & -0.5688        & 1.0000           \\
		\hline
	\end{tabular}
\end{table}

\begin{table}[htbp]
	\centering
	\caption{Pressão social do tópico Mobilidade Urbana em Niterói}
	\label{tab:eventos_populares_mobility_niteroi}
	\begin{tabular}{|l|c|c|c|c|}
		\hline
		\textbf{Tipo de Evento}                             & \textbf{Eventos} & \textbf{Score} & \textbf{Persona} \\
		\hline
		1725:Bloqueio na via                                & 15               & -0.2238        & 0.6919           \\
		\hline
		7:Ponto de infração de trânsito recorrente          & 15               & -0.2313        & 0.7299           \\
		\hline
		3:Buraco nas vias                                   & 13               & -0.2936        & 0.2642           \\
		\hline
		3938:Entulho na calçada/via pública                 & 12               & -0.2203        & 0.3056           \\
		\hline
		7561:Ponto de alagamento                            & 11               & -0.1653        & 0.4231           \\
		\hline
		7558:Ocupação irregular de área pública             & 11               & -0.2390        & 0.7246           \\
		\hline
		1727:Equipamento público danificado                 & 11               & -0.2255        & 0.4333           \\
		\hline
		1749:Manutenção de semáforo                         & 10               & -0.1963        & 0.4333           \\
		\hline
		3335:Lâmpada apagada à noite                        & 10               & -0.1376        & 0.1657           \\
		\hline
		1681:Ponto recorrente de poluição sonora            & 10               & -0.2128        & 1.0000           \\
		\hline
		3917:Calçada irregular                              & 9                & -0.2251        & 0.2913           \\
		\hline
		1741:Manutenção de ciclovia/ciclofaixa              & 9                & -0.2643        & 0.3487           \\
		\hline
		1704:Calçada inexistente                            & 8                & -0.3224        & 0.5385           \\
		\hline
		3338:Iluminação pública irregular                   & 7                & -0.1418        & 0.0323           \\
		\hline
		7574:Poda de árvore                                 & 7                & -0.1348        & 0.0882           \\
		\hline
		1746:Manutenção/implantação de placa de sinalização & 7                & -0.1914        & 0.2286           \\
		\hline
		1729:Ponto de transporte clandestino                & 7                & -0.0728        & 1.0000           \\
		\hline
		1703:Ponto de travessia irregular                   & 7                & -0.2995        & 0.3846           \\
		\hline
		1675:Bueiro sem tampa                               & 7                & -0.2707        & 0.0833           \\
		\hline
		25:Vazamento de água                                & 7                & -0.3388        & 0.2143           \\
		\hline
	\end{tabular}
\end{table}

\begin{table}[htbp]
	\centering
	\caption{Pressão social do tópico Meio Ambiente em Mesquita}
	\label{tab:eventos_populares_mesquita}
	\begin{tabular}{|l|c|c|c|c|}
		\hline
		\textbf{Tipo de Evento}                         & \textbf{Eventos} & \textbf{Score} & \textbf{Persona} \\
		\hline
		3938:Entulho na calçada/via pública             & 5                & -0.2751        & 0.2188           \\
		\hline
		1700:Ponto de queimada irregular recorrente     & 5                & -0.7196        & 0.2000           \\
		\hline
		1694:Descarte irregular de lixo                 & 4                & -0.3045        & 0.1625           \\
		\hline
		1728:Imóvel ou terreno abandonado               & 4                & -0.6117        & 0.5455           \\
		\hline
		7590:Bueiro entupido                            & 3                & -0.0517        & 0.1212           \\
		\hline
		7574:Poda de árvore                             & 3                & -0.6205        & 0.7333           \\
		\hline
		1686:Emissão de fumaça preta                    & 3                & -0.3486        & 0.2000           \\
		\hline
		1696:Mato alto                                  & 3                & 0.0146         & 0.1250           \\
		\hline
		1727:Equipamento público danificado             & 2                & -0.0626        & 0.2500           \\
		\hline
		1704:Calçada inexistente                        & 2                & -0.2328        & 0.2500           \\
		\hline
		1702:Aterro sanitário irregular                 & 2                & -0.2180        & 0.3000           \\
		\hline
		3:Buraco nas vias                               & 2                & -0.3313        & 0.1429           \\
		\hline
		1734:Retirada de árvore                         & 2                & -0.4139        & 0.0000           \\
		\hline
		9978:Conservação (via pública)                  & 2                & 0.1833         & 0.1538           \\
		\hline
		10889:Placa de sinalização quebrada/inexistente & 1                & 0.3687         & 0.0000           \\
		\hline
		7558:Ocupação irregular de área pública         & 1                & -0.3874        & 0.0000           \\
		\hline
		3917:Calçada irregular                          & 1                & -0.1806        & 0.1667           \\
		\hline
		3356:Ônibus fora do horário/rota                & 1                & -0.6554        & 1.0000           \\
		\hline
		3335:Lâmpada apagada à noite                    & 1                & -0.3444        & 0.0000           \\
		\hline
		3321:Ponto recorrente de animais na via         & 1                & -0.4447        & 1.0000           \\
		\hline
	\end{tabular}
\end{table}

\begin{table}[htbp]
	\centering
	\caption{Pressão social do tópico Saúde Pública em Santo André}
	\label{tab:eventos_populares_sa}
	\begin{tabular}{|l|c|c|c|c|}
		\hline
		\textbf{Tipo de Evento}                               & \textbf{Eventos} & \textbf{Score} & \textbf{Persona} \\
		\hline
		3353:Aglomeração de pessoas                           & 13               & -0.2489        & 0.9375           \\
		\hline
		1707:Foco de mosquito da dengue/zika                  & 11               & -0.2663        & 0.2468           \\
		\hline
		1682:Estabelecimento com condição sanitária irregular & 10               & -0.2825        & 0.6190           \\
		\hline
		1694:Descarte irregular de lixo                       & 10               & -0.2928        & 0.2857           \\
		\hline
		1681:Ponto recorrente de poluição sonora              & 10               & -0.3996        & 0.9518           \\
		\hline
		1706:Infestação animais perigosos                     & 9                & -0.2904        & 0.2340           \\
		\hline
		10901:Esgoto a céu aberto                             & 9                & -0.2725        & 0.2037           \\
		\hline
		7590:Bueiro entupido                                  & 8                & -0.1304        & 0.1429           \\
		\hline
		7558:Ocupação irregular de área pública               & 8                & -0.3090        & 0.6364           \\
		\hline
		1683:Estabelecimento sem alvará                       & 6                & -0.3304        & 1.0000           \\
		\hline
		1685:Estabelecimento com acessibilidade irregular     & 6                & -0.4091        & 1.0000           \\
		\hline
		7561:Ponto de alagamento                              & 5                & -0.2627        & 0.3939           \\
		\hline
		1732:Área com risco de deslizamento                   & 5                & -0.4430        & 0.6667           \\
		\hline
		1697:Maus tratos a animais                            & 5                & -0.2470        & 0.6000           \\
		\hline
		3338:Iluminação pública irregular                     & 3                & -0.2093        & 0.2500           \\
		\hline
		1708:Ponto de tráfico de drogas                       & 3                & -0.2767        & 0.7143           \\
		\hline
		17:Rampa de acessibilidade irregular ou inexistente   & 3                & -0.2713        & 0.5000           \\
		\hline
		3356:Ônibus fora do horário/rota                      & 2                & -0.2401        & 1.0000           \\
		\hline
		3956:Faixa de pedestre apagada                        & 2                & -0.2330        & 0.0000           \\
		\hline
		1726:Patrimônio histórico em risco                    & 2                & -0.6174        & 0.5000           \\
		\hline
	\end{tabular}
\end{table}

\end{apendicesenv}
% ---


% ----------------------------------------------------------
% Anexos
% ----------------------------------------------------------

% ---
% Inicia os anexos
% ---
%\begin{anexosenv}

%    \chapter{Páginas interessantes na Internet} 
%    \label{chapter:paginas-interessantes}
%    \input{tex/annex/paginas-interessantes}

%\end{anexosenv}
% ---

\end{document}